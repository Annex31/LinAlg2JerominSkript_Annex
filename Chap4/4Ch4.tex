\section{Das charakteristische Polynom}
\subsection{Definition}\index{Eigenwert,-vektor,-raum}
\begin{Definition}[Eigenwert,Eigenvektor,Eigenraum]
	Seien $ V $ ein $ K $-VR und $ f\in\End(V) $. Dann heißen
		\begin{enumerate}[(i)]
			\item $ x\in K $ ein Eigenwert von $ f $, falls
				\[ \exists v\in V^\times: f(v)=vx; \]
			\item $ v\in V^\times $ ein Eigenvektor von $ f $, falls
				\[ \exists x\in K:f(v)=vx; \]
			\item $ \ker(f-\id_Vx) \subset V $ ein Eigenraum, falls
				\[ \ker(f-\id_Vx) \neq \{0\}.\]
		\end{enumerate}
	\end{Definition}
\paragraph{Bemerkung}
	Der Skalar $ x\in K $ ist genau dann ein Eigenwert von $ f\in \End(V) $, wenn $ \ker(f-\id_Vx)\neq \{0\} $, d.h., wenn ein Eigenvektor $ v\in V^\times $ zu $ x $ existiert.
\paragraph{Beispiel}
	Für $ \frac{d}{ds} \in \End(C^\infty(\mathbb{R}))$ ist jedes $ x\in \mathbb{R} $ ein Eigenwert, da
		\[ \Big(\frac{d}{ds}-\id_Vx\Big)v = 0 \text{ für } v:\mathbb{R}\to\mathbb{R},s\mapsto v(s):= e^{xs}, \]
	wobei $ v\in C^\infty(\mathbb{R})\setminus \{0\} $, d.h. $ s\mapsto v(s)=e^{xs} $ ist ein Eigenvektor zum Eigenwert $ x\in\mathbb{R} $.
\paragraph{Beispiel}
	Ist $ \dim V < \infty $, so kann die Determinante zur Bestimmung von Eigenwerten von Endomorphismen $ f\in\End(V) $ benutzt werden, da
		\[ \ker(f-\id_Vx)\neq \{0\} \Leftrightarrow (f-\id_Vx) \text{ nicht injektiv}\Leftrightarrow \det(f-\id_Vx) = 0, \]
	d.h. das Auffinden von Eigenwerten $ x\in K $ von $ f $ ist reduziert auf die Bestimmung der Nullstellen der Funktion
		\[ K\ni x\mapsto \det(f-\id_Vx)\in K. \]
		
\paragraph{Beispiel}	
	Ist z.B. $ (b_1,b_2) $ Basis von $ V $ und $ f\in \End(V) $ durch $ f(B)=BX $ gegeben, so liefern die Nullstellen der Polynomfunktion
		\begin{gather*}
		\det(f-\id_Vx) = \det(X-E_2 x)= \det \begin{pmatrix}
		x_{11}-x & x_{12}\\
		x_{21} & x_{22} -x
		\end{pmatrix}\\
	= (x_{11}-x)(x_{22}-x)-x_{12}x_{21}
	= x^2 - x(x_{11}+x_{22}) + (x_{11}x_{22}-x_{12}x_{21})
		\end{gather*}
	die Eigenwerte von $ f $ -- beispielsweise erhalten wir für
		\[ X = \begin{pmatrix} 2 &3\\1 & 0 \end{pmatrix}:\ 
			\det(f-\id_Vx) = x^2-2x-4 = (x+1)(x-3), \]
	also Eigenwerte $ x_1 = -1 $ und $ x_2 = 3 $ mit zugehörigen Eigenvektoren als Lösungen von
		\[ v_i \in \ker(f-\id_Vx_i), \]
	also durch Lösungen der linearen Gleichungssysteme
		\[ \begin{pmatrix}
		2-(-1) & 3\\ 1 & -(-1)
		\end{pmatrix}
		\begin{pmatrix}
		v_1^1\\v_1^2
		\end{pmatrix} = \begin{pmatrix}
		3 & 3\\ 1 & 1
		\end{pmatrix}
		\begin{pmatrix}
		v_1^1\\v_1^2
		\end{pmatrix} \text{ und} \]
		\[ \begin{pmatrix}
		2-3 & 3\\ 1 & -3
		\end{pmatrix}
		\begin{pmatrix}
		v_2^1\\v_2^2
		\end{pmatrix}=
		\begin{pmatrix}
		-1 & 3\\ 1 & -3
		\end{pmatrix}
		\begin{pmatrix}
		v_2^1\\v_2^2
		\end{pmatrix}  \]
	sodass
		\[ v_1 = b_1-b_2 \text{ und } v_2 = b_13+b_2 \]
	Eigenvektoren zu den Eigenwerten $ x_1,x_2 $ liefert.

\paragraph{Rechenbeispiel 1}
	Für $ X = \begin{pmatrix}2&-1\\1&0\end{pmatrix} $ erhält man
		\[ \det(f-\id_Vx) = \det\begin{pmatrix}2-x&-1\\1&-x	\end{pmatrix} =x^2-2x+1 \]
	und Eigenvektoren zum Eigenwert $ x = 1 $ durch Lösung der LGS
		\[ \begin{pmatrix}
		2-1&-1\\1&-1
		\end{pmatrix}\begin{pmatrix}
		v_1^1\\v_1^2
		\end{pmatrix} =  \begin{pmatrix}
		1&-1\\1&-1
		\end{pmatrix}\begin{pmatrix}
		v_1^1\\v_1^2
		\end{pmatrix} \]
	d.h. der Eigenraum zum Eigenwert $ x $,
		\[ \ker(f-\id_V) = [\{b_1+b_2\}] \]
	hat
		\[ \dim \ker(f-\id_V)<\dim V. \]
\paragraph{Rechenbeispiel 2}
	Ist $ K=\mathbb{R} $ und
		\[ \det(f-\id_Vx)=x^2+1, \]
	so hat $ f $ keine Eigenwerte: z.B., wenn
		$ X=\begin{pmatrix} 0&1\\-1&0 \end{pmatrix} $.
		
\subsection{Definition} \index{Charakteristisches Polynom}
\begin{Definition}[Charakteristisches Polynom]
	Sei $ V $ ein $ K $-VR, für $ f\in\End(V) $ ist das \emph{charakteristische Polynom} von $ f $:
		\[ \chi_f(t) := \det (\id_Vt-f)\in K[t]. \]
	Analog definiert man für $ X\in K^{n\times n} $ das charakteristische Polynom
		\[ \chi_f(t) := \det (E_nt-X)\in K[t]. \]
\end{Definition}
\paragraph{Bemerkung}
	Oft wird auch das andere Vorzeichen in der Determinante verwendet, also $ \det(f-\id_Vt) $ bzw. $ \det(X-E_nt) $.
\paragraph{Bemerkung}
	\emph{Diese Definition ist erklärungsbedürtig!}
	
	Da $ t\notin K $ ist $ \id_Vt-f\notin \End(V) $, sondern $ \id_Vt-f\in\End(V)[t] $. Zwei Lösungsstrategien bieten sich an:
		\begin{enumerate}
			\item Erweiterung der Determinante auf $ \End(V)[t] $.
			\item Benutzung von Darstellungsmatrizen.
		\end{enumerate}
	Beide führen schließlich zur Leibniz-Formel:
	
	Ist $ B $ eine Basis von $ V $ und $ \xi_B^B(f) = X = (x_{ij})_{i,j\in\{1,\dots,n\}}$, so erhält man 
		\[ \chi_f(t)=\sum_{\sigma\in S_n}\sgn(\sigma)\prod_{j=1}^{n}\underset{\in K[t]}{\underbrace{\left(\delta_{\sigma(j)j}-x_{\sigma(j)j}\right)}} \in K[t]. \]
	Die Unabhängigkeit von der Basis $ B $ folgt aus der Transformationsformel für Darstellungsmatrizen und dem Determinanten-Multiplikationssatz (wie vorher für $ \det f = \det \xi_B^B(f) $).