\section{Der Satz von Cayley-Hamilton}
\subsection{Satz}
	Für $ f\in\End(V) $ gilt $ \chi_f(f) = 0$.
\paragraph{Unfug-Beweis}
	Durch direktes Einsetzen erhält man
		\[ \chi_f(f)=\det (\id_V f-f) = \det 0 = 0. \]
\paragraph{Zum Verständnis des Satzes}
	Ist $ V $ ein $ K $-VR mit $ n=\dim V < \infty $ und $ f\in \End (V) $, so ist
		\[ \chi_f(t) = \sum_{k=0}^{n}t^ka_k \in K[t] \]
	ein (abstraktes) Polynom in der Variablen $ t\ (= e_1\in K^\mathbb{N})$ und der Einsetzungshomomorphismus $ \psi_f:K[t]\to \End(V) $ (also ein Algebrahomomophismus) liefert
		\[ \chi_f(f) = \psi_f\left(\chi_f(t)\right) = \sum_{k=0}^{n}f^k a_k. \]
	Der Satz sagt, dass $ 0 = \chi_f(f)\in \End(V) $, d.h.
		\[ \forall v\in V: \chi_f(f)(v) = 0. \]
		
\subsection{Definition \& Lemma}\index{$ f $-zyklische Basis}
\begin{Definition}[$ f $-zyklische Basis]
	Seien $ f\in\End(V) $ und $ B $ eine \emph{$ f $-zyklische Basis} von $ V $, d.h. eine Basis der Form
		\[ B= (b_1,\dots,b_n) = \left(b,f(b),\dots,f^{n-1}(b)\right). \]		
\end{Definition}

\begin{Lemma}
	Dann existieren $ a_0,\dots,a_{n-1}\in K $ mit
		\[ f^n(b)+\sum_{k=0}^{n-1}f^k(b)a_k = 0, \]
	mit diesen Koeffizienten ist
		\[ \chi_f(t) = t^n+t^{n-1}a_{n-1}+\dots,+ta_1+a_0. \]
\end{Lemma}
\paragraph{Bemerkung}
	Im Allgemeinen existiert zu $ f\in \End(V) $ keine $ f $-zyklische Basis von $ V $, z.B. für $ f = \id_V $ und $ \dim V \geq 2 $.
\paragraph{Beweis}
	Da $ B= \left(b,f(b),\dots,f^{n-1}(b)\right) $ eine Basis ist, ist $ f^n(b)\in [B] $ und damit existieren die $ a_k $ mit
		\[ 0 = f^n(b) + \sum_{k=0}^{n-1}f^k(b)a_k. \]
	Damit ist die Darstellungsmatrix von $ f $
		\[ \xi_B^B(f) =
		\begin{pmatrix}
		0 & 0 & \cdots & 0 & -a_0 \\ 
		1 & 0 & \vdots & \vdots & -a_1 \\ 
		0 & 1 & \ddots & \vdots & \vdots \\ 
		\vdots & \vdots & \ddots & 0 & \vdots \\ 
		0 & \cdots & 0 & 1 & -a_{n-1}
		\end{pmatrix} =: X\]
	und Entwicklung von $ \chi_f(t)=\det(E_nt-\xi_B^B(f)) $ nach der ersten Zeile(nach Laplaceschem Entwicklungssatz -- dieser Satz war "`nur"' eine Methode, die Terme in der Leibniz-Formel zu sortieren) liefert
		\[ \det(E_nt-X) = \det 
		\begin{pmatrix}
		t & 0 & \cdots & 0 & a_0 \\ 
		-1 & t & \vdots & \vdots & a_1 \\ 
		0 & -1 & \ddots & \vdots & \vdots \\ 
		\vdots & \vdots & \ddots & t & \vdots \\ 
		0 & \cdots & 0 & -1 & t+a_{n-1}
		\end{pmatrix} \]
		\[ = t \det \begin{pmatrix}
		t & 0 & \cdots & 0 & a_1 \\ 
		-1 & t & \vdots & \vdots & a_2 \\ 
		0 & -1 & \ddots & \vdots & \vdots \\ 
		\vdots & \vdots & \ddots & t & \vdots \\ 
		0 & \cdots & 0 & -1 & t+a_{n-1}
		\end{pmatrix} + (-1)^{n+1} a_0 \det(X_{1n}) \]
		\[ \overset{\text{mit Ind.}}{=} t \{t^n{n-1}+t{n-2}a_{n-1}+\dots+a_1\}+a_0 = t^n+t^{n-1}a_{n-1}+\dots,+ta_1+a_0, \]
	wie behauptet.
	
\paragraph{Beispiel}
	Zur Lösung des reellen \emph{Anfangswertproblems}
		\[ y'' + 2y' - 3y = 0,\ 
		\begin{cases}
			y(0)=4 \\
			y'(0)=0
		\end{cases} \]
	schreiben wir dieses als System erster Ordnung mit dem Ansatz $ y_1 = y \text{ und }y_2 = y' $:
	
	Daraus erhält man mit $ Y= (y_1,y_2) $
	  \begin{align*}
		 Y' &= (y_1',y_2') = (y',y'') = (y',-2y'+3y)\\
		 &= (y,y') \begin{pmatrix}
		 	0 & 3\\ 1 & -2
		 \end{pmatrix} = YX
	  \end{align*}
	mit $ X= \begin{pmatrix}
	0 & 3\\ 1 & -2
	\end{pmatrix}$, d.h. wir suchen eine $ \frac{d}{ds} $-zyklische Basis $ (y,\frac{d}{ds}y) = (y,y') $ eines 2-$ \dim $ UVR $ [(y,y')]\subset C^\infty(\mathbb{R}) $ bezüglich derer $ \frac{d}{ds}\in \End(C^\infty(\mathbb{R})) $ Darstellungsmatrix $ X $ hat.
	
	Der Ansatz $ y(s) = e^{xs} (v_0,v_1)$ reduziert das AWP auf ein Eigenwertproblem.
		\[ 0 = \left(Y'-YX\right)(s) = \left(\frac{d}{ds}Y - YX\right)(s) = \underset{Y}{\underbrace{e^{xs}(v_0,v_1)}} \{E_2x-X\}\]
	bzw. (vgl. Abschnitt 3.1) mit dem zur transponierten Matrix $ X^t $ assoziierten Endomorphismus $ f_{X^t}\in \End(\mathbb{R}^2) $
		\[ f_{X^t}(v) = vx \text{ für }x\in\mathbb{R} \text{ und }v\in \mathbb{R}^2. \]
	Nach obigem Lemma sind die Eigenwerte Lösungen der Gleichung
		\[ 0 = \chi_{X^t}(x) = \chi_X(x) \overset{\text{Lemma}}{=} x^2+2x-3 = (x-1)(x+3). \]
	Also sind $ x_1 = 1 $ und $ x_2 = -3 $ die Eigenwerte; zugehörige Eigenvektoren erhält man als Lösungen der linearen Gleichungssysteme
		\[ (0,0) = (v_0,v_1)(E_2x_i-X)= (v_0,v_1)\begin{pmatrix}
		x_i&-3\\-1&x_i+2
		\end{pmatrix} = \begin{cases}
		(v_0,v_1)\begin{pmatrix}
		1&-3\\-1&3
		\end{pmatrix}& \text{ für } i = 1\\
		(v_0,v_1)\begin{pmatrix}
		-3&-3\\-1&-1
		\end{pmatrix}& \text{ für } i = 2
		\end{cases} \]
	Damit bekommt man Eigenvektoren $ (v_0,v_1) = (1,1) $ zum Eigenwert $ x=1 $ und $ (v_0,v_1) = (1,-3) $ zum Eigenwert $ x = -3 $.
	
	Die allgemeine, durch \emph{Superposition} (Linearkombination) erhaltene Lösung der Differentialgleichung ist also
		\[ s\mapsto Y(s) = e^s(1,1)c_1 + e^{-3s}(1,-3)c_2 \]
	mit Koeffizienten $ c_1,c_2 \in \mathbb{R} $. Abgleich der "`Integrationskonstanten"' $ c_1 $ und $ c_2 $ mit den Anfangsbedingungen liefert dann die Lösung
		\[ s \mapsto y(s) = 3e^s+e^{-3s}. \]
\paragraph{Bemerkung}
	Man bemerke: $ (y,y') $ ist linear unabhängig für die Lösung, ist also tatsächlich $ \frac{d}{ds} $-zyklische Basis eines 2-$ \dim $ URs $ [(y,y')]\subset C^\infty(\mathbb{R}) $ -- obwohl die den gleichen Raum aufspannenden "`Basislösungen"'
		\[ s\mapsto e^s \text{ und } s\mapsto e^{-3s} \]
	keine $ \frac{d}{ds} $-zyklischen Basen erzeugen, da sie lineare Differentialgleichungen erster Ordnung (mit konstanten Koeffizienten) lösen.
	
\subsection{Korollar}
\begin{Korollar}
	Besitzt $ V $ eine $ f $-zyklische Basis für $ f\in\End(V) $, so gilt $ \chi_f(f)=0 $.
\end{Korollar}
\paragraph{Beweis}
	Sei also $ B=(b_1,\dots,b_n) =(b,f(b),\dots,f^{n-1}(b)) $ $ f $-zyklische Basis von $ V $ und $ a_0,\dots,a_{n-1}\in K $ so, dass
		\[ 0 = f^n(b)+\sum_{k=0}^{n-1}f^k(b)a_k. \]
	Dann gilt
		\[ \chi_f(f)(b_1) = \chi_f(f)(b) = \left(f^n+\sum_{k=0}^{n-1}f^ka_k\right)(b) = f^n(b)+\sum_{k=0}^{n-1}f^k(b)a_k. \]
	Damit folgt für $ i=2,\dots,n $
		\[ \chi_f(f)(b_i) = \chi_f(f)\left(f^{i-1}(b)\right) \overset{\footnotemark}{=} f^{i-1}\left(\chi_f(f)(b) \right) = 0. \]
		\footnotetext{Aufgrund der Linearität der Endomorphismen $ \End(V) $ als unitäre Algebra.}
	Da also $ V=[B] \subset \ker {\chi_f(f)}$, folgt $ \chi_f(f) = 0. $
\paragraph{Bemerkung}
	Damit ist der Satz von Cayley-Hamilton bewiesen, sofern $ V $ eine $ f $-zyklische Basis besitzt.
	
\subsection{Lemma}
\begin{Lemma}[ $ f $-invarianter UVR endlicher Dimension besitzen eine $ f $-zyklische Basis]
	Für $ f\in\End(V) $ und $ v\in V^\times  $ sei
		\[ U := \left[\left(f^k(v)\right)_{k\in{\mathbb{N}}}\right]. \]
	Damit ist $ U $ ein $ f $-invarianter UVR von $ V $. Ist $ \dim V < \infty $, so besitzt $ U $ eine $ f $-zyklische Basis $ \left(v,f(v),\dots,f^{r-1}(v)\right) $.
\end{Lemma}
\paragraph{Beweis}
	Offenbar ist $ U $ $ f $-invarianter UR:
	\begin{itemize}
		\item $ U $ ist (als lineare Hülle einer Familie) ein UVR von $ V $;
		\item da gilt
			\[ \forall k\in \mathbb{N}: f\left(f^k(v)\right)=f^{k+1}(v)\in U \]
			folgt, dass $ f(U) = f\left(\left[\left(f^k(v)\right)_{k\in{\mathbb{N}}}\right]\right) = \left[\left(f^{k+1}(v)\right)_{k\in{\mathbb{N}}}\right]\subset U. $
	\end{itemize}
	Ist $ \dim V < \infty $ und $ v\neq 0 $, so existiert $ r\in \mathbb{N} $, sodass
		\[ \left(v,\dots,f^{r-1}(v)\right) \text{ linear unabhängig und }f^r(v)\in\left[\left(v,\dots,f^{r-1}(v)\right)\right]; \]
	damit ist $ \left(v,f(v),\dots, f^{r-1}(v)\right) $ $ f $-zyklische Basis von $ U $:
	\begin{enumerate}
		\item $ \left(v,\dots,f^{r-1}(v) \right) $ ist linear unabhängig.
		\item $ f^r(v) \in \left[\left(v,\dots,f^{r-1}(v)\right)\right]$, damit gilt
		\[ \forall k\in \mathbb{N}: k\geq r \Rightarrow f^k(v)\in \left[\left(v,\dots,f^{r-1}(v)\right)\right] \]
	\end{enumerate}
	wie man z.B. mit Induktion sehen kann: ist
		\[ f^{k-1}(v) = \sum_{j=0}^{r-1}f^j(v)x_j \in \left[\left(v,\dots,f^{r-1}(v)\right) \right], \]
	so folgt
		\[ f^k(v) = \sum_{j=1}^{r}f^j(v)x_{j-1}=f^{r}(v)x_{r-1}+\sum_{j=1}^{r-1}f^j(v)x_{j-1}\in \left[\left(v,\dots,f^{r-1}(v)\right)\right] \]
	und damit
		\[ U = \left[\left(f^k(v)\right)_{k\in \mathbb{N}}\right]\in \left[\left(v,\dots,f^{r-1}(v)\right)\right]. \]
		
\subsection{Beweis vom Satz von Cayley-Hamilton}
	Zu zeigen: für $ f\in \End(V) $ gilt $ \chi_f(f)=0 $:
		\[ \forall v\in V:\chi_f(f)(v) = 0. \]
	Sei also $ v\in V^\times $ und
		\[ U := \left[\left( f^k(v)_{k\in \mathbb{N}}\right)\right]\subset V. \]
	Mit einem zu $ U $ komplementären UVR $ U'\subset V $, $ V=U\oplus U' $, und den zugehörigen Projektionen
	
	\begin{multicols}{2}
	%------------------ Projektion ----------------
 	\begin{figure}[H]\centering
 		\tdplotsetmaincoords{0}{0} %-27
 		\begin{tikzpicture}[xscale=0.5,yscale=0.5,tdplot_main_coords]

 				\def\xstart{0} %x Koordinate der Startposition der Grafik
 				\def\ystart{-3} %y Koordinate der Startposition der Grafik
 				\def\myscale{0.5} %ändert die Größe der Grafik (Skalierung der Grafik)
                \def\myscalex{1.0}
                \def\myscaley{0.6}
                \def\maxlh{6.0}
                \def\maxlv{6.0}
                
 				\def\xstartdraw{(\xstart + \maxlh)} %xKoordinate des Referenzstartpunktes (in dieser Zeichnung: a)
 				\def\ystartdraw{(\ystart + \maxlv)}%yKoordinate des Referenzstartpunktes (in dieser Zeichnung: a)

 			    \node (pointro) at ({\xstartdraw+(\maxlh*\myscalex)},{\ystartdraw+(\maxlv*\myscaley)}) {};
 			    
			    \node (pointlu) at ({\xstartdraw-(\maxlh*\myscalex)},{\ystartdraw-(\maxlv*\myscaley)}) {};
			    \node (pointlo) at ({\xstartdraw-(\maxlh*\myscalex)},{\ystartdraw+(\maxlv*\myscaley)}) {};
                \node (pointru) at ({\xstartdraw+(\maxlh*\myscalex)},{\ystartdraw-(\maxlv*\myscaley)}) {};
                
 				%\node (pointo1) at ($(pointol)!0.2!(pointor)$) {};
 				%\node (pointo2) at ($(pointol)!0.9!(pointor)$) {};

 				\node (offsetx) at ({(3.0*\myscalex},{0.0}) {}; %just an offset
 				\node (offsety) at ({0.0},{3.0*\myscaley}) {}; %just an offset
                
                \node (unity) at ({\maxlh*0.25*\myscalex},{\maxlv*0.25*\myscaley}) {}; %just an offset
                \node (unitx) at ({\maxlh*0.25*\myscalex},{-\maxlv*0.25*\myscaley}) {}; %just an offset
                
               	\node (point00) at ($(pointlo) + 4.0*(unitx) - 1.0*(unity)$) {};
 				%\draw[fill,color=blue] (point00) circle [radius=0.18];
 				
                 
 				%Koordinatenkreuz
 				\draw[-,line width=0.2pt,color=black] ($(point00) -3.0*(unity)$) -- ($(point00) +5.0*(unity)$);
 				\draw[-,line width=0.2pt,color=black,shorten >=-20pt] ($(point00) -3.0*(unitx)$) -- ($(point00) + 3.0*(unitx)$);
 				
 				%Q2 dotted vertikal
 				\draw[-,line width=0.4pt,color=red,dotted] ($(point00) -2.0*(unitx) -2.0*(unity)$ + ) -- ($(point00) -2.0*(unitx) +3.0*(unity)$);
 				\draw[-,line width=0.4pt,color=red,dotted] ($(point00) -1.0*(unitx) -2.0*(unity)$ + ) -- ($(point00) -1.0*(unitx) +4.0*(unity)$);
 				
 				%Q1 dotted vertikal
 				\draw[-,line width=0.4pt,color=red,dotted] ($(point00) + 2.0*(unitx) -2.0*(unity)$ + ) -- ($(point00) + 2.0*(unitx) +5.0*(unity)$);
 				\draw[-,line width=0.4pt,color=red,dotted] ($(point00) +1.0*(unitx) -2.0*(unity)$ + ) -- ($(point00) +1.0*(unitx) +5.0*(unity)$);
 				
 				%Q1 und Q2 dotted horizontal
 				\draw[-,line width=0.4pt,color=red,dotted] ($(point00) - 1.0*(unitx) +4.0*(unity)$ + ) -- ($(point00) + 3.0*(unitx) +4.0*(unity)$);
 				\draw[-,line width=0.4pt,color=red,dotted] ($(point00) - 2.0*(unitx) +3.0*(unity)$ + ) -- ($(point00) + 4.0*(unitx) +3.0*(unity)$);
 				\draw[-,line width=0.4pt,color=red,dotted] ($(point00) - 3.0*(unitx) +2.0*(unity)$ + ) -- ($(point00) + 4.0*(unitx) +2.0*(unity)$);
 				\draw[-,line width=0.4pt,color=red,dotted] ($(point00) - 3.0*(unitx) +1.0*(unity)$ + ) -- ($(point00) + 4.0*(unitx) +1.0*(unity)$);	
 				\draw[-,line width=0.4pt,color=red,dotted] ($(point00) - 3.0*(unitx) -1.0*(unity)$ + ) -- ($(point00) + 3.0*(unitx) -1.0*(unity)$);	
 				
 				%Vektoren blau
 				\node (point03) at ($(point00) +3.0*(unity)$) {};
               	\node (point20) at ($(point00) +2.0*(unitx)$) {};
               	\node (point23) at ($(point00) +2.0*(unitx) +3.0*(unity) $){};
 				
 				\draw[-{>[scale=1,length=8,width=6]},shorten >=-5pt,line width=0.5pt,color=red] (point00) -- (point03);
 				\draw[-{>[scale=1,length=8,width=6]},shorten >=-5pt,line width=0.5pt,color=red] (point00) -- (point20);
 				\draw[-{>[scale=1,length=8,width=6]},shorten >=-5pt, shorten <=-5pt,line width=0.9pt,color=blue] (point00) -- (point23);
 				
 			    \node (pointvekw) at (point23) [above,color=blue]{$v$};
 			    \node (pointvekw) at (point20) [xshift=-20,color=red]{$p(v)$};
 			    \node (pointvekw) at (point03) [yshift= 10,xshift=-5,color=red]{$p'(v)$};
 			    \node (pointvekw) at ($(point00) -3.0*(unitx)$) [yshift= 8,color=green]{$U$};
 			    \node (pointvekw) at ($(point00) +5.0*(unity)$) [yshift=-5,xshift= 5,color=green]{$U'$};
 			\end{tikzpicture}
	\end{figure}
	%------------------ Projektion ----------------
	    \begin{align*}
		    &p: V\to V, p(V) = U, \ker p = U' \text{ bzw. } \\
		    &p':V\to V, p'(V) = U', \ker p' = U,
	    \end{align*}
	\end{multicols}
	ist dann
		\[ \chi_f(t) = \chi_{f'}(t)\cdot \chi_{f\mid_U}(t) \text{ mit } f' := p'\circ f\mid_{U'}\in \End(U'). \]
	Damit folgt
		\[ \chi_f(f)(v) = \chi_{f'}(f) \left(\chi_{f\mid_U}(f)(v) \right) = \chi_{f'}(f)(0)=0 \]
	nach Korollar oben, da $ U $ eine $ f $-zyklische Basis besitzt und $ v\in U $.
	
% % VO-12-04-2016 % % 
\subsection{Definition}\index{Annulatorpolynom}\index{Minimalpolynom}
\begin{Definition}[Annulatorpolynom, Minimalpolynom]
	Sei $ V $ ein $ K $-VR und $ f\in\End(V) $. Dann heißt $ p\in K[t] $
		\begin{itemize}
			\item \emph{Annulatorpolynom von $ f $}, falls $ p(f)=0 $;
			\item \emph{Minimalpolynom von $ f $}, falls $ p(t) $ normiertes Annulatorpolynom minimalen Grades ist.
		\end{itemize}
\end{Definition}
\paragraph{Bemerkung}
	Jedes (polynomiale) Vielfache 
		\[ p(t) = q(t)\mu_f(t)\in K[t] \]
	eines Minimalpolynoms $ \mu_f(t) $ von $ f $ ist ein Annulatorpolynom, da
		\[ \forall v\in V: p(f)(v) = \left(q(f)\circ \mu_f(f)\right)(v) = q(f)\left(\mu_f(f)(v)\right) = q(f)(0) = 0 \]
\paragraph{Bemerkung}
	Nach dem Satz von Cayley-Hamilton hat jeder Endomorphismus $ f\in\End(f) $ ein Annulatorpolynom, also auch ein Minimalpolynom -- wenn $ \dim V < \infty $.
	
\subsection{Lemma}
\begin{Lemma}[Jedes Minimalpolynom ist Teiler des Annulatorpolynom von $ f\in\End(V) $]
	Ist $ p(t)\in K[t] $ Annulatorpolynom von $ f\in\End(V) $, so ist jedes Minimalpolynom $ \mu_f(t)\in K[t] $ Teiler von $ p(t) $. 
\end{Lemma}

\paragraph{Beweis}
	Seien $ q(t),r(t)\in K[t] $ die (nach dem euklidischen Divisionsalgorithmus) eindeutigen Polynome mit
		\[ p(t) = q(t)\mu_f(t)+r(t) \text{ und }\deg r(t)<\deg \mu_f(t). \]
	Dies liefert
		\[ r(f) = p(f)-q(f)\circ \mu_f(f) = 0-q(f)(0) = 0, \]
	also $ r(t) = 0 $, denn andernfalls wäre $ \mu_f(t) $ nicht normiertes Annulatorpolynom minimalen Grades.
	
\subsection{Korollar}
\begin{Korollar}
	Das Minimalpolynom $ \mu_f(t)\in K[t] $ eines Endomorphismus $ f\in \End(V) $ ist eindeutig.
\end{Korollar}
\paragraph{Beweis}
	Sind $ \mu_f(t),\tilde{\mu}_f(t)\in K[t] $ Minimalpolynome von $ f\in \End(V) $, so gilt
		\[ \exists! q(t)\in K[t] : \tilde{\mu}_f(t) = q(t)\mu_f(t) \] % Nach Lemma 4.5.7
	wobei
		\begin{itemize}
			\item $ \deg q(t) = 0 $, da $ \deg \tilde{\mu}_f(t) \leq \deg \mu_f(t) $,
			\item $ q(t) = 1$, da $ \tilde{\mu}_f(t) $ und $ \mu_f(t) $ normiert sind.
		\end{itemize}
	Daher ist
		\[ \tilde{\mu}_f(t) = 1\cdot \mu_f(t) = \mu_f(t). \]
\paragraph{Bemerkung}
	Wie für Endomorphismen kann Annulatorpolynome, Minimalpolynome, usw. auch für Matrizen $ X\in K^{n\times n} $ definieren:
		\begin{itemize}
			\item mithilfe der assoziierten Endomorphismen $ f_X\in \End(K^n) $, oder 
			\item mithilfe des Einsetzungshomomorphismus $ \psi_X: K[t] \to K^{n\times n}. $ % in der Algebra der quadratischen Matrizen
		\end{itemize}
	Beide Methoden liefern das gleiche Ergebnis durch den Algebrahomomorphismus zwischen den Endomorphismen und den quadratischen Matrizen.

\paragraph{Bemerkung \& Beispiel}
	Zerfällt das charakteristische Polynom in Linearfaktoren, so zerfällt auch das Minimalpolynom in dieselben Linearfaktoren:
		\[ \chi_f(t)= \prod_{i=1}^{m}(t-x_i)^{k_i} \Rightarrow \mu_f(t) = \prod_{i=1}^{m}(t-x_i)^{m_i}, \]
	wobei für $ i= 1,\dots,m $ gilt $ 1\leq m_i\leq k_i $.
	
	Zum Beispiel: 
		\begin{itemize}
			\item $ X = \begin{pmatrix}
			1&0\\0&0
			\end{pmatrix} $: $ \chi_{f_X} = t(t-1) = \mu_{f_X}(t)$
			\item $ X = \begin{pmatrix}
			1&0\\0&1
			\end{pmatrix} $: $ \chi_{f_X} = (t-1)^2 \Rightarrow \mu_{f_X}(t) = (t-1) $
			\item $ X = \begin{pmatrix}
			1&1\\0&1
			\end{pmatrix} $: $ \chi_{f_X} = (t-1)^2 = \mu_{f_X}(t)$.
		\end{itemize}

\paragraph{Bemerkung}
	Die Definition des charakteristischen Polynoms ist etwas problematisch:
		\[ \chi_f(t) := \det (\id_Vt-f) \]
	ist "`gut"' für Polynomfunktionen, aber "`nicht korrekt"' für abstrakte Polynome; die Definition 
		\[ \chi_f(t) := \sum_{\sigma\in S_n}\sgn(\sigma)\prod_{i=1}^{n}\left(\delta_{\sigma(j)j}-x_{\sigma(j)j} \right)\in K[t] \]
	mithilfe der Darstellungsmatrix
		\[ X = (x_{ij})_{i,j\in \{1,\dots,n\}} = \xi_B^B(f) \]
	von $ f $ bzgl. einer Basis $ B $ und der Leibniz-Formel ist nicht sehr übersichtlich. Vergleiche auch [Axler, Kap. 8] zum Thema.
	
	Im Gegensatz dazu: Definitionen von "`Annulatorpolynom"' und "`Minimalpolynom"' etc. sind einfach (konzeptionell).
	
	Frage: Braucht man das charakteristische Polynom überhaupt?
	
	Man kommt auch ohne das charakteristische Polynom "`recht weit"':
		\begin{itemize}
			\item Für $ \dim V <\infty $ folgt die Existenz eines Annulatorpolynoms, und damit des Minimalpolynoms recht einfach wegen $ \dim \End(V) <\infty $.
			\item Durch Einsetzen: Jeder Eigenwert eines Endomorphismus ist Nullstelle seines Minimalpolynoms.
			\item Umgekehrt ist auch jede Nullstelle des Minimalpolynoms Eigenwert -- ist $ \mu_f(x) = 0 $, so existiert $ q(t)\in K[t] $ mit
				\[ \mu_f(t) = q(t)(t-x); \]
			wäre $ x $ kein Eigenwert, also $ f-\id_V x \in Gl(V) $, so gälte
				\[ (f-\id_Vx)(V) = V \Rightarrow \{0\} = \mu_f(f)(V) = q(f)(V), \]
			d.h. $ \mu_f(t) $ wäre nicht Minimal-Polynom.
			\item Ein Endomorphismus ist diagonalisierbar, wenn sein Minimal-Polynom in paarweise verschiedene Linearfaktoren zerfällt.
		\end{itemize}
	
	Nachteil des Minimal-Polynoms: schwierig berechenbar?