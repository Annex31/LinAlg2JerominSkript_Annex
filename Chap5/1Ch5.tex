% VO 14-04-2016 %
\chapter{Längen- und Winkelmessung}
Plan: 
	Längen und Winkel (in "`Punkträumen"' $ \cong $ affinen Räumen) verstehen.
	
Algebraisch:
	via Produkte (bilineare -- oder fast bilineare -- Abbildungen).
	
%TODO schönere Grafik...; dann auslagern.
%	\definecolor{qqwuqq}{rgb}{0.,0.39215686274509803,0.}
%	\definecolor{qqqqff}{rgb}{0.,0.,1.}
%	\begin{tikzpicture}[line cap=round,line join=round,>=triangle 45,scale=1.8]
%	\clip(0,0) rectangle (10,3);
%	\draw [shift={(5.635,1.07)},color=qqwuqq,fill=qqwuqq,fill opacity=0.1] (0,0) -- (34.85:0.14) arc (34.85:143.6:0.14) -- cycle;
%	\draw [->] (2.58,1) -- (0.56,2.3);
%	\draw [->] (6.6,0.35) -- (4.25,2.1);
%	\draw [->] (4.5,0.26) -- (6.8,1.9);
%
%	\draw [fill=qqqqff] (2.58,1) circle (1pt);
%	\draw[color=qqqqff] (2.6,1) node[below] {$A$};
%	\draw [fill=qqqqff] (0.56,2.3) circle (1pt);
%	\draw[color=blue] (0.5,2.3) node[above] {$B$};
%	\draw (1.5,0.1) node {Abstand a bis b $ \cong $ Länge b-a};
%	\draw (5.4,0.1) node {Winkel $ \cong $ Winkel zwischen Richtungsvektoren};
%	\end{tikzpicture}

\section{Bilinearformen \& Sesquilinearformen}
\paragraph{Zur Erinnerung}
	Sind $ V $ und  $W$ $ K $-VR, so nennt man eine Abbildung
		\[ \beta: V\times V\to W \]
	\emph{bilinear} oder ein \emph{Produkt}, wenn sie in jedem Argument linear ist:
		\begin{enumerate}[(i)]
			\item $ \forall w\in V :V\ni v \mapsto \beta(v,w)\in W $ ist linear;
			\item $ \forall v\in V: V\ni w\mapsto \beta(v,w)\in W $ ist linear.
		\end{enumerate}
	Zu vorgegebenen Werten $ \beta_{ij} \in W$ auf einer Basis $ (b_i)_{i\in I} $ von $ V $ existiert dann eine eindeutige Bilinearform $ \beta $ (Fortsetzungssatz Abschnitt 4.3):
		\[ \exists! \beta:V\times V\to W \text{ bilinear}: \forall i,j\in I: \beta(b_i,b_j) = \beta_{ij}. \]
\paragraph{Bemerkung}
	Man kann auch bilineare Abbildungen $ V\times V'\to W $ betrachten und, zum Beispiel, auch einen Fortsetzungssatz beweisen.
	
	Wir benötigen eine Verallgemeinerung in eine andere Richtung:
\subsection{Definition} \index{Sesquilinearform}\index{Semilinearität}
\begin{Definition}[Sesquilinearform]
Seien $ V $ ein $ K $-VR und $ K\ni x\mapsto \overline{x}\in K $ ein (Körper-) Automorphismus, d.h. eine bijektive Abbildung mit
		\[ \overline{x+y} = \overline{x}+\overline{y} \text{ und } \overline{xy} = \overline{x}\cdot \overline{y} \]
	für alle $ x,y\in K $. Eine Abbildung $ \sigma: V\times V \to K $ heißt dann \emph{Sesquilinearform} (bzgl. $ \bar{.} $), falls
		\begin{enumerate}[(i)]
			\item $ \forall v\in V: V\ni w \mapsto \sigma(v,w)\in K $ ist linear, d.h. $ \sigma(v,.)\in V^* $;
			\item $ \forall w\in V: V\ni v \mapsto \sigma(v,w)\in K $ ist \emph{semilinear}, d.h.
				\begin{enumerate}[(a)]
					\item $ \forall v,v' \in V: \sigma(v+v',w) = \sigma(v,w)+\sigma(v',w) $ und
					\item $ \forall v\in V\forall x\in K: \sigma(vx,w) = \overline{x}\sigma(v,w) $.
				\end{enumerate}
		\end{enumerate}
\end{Definition}

\paragraph{Beispiel}
	Die Identität $ K\ni x\mapsto \overline{x}:= x\in K $ ist offensichtlich ein Körperautomorphismus für jeden Körper $ K $. \emph{Bilinearformen} sind genau die Sesquilinearformen bezüglich $ \id_K $.
\paragraph{Beispiel}
	Für $ K = \C $ liefert \emph{komplexe Konjugation} einen Körperautomorphismus (keinen VR-Automorphismus, vgl. Abschnitt 1.4):
		\[ \C\ni x+iy \mapsto \overline{x+iy}:= x-iy \in \C. \]
	Dieses Beispiel ist unser Grund für die Einführung des Begriffs der Sesquilinearform.
\paragraph{Bemerkung}
	Ist $ \sigma $ Bilinearform und Sesquilinearform bezüglich $ \bar{.} $, so ist $ \sigma $ oder $ \bar{.} $ trivial:
		\[ \forall x\in K\forall v,w\in V: 0 = \sigma(vx,w) - \sigma(vx,w) = (x-\overline{x})\sigma(v,w)  \]
		\[ \Rightarrow \begin{cases}
		\forall v,w\in V: \sigma(v,w) = 0 \text{ oder}\\
		\exists v,w\in V: \sigma(v,w)\neq 0 \land \forall x\in K: \overline{x} = x.
		\end{cases} \]
\paragraph{Bemerkung}
	In $ \mathbb{Z}_p, \mathbb{Q} $ und $ \R $ gibt es nur \emph{einen} Körperautomorphismus: $ \id_K $. Ein Automorphismus $ \bar{.} $ von $ \C $ mit $ \overline{\R} = \R $ ist trivial, $ \bar{.} = \id_\C $ oder die komplexe Konjugation.
	
\subsection{Fortsetzungssatz für Sesquilinearformen}
\begin{Satz}[Fortsetzungssatz für Sesquilinearformen]
	Sind $ V $ ein $ K $-VR und $ K\ni x\mapsto \overline{x}\in K $ ein Körperautomorphismus, $ (b_i)_{i\in I} $ Basis von $ V $ und $ (s_{ij})_{i,j\in I} $ eine Familie in $ K $, so existiert eine eindeutige Sesquilinearform $ \sigma $ mit
		\[ \forall i,j\in I:\sigma(b_i,b_j) = s_{ij}. \]
\end{Satz}

% VO 19-04-2016 % 
\paragraph{Beweis}
	Wir imitieren den Beweis unseres ersten Fortsetzungssatzes für lineare Abbildungen:
	
	{Eindeutigkeit:}
	Sei $ \sigma $ eine Sesquilinearform mit der gewünschten Eigenschaft oben; gilt
		\[ v = \sum_{i\in I}b_ix_i \text{ und }w = \sum_{i\in I}b_i y_i \]
	so folgt
		\[ \sigma(v,w) = \sum_{i,j\in I}\overline{x_i}\sigma(b_i,b_j)y_j = \sum_{i,j\in I}\overline{x_i}s_{ij}y_j \]
	d.h. $ \sigma $ ist durch die Familie $ (s_{ij})_{i,j\in I} $ eindeutig bestimmt.
	
	{Existenz:}
	Da jeder Vektor $ v\in V $ eine eindeutige Basisdarstellung $ v=\sum_{i\in I}b_ix_i $ hat, wird durch
	\[ \sigma:V\times V \to K, (v,w)= \left(\sum_{i\in I}b_ix_i, \sum_{j\in I}b_jy_j\right) \]
	\[ \mapsto \sigma(v,w) := \sum_{i,j\in I}\overline{x_i}s_{ij}y_j \]
	eine Abbildung wohldefiniert. Offenbar (nachrechnen) ist $ \sigma $ dann sesquilinear. 

\paragraph{Bemerkung}
	Jede Sesquilinearform $ \sigma: V\times V\to K $ liefert eine semi-lineare Abbildung
		\[ V\ni v\mapsto \sigma(v,.)\in V^*. \]
	Mit einem "`Fortsetzungssatz für semi-lineare Abbildungen"' (Aufgabe 34) hätte man auch den früher skizzierten Beweis für bilineare Abbildungen imitieren können.

\subsection{Buchhaltung}\index{Gramsche Matrix}
\paragraph{Gramsche Matrix}
	Ist $ n=\dim V < \infty $ und $ B=(b_1,\dots,b_n) $ Basis von $ V $, so kann man eine Sesquilinearform $ \sigma: V\times V\to K $ durch eine Matrix $ S $ beschreiben:
	\[ \begin{array}{c|ccc}
	\sigma & b_1 & \dots & b_n \\ \hline
	b_1 & s_{11} &  & s_{1n} \\ 
	\vdots &  & \ddots &  \\ 
	b_n & s_{n1} &  & s_{nn}
	\end{array}  \]
	Diese Matrix
		\[ \Gamma_B(\sigma) = S = \left(\sigma(b_i,b_j)\right)_{i,j\in \{1,\dots,n\}} \]
	heißt die Darstellungsmatrix oder \emph{Gramsche Matrix} von $ \sigma $ bezüglich $ B $. Für Vektoren
		\[ v = \sum_{i=1}^{n}b_ix_i = BX \text{ und } w = \sum_{j=1}^{n}b_jy_j = BY \]
	ist dann
		\[ \sigma(v,w) = \sum_{i,j=1}^{n}\overline{x_i}s_{ij}y_j = \overline{X}^tSY \]
		\[ = (\overline{x_1},\dots,\overline{x_n})\begin{pmatrix}
		\sum_{i=1}^{n}s_{1j}y_j\\ \vdots\\ \sum_{j=1}^{n}s_{nj}y_j
		\end{pmatrix} = \sum_{i=1}^{n}\overline{x_i}\sum_{j=1}^{n}s_{ij}y_j. \]
\paragraph{Transformationsformel}
	Ein Basiswechsel $ B' = BP $ mit $ P = \xi_{B'}^B \in Gl(n)$ liefert dann
		\[ v = BX = (B'P^{-1})X = B'\underset{X'}{\underbrace{(P^{-1}X)}} \text{ und } w = B'\underset{Y'}{\underbrace{(P^{-1}Y)}} \]
	und damit für $ X,Y \in K^{n\times 1} $
		\[ \overline{X}^tSY = \overline{X'}^t\underset{S'}{\underbrace{(\overline{P}^tSP)}}Y' \]
	woraus die \emph{Transformationsformel für Gramsche Matrizen} folgt
		\[ S' = \overline{P}^tSP, \]
	wobei $ \overline{P}^t $ die Transponierte der Matrix mit Einträgen $ \overline{p_{ij}} $ ist.
\paragraph{Äquivalenz von Matrizen}
Dies liefert einen weiteren Äquivalenzbegriff für quadratische Matrizen $ S\in K^{n\times n} $:
		\[ S' \sim S :\Leftrightarrow \exists  P\in Gl(n): S' = \overline{P}^tSP. \]
	Die verschiedenen Begriffe der Äquivalenz von Matrizen (vgl. 3.1 \& 4.2) spiegeln die verschiedenen Funktionen/Bedeutungen von Matrizen wider.
	
\paragraph{Bemerkung}
	Die Menge der Sesquilinearformen auf einem $ K $-VR ist selbst ein $ K $-VR. Ist $ n=\dim V< \infty $ und $ B $ Basis von $ V $, so erhält man (Fortsetzungssatz) einen Isomorphismus
		\[ K^{V\times V}\supset \{\sigma:V\times V\to K \text{ Sesquilinearform}\}\ni \sigma \mapsto \Gamma_B(\sigma)\in K^{n\times n}. \]
\subsection{Beispiel \& Definition} \index{Sesquilinearform!kanonische}\index{Sesquilinearform!assoziierte}
\begin{Definition}[assoziierte Sesquilinearform]
	Sei $ \bar{.}:K\to K $ Körperautomorphismus; jedes $ S\in K^{n\times n} $ liefert dann eine eindeutige Sesquilinearform
		\[ \sigma_S:K^n\times K^n \to K \text{ mit } (e_i,e_j)\mapsto \sigma_S(e_i,e_j):= s_{ij}, \] 
	die zu \emph{$ S $ assoziierte Sesquilinearform}.

	Für $ S = E_n $ bezeichnet man $ \sigma_S $ auch als \emph{kanonische Sesquilinearform}.
\end{Definition}

\subsection{Definition}\index{Sesquilinearform!(schief-)symmetrische}
\begin{Definition}[symmetrische, schiefsymmetrische und alternierende Sesquilinearformen]
	Eine Sesquilinearform $ \sigma:V\times V\to K $ auf einem $ K $-VR bzgl. eines Automorphismus $ \bar{.}:K\to K $ nennen wir
		\begin{enumerate}[(i)]
			\item \emph{symmetrisch}, falls
				\[ \forall v,w\in V: \sigma(w,v) = \overline{\sigma(v,w)} \]
			\item \emph{schiefsymmetrisch}, falls
				\[ \forall v,w\in V: \sigma(w,v) = - \overline{\sigma(v,w)} \]
			\item \emph{alternierend}, falls
				\[ \forall v \in V: \sigma(v,v) = 0. \]
		\end{enumerate}
\end{Definition}

\begin{Definition}[Hermitesche Sesquilinearform]\index{Sesquilinearform!Hermitesche}
	Falls $ K =\C $ und $ \bar{.} $ komplexe Konjugation sind, so nennt man eine symmetrische Sesquilinearform auch \emph{Hermitesche Sesquilinearform}.
\end{Definition}

\paragraph{Bemerkung}
	Ist $ \sigma $ nicht-trivial und (schief-)symmetrisch, so muss $ \bar{.} $ eine Involution sein.
	
	Nämlich: Wähle $ v,w\in V $ mit $ \sigma(v,w) = 1$; dann gilt
		\[ \forall x\in K: \overline{\overline{x}} = \overline{\sigma(vx,w)} = \pm \sigma(w,vx) = \overline{\overline{x}\sigma(v,w)} = \pm \sigma(w,v)x = \overline{\sigma(v,w)}x = x. \]

% VO 21-04-2016 %
	Ist $ \Char(K) \neq 2 $ und $ \bar{.}  $ Involution, so kann jede Sesquilinearform in einen symmetrischen und einen schiefsymmetrischen Anteil zerlegt werden:
		\[ \forall v,w\in V: \sigma(v,w) = \frac{1}{2}\left(\sigma(v,w)+\overline{\sigma(w,v)}\right) +\frac{1}{2}\left(\sigma(v,w)-\overline{\sigma(w,v)} \right).\]
\paragraph{Bemerkung}
	Ist $ \Char(K)\neq 2 $ und $ \bar{.} = \id_K $, so sind "`alternierend"' und "`schiefsymmetrisch"' äquivalent für eine Sesquilinearform $ \sigma $.
	
	Andererseits ist jede alternierende Sesquilinearform bilinear, d.h. $ \bar{.} = \id_K $ oder $ \sigma = 0 $.
	
\paragraph{Buchhaltung}
	Unter den folgenden Annahmen:
		\begin{itemize}
			\item $ \Char(K)\neq 2 $ und $ \bar{.} $ Involution;
			\item $ n=\dim V <\infty $ und $ B $ ist Basis von $ V $;
		\end{itemize}
	gilt für die Gramsche Matrix $ S = \Gamma_B(\sigma) $ einer Sesquilinearform $ \sigma $ auf $ V $:
		\begin{itemize}
			\item $ 0 = \overline{S}^t-S\Leftrightarrow \sigma $ symmetrisch;\footnote{bis auf Faktor 2: Gramsche Matrix des schiefsymmetrischen Anteils von $ \sigma $}
			\item $ 0 = S + \overline{S}^t \Leftrightarrow \sigma $ schiefsymmetrisch.
		\end{itemize}
	Nämlich:
		\[ \overline{S}^t = \begin{pmatrix}
		\overline{\sigma(b_1,b_1)} & \overline{\sigma(b_1,b_2)} &\cdots& \overline{\sigma(b_1,b_n)} \\ 
		\overline{\sigma(b_2,b_1)} &  & & \vdots \\ 
		\vdots &  & & \vdots \\ 
		\overline{\sigma(b_n,b_1)} & \cdots & & \overline{\sigma(b_n,b_n)}
		\end{pmatrix}^t =  \begin{pmatrix}
		\overline{\sigma(b_1,b_1)} & \overline{\sigma(b_2,b_1)} &\cdots& \overline{\sigma(b_n,b_1)} \\ 
		\overline{\sigma(b_1,b_2)} &  & & \vdots \\ 
		\vdots &  & & \vdots \\ 
		\overline{\sigma(b_1,b_n)} & \cdots & & \overline{\sigma(b_n,b_n)}
		\end{pmatrix} \]
		\[ S = \begin{pmatrix}
		\sigma(b_1,b_1) & \sigma(b_1,b_2) &\cdots& \sigma(b_1,b_n) \\ 
		\sigma(b_2,b_1) &  & & \vdots \\ 
		\vdots &  & & \vdots \\ 
		\sigma(b_n,b_1) & \cdots & & \sigma(b_n,b_n)
		\end{pmatrix} \]
		
\subsection{Definition} \index{Orthogonal!-raum}\index{Orthogonal}
\begin{Definition}[orthogonal, Orthogonalraum]
	Sei $ \sigma $ symmetrische Sesquilinearform auf einem Vektorraum $ V $. Zwei Vektoren $ v,w\in V $ heißen \emph{orthogonal} (bzgl. $ \sigma $),
		\[ w \perp v, \text{ falls } \sigma(v,w) = 0. \]
	Der \emph{Orthogonalraum} einer Menge $ \emptyset \neq S\subset V $ ist der UVR
		\[ S^\perp := \bigcap_{s\in S} \ker \underset{\in V^*}{\underbrace{\sigma(s,.)}}. \]
\end{Definition}
\paragraph{Bemerkung}
	Wegen der Symmetrie von $ \sigma $ ist die \emph{Orthogonalitätsrelation} symmetrisch,
		\[ w \perp v \Leftrightarrow v \perp w. \]
\paragraph{Bemerkung}
	Da $ \forall v\in V: \sigma(v,.) \in V^* $, ist der Orthogonalraum wohldefiniert und (als Schnitt von UVR) ein UVR. Offenbar gilt
		\[ \tilde{S} \subset S \Rightarrow \tilde{S}^\perp \supset S^\perp. \]
	Damit folgt direkt $ S^\perp \supset [S]^\perp $, sind andererseits $ w\in S^\perp $ und $ v\in [S] $, so gilt
		\[ v = \sum_{s\in S}sx_s \Rightarrow \sigma(v,w)= \sum_{s\in S}\overline{x_s}\sigma(s,w) = 0, \text{ da } \forall s\in S: w\perp s \]
	d.h. $ w\in S^\perp \Rightarrow w\in [S]^\perp. $ Insgesamt ist also
		\[ \forall S \subset V: [S]^\perp=S^\perp.\]
	Ähnlich zeigt man für jede Familie $ (U_i)_{i\in I} $ von UVR $ U_i\subset V $:
		\[ \left(\sum_{i\in I}U_i \right)^\perp= \bigcap_{i\in I} U_i^\perp. \]
\paragraph{Bemerkung \& Beispiel}
	Für $ S\subset V $ kann man $ S^{\perp\perp} = \left(S^\perp\right)^\perp $ betrachten; im Allgemeinen gilt
		\[ S\subset S^{\perp\perp} \text{ aber } S\neq S^{\perp\perp}. \]
	Ist etwa $ \sigma = 0 $, so ist $ S^\perp = V $ für jede Menge $ \emptyset \neq S\subsetneq V $; also ist
		\[ S^{\perp\perp} = V^\perp = V \neq S. \]

\subsection{Definition}\index{Radikal!-raum}\index{Radikal!-frei}
\begin{Definition}[Radikal(-raum),radikalfrei,nicht-degeneriert,degeneriert]
$ V^\perp $ ist der \emph{Radikal(-raum)} eines VR mit symmetrischer Sesquilinearform $ \sigma $; ist $ V^\perp = \{0 \} $, so heißt $ \sigma $ \emph{radikalfrei} oder \emph{nicht-degeneriert}, andernfalls \emph{degeneriert}.
\end{Definition}
\paragraph{Beispiel}
	Betrachte $ V=\R^2 $ mit Standardbasis $ (e_1,e_2) $.
	
	Ist für eine symmetrische Sesquilinearform (Bilinearform) $ \sigma $ auf $ V $
		\[ \sigma(e_1,e_1) = 0, \sigma(e_1,e_2) = 1, \sigma(e_2,e_2) = 0 \]
	so ist $ \sigma $ nicht-degeneriert, $ V^\perp = \{0\} $, da
		\[ v = e_1x_1 + e_2x_2 \perp e_1,e_2 \Rightarrow
		\begin{cases}
		0 = \sigma(e_1,v) = x_2\\
		0 = \sigma(e_2,v) = x_1
		\end{cases} \]
		\[ \Rightarrow x_1 = x_2 = 0, \]
	also $ V^\perp = \{0\} $, d.h. $ \sigma $ ist nicht-degeneriert.
	
	% Grafik R^2
	
	Ist aber
		\[ \sigma(e_1,e_1) = 1, \sigma(e_1,e_2) = 1, \sigma(e_2,e_2) = 1, \]
	so ist $ V^\perp = [e_1-e_2] $, d.h. $ \sigma $ ist degeneriert.
	
	% Grafik R^2 mit UVR [e_1-e_2]-> Gerade y = -x

\subsection{Lemma}
\begin{Lemma}[]
	Ist $ U\subset V $ ein zum Radikal von $ (V,\sigma) $ komplementärer UVR, $ V = V^\perp \oplus U $, so ist
		\[ \sigma\big|_{U\times U}:U\times U \to K \]
	radikalfrei.
\end{Lemma}
\paragraph{Beweis}
	Sei $ u\in U $ im Radikal von $ (U,\sigma\big|_{U\times U}) $, d.h. es gelte $ \forall v\in U: \sigma(v,u) = 0 $.
	Weil
		\[ \forall v\in V^\perp\forall w\in V: v\perp w \Rightarrow \forall v\in V^\perp: v\perp u \]
	erhalten wir $ u\in U\cap V^\perp = \{0\} $.
\paragraph{Beispiel}
	Die Einschränkung von $ \sigma $ mit (wie oben)
		\[ \forall i,j \in \{1,2\}: \sigma(e_i,e_j) = 1 \]
	auf jeden UVR $ U = [e_1x_1 + e_2x_2] $ mit $ x_1 + x_2 \neq 0 $ ist radikalfrei, denn
			\[ \sigma(e_1x_1+e_2x_2, e_1x_1+e_2x_2) = (x_1+x_2)^2 \neq 0. \]
	