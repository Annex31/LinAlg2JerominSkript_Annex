\section{Euklidische \& unitäre Vektorräume}

% GRAFIK-MOTIVATION %
	% (A,V,\tau) reelle affine Ebene
	% (o; e_1,e_2) affines Bezugssystem
	% Abstand/Länge von b-a ist (falls e_1 \perp e_2 und |e_1| = |e_2| = 1):
	% d(a,b) = \|b-a\| = \sqrt{(y_1-x_1)^2+(y_2-x_2)^2}

\paragraph{Bemerkung}
	Die folgende Definition ist nur für Skalarprodukte $ \Skl{.}{.} $ sinnvoll, für die
		\[ v\mapsto \Skl{ v}{v} \in T \]
	mit einem angeordneten Teilkörper $ T \subset K $ des Körpers $ K $ (vgl. Abschnitt 1.2).
	Ein nicht-triviales Beispiel, mit $ T=\R\subset \C = K $, ist ein Hermitesches Skalarprodukt:
		\[ \forall v\in V: \Skl{ v}{v} = \overline{\Skl{ v}{v}} \Rightarrow \forall v\in V: \Skl{ v}{v} \in \R\subset \C. \]
\paragraph{Vereinbarung}
	Im Folgenden beschränken wir uns bis auf Weiteres auf $ \K $-VR mit $ \Skl{.}{.} $ Hermitsche Sesquilinearform, falls $ \K = \C $ (vgl. Satz von Sylvester).
	
\subsection{Definition}\index{positiv definit}\index{induzierte Norm}\index{Vektorraum!Euklidischer}\index{Vektorraum!unitärer}
\begin{Definition}[positiv definit, euklidischer-, unitärer Vektorraum ]
	Ein Skalarprodukt $ \Skl{.}{.} $ auf einem $ \K $-VR $ V $ heißt \emph{positiv definit}, falls
		\[ \forall v\in V^\times:\Skl{ v}{v} >0; \]
	die \emph{induzierte Norm} eines positiv-definiten Skalarprodukts $ \Skl{.}{.} $ ist die Abbildung
		\[ \|.\|: V\to \R, v\mapsto \|v\| := \sqrt{\Skl{ v}{v}}\geq 0. \]
	Ein $ \K $-VR $ (V,\Skl{.}{.} ) $ mit positiv-definitem Skalarprodukt ist
		\begin{itemize}
			\item ein \emph{Euklidischer Vektorraum}, falls $ \K=\R $, und
			\item ein \emph{unitärer Vektorraum}, falls $ \K = \C $ und $ \Skl{.}{.} $ Hermitesche Sesquilinearform ist. 
		\end{itemize}
\end{Definition}

\subsection{Bemerkung \& Definition}
\begin{Definition}[negativ definit, indefinit]
	Ebenso definiert man ein Skalarprodukt als \emph{negativ definit}, falls
		\[ \forall v\in V^\times: \Skl{ v}{v} < 0; \]
	$ \Skl{.}{.} $ heißt \emph{indefinit}, falls es weder positiv, noch negativ definit ist.
	Die Definition der induzierten Norm ist nur im positiv definiten Fall sinnvoll.
\end{Definition}

% VO 10-05-2016 %

\paragraph{Beispiel}	
	Der \emph{Betrag} einer komplexen Zahl
		$ z=x+iy\in \C\cong_\R \R^2 $
			% Vektorraum-Isomorphismus - R^2 ist kein Körper! %
	ist 
		\[ |z| = \sqrt{\overline{z}z} = \sqrt{x^2+y^2} \]
	die vom Standardskalarprodukt auf $ \R^2 $ induzierte Norm.
	Insbesondere gilt
		\[ \forall z\in \C: |\Re z| \leq |z| \]	
\subsection{Bemerkung zum Zusammenhang von reellen und komplexen VR}
	Da $ \R\subset \C $ ein Teilkörper ist, kann jeder $ \C $-VR $ V $ auch als $ \R $-VR aufgefasst werden (Einschränkung der Skalarmultiplikation).
	
	Ist nun $ S\subset V $ linear unabhängig über $ \C $ (in $ V $ als $ \C $-VR), so ist
		\[ S' := S\cup Si = S \cup \{si\mid s\in S\} \]
	linear unabhängig über $ \R $, denn
		\[ 0 = \sum_{s\in S}sx_s + \sum_{s\in S}siy_s = \sum_{s\in S} s (x_s+iy_s) \]
		\[ \Rightarrow \forall s\in S: x_s + iy_s = 0\Rightarrow \forall s\in S: x_s = y_s = 0, \]
	d.h. $ S' = S\cup S_i $ ist linear unabhängig über $ \R $.
	Insbesondere folgt
		\[ \dim_\R V = 2\dim_\C V.  \]

	Weiters definiert für ein Hermitesches Skalarprodukt $ \Skl{ .}{. } $ auf $ V $ (als $ \C $-VR)
		\[ \Skl{ .}{. } :V\times V\to \R, (v,w)\mapsto \Skl{ v}{w }_\R := \Re \Skl{v}{w } \]
	ein reelles Skalarprodukt auf $ V $ (als $ \R $-VR), das genau dann positiv definit ist, wenn $ \Skl{.}{. } $ positiv definit ist:
		\[ \forall v\in V: \Skl{v}{v }_\R = \Skl{v}{v }. \]
	Damit kann man jeden unitären Vektorraum als Euklidischen Vektorraum auffassen:
		\begin{itemize}
			\item mit verschiedenen Skalarprodukten $ \Skl{.}{.} $ bzw. $ \Skl{.}{.}_\R $, aber
			\item mit gleichen induzierten Normen. % denn diese sind ohnehin immer reell
		\end{itemize}
\index{Komplexifizierung}
\paragraph{Komplexifizierung}
	Fasst man einen $ \C $-VR $ V $ als $ \R $-VR auf, so liefert Multiplikation mit $ i\in \C $ einen Endomorphismus
		\[ J:V\to V, v\mapsto J(v):= vi \] % Skalarmultiplikation in $ V $ als $ \C $-VR.
	mit
		\[ J^2 = -\id_V. \]
	Insbesondere besitzt $ J $ keine reellen Eigenwerte;
	ist $ \dim V < \infty $, so folgt damit
		\[ \dim V = \deg{\chi_J}(t) = 0 \mod 2. \]
	Umgekehrt: Ist $ V $ ein $ \R $-VR und $ J\in \End(V) $ mit $ J^2 = -\id_V $ gegeben, so erhält man eine komplexe Skalarmultiplikation
		\[ \cdot :\C\times V \to V, (z,v)\mapsto vz := vx+J(v)y, \]
	für $ z = x+iy $.
	Ist weiter $ \Skl{.}{.} $ ein (reelles) Skalarprodukt auf $ V $, das von $ J $ erhalten wird, 
		\[ \forall v,w\in V: \Skl{Jv}{Jw} = \Skl{v}{w}, \]
	so definiert
		\[ \Skl{v}{w}_\C := \Skl{v}{w} -i \Skl{v}{Jw} \]
	ein Hermitesches Skalarprodukt auf dem so konstruierten $ \C $-VR.

\paragraph{Beispiel}\label{JDrehung}
	Ist $ \Skl{.}{.} $ das kanonische Skalarprodukt auf $ \R^2 $, mit der Standardbasis $ (e_1,e_2) $ als ONB, so definiert (Fortsetzungssatz)
		\[ J(e_1) = e_2 \text{ und } J(e_2) = -e_1, \]
	einen Endomorphismus $ J\in\End(\R^2) $ mit
		\[ J^2 = -\id_{\R^2} \text{ und } \Skl{Je_i}{Je_j} = \Skl{e_i}{e_j}. \]
	Vermöge
		\[ e_1i := J(e_1) = e_2\ \text{ und }\ e_2i := J(e_2) = J^2(e_1) = -e_1 = e_1 i^2 \]
	wird $ \R^2 $ zu einem eindimensionalen $ \C $-VR, $ \R^2 = [\{e_1\}]_\C $, da
		\[ e_1x+e_2y = e_1x+J(e_1)y = e_1 (x+iy); \]
	und
		\[ \Skl{e_1x+e_2y}{e_1x'+e_2y'}_\C = \Skl{e_1(x+iy)}{e_1(x'+iy')}_\C = \overline{(x+iy)}(x'+iy') \]
	liefert das kanonische Skalarprodukt auf $ \C $, mit dem kanonischen Euklidischen Skalarprodukt von $ \R^2 $ als Realteil.

\subsection{Komplexifizierungslemma}
\begin{Lemma}[Komplexifizierungslemma]
	Ist $ (V,\Skl{.}{.}) $ ein Euklidischer Vektorraum, so liefert
		\[ (v,w)(x+iy) := (vx-wy,wx+vy) \]
	eine komplexe Skalarmultiplikation auf $ V_\C := V\times V $, und
		\[ \SSkl{((v,w))}{(v',w')}_\C := \left(\Skl{v}{v'}+\Skl{w}{w'} \right)+i\left(\Skl{v}{w'}-\Skl{w}{v'} \right) \]
	ein Hermitesches Skalarprodukt, das $ (V_\C,\SSkl{.}{.}_\C) $ zu einem unitären VR macht.
\end{Lemma}

\paragraph{Beweis}
	Auf dem Euklidischen VR $ (V^2,\SSkl{.}{.}) $, wobei
		\[ \SSkl{.}{.}:V^2\times V^2 \to \R, \left((v,w),(v',w')\right)\mapsto \SSkl{(v,w)}{(v',w')} := \Skl{v}{v'}+\Skl{w}{w'}, \]
	definiere $ J\in \End(V^2) $ durch
		\[ J:V^2 \to V^2, (v,w)\mapsto J\left((v,w)\right):= (-w,v). \]
	Offenbar gilt $ J^2 = -\id_{V^2} $ und
		\[ \SSkl{J(v,w)}{J(v',w')} = \Skl{w}{w'}+\Skl{v}{v'} = \SSkl{(v,w)}{(v',w')}, \]
	sodass
		\[ (v,w)(x+iy) = (v,w)x+J(v,w)y= (vx-wy,wx+vy) \]
	und
		\[ \SSkl{(v,w)}{(v',w')}_\C = \SSkl{(v,w)}{(v',w')}-i\SSkl{(v,w)}{J(v',w')}  \]
		\[ = \left(\Skl{v}{v'}+\Skl{w}{w'}\right) -i\left(-\Skl{v}{w'}+\Skl{w}{v'}\right) \]
	$ (V^2,\SSkl{.}{.}) $ zu einem unitären VR machen, wie vorher.
\paragraph{Bemerkung}
	Mit dem "`Komplexifizierungslemma"' kann man jeden Euklidischen VR in einen unitären VR gleicher (komplexer) Dimension einbetten:
		\[ \dim_\C V_\C = \frac{1}{2}\dim_\R V^2 = \dim_\R V. \]
	Wichtig für den Zusammenhang zwischen unitären und Euklidischen VR:
	Die induzierte Norm des Hermitschen Skalarprodukts kann als die eines Euklidischen Skalarprodukts aufgefasst werden.
	
% VO 12-05-2016 %
\subsection{Definition}\index{Norm}
\begin{Definition}[Norm , normierter Vektorraum ]
	Eine Abbildung $ \|.\|:V\to \R $ auf einem $ \K $-VR $ V $ heißt \emph{Norm}, falls
	\begin{enumerate}[(i)]
		\item $ \forall v\in V^\times: \|v\| > 0 $, d.h. $ \|.\| $ ist \emph{positiv definit};
		\item $ \forall v\in V \forall x\in \K: \|vx\| = \|v\|\cdot |x| $, d.h. $ \|.\| $ \emph{positiv homogen};
		\item $ \forall v,w\in V: \| v+w\|\leq \|v\| + \|w\| $, d.h. $ \|.\| $ erfüllt die \emph{Dreiecksungleichung}.
	\end{enumerate}
	Ein Vektorraum mit Norm, $ (V,\|.\|) $ heißt \emph{normierter Vektorraum}.
\end{Definition}

\paragraph{Bemerkung}
	Die von einem positiv definiten Skalarprodukt $ \Skl{.}{.} $ induzierte Norm $ \|.\| $ erfüllt offenbar (i) und (ii); die Dreiecksungleichung zeigen wir unten.
\paragraph{Cauchy-Schwarzsche Ungleichung}
\begin{Satz}[Cauchy-Schwarzsche Ungleichung]
	Ist $ (V,\Skl{.}{.}) $ Euklidisch oder unitär, so gilt\footnote{Im Euklidischen Fall ist der Betrag offenbar überflüssig.}
		\[ \forall v,w\in V: |\Skl{v}{w}|^2 \leq \Skl{v}{v}\Skl{w}{w} \]
\end{Satz}
\paragraph{Beweis}
	Seien $ v,w\in V $, o.B.d.A $ v\neq 0 $. Wir bestimmen das Minimum der Funktion im Euklidischen Fall (unitärer Fall in der Übung)
		\[ \R \ni s\mapsto g(s):= \Skl{vs-w}{vs-w}. \]
	Einsetzen des kritischen Punktes,
		\[ 0 = g'(s) = 2\Skl{v}{v}s -(\Skl{v}{w}+\Skl{w}{v}) = 2(\Skl{v}{v}s-\Re \Skl{v}{w}) \]
		\[ \Rightarrow s = \frac{\Skl{v}{w}}{\Skl{v}{v}} \]
	liefert
		\[ 0\leq g(s) = \Skl{v}{v}\frac{\Skl{v}{w}^2}{\Skl{v}{v}^2}-2\Skl{v}{w}\frac{\Skl{v}{w}}{\Skl{v}{v}}+\Skl{w}{w} \]
		\[ = \frac{1}{\Skl{v}{v}}\left(-\Skl{v}{w}^2+\Skl{v}{v}\Skl{w}{w} \right) \Leftrightarrow 0\leq -\Skl{v}{w}^2+\Skl{v}{v}\Skl{w}{w}. \ \]
		
\subsection{Korollar}
\begin{Korollar}[]
	Die induzierte Norm in $ (V,\Skl{.}{.}) $ erfüllt die Dreiecksungleichung.
\end{Korollar}
\paragraph{Beweis}
	Ist $ (V,\Skl{.}{.}) $ Euklidischer VR, so gilt für $ v,w\in V $:
		\begin{align*}
		\|v+w\|^2 = \Skl{v+w}{v+w} &= \|v\|^2+2\Skl{v}{w}+\|w\|^2 \\
		&\overset{C.S.}{\leq} \|v\|^2+2\|v\|\cdot \|w\|+\|w\|^2 = (\|v\|+\|w\|)^2.
		\end{align*}
\paragraph{Bemerkung}
	Das Skalarprodukt eines Euklidischen VR $ (V,\Skl{.}{.}) $ kann (Polarisation) aus seiner induzierten Norm rekonstruiert werden.
	
	Nicht jede Norm ist jedoch von einem Skalarprodukt induziert (vgl. Aufgabe 56).
	Hinreichende (Satz von Jordan-von Neumann) und notwendige Bedingung ist die Parallelogrammgleichung:
	
\subsection{Parallelogrammgleichung}
\begin{Satz}[Parallelogrammgleichung]
	Für die induzierte Norm $ \|.\| $ von $ (V,\Skl{.}{.}) $ gilt:
		\[ \forall v,w\in V: \|v+w\|^2+\|v-w\|^2=2\|v\|^2+2\|w\|^2 \]
\end{Satz}		

	%------------------ Parallelogrammgleichung ----------------
 	\begin{figure}[H]\centering
 		\tdplotsetmaincoords{0}{0} %-27
 	\begin{tikzpicture}[yscale=1,tdplot_main_coords]

 		\def\xstart{0} %x Koordinate der Startposition der Grafik
 		\def\ystart{0} %y Koordinate der Startposition der Grafik
 		\def\myscale{1.0} %�ndert die Gr��e der Grafik (Skalierung der Grafik)
        \def\myscalex{(\myscale)}
        \def\myscaley{(\myscale)}
                
 		\def\xstartdraw{(\xstart + 1.5)} %xKoordinate des Referenzstartpunktes (in dieser Zeichnung: a)
 		\def\ystartdraw{(\ystart + 1.0)}%yKoordinate des Referenzstartpunktes (in dieser Zeichnung: a)

 		\def\balkenhoehe{(4.0)}% L�nge des vertikalen blauen Balkens
 		\def\balkenlaenge{(7)}% L�nge des horizontalen blauen Balkens
 		\def\balkenbreite{0.4} %Balkenbreite

 		%---------Begin Balken----------
 		\def\drehwinkel{0}
 		\node (VekV) at ({\xstart+0.2*cos(\drehwinkel)-\balkenbreite*sin(\drehwinkel)},{\ystart+0.5*sin(\drehwinkel)+\balkenbreite*cos(\drehwinkel)})[right, xshift=1,color=blue] {$V$};
 		\node (AffA) at ({\xstart+(\balkenlaenge-1)*cos(\drehwinkel)},{\ystart+(\balkenlaenge-1)*sin(\drehwinkel)+\balkenbreite*cos(\drehwinkel)})[color=red] {$A$};

 		\path[ shade, top color=white, bottom color=blue, opacity=.6]
 		({\xstart},{\ystart},0)  -- ({\xstart - \balkenbreite * cos(\drehwinkel)- (-\balkenbreite+0)*sin(\drehwinkel)},{\ystart - \balkenbreite * sin(\drehwinkel)+ (-\balkenbreite+0)*cos(\drehwinkel)},0)  -- ({\xstart - \balkenbreite * cos(\drehwinkel)- (\balkenhoehe+0.5)*sin(\drehwinkel)},{\ystart - \balkenbreite * sin(\drehwinkel)+ (\balkenhoehe+0.5)*cos(\drehwinkel)},0) -- ({\xstart - 0 * cos(\drehwinkel)- (\balkenhoehe+0)*sin(\drehwinkel)},{\ystart - 0 * sin(\drehwinkel)+ (\balkenhoehe+0)*cos(\drehwinkel)},0) -- cycle;

 		\path[ shade, right color=white, left color=blue, opacity=.6]
 		({\xstart},{\ystart},0)  -- ({\xstart - \balkenbreite * cos(\drehwinkel)- (-\balkenbreite+0)*sin(\drehwinkel)},{\ystart - \balkenbreite * sin(\drehwinkel)+ (-\balkenbreite+0)*cos(\drehwinkel)},0) --
 		({\xstart + (\balkenlaenge+0.5) * cos(\drehwinkel)- (-\balkenbreite+0)*sin(\drehwinkel)},{\ystart + (\balkenlaenge+0.5) * sin(\drehwinkel)+ (-\balkenbreite+0)*cos(\drehwinkel)},0) --
 		({\xstart + \balkenlaenge * cos(\drehwinkel)},{\ystart + \balkenlaenge * sin(\drehwinkel)},0)--
 		cycle;
 		%---------End Balken----------
 		\def\lightoffset{0.2*\myscale} %offeset der Vektoren

 		% rote Punkte Definition
 		
 		\node (offsetx) at ({(3.5*\myscalex},{0.0}) {}; %just an offset
 		\node (offsety) at ({0.0},{2.5*\myscaley}) {}; %just an offset
 		
 		\node (pointa1) at ({\xstartdraw},{\ystartdraw}) {};
 		\node[ xshift=-2mm, yshift=-3mm,color=red] (labela1) at (pointa1) {$0$};
 		
 		\node (pointa2) at ($(pointa1) + 0.1*(offsetx) + 1.0*(offsety)$) {};
 		
 		\node (pointb1) at ($(pointa1) + 1.0*(offsetx) + 0.1*(offsety)$ ) {};
 		\node (pointb2) at ($(pointb1) + 0.1*(offsetx) + 1.0*(offsety)$) {};
 	
 		%Vektoren blau
 	    %waagrecht
 		\draw[-{>[scale=1,length=10,width=6]},shorten >=2pt, shorten <=2pt,line width=0.2pt,color=blue] (pointa1) -- (pointb1);
 		\draw[-{>[scale=1,length=10,width=6]},shorten >=2pt, shorten <=2pt,line width=0.2pt,color=blue,dashed] (pointa2) -- (pointb2);
 		
 		%senkrecht
 		\draw[-{>[scale=1,length=10,width=6]},shorten >=2pt, shorten <=2pt,line width=0.2pt,color=blue] (pointa1) -- (pointa2);
 		\draw[-{>[scale=1,length=10,width=6]},shorten >=2pt, shorten <=2pt,line width=0.2pt,color=blue,dashed] (pointb1) -- (pointb2);
 		
 		%diagonal
 		\draw[-{>[scale=1,length=10,width=6]},shorten >=2pt, shorten <=2pt,line width=0.2pt,color=blue] (pointa2) -- (pointb1);
 		\draw[-{>[scale=1,length=10,width=6]},shorten >=2pt, shorten <=2pt,line width=0.2pt,color=blue] (pointa1) -- (pointb2);
 		
 		%Beschriftung der Vektoren
 		\node [color=blue] (pointlabelvu) at ($(pointa1)!0.5!(pointb1)$) [above, xshift=0, yshift=-5mm] {$v$} ;
 		\node [color=blue] (pointlabelvo) at ($(pointa2)!0.5!(pointb2)$) [above, xshift=0, yshift=0mm] {$v$} ;
 		
 		\node [color=blue] (pointlabelwl) at ($(pointa1)!0.5!(pointa2)$) [above, xshift=-4mm, yshift=-5mm] {$w$} ;
 		\node [color=blue] (pointlabelwr) at ($(pointb1)!0.5!(pointb2)$) [above, xshift=2mm, yshift=-5mm] {$w$} ;
 		
 		%Beschriftung der Diagonalvektoren
 		def\drehwinkel{30}
 		\node [color=blue,rotate=35] (pointlabeldo) at ($(pointa1)!0.2!(pointb2)$) [above, xshift=3mm, yshift=-0.5mm] {$v+w$} ;
        \node [color=blue,rotate=-35] (pointlabeldu) at ($(pointa2)!0.3!(pointb1)$) [above, xshift=2mm, yshift=-1mm] {$v-w$} ;


 		%Punkte malen
 		\draw[fill,color=red] (pointa1) circle [x=1cm,y=1cm,radius=0.08]node[above, xshift=0, yshift=0]{};
 		\draw[fill,color=red] (pointb1) circle [x=1cm,y=1cm,radius=0.08]node[above, xshift=0, yshift=0]{};
 		\draw[fill,color=red] (pointa2) circle [x=1cm,y=1cm,radius=0.08]node[below, xshift=5, yshift=0]{};
 		\draw[fill,color=red] (pointb2) circle [x=1cm,y=1cm,radius=0.08]node[below, xshift=5, yshift=0]{};
 		
\end{tikzpicture}
	\end{figure}
	%------------------ Parallelogrammgleichung ----------------
		
\paragraph{Beweis}
	Rechnung, wie bei Polarisation.
\paragraph{Beispiel}
	Für die induzierte Norm des kanonischen Skalarprodukts auf $ \R^n $ gilt die Parallelogrammgleichung:
		\[ \sum_{i=1}^{n}(x_i+y_i)^2 + \sum_{i=1}^{n}(x_i-y_i)^2 = 2\sum_{i=1}^{n}x_i^2+2\sum_{i=1}^{n}y_i^2. \]
	Für die durch
		\[ \|(x_i)_{i\in \{1,\dots,n\}}\|_1 = \sum_{i=1}^{n}|x_i| \]
	auf $ \R^n $ definierte Norm $ \|.\|_1 $ gilt sie nicht; diese Norm ist also nicht induzierte Norm eines Euklidischen Skalarprodukts auf $ \R^n $.
\paragraph{Beispiel}
	Auf dem Raum $ C^0([0,1]) $ der stetigen Funktionen auf $ [0,1] $ definiert
		\[ \|.\|_\infty: C^0([0,1]) \to \R, f\mapsto \|f\|_\infty := \max_{x\in [0,1]}|f(x)| \]
	die \emph{Maximumsnorm} (vgl. gleichmäßige Konvergenz).
	
	Für $ f,g\in C^0([0,1]) $,
		\[ f(x):= 1-x \text{ und } g(x) = x \]
	ist dann
		\[ \|f\|_\infty = \|g\|_\infty = \|f+g\|_\infty = \|f-g\|_\infty = 1 \]
	womit die Parallelogrammgleichung offenbar nicht erfüllt, und die Norm keine induzierte Norm eines Skalarprodukts ist.