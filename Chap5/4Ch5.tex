\section{Euklidische Geometrie}
\subsection{Definition}\index{Euklidischer Raum}\index{Länge}\index{Abstand}\index{Winkel}
\begin{Definition}[Euklidischer Raum, Abstand, Länge, Winkel ]
	Ein \emph{Euklidischer Raum} ist eine affiner Raum $ (A,V,\tau) $ über einem Euklidischen Vektorraum $ (V,\Skl{.}{.}) $ mit induzierter Norm $ \|.\| $.
		\begin{itemize}
			\item Die \emph{Länge} eines Vektors $ v\in V $ ist seine Norm, der \emph{Abstand} zweier Punkte $ a,b\in A $ ist die Länge ihres Verbindungsvektors,
				\[ d(a,b) := \|b-a\| = \sqrt{\Skl{b-a}{b-a}}. \]
			\item Der \emph{Winkel} $ \alpha\in [0,\pi] $ zweier Vektoren $ v,w\in V^\times $ ist durch die Gleichung
				\[ \Skl{v}{w} = \|v\|\cdot \|w\|\cdot \cos \alpha \]
			definiert; der \emph{Winkel} (am Punkt $ a $) in einem nicht-degenerierten Dreieck $ \{a,b,c\} \subset A$ ist der Winkel der beiden Seitenvektoren $ v=b-a $ und $ w = c-a $.
		\end{itemize}
\end{Definition}

\begin{figure}[H]
\centering	
\definecolor{qqwuqq}{rgb}{0.,0.39215686274509803,0.}
\definecolor{qqqqff}{rgb}{0.,0.,1.}
\begin{tikzpicture}[line cap=round,line join=round,>=triangle 45,x=1.0cm,y=1.0cm]
\draw[->,color=black] (-1.9743290273232554,0.) -- (5.524000640987365,0.);
\foreach \x in {-1.,1.,2.,3.,4.,5.}
\draw[shift={(\x,0)},color=black] (0pt,2pt) -- (0pt,-2pt) node[below] {\footnotesize $\x$};
\draw[->,color=black] (0.,-1.133095836146613) -- (0.,4.526265243390068);
\foreach \y in {-1.,1.,2.,3.,4.}
\draw[shift={(0,\y)},color=black] (2pt,0pt) -- (-2pt,0pt) node[left] {\footnotesize $\y$};
\draw[color=black] (0pt,-10pt) node[right] {\footnotesize $0$};
\clip(-1.9743290273232554,-1.133095836146613) rectangle (5.524000640987365,4.526265243390068);
\draw [shift={(1.,1.)},color=qqwuqq,fill=qqwuqq,fill opacity=0.1] (0,0) -- (13.366930696316846:0.38851449058604254) arc (13.366930696316846:70.38212059188172:0.38851449058604254) -- cycle;
\draw [->] (1.,1.) -- (1.72,3.02);
\draw [->] (1.,1.) -- (3.02,1.48);
\begin{scriptsize}
\draw [fill=qqqqff] (1.,1.) circle (2.5pt);
\draw[color=qqqqff] (0.9654306181111331,0.848328065842202) node {$A$};
\draw [fill=qqqqff] (1.72,3.02) circle (2.5pt);
\draw[color=qqqqff] (1.8072120143808919,3.257117907475663) node {$B$};
\draw [fill=qqqqff] (3.02,1.48) circle (2.5pt);
\draw[color=qqqqff] (3.115210799353902,1.7160104281510296) node {$C$};
\draw[color=black] (1.1726383464236891,2.130425884776141) node {$v$};
\draw[color=black] (2.014419742693448,1.1202882092524316) node {$w$};
\draw[color=qqwuqq] (1.5093509049315927,1.495852216818939) node {$\alpha$};
\end{scriptsize}
\end{tikzpicture}
\end{figure}

\paragraph{Bemerkung}
	Nach der Cauchy-Schwarzschen Ungleichung ist für $ v,w\in V^\times $
		\[ \frac{\Skl{v}{w}}{\|v\|\cdot \|w\|}\in [-1,1]; \]
	andererseits ist 
		\[ \cos:[0,\pi]\to [-1,1]\text{ bijektiv} \]
	Damit ist der Winkel von Vektoren bzw. im Dreieck wohldefiniert.

\subsection{Definition}\index{Kongruenzabbildung}\index{Isometrie}\index{Ähnlichkeitstransformation}
\begin{Definition}[Kongruenzabbildung, Ähnlichkeitstransformation]
    Eine affine Transformation eines Euklidischen Raumes heißt
		\begin{itemize}
			\item \emph{Kongruenzabbildung} oder \emph{Isometrie}, falls sie Abstandstreu ist,
			\item \emph{Ähnlichkeitstransformation}, falls sie winkeltreu ist.
		\end{itemize}
\end{Definition}
\paragraph{Bemerkung}
	Jede Kongruenzabbildung ist Ähnlichkeitstransformation (Polarisation).
\paragraph{Bemerkung}
	Offenbar bilden die Kongruenz- bzw. Ähnlichkeitsabbildungen eines Euklidischen Raumes $ A $ auf $ A $ operierende (Transformations-)Gruppen.
	
\subsection{Definition (Geometrie)}\index{Euklidische Geometrie}\index{Ähnlichkeitsgeometrie}
\begin{Definition}[Euklidische Geometrie, Ähnlichkeitsgeometrie]
	Die auf einem Euklidischen Raum operierende Gruppe der Kongruenzabbildungen bestimmt eine Euklidische Geometrie.
	
	Die Gruppe der Ähnlichkeitstransformationen eines Euklidischen Raumes $ A $ bestimmt eine Ähnlichkeitsgeometrie.
\end{Definition}

% VO 19-05-2016 %

\paragraph{Beispiel}
	Jede Translation $ \tau_v:A\to A $ ist eine Isometrie:
	Für $ a,b\in A $ gilt
		\[ \exists!w\in V: b=\tau_w(a) \]
	d.h. $ w=b-a $; also
		\[ \tau_v(b) = \tau_v(\tau_w(a)) = \tau_{v+w}(a) = \tau_w(\tau_v (a)) \]
	d.h. $ w = \tau_v(b)-\tau_v(a) $.
	Damit folgt:
		\[ \|\tau_v(b)-\tau_v(a)\| = \|w\| = \|b-a\| \]
	d.h. $ \tau_v $ ist abstandstreu, da $ a,b\in A $ beliebig waren.
\paragraph{Beispiel}
	Die Streckung mit Zentrum $ o\in A $ um den Faktor $ s\in \mathbb{R}^\times $,
		\[ o+v=a\overset{\delta_s}{\mapsto}\delta_s(a) = \delta_s(o+v):= o+vs \]
	ist winkeltreu, denn für $ a=o+v, b=o+w $ gilt
		\[ \delta_s(b)-\delta_s(a) = (o+ws)-(o+vs) = \dots = (w-v)s \]
	und damit für drei paarweise verschiedene Punkte $ a,b,c\in A $
		\[ \cos \alpha = \frac{\Skl{\delta_s(b)-\delta_s(a)}{\delta_s(c)-\delta_s(a)}}{\|\delta_s(b)-\delta_s(a)\|\|\delta_s(c)-\delta_s(a)\|} = \frac{\Skl{(b-a)s}{(c-a)s}}{\|(b-a)s\|\|(c-a)s\|} =\frac{s^2}{|s^2|} \cdot \frac{\Skl{b-a}{c-a}}{\|b-a\|\|c-a\|} \]
	d.h. $ \delta_s $ ist winkeltreu; andererseits ist $ \delta_s $ für $ s\neq \pm 1 $ nicht abstandstreu.
	Ist $ a \neq b $, so gilt dann
		\[ \|\delta_s(b)-\delta_s(a)\| = \|b-a\|\cdot |s| \neq \|b-a\|. \] 
\paragraph{Zur Erinnerung}
	Jede affine Abbildung $ \alpha:A\to A' $ besitzt einen (eindeutigen) \emph{linearen Anteil} $ \lambda:V\to V' $, sodass
		\[ \forall a\in A\forall v\in V: \alpha(a+v) = \alpha(a)+\lambda(v); \]
	ist $ \alpha $ eine affine Transformation, so ist $ \lambda \in Gl(V) $.
\paragraph{Bemerkung}
	Jede Ähnlichkeitstransformation ist Komposition einer Streckung und einer Kongruenzabbildung.
	
	Nämlich: Ist $ \alpha $ Ähnlichkeitstransformation mit linearem Anteil $ \lambda\in Gl(V) $, so erhält $ \lambda $ Winkel von Vektoren, insbesondere also Orthogonalität.
	Nun wähle $ w\in V^\times $ und setze
		\[ s := \frac{\|w\|}{\|\lambda w\|}. \]
	Ist dann $ v\in V $ mit $ \|v\|=\|w\| $, so folgt
		\[ v+w \perp v-w \Rightarrow \lambda(v+w)\perp \lambda(v-w) \Rightarrow \|\lambda(v)\| = \|\lambda(w)\|, \]
	also 
		\[ \forall v\in V^\times: \frac{\|\lambda(v)\|}{\|v\|} = \|\lambda(v\frac{\|w\|}{\|v\|})\|\frac{1}{\|w\|} = \frac{\|\lambda (w)\|}{\|w\|} = \frac{1}{s}. \]
	Mit einem beliebigen Streckungszentrum $ o\in A $ erhält man also eine Isometrie durch
		\[ \delta_s\circ \alpha :A\to A. \]
\paragraph{Beispiel}
	Eine \emph{nicht-triviale} Scherung ist \emph{keine} Ähnlichkeitstransformation. Beweis in der Übung. % 3 Zeilen Rechnung, 5 Zeilen Begründung!

\subsection{Lemma \& Definition}
	Eine affine Transformation $ \alpha:A\to A $ eines Euklidischen Raumes $ A $ ist genau dann eine Kongruenzabbildung, wenn ihr linearer Anteil $ \lambda $ \emph{orthogonal} ist:
		\[ \lambda\in O(V):= \{f\in Gl(V)\mid \forall v,w\in V: \Skl{f(v)}{f(w)} = \Skl{v}{w}\}. \]
	$ O(V) $ heißt die \emph{orthonogale Gruppe} von $ (V,\Skl{.}{.}) $.
\paragraph{Bemerkung}
	$ O(V)\subset Gl(V) $ ist eine Gruppe. Beweis in der Übung.
\paragraph{Bemerkung}
	Ist $ f\in \End(V) $, so folgt die Injektivität von $ f $ aus
		\[ \forall v,w\in V: \Skl{f(v)}{f(w)} = \Skl{v}{w}. \]
	Aus $ f(v) = 0 $ folgt nämlich
		\[ 0 = \|f(v)\| = 0 = \|v\| \Rightarrow v = 0, \text{ da } \Skl{.}{.}\text{ pos. definit.} \]
	Ist $ \dim V <\infty $, so folgt mit dem Rangsatz, $ \dim V = \rg f + \dfkt f = \rg $, dass $ f\in Gl(V) $.
	
	Im Fall $ \dim V = \infty $ ist $ f $ nicht notwendigerweise surjektiv, wie der \emph{Shiftoperator}
		\[ f\in \End(\R^\N), \forall n\in \N: f(e_n) = e_{n+1} \]
	zeigt.
\paragraph{Beweis (Lemma)}
	Sei $ (A,V,\tau) $ Euklidischer Raum über einem Euklidischen VR $ (V,\Skl{.}{.}) $ und $\alpha:A\to A $ Affinität mit linearem Anteil $ \lambda\in Gl(V) $. Dann ist $ \alpha $ genau dann Isometrie, wenn
		\[ \forall a,b\in A: \|\lambda(b-a)\| = \|\alpha(b)-\alpha(a)\| = \|b-a\|,  \]
	also (Polarisation), wenn $ \lambda\in O(V) $.
\subsection{Definition}
	Ist $ (V,\Skl{.}{.}) $ unitärer VR, so heißt $ f\in Gl(V) $ mit
		\[ \forall v,w\in V: \Skl{f(v)}{f(w)} = \Skl{v}{w} \]
	\emph{unitär}; die \emph{unitäre Gruppe} von $ (V,\Skl{.}{.}) $ ist die Gruppe
		\[ U(V) := \{f\in Gl(V)\mid \forall v,w\in V: \Skl{f(v)}{f(w)} = \Skl{v}{w} \}. \]