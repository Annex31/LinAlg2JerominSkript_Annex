\tdplotsetmaincoords{0}{0} %-27
 	\begin{tikzpicture}[yscale=1,tdplot_main_coords]

 		\def\xstart{-1.0} %x Koordinate der Startposition der Grafik
 		\def\ystart{-1.9} %y Koordinate der Startposition der Grafik
 		\def\myscale{1.0} %ändert die Größe der Grafik (Skalierung der Grafik)
        \def\myscalex{(\myscale)}
        \def\myscaley{(\myscale)}
                
 		\def\xstartdraw{(\xstart + 4.2)} %xKoordinate des Referenzstartpunktes (in dieser Zeichnung: a)
 		\def\ystartdraw{(\ystart + 1.0)}%yKoordinate des Referenzstartpunktes (in dieser Zeichnung: a)

 		\def\balkenhoehe{(5.0)}% Länge des vertikalen blauen Balkens
 		\def\balkenlaenge{(7.5)}% Länge des horizontalen blauen Balkens
 		\def\balkenbreite{0.4} %Balkenbreite

 		%---------Begin Balken----------
 		\def\drehwinkel{0}
 		\node (VekV) at ({\xstart+0.2*cos(\drehwinkel)-\balkenbreite*sin(\drehwinkel)},{\ystart+0.5*sin(\drehwinkel)+\balkenbreite*cos(\drehwinkel)})[right, xshift=1,color=blue] {$\mathbb{R}^2$};
 		\node (AffA) at ({\xstart+(\balkenlaenge-0.5)*cos(\drehwinkel)},{\ystart+(\balkenlaenge-0.5)*sin(\drehwinkel)+\balkenbreite*cos(\drehwinkel)})[color=red] {$A$};

 		\path[ shade, top color=white, bottom color=blue, opacity=.6]
 		({\xstart},{\ystart},0)  -- ({\xstart - \balkenbreite * cos(\drehwinkel)- (-\balkenbreite+0)*sin(\drehwinkel)},{\ystart - \balkenbreite * sin(\drehwinkel)+ (-\balkenbreite+0)*cos(\drehwinkel)},0)  -- ({\xstart - \balkenbreite * cos(\drehwinkel)- (\balkenhoehe+0.5)*sin(\drehwinkel)},{\ystart - \balkenbreite * sin(\drehwinkel)+ (\balkenhoehe+0.5)*cos(\drehwinkel)},0) -- ({\xstart - 0 * cos(\drehwinkel)- (\balkenhoehe+0)*sin(\drehwinkel)},{\ystart - 0 * sin(\drehwinkel)+ (\balkenhoehe+0)*cos(\drehwinkel)},0) -- cycle;

 		\path[ shade, right color=white, left color=blue, opacity=.6]
 		({\xstart},{\ystart},0)  -- ({\xstart - \balkenbreite * cos(\drehwinkel)- (-\balkenbreite+0)*sin(\drehwinkel)},{\ystart - \balkenbreite * sin(\drehwinkel)+ (-\balkenbreite+0)*cos(\drehwinkel)},0) --
 		({\xstart + (\balkenlaenge+0.5) * cos(\drehwinkel)- (-\balkenbreite+0)*sin(\drehwinkel)},{\ystart + (\balkenlaenge+0.5) * sin(\drehwinkel)+ (-\balkenbreite+0)*cos(\drehwinkel)},0) --
 		({\xstart + \balkenlaenge * cos(\drehwinkel)},{\ystart + \balkenlaenge * sin(\drehwinkel)},0)--
 		cycle;
 		%---------End Balken----------
 	
 		\node (pointz1) at ({\xstartdraw},{\ystartdraw}) {};
 		
        \def\cradius{2.5}
 		
 		%\draw[color=green] (pointz1) circle [x=1cm,y=1cm,radius=\cradius cm]node[below, xshift=5, yshift=0]{};
 		
 		\node (pointa1) at ($(pointz1) + (180:\cradius)$) {};
 		%\node (pointb1) at ($(pointz1) + (140:2.5)$) {};
 		\node (pointb1) at ($(pointz1) + (35:\cradius)$) {};
 		\node (pointc1) at ($(pointz1) + (100:\cradius)$) {};
 		%\node (pointb12) at ($(pointz1) + (140:3.0)$) {};
 		%\node (pointb2) at ($(pointz1) + (140:-2.0)$) {};
 		\node (pointsa1b1) at ($(pointa1)!(pointz1)!(pointb1)$) {};
 		\node (pointsb1c1) at ($(pointb1)!(pointz1)!(pointc1)$) {};
 		\node (pointsa1c1) at ($(pointa1)!(pointz1)!(pointc1)$) [] {};
 		
 		\node (pointha1b1) at ($(pointa1)!(pointc1)!(pointb1)$) {};
 		\node (pointhb1c1) at ($(pointc1)!(pointa1)!(pointb1)$) {};
 		\node (pointha1c1) at ($(pointc1)!(pointb1)!(pointa1)$) {};
 		
 		\node (offset_for_p) at ($(pointsa1b1) - (pointz1)$){};
 		\node (pointp) at ($(pointz1) - (offset_for_p)$){};
 
 		\draw[name path=a1--b1,-,shorten >=-30pt, shorten <=-30pt,line width=0.1pt,color=red] (pointa1) -- (pointb1);
 	    \draw[name path=c1--b1,-,shorten >=-30pt, shorten <=-45pt,line width=0.1pt,color=red] (pointc1) -- (pointb1);
 	    \draw[name path=a1--c1,-,shorten >=-40pt, shorten <=-30pt,line width=0.pt,color=red] (pointa1) -- (pointc1);
 	    
 	    \node (offset_for_hc) at ($(pointha1b1) - (pointc1)$){};
 	    \draw [name path=linehc, color=brown!70!black, -, line width=0.5pt, shorten >=-2mm, shorten <=-8mm, ] (pointha1b1) -- ($(pointc1) - (offset_for_hc)$);
 	   
 	    \node (offset_for_ha) at ($(pointhb1c1) - (pointa1)$){};
 	    \draw [name path=lineha, color=brown!70!black, -, line width=0.5pt, shorten >=0mm, shorten <=-8mm, ] (pointa1) -- ($(pointhb1c1) + 0.5*(offset_for_ha)$);
 	    
 	    \node (offset_for_hb) at ($(pointha1c1) - (pointb1)$){};
 	    \draw [name path=linehb, color=brown!70!black, -, line width=0.5pt, shorten >=-2mm, shorten <=-8mm, ] (pointb1) -- ($(pointha1c1) + 0.6*(offset_for_hb)$);
 	    (pointha1c1) -- (pointb1);
 	    
 	    
        \path [name intersections={of = linehc and lineha, by=HS }];
 	    
 	    %Beschriftung der Hoehen
 	    \node[ xshift=-3mm, yshift=0mm,color=brown!70!black,rotate=0] (label_ha) at ($(pointa1) !0.6!(HS) $) {{\small $h_{a}$}};
 	    \node[ xshift=4mm, yshift=0mm,color=brown!70!black,rotate=0] (label_hb) at ($(pointb1) !0.4!(HS) $) {{\small $h_{b}$}};
 	    \node[ xshift=0mm, yshift=0mm,color=brown!70!black,rotate=0] (label_hc) at ($(pointha1b1) + 0.5*(offset_for_hc) $) {{\small $h_{c}$}};
 	    
 	    %rechter Winkel
 	    \draw[line width=0.2pt,color=blue] ($(pointha1c1) + (-40:0.3)$) arc[radius=0.3, start angle=-40, end angle=50] ($(pointha1c1) + (50:0.3)$);
 	    
 	     \draw[line width=0.2pt,color=blue] ($(pointhb1c1) + (155:0.3)$) arc[radius=0.3, start angle=155, end angle=245] ($(pointha1c1) + (245:0.3)$);
 	     
 	     \draw[line width=0.2pt,color=blue] ($(pointha1b1) + (15:0.3)$) arc[radius=0.3, start angle=15, end angle=105] ($(pointha1c1) + (105:0.3)$);
 	    
 	    %Vektoren blau
 	    \draw[name path=p--a,-{>[scale=1,length=8,width=8]},shorten >=4pt, shorten <=4pt,line width=0.2pt,color=blue] (pointa1)  -- ($(pointa1)!0.4!(HS)$);
 	    \draw[name path=p--a,-{>[scale=1,length=8,width=8]},shorten >=2pt, shorten <=2pt,line width=0.2pt,color=blue] (pointc1)  -- (pointb1);
 	    
 	   	%Punkte malen
 		%rechter Winkel
 		\node (pointrw1) at ($(pointha1b1) + (60:0.15)$) {};
 		\node (pointrw2) at ($(pointhb1c1) + (200:0.15)$) {};
 		\node (pointrw3) at ($(pointha1c1) + (5:0.15)$) {};
 		
 		%Punkte rechter Winkel
 		\draw[fill,color=blue] (pointrw1) circle [radius=0.02]node[above, xshift=0, yshift=0]{};
 		\draw[fill,color=blue] (pointrw2) circle [radius=0.02]node[above, xshift=0, yshift=0]{};
 		\draw[fill,color=blue] (pointrw3) circle [radius=0.02]node[above, xshift=0, yshift=0]{};
 		
 		%Punkte weiss
 		\draw[fill,color=white] (pointa1) circle [radius=0.11] node[below, xshift=5, yshift=0]{};
 		\draw[fill,color=white] (pointb1) circle [radius=0.11] node[below, xshift=5, yshift=0]{};
 		\draw[fill,color=white] (pointc1) circle [radius=0.11] node[below, xshift=5, yshift=0]{};
 		\draw[fill,color=white] (HS) circle [radius=0.11] node[below, xshift=5, yshift=0]{};
 		
 		\draw[fill,color=white] ($(pointa1)!0.4!(HS)$) circle [radius=0.11] node[below, xshift=5, yshift=0]{};
 		
 		%Punkte rot
 		\draw[fill,color=red] (pointa1) circle [radius=0.06]node[below, xshift=5, yshift=0]{};
 		\draw[fill,color=red] (pointb1) circle [radius=0.06]node[below, xshift=5, yshift=0]{};
 		\draw[fill,color=red] (pointc1) circle [radius=0.06]node[below, xshift=5, yshift=0]{};
 		\draw[fill,color=brown!70!black] (HS) circle [radius=0.06]node[below, xshift=5, yshift=0]{};
 	
 	
        \draw[fill,color=brown!70!black] ($(pointa1)!0.4!(HS)$) circle [radius=0.06] node[below, xshift=5, yshift=0]{};
        
 		%Beschriftung der Punkte
 		\node[ xshift=-2mm, yshift=1mm,color=red] (labela1) at (pointa1) {$a$};
 		\node[ xshift=1mm, yshift=3mm,color=red] (labelb1) at (pointb1) {$b$};
 		\node[ xshift=0.5mm, yshift=3.5mm,color=red] (labelc1) at (pointc1) {$c$};
 		\node[ xshift=19mm, yshift=1mm,color=brown!70!black] (labelhs) at (HS) {\small Höhenschnittpunkt};
 	
 	    \node[ xshift=2.5mm, yshift=-1mm,color=brown!70!black] (labela1) at ($(pointa1)!0.4!(HS)$) {\small $p$};
 	     
 	    \node[ xshift=-5mm, yshift=1mm,color=blue] (labela1) at ($(pointa1)!0.2!(HS)$) {\small $p-a$};
 	    \node[ xshift=-3mm, yshift=-2mm,color=blue] (labela1) at ($(pointc1)!0.5!(pointb1)$) {\small $b-c$};
 	
\end{tikzpicture}