\chapter{Struktursätze für Endomorphismen}
\section{Adjungierte \& duale Abbildungen}
	Zunächst: Ziel dieses Kapitels ist besseres, strukturelles Verständnis der Bedingungen für orthogonale Transformationen bzw. Orthogonalprojektionen:
		\[ \forall v,w\in V: \Skl{f(v)}{f(w)} = \Skl{v}{w} \text{ bzw. } \Skl{p(v)}{w} = \Skl{v}{p(w)}. \]
\paragraph{Motivation}
	Jede Sesquilinearform $ \sigma: V\times V\to K $ liefert  eine (semi-)lineare Abbildung
		\[ V\ni v\mapsto \sigma(v,.)\in V^*. \]
	Ist die Sesquilinearform $ \sigma $ symmetrisch und nicht-degeneriert, d.h. ein Skalarprodukt auf $ V $, so ist die Abbildung injektiv:
		\[ \sigma(v,.) = 0 \Rightarrow v = 0 \Rightarrow v\in V^\perp = \{0\}. \]
	Ist die Abbildung auch surjektiv, so kann man sie benutzen, um $ V^* $ und $ V $ zu identifizieren,
		\[ V^* \mathop{\overline{\simeq}} V. \]
\subsection{Rieszsches Darstellungslemma}
	Sei $ (V,\Skl{.}{.}) $ ein $ K $-VR mit Skalarprodukt. Die \emph{kanonische Injektion}
		\[ \phi : V\to V^*, v\mapsto \phi(v):= \Skl{v}{.}. \]
	ist semi-lineare und injektiv. Ist $ \dim V < \infty $, so ist $ \phi $ auch surjektiv; wir nennen dann
		\begin{itemize}
			\item $\nabla w := \phi^{-1}(w) $ den \emph{Gradienten} von $ w\in V^* $, und
			\item $ \phi: V\to V^* $ die \emph{kanonische Identifikation} von $ (V,\Skl{.}{.}) $ mit $ V^* $.
		\end{itemize}