\chapter{Struktursätze für Endomorphismen}
\section{Adjungierte \& duale Abbildungen}
	Zunächst: Ziel dieses Kapitels ist besseres, strukturelles Verständnis der Bedingungen für orthogonale Transformationen bzw. Orthogonalprojektionen:
		\[ \forall v,w\in V: \Skl{f(v)}{f(w)} = \Skl{v}{w} \text{ bzw. } \Skl{p(v)}{w} = \Skl{v}{p(w)}. \]
\paragraph{Motivation}
	Jede Sesquilinearform $ \sigma: V\times V\to K $ liefert  eine (semi-)lineare Abbildung
		\[ V\ni v\mapsto \sigma(v,.)\in V^*. \]
	Ist die Sesquilinearform $ \sigma $ symmetrisch und nicht-degeneriert, d.h. ein Skalarprodukt auf $ V $, so ist die Abbildung injektiv:
		\[ \sigma(v,.) = 0 \Rightarrow v = 0 \Rightarrow v\in V^\perp = \{0\}. \]
	Ist die Abbildung auch surjektiv, so kann man sie benutzen, um $ V^* $ und $ V $ zu identifizieren,
		\[ V^* \mathop{\overline{\simeq}} V. \]
\subsection{Rieszsches Darstellungslemma}
\begin{Lemma}[Rieszsches Darstellungslemma]
	Sei $ (V,\Skl{.}{.}) $ ein $ K $-VR mit Skalarprodukt. Die \emph{kanonische Injektion}
		\[ \phi : V\to V^*, v\mapsto \phi(v):= \Skl{v}{.}. \]
	ist semi-linear und injektiv. Ist $ \dim V < \infty $, so ist $ \phi $ auch surjektiv; wir nennen dann
		\begin{itemize}
			\item $\nabla w := \phi^{-1}(w) $ den \emph{Gradienten} von $ w\in V^* $, und
			\item $ \phi: V\to V^* $ die \emph{kanonische Identifikation} von $ (V,\Skl{.}{.}) $ mit $ V^* $.
		\end{itemize}
\end{Lemma}
% VO 02-06-2016 %
\paragraph{Beweis}
	Semi-Linearität und Injektivität folgen sofort aus den Eigenschaften des Skalarprodukts.\footnote{Semi-Linearität in der linken Komponente; Nicht-Degeneriertheit impliziert Injektvität.}
	
	Ist $ \dim V <\infty $, so ist $ \phi:V\to V^* $ wegen $ \dim V^* = \dim V $ und der Injektivität auch surjektiv.
\paragraph{Bemerkung}
	Dies ist eine "`kleine"' Version des Rieszschen Darstellungssatzes
	\[ \forall \omega\in V^*\exists! w\in V: \omega = \Skl{w}{.}. \]
	Der "`richtige"' Satz schränkt die Dimension nicht ein, und ist ein wichtiges Hilfsmittel in der Funktional-Analysis.
\paragraph{Bemerkung}
	Ist $ E = (e_1,\dots,e_n) $ ONB von $ (V,\Skl{.}{.}) $, so gilt für die Vektoren der dualen Basis $ E^* = (e_1^*,\dots,e_n^*) $
		\[ \nabla e_i^* = e_i\frac{1}{\Skl{e_i}{e_i}} = \pm e_i, \]
	da für $ j=1,\dots,n $ gilt:
		\[ \Skl{e_i\frac{1}{\Skl{e_i}{e_i}}}{e_j} = \delta_{ij} = e_i^*(e_j), \]
	also
		\[ \phi(e_i\frac{1}{\Skl{e_i}{e_i}}) = e_i^*. \]
	Insbesondere gilt im Falle eines Euklidischen VR
		\[ \forall i=1,\dots,n: \nabla e_i^* = e_i \Leftrightarrow \phi(e_i)= e_i^*, \]
	d.h. $ \phi $ realisiert den früher diskutierten (vgl. Abschnitt 1.4) durch duale Basen gegebenen
	Isomorphismus -- im Falle von ONB.

\subsection{Korollar \& Definition}\index{Adjungierte}
\begin{Definition}[Adjungierte]
	Sind $ (w,\SSkl{.}{.}) $ und $ (V,\Skl{.}{.}) $ Vektorräume mit Skalarprodukten, $ \dim W < \infty $ und $ f\in \hom(W,V) $, so hat $ f $ eine eindeutige \emph{Adjungierte} $ f^* \in \hom(V,W) $;
	dabei ist $ f^* $ \emph{adjungiert zu $ f $}, falls
		\[ \forall v\in V\forall w\in W: \SSkl{f^*(v)}{w} = \Skl{v}{f(w)}. \]
\end{Definition}
\paragraph{Achtung:}
$ V $ und $ W $ sind VR über dem gleichen Körper $ K $; die Skalarprodukte sind sesquilinear bzgl. des gleichen Körperautomorphismus!
\paragraph{Beweis}
	Für jedes $ v\in V $ definiert
		\[ \omega_v:W\to K, w\mapsto \omega_v(w) := \Skl{v}{f(w)} \]
	eine Linearform $ \omega_v\in W^* $; nach Rieszschem Darstellungslemma erhält man daher eine eindeutige Abbildung 
		\[ f^*:V\to W, v\mapsto f^*(v):= \nabla \omega_v. \]
	Die Linearität von $ f^* $ folgt aus der dualen Abbildung, siehe unten.
\paragraph{Bemerkung}
	Offenbar (Symmetrie) ist $ f^{**} = f $, wenn $ f^{**} := (f^*)^* $ existiert.

\subsection{Definition \& Lemma}
\begin{Definition}[transponiert, dual]
	Ist $ f\in \hom(W,V) $, so heißt $ f^t\in \hom(V^*,W^*) $,
		\[ f^t:V^*\to W^*,\nu\mapsto f^t(\nu):= \nu\circ f \]
	zu $ f $ \emph{transponiert} oder \emph{dual}. Sind $ \SSkl{.}{.} $ und $ \Skl{.}{.} $ Skalarprodukte auf $ W $ bzw. $ V $ und
		\[ \psi:W\to W^*\text{ und }\phi:V\to V^* \]
	die zugehörigen kanonischen Injektionen, und ist $ f^*\in \hom(V,W) $ adjungiert zu $ f $, so gilt
		\[ \psi \circ f^* = f^t\circ \phi. \]
\end{Definition}
\paragraph{Beweis}
	Für $ v\in V $ und $ w\in W $ gilt:
		\[ \left((\psi\circ f^*)(v)\right)(w) = \SSkl{f^*(v)}{w} \]
		\[ \left((f^t\circ \phi)(v)\right)(w) = (\phi(v)\circ f)(w) = \Skl{v}{f(w)} \]
	und nach Definition der Adjungierten folgt die Gleichheit.
\paragraph{Bemerkung}
	Ist $ \dim W<\infty $, so ist $ \psi $ bijektiv und das Resultat des Lemmas kann als Definition dienen:
		\[ f^* := \psi^{-1}\circ f^t\circ \phi. \]
	Wegen $ f^t\in \hom(V^*,W^*) $ folgt damit auch $ f^*\in \hom(V,W) $.
\paragraph{Bemerkung}
	$ f\in \hom(W,V) $ hat \emph{immer} eine Transponierte, eine Adjungierte aber nur unter bestimmten Voraussetzungen, z.B. wenn $ \dim W < \infty $.
\paragraph{Bemerkung}
	Oft wird die Transponierte/Duale $ f^t $ auch mit $ f^* $ bezeichnet.
\paragraph{Buchhaltung}
	Sind $ B=(b_1,\dots,b_n) $ und $ C=(c_1,\dots,c_m) $ Basen von $ V $ bzw. $ W $ und $ f\in \hom(W,V) $, so gilt:
		\[ \xi_{B^*}^{C^*}(f^t) = \left(\xi_C^B(f)\right)^t. \]
	Sind $ (V,\Skl{.}{.}) $ und $ (W,\SSkl{.}{.}) $ unitär (oder Euklidisch) und $ B, C $ ONB, so gilt
		\[ \xi_B^C(f^*) = \left(\xi_C^B(f)\right)^*, \]
	wobei 
		\[ X^*:= \overline{X}^t \text{ für } X\in K^{n\times m}. \]
	In diesem Falle gilt nämlich $ b_j^*=\phi(b_j) $ und $ c_i^* = \psi(c_i) $ und damit
		\[ x_{ij}^* = c_i^*(f^*(b_j)) = \SSkl{c_i}{f^*(b_j)} = \overline{\Skl{b_j}{f(c_i)}} = \overline{x_{ji}}. \]
\paragraph{Bemerkung}
	Sind $ f,g \in \hom(W,V)$ und $ x\in K $, so gilt
		\[ (f+gx)^t = f^t+g^tx \text{ und } (f+gx)^* = f^*+g^*\overline{x}. \]
		
\subsection{Lemma}
\begin{Lemma}
	Sind $ f\in \hom(W,V) $ und $ g\in \hom(V,U) $, so gilt
		\[ (g\circ f)^t = f^t \circ g^t \text{ und } (g\circ f)^* = f^* \circ g^*. \]
\end{Lemma}
\paragraph{Beweis}
	Nachrechnen/-lesen oder über ein kommutatives Diagramm.
	%------------------ KommutativesDiagramm ----------------
	% https://de.wikipedia.org/wiki/Kommutatives_Diagramm
     	\begin{figure}[H]\centering
     			\begin{tikzpicture}[yscale=1]

 		\def\xstart{0} %x Koordinate der Startposition der Grafik
 		\def\ystart{0} %y Koordinate der Startposition der Grafik
 		\def\myscale{1.0} %ändert die Größe der Grafik (Skalierung der Grafik)
        \def\myscalex{(\myscale)}
        \def\myscaley{(\myscale)}
                
 		\def\xstartdraw{(\xstart + 0.0)} %xKoordinate des Referenzstartpunktes (in dieser Zeichnung: a)
 		\def\ystartdraw{(\ystart + 0.0)}%yKoordinate des Referenzstartpunktes (in dieser Zeichnung: a)
 		\def\abstandx{2.8}
 		\def\abstandy{2.0}
 		  
        \node (pointw) at ({\xstartdraw},{\ystartdraw}) {$W$};
        \node (pointv) at ($(pointw) + (0:\abstandx)$) {$V$};
        \node (pointu) at ($(pointv) + (0:\abstandx)$) {$U$};
 		
 		\node (pointwst) at ($(pointw) + (270:\abstandy)$) {$W^*$};
 		\node (pointvst) at ($(pointwst) + (0:\abstandx)$) {$V^*$};
        \node (pointust) at ($(pointvst) + (0:\abstandx)$) {$U^*$};
        
        \draw[-{>[scale=1,length=6,width=6]},shorten >=4pt, shorten <=4pt,line width=0.2pt,color=black] (pointw)  -- (pointwst);
        \draw[-{>[scale=1,length=6,width=6]},shorten >=4pt, shorten <=4pt,line width=0.2pt,color=black] (pointv)  -- (pointvst);
        \draw[-{>[scale=1,length=6,width=6]},shorten >=4pt, shorten <=4pt,line width=0.2pt,color=black] (pointu)  -- (pointust);
        
        \draw[-{>[scale=1,length=6,width=6]},shorten >=4pt, shorten <=4pt,line width=0.2pt,color=black] (pointust)  -- (pointvst);
        \draw[-{>[scale=1,length=6,width=6]},shorten >=4pt, shorten <=4pt,line width=0.2pt,color=black] (pointvst)  -- (pointwst);
        
        \draw [-{>[scale=1,length=6,width=6]},shorten >=4pt, shorten <=4pt,line width=0.2pt,color=black] (pointw) to [bend right=-25] (pointv);
        \draw [-{>[scale=1,length=6,width=6]},shorten >=4pt, shorten <=4pt,line width=0.2pt,color=black] (pointv) to [bend right=-25] (pointw);
        
        \draw [-{>[scale=1,length=6,width=6]},shorten >=4pt, shorten <=4pt,line width=0.2pt,color=black] (pointv) to [bend right=-25] (pointu);
        \draw [-{>[scale=1,length=6,width=6]},shorten >=4pt, shorten <=4pt,line width=0.2pt,color=black] (pointu) to [bend right=-25] (pointv);
        
        \draw [-{>[scale=1,length=6,width=6]},shorten >=4pt, shorten <=4pt,line width=0.2pt,color=black] (pointust) to [bend right=-25] (pointwst);
        
        \draw [-{>[scale=1,length=6,width=6]},shorten >=4pt, shorten <=4pt,line width=0.2pt,color=black] (pointw.north) to [bend right=-35] (pointu.north);
        
        \draw [-{>[scale=1,length=6,width=6]},shorten >=4pt, shorten <=4pt,line width=0.2pt,color=black] [bend right=-35] (pointu.east) to [bend right=-46]($(pointust) + (-25:5mm)$) to [bend right=-46]($(pointwst) + (195:5mm)$) to [bend right=-45] (pointw.west);
        
        \node[ xshift=2mm, yshift=0mm,color=black!70!black] (labelpsi) at ($(pointw)!0.5!(pointwst)$) {$\psi$};
        \node[ xshift=2mm, yshift=0mm,color=black!70!black] (labelphi) at ($(pointv)!0.5!(pointvst)$) {$\phi$};
        \node[ xshift=2mm, yshift=0mm,color=black!70!black] (labelmu) at ($(pointu)!0.5!(pointust)$) {$\mu$};
        
        \node[ xshift=0mm, yshift=6.5mm,color=black!70!black] (labelf) at ($(pointw)!0.5!(pointv)$) {\small $f$};
        \node[ xshift=0mm, yshift=-6.5mm,color=black!70!black] (labelfst) at ($(pointw)!0.5!(pointv)$) {\small $f^*$};
        
        \node[ xshift=0mm, yshift=6.5mm,color=black!70!black] (labelg) at ($(pointv)!0.5!(pointu)$) {\small $g$};
        \node[ xshift=0mm, yshift=-6.5mm,color=black!70!black] (labelgst) at ($(pointv)!0.5!(pointu)$) {\small $g^*$};
        
        \node[ xshift=0mm, yshift=2.5mm,color=black!70!black] (labelfst) at ($(pointwst)!0.5!(pointvst)$) {\small $f^t$};
        \node[ xshift=0mm, yshift=2.5mm,color=black!70!black] (labelgst) at ($(pointvst)!0.5!(pointust)$) {\small $g^t$};
        
        \node[ xshift=0mm, yshift=10mm,color=black!70!black] (labelgf) at ($(pointv)$) {\small $g \circ f$};
        \node[ xshift=0mm, yshift=-10.5mm,color=black!70!black] (labelgft) at ($(pointvst)$) {\small $(g \circ f)^{t} = f^t \circ g^t$};
        \node[ xshift=0mm, yshift=-17.5mm,color=black!70!black] (labelgft) at ($(pointvst)$) {\small $(g \circ f)^{*} = f^* \circ g^*$};
\end{tikzpicture}
    	\end{figure}
   %------------------ KommutativesDiagramm ----------------	

