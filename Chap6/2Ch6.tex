\section{Normale Endomorphismen}
\paragraph{Motivation}
	Für orthogonale/unitäre selbst- und schiefadjungierte Endomorphismen $ f\in \End(V) $ gilt stets
		\[ f^*\circ f = f\circ f^*. \]
	Dies ist eine "`gute"' Eigenschaft: sie liefert viele/wichtige strukturelle Aussagen über Endomorphismen.
\paragraph{Generalvoraussetzung}
	In diesem Abschnitt ist $ (V,\Skl{.}{.}) $ Euklidisch oder unitär.

\subsection{Definition}\index{normal (Endomorphismen)}
\begin{Definition}[normal]
	$ f\in \End(V) $ heißt \emph{normal}, wenn $ f $ eine Adjungierte $ f^*\in \End(V) $ besitzt und 
		\[ f^*\circ f = f\circ f^*. \]
\end{Definition}
\subsection{Buchhaltung}
	Ist $ \dim V < \infty $ und $ X=\xi_B^B(f) $ Darstellungsmatrix von $ f\in \End(V) $ bzgl. einer ONB $ B $ von $ (V,\Skl{.}{.}) $, so gilt
		\[ f \text{ normal}\Leftrightarrow X^*X=XX^*, \]
	d.h. wenn $ X\in \K^{n\times n} $ \emph{normal} ist.
	
\subsection{Lemma}
\begin{Lemma}[]
	Ist $ f\in \End(V) $ normal, so gilt:
		\begin{enumerate}[(i)]
			\item $ \ker f = \ker f^* = f(V)^\perp $;
			\item $ \forall v,w\in V: \Skl{f^*(v)}{f^*(w)}=\Skl{f(v)}{f(w)} $;
			\item $ \forall x\in \K\forall v\in V: f(v) = vx \Rightarrow f^*(v) = v\overline{x} $.
			\item Sind $ v,w\in V $ Eigenvektoren zu EW $ x,y\in \K $ von $ f $, so gilt
				\[ x = y \text{ oder } v\perp w. \]
		\end{enumerate}
\end{Lemma}
\paragraph{Beweis}
	Sei $ f\in \End(V) $ normal.
		\begin{enumerate}
			\item[(ii)] Wegen $ f^{**} = f $ gilt für $ v,w\in V $:
				\[ \Skl{f^*(v)}{f^*(w)}-\Skl{f(v)}{f(w)} = \Skl{(f^{**}\circ f^*)(v)}{w}-\Skl{(f^*\circ f)(v)}{w} = \Skl{(f\circ f^*-f^*\circ f)(v)}{w} = 0 \]
			\item[(i)] Wegen (ii) gilt für $ v\in V $
				\[ f^*(v) = 0 \Rightarrow 0 = \|f^*(v)\|^2 = \|f(v)\|^2\Rightarrow f(v) = 0 \]
				und umgekehrt, und damit
				\[ \ker f^* = \ker f. \]
				Nach früherem Lemma ist
					\[ \ker f^* = f(V)^\perp \]
			\item[(iii)] Nach (i) ist für $ x\in \K $
				\[ \ker(f-\id_Vx) = \ker(f-\id_Vx)^* = \ker(f^*-\id_v\overline{x}), \]
				da $ f-\id_Vx $ mit $ f $ normal ist.
			\item[(iv)] Mit (iii) folgt für $ v,w\in V $ mit $ f(v) = vx $ und $ f(w) = wy $	\[ (x-y)\Skl{v}{w}=\Skl{v\overline{x}}{w}-\Skl{v}{wy} = \Skl{f^*(v)}{w}-\Skl{v}{f(w)} = 0. \] 
		\end{enumerate}

% VO 09-06-2016 %
\subsection{Lemma}
\begin{Lemma}[]
	Ist $ f\in \End(V) $ normal und $ U\subset V $ UVR, so gilt
		\begin{enumerate}[(i)]
			\item Ist $ U\ f $-invariant, so ist $ U^\perp\ f^* $-invariant;
			\item Ist $ U\ f $- und $ f^* $-invariant, so liefert Einschränkng normale Endomorphismen
				\[ f\big|_U \in \End(U) \text{ und } f\big|_{U^\perp}\in \End(U^\perp). \]
		\end{enumerate}
\end{Lemma}
\paragraph{Beweis}
	Für (i) wird nur die Existenz der Adjungierten benutzt.
		\begin{enumerate}[(i)]
			\item Sei $ v\in U^\perp $, dann gilt:
				\[ \forall u\in U: \Skl{f^*(v)}{u} = \Skl{v}{f(u)} = 0 \]
			\item Da $ U\ f $- und $ f^* $-invariant ist, ist (nach(i)) $ U^\perp\ f^*$- und $ f^{**} = f $-invariant.		
			Damit ist es sinnvoll
				\[ f\big|_U \in \End(U), f\big|_{U^\perp}\in \End(U^\perp) \]
				\[ f^*\big|_U\in \End(U), f^*\big|_{U^\perp}\in \End(U^\perp) \]
			zu betrachten. Nun gilt:
				\[ \forall u,v\in U: \Skl{f\big|_U^*(u)}{v} = \Skl{u}{f\big|_U(v)} = \Skl{u}{f(v)} = \Skl{f^*(u)}{v} = \Skl{f^*\big|_U(u)}{v} \]
			und analog für $ v,w\in U^\perp $. Damit folgt: $ f\big|_U^* = f^*\big|_U $ und $ f\big|^*_{U^\perp} = f^*\big|_{U^\perp}$ und also
				\[ f\big|_U^* \circ f\big|_U = f^*\circ f\big|_U = f\circ f^*\big|_U = f\big|_U \circ f\big|_U^*. \]
		\end{enumerate}
\paragraph{Bemerkung \& Beispiel}
	 Eine Orthogonalprojektion ist selbstadjungiert, $ p\in \End(V) $ mit $ p^2=p, p^*=p $ und damit normal. Ist $ p\neq \id_V, 0 $, so ist
		 \[ V=U\oplus_\perp U^\perp \text{ mit }
			 \begin{cases}
			 U:= p(V),\\ U^\perp = \ker p.
			 \end{cases} \]
	Wir definieren:
		\[ \pi:V\to U, v\mapsto \pi(v):= p(v) \text{ und } \iota : U\to V, u\mapsto \iota(u):= u; \]
	man nennt die isometrische Abbildung $ \iota $ auch die \emph{Inklusion} von $ U $ in $ V $. Dann sind $ \pi $ und $ \iota $ adjungiert:
		\[ \forall u\in U\forall v\in V: \Skl{\iota(u)}{v} = \Skl{u}{\pi(v)}\big|_U \]
	Insbesondere ist die "`Projektionsabbildung"' $ \pi $ \emph{nicht} selbstadjungiert.
\paragraph{Achtung:}
	Die Adjungierte hängt von Definitions- und Wertebereich ab!
	
\subsection{Spektralsatz (unitärer Fall)}\index{Spektralsatz}
\begin{Satz}[Spektralsatz]
	Sei $ (V,\Skl{.}{.}) $ unitär, $ \dim V<\infty $, und sei $ f\in \End(V) $ normal. Dann besitzt $ V $ eine ONB aus Eigenvektoren von $ f $.
\end{Satz}
\paragraph{Bemerkung}
	Es gilt auch die Umkehrung: Ist $ (e_1,\dots,e_n) $ ONB mit
		\[ f(e_i) = e_ix_i, \text{ also } f^*(e_i) = e_i\overline{x_i},\ i=1,\dots,n, \]
	so gilt
		\[ (f^*\circ f)(e_i) = e_i\overline{x_i}x_i = e_ix_i\overline{x_i} = (f\circ f^*)(e_i),\ i=1,\dots,n, \]
	d.h. $ f $ ist normal.
\paragraph{Beweis}
	Induktion über $ n=\dim V $.
	
	Für $ n=1 $ ist die Aussage trivial. Sei die Aussage für $ n\in\N $ wahr. Für $ n+1 $ gilt dann:
	
	$ f $ hat einen Eigenwert $ x\in \C $, da das charakteristische Polynom $ \chi_f(t)\in \C[t] $ nach Fundamentalsatz der Algebra in Linearfaktoren zerfällt.
	
	Sei $ e\in V^\times $ ein zugehöriger Eigenvektor,
		\[ f(e)=ex, \text{ o.B.d.A. } \|e\| = 1. \]
	Wegen $ f^*(e)=e\overline x $ ist $ [e] $ dann $ f $- und $ f^* $- invariant und damit
		\[ V=[e] \oplus_\perp [e]^\perp, \]
	wobei $ f\big|_{[e]}\in \End([e]) $ und $ f\big|_{[e]^\perp}\in \End([e]^\perp) $ normal sind (Lemma).
	
	Da $ \dim [e]^\perp = n$ liefert die Induktions-Annahme eine ONB $ (e_1,\dots,e_n) $ von $ [e]^\perp $ aus Eigenvektoren von $ f\big|_{[e]^\perp} $. Damit ist $ (e,e_1,\dots,e_n) $ eine ONB aus Eigenvektoren von $ f $.

\subsection{Buchhaltung}
	Ein normaler Endomorphismus $ f\in \End(V) $ eines unitären VR $ (V,\Skl{.}{.}) $ mit $ \dim V < \infty $ ist also \emph{orthogonal diagonalisierbar}, d.h. es existiert eine ONB aus Eigenvektoren von $ f $.
	
	Also, bezüglich einer solchen ONB $ B $ ist
		\[ \xi_B^B(f) = \operatorname{diag}(x_1,\dots,x_n). \]
	Da $ \xi_B^B(f^*) = (\xi_B^B(f))^* $ gilt:
		\begin{itemize}
			\item ist $ f $ selbstadjungiert, so sind alle Eigenwerte reell, $ x_i = \overline{x_i} $;
			\item ist $ f $ schiefadjungiert, so sind alle Eigenwerte imaginär, $ x_i = -\overline{x_i} $
			\item ist $ f $ unitär, so sind alle Eigenwerte \emph{unitär}, d.h. für $ j=1,\dots,n $ ist
				\[ x_j\in S^1 := \{x\in \C:\overline{x}x = 1\} = \{e^{iy}\mid y\in \R\}. \]
		\end{itemize}

\subsection{Korollar \& Definition}	
\begin{Korollar}[]
	Ist $ X\in \C^{n\times n} $ normal, $ X^*X = XX^* $, so gilt
		\[ \exists P\in U(n): P^{-1}XP=\operatorname{diag}(x_1,\dots,x_n). \]
	Dabei gilt:
		\begin{itemize}
			\item ist $ X $ \emph{selbstadjungiert}, $ X^*=X $, so sind $ x_1,\dots,x_n \in \R $;
			\item ist $ X $ \emph{schiefadjungiert}, $ X^*=-X $, so sind $ x_1,\dots,x_n\in i\R $;
			\item ist $ X $ \emph{unitär}, $ X\in U(n)= \{Y\in \C^{n\times n}\mid Y^*Y=E_n\} $, so gilt $ |x_1|,\dots,|x_n| = 1. $
		\end{itemize}
\end{Korollar}
\paragraph{Beweis}
	Sei $ X\in \C^{n\times n} $, betrachte $ \C^n $ als unitären VR mit Standardbasis $ E $ als ONB.
		\[ \Gamma_E(\Skl{.}{.}) = E_n, \]
	und den assoziierten Endomorphismus $ f_X\in \End(\C^n) $. Orthonormale Basiswechsel $ B=EP $ sind dann durch unitäre Matrizen $ P\in U(n) $ gegeben:
		\[ E_n = \Gamma_B(\Skl{.}{.}) = P^*\Gamma_E(\Skl{.}{.})P = P^*P \Leftrightarrow P\in U(n). \]
	Anwendung des Spektralsatzes liefert also die Behauptung.

\paragraph{Beispiel}
	Für  $ X = \begin{pmatrix}
	\cos s & -\sin s\\ \sin s & \cos s
	\end{pmatrix}\in \C^{2\times 2} $ mit $ s\in \R $ ist $ X^*=X^t = X^{-1} $, d.h. $ X $ ist unitär, also normal, und damit
		\[ \exists P\in (2): P^{-1}\begin{pmatrix}
		\cos s & -\sin s\\ \sin s & \cos s
		\end{pmatrix}P = \begin{pmatrix}
		x_1 & 0 \\ 0 & x_2
		\end{pmatrix} \]
	wobei $ x_i $ die Eigenwerte von $ f_X $ sind, d.h. Nullstellen des charakteristischen Polynoms
		\[ \chi_X(t) = (t-\cos s)^2 + \sin^2 s = t^2-2t\cos s + 1 = (t-e^{is})(t-e^{-is}). \]
	Bemerke: $ |e^{\pm is}| = 1 $, d.h. $ x_{1,2} $ sind unitär. Eigenvektoren bzw. P:
		\[ P = \frac{1}{\sqrt{2}}\begin{pmatrix} i & -i \\ 1 & 1 \end{pmatrix} \]

\subsection{Spektralzerlegung (unitärer Fall)}
\begin{Lemma}[]
	Sei $ (V,\Skl{.}{.}) $ unitär, $ \dim V <\infty $, und sei $ f\in \End(V) $ normal; dann zerfällt $ V $ als orthogonale direkte Summe der Eigenräume von $ f $,
		\[ V = \bigoplus_{x\in \chi_f^{-1}(\{0\})} \ker(\id_Vx-f). \]
\end{Lemma}
\paragraph{Beweis}
	Folgt direkt aus dem Spektralsatz.
\paragraph{Bemerkung}
	Mit gewissen Voraussetzungen gilt der Satz auch für $ \dim V = \infty $.
% VO 14-06-2016 %

\subsection{Definition \& Lemma}
\begin{Definition}[komplexe Erweiterung]
	Seien $ (V,\Skl{.}{.}) $ Euklidischer VR und $ f\in \End(V) $ normal. Die \emph{komplexe Erweiterung} 
		\[ f_\C:V_\C \to V_\C, (v,w)\mapsto f_\C(v,w) := (f(v),f(w)) \]
\end{Definition}
\begin{Lemma}[]
	von $ f $ ist dann ein normaler Endomorphismus von $ (V_\C,\SSkl{.}{.}_\C) $.
\end{Lemma}
\paragraph{Bemerkung}
	Die komplexe Erweiterung für $ f\in\Hom(V,W) $ definiert man analog.
\paragraph{Bemerkung}
	Auf $ V_\C = V\times V $ ist die (komplexe) Skalarmultiplikation
		\[ (v,w)(x+iy) = (v,w)x+J(v,w)y \]
	wobei $ J(v,w) = (-w,v) $; damit ist $ f_\C $ komplex linear, da 
		\[ f_\C \circ J = J\circ f_\C. \]
\paragraph{Beweis}
	Nach Komplexifizierungslemma ist $ (V_\C,\SSkl{.}{.}) $ unitär, wobei
		\[ \SSkl{(v,w)}{(v',w')} = \left(\Skl{v}{v'}+\Skl{w}{w'}\right)+i\left(\Skl{v}{w'}-\Skl{w}{v'}\right); \]
	offenbar gilt für $ v,w,v',w'\in V_\C $
		\[ \SSkl{\left(f^*(v),f^*(w)\right)}{(v',w')}_\C = \SSkl{(v,w)}{\left(f(v'),f(w')\right)} = \SSkl{(v,w)}{f_\C(v',w')} \]
	also $ (f_\C)^*=(f^*)_\C $ und damit
		\[ f_\C^* \circ f_\C = \left(f^* \circ f\right)_\C = \left(f_\C \circ f^*_\C \right), \]
	d.h. $ f_\C $ ist normal.
\paragraph{Bemerkung}
	Ist $ (v,w)\in V_\C^\times $ Eigenvektor zum Eigenwert $ (x-iy) \in \C $ von $ f_\C $,
		\[ f_\C(v,w) = \left(f(v),f(w)\right) = (v,w)(x-iy) = (vx+wy,-vy+wx) = (v,w)
			\begin{pmatrix}
				x & -y\\ y & x
			\end{pmatrix} \]
	so können zwei Fälle eintreten:
		\begin{enumerate}
			\item $ y=0 $ und $ [\{v,w\}] \subset \ker (\id_Vx-f) $, oder
			\item $ y\neq 0 $ und $ \dim [\{v,w\}] = 2 $ und\footnote{$ [\{v,w\}] $ ist $ f $-invarianter UVR, also $ f|_{[\{v,w\}]} \in \End([\{v,w\}])$} $ f|_{[\{v,w\}]} $ ist \emph{Drehstreckung}.
		\end{enumerate}
	Im zweiten Fall $ (y\neq 0) $ ist $ (v,-w) $ ebenfalls Eigenvektor zum Eigenwert $ (x+iy) $ von $ f_\C $, d.h. komplexe Eigenwerte/-vektoren treten in "`komplex konjugierten Paaren"' auf.
	
\subsection{Spektralzerlegung (Euklidischer Fall)}\index{Spektralzerlegung}
\begin{Satz}[Spektralzerlegung (Euklidischer Fall)]
	Seien $ (V,\Skl{.}{.}) $ Euklidischer VR mit $ \dim V <\infty $ und $ f\in \End(V) $ normal; dann zerfällt $ V $ als orthogonale direkte Summe $ f $- und $ f^* $-invarianter UVR $ V_i $
		\[ V = \bigoplus_{i=1}^mV_i \text{ mit } V_i\perp V_j \text{ für } i\neq j \]
	und
		\[ \begin{cases}
			\dim V_i = 1& \text{und } f|_{V_i} \text{ Streckung für } i\leq k\leq m\\
			\dim V_i = 2& \text{und } f|_{V_i} \text{Drehstreckung für }k < i\leq m,
		\end{cases} \]
	für geeignetes $ k\in\{0,\dots,m\} $. Beweis nach Lemma.
\end{Satz}
\subsection{Buchhaltung}
	Zu einem normalen $ f\in \End(V) $ eines Euklidischen VR $ (V,\Skl{.}{.}) $ mit $ \dim V < \infty $
	gibt es also eine ONB $ E $ von $ (V,\Skl{.}{.}) $, sodass 
		\[ \xi_E^E(f) = \begin{pmatrix}
		x_1& 0&\cdots & & &\\
		0 & \ddots & \ddots& & 0& \\
		\vdots & \ddots & x_k& & &\\
		& & & X_{n+1} & \ddots & \vdots\\
		&0 & & \ddots & \ddots & 0\\
		& & & \cdots&0 & X_m
		\end{pmatrix}
		\text{ mit } X_i = \begin{pmatrix}
		x_i & -y_i\\ y_i & x_i
		\end{pmatrix} \text{ für } i=k+1,\dots,m \]
	und $ x_1,\dots,x_m,y_{k+1},\dots,y_m\in \R $. Da $ \xi_E^E(f^*) = (\xi_E^E(f))^* = (\xi_E^E(f))^* $ gilt
		\begin{itemize}
			\item ist $ f $ selbstadjungiert, so sind alle Eigenwerte reell, $ k=m $;
			\item ist $ f $ schiefadjungiert, so sind alle Eigenwerte imaginär, $ k=0 $ und $ x_1=\dots=x_m=0 $;
			\item ist $ f $ orthogonal, so sind alle Eigenwerte unitär, $ x_i^2+y_i^2=1 $ (insbesondere $ x_1^2=\dots=x_k^2=1 $).
		\end{itemize}
	Entsprechendes gilt für normale Matrizen. Insbesondere erhält man den Satz über die
\subsection{Hauptachsentransformation}\index{Hauptachsentransformation}
\begin{Satz}[Hauptachsentransformation]
	Ist $ (V,\Skl{.}{.}) $ Euklidischer VR mit $ \dim V<\infty $ und $ f\in \End(V) $ selbstadjungiert, so ist $ f $ orthogonal diagonalisierbar.
\end{Satz}
\paragraph{Bemerkung}
	Die Hauptachsentransformation kann zur Bestimmung der Signatur einer symmetrischen Bilinearform $ \sigma:V\times V\to \R $ auf einem $ \R $-VR $ V $ mit $ \dim V <\infty $ dienen:
		\begin{itemize}
			\item Wähle (beliebig) ein Euklidisches (Referenz-) Skalarprodukt $ \Skl{.}{.}:V\times V\to \R $;
			\item Definiere $ b\in \End(V) $ (Rieszsches Darstellungslemma) durch
				\[ \forall v,w\in V: \sigma(v,w) = \Skl{v}{b(w)}; \]
			da $ \sigma $ symmetrisch ist, ist $ b $ selbstadjungiert.
			\item Bestimme Eigenwerte $ x_i\in \R $ von $ b $ mit Vielfachheiten\footnote{Geometrische und algebraische Vielfachheiten sind gleich, da $ b $ diagonalisierbar ist.} $ k_i\in \N $ (Hauptachsentransformation).
			\item Dann ist
				\[ \sgn(\sigma) = \left(\sum_{x_i>0}k_i, \sum_{x_i<0}k_i,\dfkt b\right). \]
		\end{itemize}
	Ist $ E $ ONB aus Eigenvektoren von $ b $, so gilt\footnote{Gleichheit der Einträge; sonst sinnlos!}
		\[ \Gamma_E(\sigma) = \xi_E^E(b). \]
\subsection{Quadratwurzelsatz}
\begin{Satz}[Quadratwurzelsatz]
	Ist $ (V,\Skl{.}{.}) $ Euklidischer VR und $ f\in \End(V) $ selbstadjungiert, so heißt
		\begin{enumerate}[(i)]
			\item $ f $ \emph{positiv semi-definit} $ (f\geq 0) $, falls
				\[ \forall v\in V :\Skl{v}{f(v)}\geq 0;\]
			\item \emph{positiv definit} $ (f>0) $, falls
				\[ \forall v\in V^\times: \Skl{v}{f(v)}>0. \]
		\end{enumerate}
	Ist $ \dim V < \infty $ und $ f $ positiv semi-definit, so gilt
		\[ \exists! g\in \End(V): \begin{cases}
		g\geq 0\\f = g\circ g =g^2.
		\end{cases} \]
\end{Satz}
\paragraph{Beweis}
	Mit Hauptachsentransformation: Ist $ E = (e_1,\dots,e_n) $ ONB aus Eigenvektoren von $ f $,
		\[ f(e_i) = e_ix_i \text{ mit } x_i = \Skl{e_i}{f(e_i)}\geq 0 \]
	für $ i=1,\dots,n $, dann liefert
		\[ g\in\End(V) \text{ mit } \forall i\in \{1,\dots,n\}: g(e_i) = e_i\sqrt{x_i} \]
	eindeutig die gesuchte "`Quadratwurzel"' von $ f $.
\subsection{Polarzerlegung}
\begin{Satz}[Polarzerlegung]
	Ist $ (V,\Skl{.}{.}) $ Euklidischer VR mit $ \dim V <\infty $, so gilt
		\[ \forall f\in Gl(V) \exists! h > 0 \exists! k\in O(V) : f = h\circ k. \]
\end{Satz}
\paragraph{Beweis}
	
	\begin{itemize}
		\item \emph{Eindeutigkeit:\quad} Ist $ f=h\circ k $ mit $ k\in O(V), h>0 $, so gilt
		\[ H:= f\circ f^* = h\circ \underbrace{k \circ k^*}_{=\id_V} \circ\, h^* = h^2 \]
		nach Quadratwurzelsatz ist also $ h $, und damit $ k $ eindeutig bestimmt.
		\item \emph{Existenz:\quad} Wegen $ \ker f^* = f(V)^\perp = V^\perp = \{0\} $ gilt für $ H:= f\circ f^* $
			\[ \forall v\in V^\times: \Skl{v}{H(v)} = \Skl{v}{(f\circ f^*)(v)} = \Skl{f^*(v)}{f^*(v)}>0  \]
		also $ H>0 $. Definiere (Quadratwurzelsatz)
			\[ h:= \sqrt{H}>0 \text{ und } k:= h^{-1}\circ f; \]
		dann ist
			\[ \forall v\in V: \Skl{k(v)}{k(v)} = \Skl{(h^{-1}\circ f)(v)}{(h^{-1}\circ f)(v)} = \Skl{(H^{-1}\circ f)(v)}{f(v)} \]
			\[= \Skl{\left((f^*)^{-1}\circ f^{-1}\circ f\right)(v)}{f(v)} = \Skl{v}{\left(f^{-1}\circ f\right)(v)} = \Skl{v}{v}, \]
		also ist $ k\in O(V) $.
	\end{itemize}
\paragraph{Bemerkung}
	Quadratwurzelsatz und Polarzerlegung gelten auch in unitären VR -- "`positiv (semi-)definit"' ist auch im unitären Fall sinnvoll:
		\[ \forall v\in V: \overline{\Skl{v}{f(v)}} = \Skl{f(v)}{v}=\Skl{v}{f(v)} \]
	für selbstadjungierte $ f $, also $ \forall v\in V: \Skl{v}{f(v)}\in \R $.