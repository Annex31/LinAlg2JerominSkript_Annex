% VO 21-06-2016 %
\section{Quadriken}
\paragraph{Generalvoraussetzung}
	In diesem Abschnitt ist $ (A,V,\tau) $ ein \emph{reeller} affiner Raum über einem $ \R $-VR $ V $; $ \Skl{.}{.} $ ist ein Euklidisches Skalarprodukt.
\subsection{Definition}\index{Quadrik}
\begin{Definition}[Quadrik]
	Eine \emph{Quadrik} $ Q\subset A $ ist die Lösungsmenge einer quadratischen Gleichung
		\[ Q = \{q=o+v\mid \beta(v,v)+2\lambda(v)+\rho = 0 \} \]
	wobei $ o\in A $ ein Ursprung ist und
	\begin{itemize}
		\item $ \beta: V\times V\to \R $ eine symmetrische Bilinearform, $ \beta \neq 0 $;
		\item $ \lambda: V\to \R $ eine Linearform; und
		\item $ \rho \in \R$ sind.
	\end{itemize}
	Diese Definition hängt nicht vom Ursprung $ o\in A $ ab:
\end{Definition}
\subsection{Lemma}
\begin{Lemma}[]
	Ist $ Q=\{o+v\in A \mid \beta(v,v)+2\lambda(v)+\rho=0 \} $ eine Quadrik und $ o' = o+w\in A $ ein anderer Ursprung, so ist
		\[ Q=\{o'+v\in A\mid \beta'(v,v)+2\lambda'(v)+\rho'=0 \} \]
	mit $ \beta'=\beta $, $ \lambda'=\lambda+\beta(w,.) $, $ \rho' = \rho+2\lambda(w)+\beta(w,w) $.
\end{Lemma}
\paragraph{Beweis}
	Mit $ q=o'+v=o+(w,v) $ nachrechnen, vgl. Aufgabe 91.
\paragraph{Bemerkung}
	Insbesondere ist $ \beta' = \beta $ unabhängig vom gewählten Ursprung, der lineare Term ändert sich mit $ \beta(w,.) $ -- unter "`guten Umständen"' kann man also $ \lambda $ durch geeignete Wahl von $ o' $ verschwinden lassen (quadratische Ergänzung).
\paragraph{Beispiel}
	Für $ \lambda\in V^*\setminus\{0\} $ liefert
		\[ \beta:V\times V\to \R, (v,w)\mapsto \beta(v,w):=\lambda(v)\lambda(w) \]
	eine symmetrische Bilinearform $ \beta\neq 0 $ und daher
		\[ Q=\{o+v\in A\mid \lambda^2(v)=\beta(v,v) = 0 \} \]
	eine Quadrik; andererseits ist $ Q = o+\ker \lambda $ eine Hyperebene (AUR mit $ \dim = \dim A-1 $).
\subsection{Bemerkung \& Definition}
\begin{Definition}[echte Quadriken]
	Im Folgenden betrachten wir nur \emph{echte Quadriken}, d.h. Quadriken $ Q\subset A $, die nicht in einer affinen Hyperebene enthalten sind.
	Insbesondere schließen wir $ Q=\emptyset $ aus. 
\end{Definition}

	Nach Wahl des Ursprungs $ o\in A $ bestimmt eine echte Quadrik die zugehörige Gleichung bis auf Vielfache: Das Tupel $ (\beta,\lambda,\rho) $ ist bis auf (gemeinsame) Skalarmultiplikation mit $ x\in\R^\times  $ eindeutig bestimmt.
\subsection{Definition}\index{Mittelpunkt}\index{Spitze}
\begin{Definition}[Mittelpunkt, Spitze]
	Ein Punkt $ z\in A $ heißt \emph{Mittelpunkt} einer Quadrik $ Q\subset A $, falls
		\[ \forall q=z+v\in A: q\in Q\Rightarrow z-v\in Q, \]
	ein Mittelpunkt $ z $ einer Quadrik $ Q $ heißt \emph{Spitze}, falls $ z\in Q $.
\end{Definition}	
\begin{Definition}[Mittelpunktsquadrik, Kegel, Paraboloid]

	Eine Quadrik $ Q $ heißt
	\begin{itemize}
		\item \emph{Mittelpunktsquadrik}, falls sie einen Mittelpunkt $ z\in A $ hat;
		\item \emph{Kegel}, falls sie eine Spitze hat; 
		\item \emph{Paraboloid} (oder Parabel für $ \dim A = 2 $), falls sie keinen Mittelpunkt hat.
	\end{itemize}
\end{Definition}	
\paragraph{Bemerkung \& Beispiel}
	Eine Quadrik $ Q $ kann mehr als einen Mittelpunkt oder eine Spitze haben.
	Beispielsweise liefert für $ \lambda\in V^*\setminus\{0\} $
		\[ Q = \{q=o+v\in A\mid \beta(v,v) = \lambda^2(v) = 1\} \]
	ein Paar paralleler Hyperebenen, eine Quadrik, für die jeder Punkt $ z = o+w \in o+\ker \lambda $ ein Mittelpunkt ist, da für $ o+v = (o+w)+(v-w) = z+(v-w)\in Q $ gilt
		\[ \lambda\left((z-(v-w))-o\right) = \lambda(2w-v) = -\lambda(v)\implies o+v\in Q\Rightarrow z-(v-w)\in Q. \]
\subsection{Lemma}
\begin{Lemma}[]
	Seien $ Q\subset A $ eine echte Quadrik und $ z\in A $. Dann ist
	\begin{itemize}
		\item $ z\in A $ Mittelpunkt von $ Q $, falls
			\[ \exists c\in \R: Q=\{q\in A\mid \beta(q-z,q-z)=c \}; \]
		\item $ z\in A $ Spitze von $ Q $, falls
			\[ Q=\{q\in A\mid \beta(q-z,q-z) = 0\}. \]
	\end{itemize}
\end{Lemma}
\paragraph{Beweis}
	Da eine Spitze ein Mittelpunkt auf $ Q $ ist, folgt die zweite Aussage direkt aus der ersten.
	Sei $ z\in A $ Mittelpunkt von $ Q $; mit $ z $ als Ursprung und geeigneten $ (\beta,\lambda,\rho) $ ist dann
		\[ Q=\{q=z+v\mid \beta(v,v)+2\lambda(v)+\rho = 0\}. \]
	Da $ z $ Mittelpunkt von $ Q $ ist, gilt
		\[ \forall q=z+v\in Q: \begin{cases}
		0 = \beta(v,v)+2\lambda(v)+\rho\\
		0 = \beta(v,v)-2\lambda(v)+\rho
		\end{cases} \]
	mithin
		\[ \forall q=z+v\in Q: \lambda(v) = 0,\quad \text{ also }\quad Q\subset z +\ker \lambda. \]
	Da $ Q $ echte Quadrik ist, folgt also $ \ker \lambda = V $ bzw. $ \lambda = 0 $. Die Behauptung folgt dann mit der Wahl von $ c=-\rho $. Umgekehrt: Ist für ein $ c\in \R $
		\[ Q=\{q=z+v\in A\mid \beta(v,v)=c \}, \]
	so ist $ z $ offenbar Mittelpunkt von $ Q $.
\subsection{Bemerkung \& Definition}\index{Erzeugende}
	Ist $ Q\subset A $ ein Kegel mit Spitze $ z\in Q $, so ist für $ q\in Q\setminus\{z\} $ und $ v:= q-z $
		\[ \forall x\in \R: \beta(vx,vx)= \beta(v,v)x^2 = 0,  \]
	also ist mit $ q $ auch die gesamte Gerade $ [\{z,q\}] = \{z+vx\mid x\in \R\}\subset Q $. Diese in $ Q $ enthaltenen Geraden heißen auch \emph{Erzeugende} des Kegels.
\subsection{Affine Klassifikation der Mittelpunktsquadriken}
\begin{Lemma}[]
	Ist $ Q\subset A $ echte Mittelpunktsquadrik eines affinen Raumes $ A $, so existieren
	\begin{itemize}
		\item affines Bezugssystem $ (o,e_1,\dots,e_n) $ von $ A $ und
		\item $ c\in \{0,1\} $ und $ p,r\in \N $ mit $ 1\leq p\leq r \leq n $, 
	\end{itemize}
	sodass 
		\[ Q=\left\{q=o+\sum_{i=1}^{n}e_ix_i\mid \sum_{i=1}^{p}x_i^2-\sum_{i=p+1}^{r}x_i^2 = c \right\} \]
	und $ p $ ist der Positivitätsindex von $ \beta $, $ r-p $ der Negativitätsindex ($ n-r $ Radikaldimension). 
\end{Lemma}
\paragraph{Beweis}
	Folgt direkt aus dem Satz von Sylvester.
\subsection{Bemerkung \& Definition}
	Zwei echte Mittelpunktsquadriken $ Q,Q' $ sind also genau dann \emph{affin äquivalent}, d.h. $ Q' = \alpha(Q) $ für eine Affinität $ \alpha:A\to A $, wenn $ \sgn(\beta')=\sgn(\beta) $, bzw. $ \sgn(\beta')=\sgn(\pm \beta) $ im Fall eines Kegels.
% VO 23-06-2016 %
\subsection{Euklidische Klassifikation der Mittelpunktsquadriken}
\begin{Lemma}[]
	Ist $ Q\subset A $ eine echte Mittelpunktsquadrik eines Euklidischen Raumes, $ \dim A <\infty $, so existierten 
	\begin{itemize}
		\item ein kartesisches Bezugssystem $ (o;e_1,\dots,e_n) $ von $ A $,
		\item $ c\in \{0,1\} $ und $ p,r\in \N $ mit $ 1\leq p\leq r\leq n $ und
		\item $ a_i\in (0,\infty) $ für $ i=1,\dots,r $
	\end{itemize}
	sodass
		\[ Q = \left\{o+\sum_{i=1}^{n}e_ix_i\in A \mid \sum_{i=1}^{p}\left(\frac{x_i}{a_i}\right)^2-\sum_{i=p+1}^{r}\left(\frac{x_i}{a_i}\right)^2 = c \right\}. \]
\end{Lemma}
\paragraph{Beweis}
	Folgt mit der Hauptachsentransformation (daher ihr Name).
	Sei $ b\in \End(V) $ so, dass
		\[ \forall v,w\in V: \beta(v,w) = \Skl{v}{b(w)}, \]
	nach Riesz existiert ein eindeutiges solches $ b $.
	
	Da $ \beta $ symmetrisch ist, gilt $ b^* = b $. Da $ Q $ Mittelpunktsquadrik ist, kann sie mithilfe des Mittelpunkts $ z\in A $ geschrieben werden als
		\[ Q=\left\{z+v\in A\mid \Skl{v}{b(v)} = c \right\} \]
	mit o.B.d.A. (Multiplikation der Gleichung mit $ c^{-1} $ im Fall $ c\neq 0 $) $ c\in \{0,1\} $. Nach Hauptachsentransformation existiert eine ONB $ (e_1,\dots,e_n) $ aus Eigenvektoren von $ b $, wobei o.B.d.A die Eigenwerte zu $ e_1,\dots,e_p $ positiv, $ e_{p+1},\dots,e_r $ negativ und zu $ e_{r+1},\dots,e_n $ gleich 0 sind, für $ 0\leq p\leq r\leq n $.
	
	Also existieren $ a_1,\dots,a_r\in (0,\infty) $, sodass
		\[ b(e_i)=\begin{cases}
			e_i\frac{1}{a_i^2} &\text{für } i = 1,\dots,p\\
			-e_i\frac{1}{a_i^2} &\text{für } i = p+1,\dots,r\\
			0 & \text{für } i = r+1,\dots,n.
		\end{cases} \]
	Damit gilt mit dem kartesischen Bezugssystem $ (z;e_1,\dots,e_n) $
		\begin{align*}
		Q=\left\{z+\sum_{i=1}^{n}e_ix_i\in A\mid c\right.&=\left\langle{\sum_{i=1}^{n}e_ix_i},{b\left(\sum_{j=1}^{n}e_jx_j\right)}\right\rangle \\
		 &= \left.\sum_{i,j=1}^{n}x_i\Skl{e_i}{b(e_j)}x_j = \sum_{i=1}^{p}\left(\frac{x_i}{a_i}\right)^2-\sum_{i=p+1}^{r}\left(\frac{x_i}{a_i}\right)^2 \right\}.
		\end{align*}
	Da $ \#Q \leq 1 $ für $ p=0 $, muss $ p\geq 1 $ sein, denn $ Q $ war als echt vorausgesetzt.
\paragraph{Bemerkung}
	Diese beiden Sätze liefern "`Klassifikationen"' in den jeweiligen Geometrien, d.h. eine Einteilung der Menge der Quadriken in Äquivalenzklassen, wobei zwei Quadriken $ Q,Q'\subset A $ äquivalent sind, wenn $ Q $ durch eine Transformation der jeweiligen Geometrie auf $ Q' $ abgebildet werden kann, d.h. $ \alpha $ ist affine Transformation oder Kongruenzabbildung:
		\[ Q\sim Q' :\Leftrightarrow \exists \alpha:A\to A\: Q' = \alpha(Q). \]
	Also: Haben (nach Klassifikationssätzen) zwei Quadriken $ Q,Q'$ die gleiche Gleichung (bzgl. affiner/kartesischer Bezugssysteme $ (z;e_1,\dots,e_n) $ bzw. $ (z',e_1',\dots,e_n') $), so existiert eine affine Transformation $ \alpha:A\to A $, definiert durch
		\[  \alpha(z) = z' \text{ und } \alpha(z+e_i) = z+e_i' \text{ für } i=1,\dots,n \]
	mit $ \alpha(Q) = Q' $. Die Umkehrung folgt ähnlich (Vorsicht bei Kegeln!).
\subsection{Beispiel \& Definition}\index{Ellipse}\index{Hyperbel}\index{Asymptotenkegel}
	In einer Euklidischen Ebene $ E^2 $ ergeben sich als echte Mittelpunktsquadriken:
		\begin{itemize}
			\item $ p=r=2 : c=1 $ ($ c=0 $ liefert nur einen Punkt) und
				\[ Q=\left\{z+e_1x_1+e_2x_2\in E^2\mid \left(\frac{x_1}{a_1}\right)^2+\left(\frac{x_2}{a_2}\right)^2 = 1\right\} \]
				liefert eine \emph{Ellipse} mit \emph{Halbachsenlängen} $ a_1,a_2 > 0$.
			\item $ p=1 $ und $ r=2 $
				\begin{itemize}
					\item mit $ c=1 $ liefert
						\[ Q=\left\{z+e_1x_1+e_2x_2\in E^2\mid \left(\frac{x_1}{a_1}\right)^2-\left(\frac{x_2}{a_2}\right)^2 = 1\right\} \]
						eine \emph{Hyperbel} durch die Scheitel $ z=\pm e_1a_1 $ und sich in $ z $ schneidenden Asymptoten.
    		%------------------ Hyperbel.tikz ----------------
	        \begin{figure}[H]\centering
     		    \tdplotsetmaincoords{0}{0} %-27
 	\begin{tikzpicture}[yscale=1,tdplot_main_coords]

 		\def\xstart{0} %x Koordinate der Startposition der Grafik
 		\def\ystart{0} %y Koordinate der Startposition der Grafik
 		\def\myscale{1.0} %ändert die Größe der Grafik (Skalierung der Grafik)
        \def\myscalex{(\myscale)}
        \def\myscaley{(\myscale)}
                
 		\def\xstartdraw{(\xstart + 3.5)} %xKoordinate des Referenzstartpunktes (in dieser Zeichnung: a)
 		\def\ystartdraw{(\ystart + 2.0)}%yKoordinate des Referenzstartpunktes (in dieser Zeichnung: a)

 		\def\balkenhoehe{(3.5)}% Länge des vertikalen blauen Balkens
 		\def\balkenlaenge{(7.5)}% Länge des horizontalen blauen Balkens
 		\def\balkenbreite{0.4} %Balkenbreite

 		%---------Begin Balken----------
 		\def\drehwinkel{0}
 		\node (VekV) at ({\xstart+0.2*cos(\drehwinkel)-\balkenbreite*sin(\drehwinkel)},{\ystart+0.5*sin(\drehwinkel)+\balkenbreite*cos(\drehwinkel)})[right, xshift=1,color=blue] {$\mathbb{R}^2$};
 		\node (AffA) at ({\xstart+(\balkenlaenge-0.5)*cos(\drehwinkel)},{\ystart+(\balkenlaenge-0.5)*sin(\drehwinkel)+\balkenbreite*cos(\drehwinkel)})[color=red] {$A$};

 		\path[ shade, top color=white, bottom color=blue, opacity=.6]
 		({\xstart},{\ystart},0)  -- ({\xstart - \balkenbreite * cos(\drehwinkel)- (-\balkenbreite+0)*sin(\drehwinkel)},{\ystart - \balkenbreite * sin(\drehwinkel)+ (-\balkenbreite+0)*cos(\drehwinkel)},0)  -- ({\xstart - \balkenbreite * cos(\drehwinkel)- (\balkenhoehe+0.5)*sin(\drehwinkel)},{\ystart - \balkenbreite * sin(\drehwinkel)+ (\balkenhoehe+0.5)*cos(\drehwinkel)},0) -- ({\xstart - 0 * cos(\drehwinkel)- (\balkenhoehe+0)*sin(\drehwinkel)},{\ystart - 0 * sin(\drehwinkel)+ (\balkenhoehe+0)*cos(\drehwinkel)},0) -- cycle;

 		\path[ shade, right color=white, left color=blue, opacity=.6]
 		({\xstart},{\ystart},0)  -- ({\xstart - \balkenbreite * cos(\drehwinkel)- (-\balkenbreite+0)*sin(\drehwinkel)},{\ystart - \balkenbreite * sin(\drehwinkel)+ (-\balkenbreite+0)*cos(\drehwinkel)},0) --
 		({\xstart + (\balkenlaenge+0.5) * cos(\drehwinkel)- (-\balkenbreite+0)*sin(\drehwinkel)},{\ystart + (\balkenlaenge+0.5) * sin(\drehwinkel)+ (-\balkenbreite+0)*cos(\drehwinkel)},0) --
 		({\xstart + \balkenlaenge * cos(\drehwinkel)},{\ystart + \balkenlaenge * sin(\drehwinkel)},0)--
 		cycle;
 		%---------End Balken----------
 	
 	\node (pointz1)[color=red] at ({\xstartdraw},{\ystartdraw}) {};
 		
    \def\cradius{2.5}
 	\def\cwinkel{14}
 	
 		
 	\node (pointlo1) at ($(pointz1) + (135+\cwinkel:\cradius)$) {};
 	\node (pointro1) at ($(pointz1) + (45-\cwinkel:\cradius)$) {};
 	\node (pointlu1) at ($(pointz1) + (225-\cwinkel:\cradius)$) {};
 	\node (pointru1) at ($(pointz1) + (315+\cwinkel:\cradius)$) {};
 	
 
 	\draw[name path=lo--ru,-,shorten >=0pt, shorten <=0pt,line width=0.1pt,color=blue!50!white] (pointlo1.center) -- (pointru1.center);
 	\draw[name path=lu--ro,-,shorten >=0pt, shorten <=0pt,line width=0.1pt,color=blue!50!white] (pointro1.center) -- (pointlu1.center);

 	
    \draw [red, thick, domain=-1.1:1.1, samples=50] plot ({\xstartdraw + sqrt( ((\x*\x)*2.3 +1)*1.2 ) },{\ystartdraw +\x});
    \draw [red, thick, domain=-1.1:1.1, samples=50] plot ({\xstartdraw - sqrt( ((\x*\x)*2.3 +1)*1.2 ) },{\ystartdraw +\x});
   
   \draw [green!70!black, thick, domain=-1.3:1.3, samples=50] plot ({\xstartdraw + sqrt( ((\x*\x)*2.3 )*1.2 ) },{\ystartdraw +\x});
    \draw [green!70!black, thick, domain=-1.3:1.3, samples=50] plot ({\xstartdraw - sqrt( ((\x*\x)*2.3 )*1.2 ) },{\ystartdraw +\x});
    
 	
 	%Vektoren blau
 	\draw[name path=pe2,-{>[scale=1,length=6,width=5]},shorten >=0pt, shorten <=0pt,line width=0.2pt,color=blue] (pointz1)  -- ($(pointz1) + (90:1)$);
 	\draw[name path=pe1,-{>[scale=1,length=6,width=5]},shorten >=0pt, shorten <=0pt,line width=0.2pt,color=blue] (pointz1)  -- ($(pointz1) + (0:1)$);
 	 
 	\node (pointScheitel1) at ({\xstartdraw + sqrt( ((1.2 ) },{\ystartdraw }) {};
 	 
 	\path[name path=line1] (pointScheitel1) -- +(0,3);
 	\draw[thick,color=red,line width=0.3pt, name intersections={of=lu--ro and line1,by={Int1}}, dotted] (pointScheitel1) -- (Int1);
 	 
 	\path[name path=line2] (Int1) -- +(-3,0);
 	\node(helppoint) at ($(pointz1.center) + (90:1)$){};
 	\draw[thick,color=red,line width=0.3pt, dotted] (Int1) -- ($(pointz1)!(Int1)!(helppoint)$);

 	%Punkte malen

 	\node [color=blue] (pointle1) at ($(pointz1) + (-15:0.7)$) {\small $e_1$};
 	\node [color=blue] (pointle1) at ($(pointz1) + (105:0.7)$) {\small $e_2$};
 
 	%Punkte weiss
 	%\draw[fill,color=white] (pointz1.center) circle [radius=0.11] node[below, xshift=0, yshift=0]{};
    %Punkte rot
 	\draw[fill,color=green!70!black] (pointz1.center) circle [radius=0.06]node[below, xshift=0, yshift=0]{};
 		
 	
 	\draw[fill,color=red]  (pointScheitel1) circle [radius=0.06]node[below, xshift=0, yshift=0]{};
 	\draw[fill,color=red]  ({\xstartdraw - sqrt( ((1.2 ) },{\ystartdraw }) circle [radius=0.06]node[ xshift=-8mm, yshift=0mm]{\small Scheitel};
 		
 	\node [color=red] (pointla1) at ($(pointz1) + (20:1.0)$) {\small $a_2$};
 	\node [color=red] (pointla2) at ($(pointz1) + (55:1.0)$) {\small $a_1$};
 		
 	\node [color=red] (pointlaHaupt) at ($(pointz1) + (90:1.3)$) {\small Q für $c=1$};
 	\node [color=green!70!black] (pointlaHaupt) at ($(pointz1) + (-90:1.3)$) {\small Q für $c=0$};
 		
 	\node [color=red] (pointlz1) at ($(pointz1) + (-90:0.25)$) {\small $z$};
\end{tikzpicture}
    	    \end{figure}
            %------------------ Hyperbel.tikz ----------------
   
					\item mit $ c=0 $ liefert
						\[ Q=\left\{z+e_1x_1+e_2x_2\in E^2\mid \left(\frac{x_1}{a_1}\right)^2=\left(\frac{x_2}{a_2}\right)^2 \right\} \]
						den \emph{Asymptotenkegel} der obigen Hyperbel.
				\end{itemize}
			\item $ p=r=1 : c=1 $ ($ c=0 $ liefert eine Gerade, also keine echte Mittelpunktsquadrik) liefert
				\[ Q=\left\{z+e_1x_1+e_2x_2\in E^2\mid \left(\frac{x_1}{a_1}\right)^2 = 1\right\} \]
			zwei parallele Geraden im Abstand $ 2a_1 > 0$.
		\end{itemize}
	