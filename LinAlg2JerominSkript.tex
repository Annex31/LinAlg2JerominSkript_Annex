\documentclass[11pt, DIV=14, parskip=half]{scrreprt}
\usepackage{microtype}
\usepackage[utf8]{inputenc}
\usepackage[T1]{fontenc}
\usepackage{lmodern}%schoeneres Schriftbild
\usepackage[ngerman]{babel}%deutsche Silbentrennung
\usepackage[onehalfspacing]{setspace}
%\usepackage{tikz}%Zeichnungen
\usepackage{amsmath,amsfonts,amssymb}%Mathematik-Pakete
\usepackage{graphicx}
\usepackage{enumerate}%enumerate mit roemischen Zahlen
%\usepackage{float}%fuer H Positionierung
\usepackage{hyperref} %Verlinktes Inhaltsverzeichnis
\usepackage{makeidx} %Stichwortverzeichnis
\makeindex

\author{Studierendenmitschrift}
\title{Skript Lineare Algebra \& Geometrie 2, Hertrich-Jeromin}

\setcounter{tocdepth}{1}

% Eigene Operatoren:
\let\hom\relax
\DeclareMathOperator{\Char}{Char}
\DeclareMathOperator{\End}{End}
\DeclareMathOperator{\Aut}{Aut}
\DeclareMathOperator{\Iso}{Iso}
\DeclareMathOperator{\hom}{Hom}
\DeclareMathOperator{\rg}{rg}
\DeclareMathOperator{\dfkt}{def}
\DeclareMathOperator{\id}{id}
\DeclareMathOperator{\sgn}{sgn}
\DeclareMathOperator{\vol}{vol}
\DeclareMathOperator{\ggT}{ggT} 

\newenvironment{Satz}[1][]{}{\par\addvspace{\baselineskip}}
\newenvironment{Lemma}[1][]{}{\par\addvspace{\baselineskip}}
\newenvironment{Definition}[1][]{}{\par\addvspace{\baselineskip}}
\newenvironment{Korollar}[1][]{}{\par\addvspace{\baselineskip}}

\begin{document}
\maketitle
\tableofcontents
% Einbinden der Kapitel
%VO1-2016-03-01
\setcounter{chapter}{3}
\chapter{Volumenmessung}
\setcounter{section}{2}
\section{Polynome \& Polynomfunktionen}
	Warum? (Vielleicht eher "`Algebra"' -- allgemein -- als "`lineare"' Algebra) Wichtig: das charakteristische Polynom eines Endomorphismus -- wichtiges Hilfsmittel im Kontext der Struktursätze.
\paragraph{Beispiel}
	Wir definieren Polynomfunktionen $ p,q: K\to K $ eines Körpers $ K $ in sich durch 
		\begin{align*}
		p:\ & K\to K,\ x\mapsto p(x):= 1+x+x^2\\
		q:\ & K\to K,\ x\mapsto q(x):= 1
		\end{align*}
	Falls $ K=\mathbb{Z}_2 $ so gilt dann
		\begin{align*}
		&\forall x\in K: x(x+1)=0\\
		\Rightarrow\ &\forall x\in K: p(x) = q(x)
		\end{align*}
	d.h., unterschiedliche "`Polynome"' liefern die gleiche Polynomfunktion: Koeffizientenvergleich funktioniert nicht.
\paragraph{Wiederholung}
	Auf dem Folgenraum $ K^\mathbb{N} $ betrachten wir die Familie $ (e_k)_{k\in \mathbb{N}} $ mit
		\[ e_k :\mathbb{N}\to K,\ j\mapsto e_k(j):= \delta_{jk}; \]
	wir wissen: $ (e_k)_{k\in \mathbb{N}} $ ist linear unabhängig, aber kein Erzeugendensystem:
		\[ \forall k\in \mathbb{N}: e_k \notin [(e_j)_{j\neq k}] \text{ und }
		[(e_j)_{j\in\mathbb{N}}]\neq K^{\mathbb{N}}\]
	Insbesondere gilt:
		\[ \forall x\in [(e_j)_{j\in \mathbb{N}}]\ \exists n\in \mathbb{N}\ \forall k>n : x_k = 0 \]
\subsection{Idee \& Definition} \index{Polynom}\index{Cauchyprodukt}
	\begin{Definition}[Cauchyprodukt]
		Wir fassen ein Polynom als (endliche) Koeffizientenfolge auf,
		\[ \sum_{k=0}^{n} t^ka_k \cong
		\sum_{k\in \mathbb{N}}e_ka_k \text{ mit } a_k = 0 \text{ für } k>n \]
	und führen darauf das \emph{Cauchyprodukt} (vgl. Analysis) als Multiplikation ein:
		\[ (a_k)_{k\in \mathbb{N}} \odot (b_k)_{k\in \mathbb{N}} := (c_k)_{k\in \mathbb{N}} \]
	wobei
		\[ c_k := \sum_{j=0}^{k}a_jb_{k-j}. \]
	\end{Definition}
	Insbesondere gilt damit
		\begin{gather*}
		\forall j,k\in \mathbb{N}: e_j \odot e_k = e_{j+k}
		\Rightarrow \forall k\in \mathbb{N}:
			\begin{cases}
				e_0 \odot e_k = e_k\\
				e_1^k = \underset{k \text{ mal}}{\underbrace{e_1 \odot \cdots \odot e_1}} = e_k
			\end{cases}
		\end{gather*}
	Mit $ 1:= e_0,\ t:= e_1 $ und $ t^0 := 1 $, wie üblich, liefert dies:
		\[ \sum_{k=0}^{n}t^ka_k = \sum_{k\in \mathbb{N}}e_ka_k \in [(e_k)_{k\in N}]\subset K^\mathbb{N} \]
\subsection{Definition}\index{Polynom!-algebra}\index{Polynom!Grad}\index{Polynom!normiertes}
		\begin{Definition}[Polynomalgebra]
			\[ K[t] := ([(e_k)_{k\in \mathbb{N}}],\odot) ,\]
	mit dem Cauchyprodukt $ \odot $, ist die \emph{Polynomalgebra} über dem Körper $ K $; die Elemente von $ K[t] $,
		\[ p(t) = \sum_{k=0}^{n}t^ka_k = \sum_{k\in\mathbb{N}}e_ka_k, \]
	heißen \emph{Polynome in der Variablen} $ t:= e_1 $.
	Der \emph{Grad} eines Polynoms ist
		\[  \deg\sum_{k=0}^{n}t^ka_k := \max \{k\in \mathbb{N}\mid a_k \neq 0\}  \quad \left( \text{bzw. } \deg 0 := -\infty \right) \]
	Ist (der "`höchste"' Koeffizient) $ a_n = 1 $ für $ \deg p(t) = n $, so heißt das Polynom $ p(t) $ \emph{normiert}.
		\end{Definition}
\paragraph{Notation}
	Mit $t^k = e_{k}$, also $ K[t] = [(e_k)_{k\in N}] $
	wird das Cauchyprodukt auf $ K[t] $ eine "`normale"' Multiplikation, gefolgt von einer Sortierung nach den Potenzen der Variablen $ t $. Wir werden das $ \odot $ daher oft unterdrücken, und z.B. $ p(t)q(t) $ schreiben, anstelle von $ p(t) \odot q(t) $.
\paragraph{Bemerkung (Koeffizientenvergleich)}
	Mit dieser Definition von "`Polynom"' gilt
		\begin{align*}
			p(t)=\sum_{k=0}^{n}t^ka_k = 0
			\Rightarrow \forall k\in \mathbb{N}: a_k = 0,
		\end{align*}
	da $ (t^k)_{k\in \mathbb{N}} = (e_k)_{k\in \mathbb{N}}$ linear unabhängig ist.
	Koeffizientenvergleich funktioniert!
\paragraph{Bemerkung}
	Die Polynomalgebra $ K[t] $ über $ K $ ist eine assoziative und kommutative $ K $-Algebra, weiters ist $ K[t] $ unitär mit Einselement $ 1=e_0 $.
\subsection{Definition}\index{Algebra}
	\begin{Definition}[Algebra]
		Eine $ K $-Algebra ist ein $ K $-VR mit einer \emph{bilinearen Abbildung},
		\[ \odot: V\times V \to V,\ (v,w)\mapsto v\odot w, \]
	d.h. es gilt
		\begin{enumerate}[(i)]
			\item $ \forall w\in V:\ V\ni v\mapsto v\odot w\in V $ ist linear;
			\item $ \forall v\in V: V\ni w\mapsto v\odot w\in V $ ist linear.
		\end{enumerate}
	
	Eine $ K $-Algebra heißt
		\begin{itemize}
			\item unitär (mit Einselement 1), falls  \hfill$ \exists 1\in V\forall v\in V: 1\odot v = v\odot 1 = v; $
			\item assoziativ, falls \hfill$ \forall u,v,w\in V: (u\odot v)\odot w = u\odot (v\odot w); $
			\item kommutativ, falls \hfill$ \forall v,w,\in V: v\odot w = w\odot v $
		\end{itemize}
	\end{Definition}
\paragraph{Beispiel}
	$ \End(V) $ ist (mit Komposition) eine unitäre assoziative Algebra.
\paragraph{Bemerkung}
	In jeder Algebra $ (V,\odot) $ gilt:
		\[ \forall v\in V: 0\odot v = v\odot 0 = 0 \]
	da z.B. für $ v\in V $ gilt
		\[ v\odot 0 = v\odot (0+0) = v\odot 0 + v\odot 0 \ \Rightarrow\  0=v\odot 0\]
	Ist $ (V,\odot) $ unitär, so folgt $ [1]\in V $ wegen $ 1\odot 1 = 1 $
		\[ ([1], +\mid_{[1]\times [1]},\odot\mid_{[1]\times [1]} ) \cong K \]
	vermöge $ K\ni x \mapsto 1 \cdot x\in [1] $ (siehe Aufgabe 5).
\subsection{Definition}\index{Algebra!-Homomorphismus}
	\begin{Definition}[Algebra-Homomorphismus]
		Ein \emph{Algebra-Homomorphismus} zwischen $ K $-Algebren $ (V,\odot) $ und $ (W,*) $ ist eine lineare Abbildung $ \psi\in \hom(V,W) $, für die gilt:
		\[ \forall v,v' \in V: \psi(v\odot v')=\psi(v)*\psi(v') \]
	\end{Definition}
\paragraph{Bemerkung}
	$ \hom(V,W) $ wird oft auch für den (Vektor-)Raum der Algebra-Homomorphismen verwendet. In dieser LVA bedeutet $ \hom(V,W) $ immer VR-Homomorphismen, bei allen "`anderen"' Homomorphismen wird extra erwähnt, was gemeint ist.

%VO2-2016-03-03

\subsection{Einsetzungssatz \& Definitionen}
	\begin{Satz}[Einsetzungssatz]\index{Einsetzungshomomorphismus}\index{Polynom!funktion}
		Seien $ (V,\odot) $ eine unitäre assoziative Algebra und $ v\in V $. Dann ist
			\[ \psi_v: K[t]\to V,\ \sum_{k=0}^{n}t^ka_k = p(t)\mapsto \psi_v(p(t)) := \sum_{k=0}^{n}v^ka_k \]
		-- wobei $ v^0 = 1 $ sinnvoll ist, da die Algebra unitär ist -- ein Algebra-Homomorphismus; $ \psi_v $ heißt \emph{Einsetzungshomomorphismus}. 
			\[ p:V\to V,\ v\mapsto p(v) := \psi_v(p(t)) \]
		heißt die zu $ p(t)\in K[t] $ gehörige \emph{Polynomfunktion} auf $ V $.
	\end{Satz}
\paragraph{Bemerkung}
	Wie üblich: $ v^k := \underset{k-\text{mal}}{\underbrace{v\odot\cdots \odot v}} $ und $ v^0 := 1 $.
\paragraph{Beweis}
	\begin{enumerate}
		\item $ \psi_v $ ist linear:
			\begin{itemize}
				\item für $ p(t) = \sum_{k\in\mathbb{N}} t^ka_k $ und $ a\in K $ gilt:
					\[ \psi_v(p(t)a) = \psi_v(\sum_{k\in\mathbb{N}} t^ka_ka) = \sum_{k\in\mathbb{N}} v^ka_ka = \psi_v(p(t))a; \]
				\item für $ p(t) = \sum_{k\in\mathbb{N}} t^ka_k $ und $ q(t) = \sum_{k\in\mathbb{N}} t^k b_k $ gilt:
					\[ \psi_v(p(t)+q(t)) = \psi_v(\sum_{k\in\mathbb{N}}t^k(a_k+b_k)) = \sum_{k\in\mathbb{N}} v^k(a_k+b_k) = \psi_v(p(t))+\psi_v(q(t)) \]
			\end{itemize}
		\item $ \psi_v $ ist verträglich mit der Multiplikation:
		
			Für die Vektoren der Basis $ (t^k)_{k\in\mathbb{N}} $ von $ K[t] $ gilt, da $ (V,\odot) $ assoziativ ist,
				\[ \psi_v(t^mt^n) = \psi_v(t^{m+n}) = v^{m+n} = v^m\odot v^n = \psi_v(t^m)\odot \psi_v(t^n). \]
			Da aber $ \psi_v $ linear und die Multiplikation in $ K[t] $ und in $ (V,\odot) $ bilinear sind, folgt die Behauptung.
	\end{enumerate}
\paragraph{Bemerkung (Fortsetzungssatz für bilineare Abbildungen)}
	Im Beweis haben wir verwendet: Die Abbildungen
		\[ K[t]\times K[t]\to V,\ (p(t),q(t))\mapsto
			\begin{cases}
			\psi_v(p(t)q(t))& \text{ (Cauchyprodukt)}\\
			\psi_v(p(t))\odot \psi_v(q(t)) & \text{(Produkt in $ (V,\odot) $)}
			\end{cases} \]
	sind bilinear (da $ \psi_v $ linear ist), sind also gleich, sobald sie auf einer Basis übereinstimmen.
	Dies ist die Eindeutigkeit eines Fortsetzungssatzes für bilineare Abbildungen:
	
	Sind $ V,W\ K$-VR, $ (b_i)_{i\in I} $ eine Basis von $ V $ und $ (\beta_{ij})_{i,j\in I} $ eine Familie in $ W $, so gibt es eine eindeutige bilineare Abbildung
		\[ \beta: V\times V\to W \]
	mit
		\[ \forall i,j\in I: \beta(b_i,b_j) = \beta_{ij} \]
	Dieser Fortsetzungssatz folgt direkt aus dem Fortsetzungssatz für lineare Abbildungen, da
		\[ \{\beta:V\times V\to W \text{ bilinear}\} \cong \hom(V,\hom(V,W))\]
	vermittels des Isomorphismus
		\[ \beta \mapsto \big(v\mapsto\underset{\in \hom(V,W)}{\underbrace{\beta(v,.)}}\big), \]
	d.h. durch Nacheinandereinsetzen der Argumente.
\paragraph{Bemerkung}
	Die Abbildung eines Polynoms auf seine Polynomfunktion auf dem Körper,
		\[ K[t]\ni p(t)\mapsto (x\mapsto p(x))=\psi_x(p(t))\in K^K \]
	ist für $ \Char K\neq 0 $ nicht injektiv, das heißt: Koeffizientenvergleich kann nur funktionieren, wenn $ \Char K = 0 $
\paragraph{Beispiel \& Bemerkung}
	Ist $ V\ K $-VR, so ist $ \End(V) $ eine $ K $-Algebra (mit Komposition $ \circ $). Man erhält also für $ f\in \End(V) $ einen Einsetzungshomomorphismus
		\[ \psi_f: K[t]\to \End(V),\ p(t) \mapsto \psi_f(p(t)) = p(f); \]
	und für jedes Polynom $ p(t)\in K[t] $ eine zugehörige Polynomfunktion
		\[ p: \End(V)\to\End(V),\ f\mapsto \psi_f(p(t))= p(f). \]
	Dieses Beispiel ist der Schlüssel zum Satz von Cayley-Hamilton (im nächsten Abschnitt).
\subsection{Lemma}
	\begin{Lemma}
		Für Polynome $ p(t), q(t)\in K[t] $ gilt:
			\begin{itemize}
				\item $ \deg p(t)\odot q(t) = \deg p(t)+\deg q(t) $,
				\item $ \deg p(t)+q(t) \leq \max\{\deg p(t), \deg q(t)\} $.
			\end{itemize}
	\end{Lemma}
\paragraph{Beweis}
	Für $ p(t) = \sum_{k\in\mathbb{N}}t^ka_k $ und $ q(t) = \sum_{k\in\mathbb{N}}t^kb_k $ ist
		\[ p(t)\odot q(t) = \sum_{k\in\mathbb{N}}t^kc_k \text{ mit } c_k = \sum_{j=0}^{k}a_jb_{k-j} \]
	Gilt nun $ \deg p(t) = n $ und $ \deg q(t) = m $, d.h.
		\[ a_n,b_n \neq 0 \land \forall k>n, k'>m:a_k = b_{k'}=0 \] 
	so folgt
		\[ \left.
		\begin{aligned}
		\forall k>m+n : c_k = 0\ \\
		        c_{m+n} = a_nb_m\ \\
		\end{aligned}
		 \right\}
		\Rightarrow \deg p(t)\odot q(t) = m+n \]
	Gilt andererseits $ \deg p(t) = -\infty $ oder $ \deg q(t) = -\infty $, also $ p(t) = 0 \lor q(t) = 0 $,
	so folgt
		\[ p(t)\odot q(t) = 0 \Rightarrow \deg p(t)\odot q(t) = -\infty. \]
	Die zweite Behauptung ist offensichtlich wahr.

\section{Das charakteristische Polynom}
\subsection{Definition}\index{Eigenwert,-vektor,-raum}
\begin{Definition}[Eigenwert,Eigenvektor,Eigenraum]
	Seien $ V $ ein $ K $-VR und $ f\in\End(V) $. Dann heißen
		\begin{enumerate}[(i)]
			\item $ x\in K $ ein Eigenwert von $ f $, falls
				\[ \exists v\in V^\times: f(v)=vx; \]
			\item $ v\in V^\times $ ein Eigenvektor von $ f $, falls
				\[ \exists x\in K:f(v)=vx; \]
			\item $ \ker(f-\id_Vx) \subset V $ ein Eigenraum, falls
				\[ \ker(f-\id_Vx) \neq \{0\}.\]
		\end{enumerate}
	\end{Definition}
\paragraph{Bemerkung}
	Der Skalar $ x\in K $ ist genau dann ein Eigenwert von $ f\in \End(V) $, wenn $ \ker(f-\id_Vx)\neq \{0\} $, d.h., wenn ein Eigenvektor $ v\in V^\times $ zu $ x $ existiert.
\paragraph{Beispiel}
	Für $ \frac{d}{ds} \in \End(C^\infty(\mathbb{R}))$ ist jedes $ x\in \mathbb{R} $ ein Eigenwert, da
		\[ \Big(\frac{d}{ds}-\id_Vx\Big)v = 0 \text{ für } v:\mathbb{R}\to\mathbb{R},s\mapsto v(s):= e^{xs}, \]
	wobei $ v\in C^\infty(\mathbb{R})\setminus \{0\} $, d.h. $ s\mapsto v(s)=e^{xs} $ ist ein Eigenvektor zum Eigenwert $ x\in\mathbb{R} $.
\paragraph{Beispiel}
	Ist $ \dim V < \infty $, so kann die Determinante zur Bestimmung von Eigenwerten von Endomorphismen $ f\in\End(V) $ benutzt werden, da
		\[ \ker(f-\id_Vx)\neq \{0\} \Leftrightarrow (f-\id_Vx) \text{ nicht injektiv}\Leftrightarrow \det(f-\id_Vx) = 0, \]
	d.h. das Auffinden von Eigenwerten $ x\in K $ von $ f $ ist reduziert auf die Bestimmung der Nullstellen der Funktion
		\[ K\ni x\mapsto \det(f-\id_Vx)\in K. \]
		
\paragraph{Beispiel}	
	Ist z.B. $ (b_1,b_2) $ Basis von $ V $ und $ f\in \End(V) $ durch $ f(B)=BX $ gegeben, so liefern die Nullstellen der Polynomfunktion
		\begin{gather*}
		\det(f-\id_Vx) = \det(X-E_2 x)= \det \begin{pmatrix}
		x_{11}-x & x_{12}\\
		x_{21} & x_{22} -x
		\end{pmatrix}\\
	= (x_{11}-x)(x_{22}-x)-x_{12}x_{21}
	= x^2 - x(x_{11}+x_{22}) + (x_{11}x_{22}-x_{12}x_{21})
		\end{gather*}
	die Eigenwerte von $ f $ -- beispielsweise erhalten wir für
		\[ X = \begin{pmatrix} 2 &3\\1 & 0 \end{pmatrix}:\ 
			\det(f-\id_Vx) = x^2-2x-4 = (x+1)(x-3), \]
	also Eigenwerte $ x_1 = -1 $ und $ x_2 = 3 $ mit zugehörigen Eigenvektoren als Lösungen von
		\[ v_i \in \ker(f-\id_Vx_i), \]
	also durch Lösungen der linearen Gleichungssysteme
		\[ \begin{pmatrix}
		2-(-1) & 3\\ 1 & -(-1)
		\end{pmatrix}
		\begin{pmatrix}
		v_1^1\\v_1^2
		\end{pmatrix} = \begin{pmatrix}
		3 & 3\\ 1 & 1
		\end{pmatrix}
		\begin{pmatrix}
		v_1^1\\v_1^2
		\end{pmatrix} \text{ und} \]
		\[ \begin{pmatrix}
		2-3 & 3\\ 1 & -3
		\end{pmatrix}
		\begin{pmatrix}
		v_2^1\\v_2^2
		\end{pmatrix}=
		\begin{pmatrix}
		-1 & 3\\ 1 & -3
		\end{pmatrix}
		\begin{pmatrix}
		v_2^1\\v_2^2
		\end{pmatrix}  \]
	sodass
		\[ v_1 = b_1-b_2 \text{ und } v_2 = b_13+b_2 \]
	Eigenvektoren zu den Eigenwerten $ x_1,x_2 $ liefert.

\paragraph{Rechenbeispiel 1}
	Für $ X = \begin{pmatrix}2&-1\\1&0\end{pmatrix} $ erhält man
		\[ \det(f-\id_Vx) = \det\begin{pmatrix}2-x&-1\\1&-x	\end{pmatrix} =x^2-2x+1 \]
	und Eigenvektoren zum Eigenwert $ x = 1 $ durch Lösung der LGS
		\[ \begin{pmatrix}
		2-1&-1\\1&-1
		\end{pmatrix}\begin{pmatrix}
		v_1^1\\v_1^2
		\end{pmatrix} =  \begin{pmatrix}
		1&-1\\1&-1
		\end{pmatrix}\begin{pmatrix}
		v_1^1\\v_1^2
		\end{pmatrix} \]
	d.h. der Eigenraum zum Eigenwert $ x $,
		\[ \ker(f-\id_V) = [\{b_1+b_2\}] \]
	hat
		\[ \dim \ker(f-\id_V)<\dim V. \]
\paragraph{Rechenbeispiel 2}
	Ist $ K=\mathbb{R} $ und
		\[ \det(f-\id_Vx)=x^2+1, \]
	so hat $ f $ keine Eigenwerte: z.B., wenn
		$ X=\begin{pmatrix} 0&1\\-1&0 \end{pmatrix} $.
		
\subsection{Definition} \index{Charakteristisches Polynom}
\begin{Definition}[Charakteristisches Polynom]
	Sei $ V $ ein $ K $-VR, für $ f\in\End(V) $ ist das \emph{charakteristische Polynom} von $ f $:
		\[ \chi_f(t) := \det (\id_Vt-f)\in K[t]. \]
	Analog definiert man für $ X\in K^{n\times n} $ das charakteristische Polynom
		\[ \chi_f(t) := \det (E_nt-X)\in K[t]. \]
\end{Definition}
\paragraph{Bemerkung}
	Oft wird auch das andere Vorzeichen in der Determinante verwendet, also $ \det(f-\id_Vt) $ bzw. $ \det(X-E_nt) $.
\paragraph{Bemerkung}
	\emph{Diese Definition ist erklärungsbedürftig!}
	
	Da $ t\notin K $ ist $ \id_Vt-f\notin \End(V) $, sondern $ \id_Vt-f\in\End(V)[t] $. Zwei Lösungsstrategien bieten sich an:
		\begin{enumerate}
			\item Erweiterung der Determinante auf $ \End(V)[t] $.
			\item Benutzung von Darstellungsmatrizen.
		\end{enumerate}
	Beide führen schließlich zur Leibniz-Formel:
	
	Ist $ B $ eine Basis von $ V $ und $ \xi_B^B(f) = X = (x_{ij})_{i,j\in\{1,\dots,n\}}$, so erhält man 
		\[ \chi_f(t)=\sum_{\sigma\in S_n}\sgn(\sigma)\prod_{j=1}^{n}\underset{\in K[t]}{\underbrace{\left(\delta_{\sigma(j)j}-x_{\sigma(j)j}\right)}} \in K[t]. \]
	Die Unabhängigkeit von der Basis $ B $ folgt aus der Transformationsformel für Darstellungsmatrizen und dem Determinanten-Multiplikationssatz (wie vorher für $ \det f = \det \xi_B^B(f) $).

% VO 2016-03-15

\subsection{Bemerkung \& Definition}\index{Spur}
\begin{Definition}[Spur]
	Ist $ \dim V=n $, so ist $ \chi_f(t) $ ein normiertes Polynom vom Grad $ \deg\left(\chi_f(t)\right)=n $,
		\[ \chi_f(t)=t^n-t^{n-1}\tr f + \dots + (-1)^n\det f,\] % = \det(-f) = \chi_f(0)
	wobei die \emph{Spur} $ \tr f $ (\glqq tr \grqq $\widehat{=}$ trace) von $ f $ durch diese Gleichung (wohl-)defininiert ist.
\end{Definition}	
	Ist $ (x_{ij})_{i,j\in\{1,\dots,n\}} = X = \xi_B^B(f) $ Darstellungsmatrix von $ f $, so gilt
		\[ \tr f = \sum_{j=1}^{n}x_{jj} = \sum_{j=1}^{n} b_j^*f(b_j). \]
	Oft wird $ \det(f-\id_vt)=(-1)^n\chi_f(t) $ als charakteristisches Polynom definiert -- dieses Polynom ist dann nur für gerade $ n $ normiert.
\subsection{Korollar}
\begin{Korollar}[Eigenwerte sind Nullstellen des char. Polynoms]
	Ein $ x\in K $ ist genau dann Eigenwert von $ f $, wenn $ \chi_f(x)=0 $.
	
	Also: Die Eigenwerte von $ f $ sind genau die Nullstellen des charakteristischen Polynoms $ \chi_f(t) $.
\end{Korollar}
\paragraph{Beweis}
	Klar -- das war die Idee hinter der Definition des charakteristischen Polynoms.
\subsection{Korollar \& Definition}\index{Algebraische/geometrische Vielfachheit}
\begin{Korollar}[Eigenwert ist Nullstelle des charakteristischen Polynoms]
	Ist $ x\in K $ Eigenwert von $ f\in\End(V) $, so ist $ (t-x) $ Teiler des charakteristischen Polynoms. Insbesondere gilt:
		\[ \exists!k\in \mathbb{N}^\times:
			\begin{cases}
				(t-x)^k\mid \chi_f(t)\\
				(t-x)^{k+1}\nmid \chi_f(t)
			\end{cases} \]
\end{Korollar}
\begin{Definition}[algebraische Vielfachheit, geometrische Vielfachheit]
	Diese Zahl $ k $ heißt die \emph{algebraische Vielfachheit} von $ x $;
		\[ g:= \dfkt(\id_Vx-f) \leq k \]
	ist die \emph{geometrische Vielfachheit} von $ x $.
\end{Definition}
\paragraph{Beweis}
	Da $ x $ Eigenwert von $ f $ ist, ist die Existenz und Eindeutigkeit von $ k $ klar. Außerdem gilt analog auch $ g\geq 1 $.
	
	Zu zeigen bleibt: $ g\leq k $, d.h. $ (t-x)^g \mid \chi_f(t) $:
	
	Für eine Basis $ B = (b_1,\dots,b_n) $ von $ V $ mit
	$ \ker (\id_v x - f) = [(b_1,\dots,b_g)]$
	hat
		\[ \xi_B^B(f) =
		\begin{pmatrix}
			E_gx & Y\\
			0 & X
		\end{pmatrix}
		\text{ mit } Y\in K^{g\times (n-g)}, X\in K^{(n-g)\times(n-g)} \]
	Blockgestalt, also ist
		\[ \chi_f(t)=(t-x)^g\cdot \chi_X(t), \]
	d.h. $ (t-x)^g \mid \chi_f(t)$, da $ (t-x)^{k+1}\nmid \chi_f(t) $, gilt also $ g\leq k $.
\paragraph{Beispiel}
	Ist $ f\in\End(V) $ wie oben durch $ f(B)=BX $ gegeben, so haben die Eigenwerte
		\[ x_1 = -1 \text{ und } x_2 = 3 \text{ für }
		X=\begin{pmatrix} 2 &3\\1 & 0 \end{pmatrix} \]
	algebraische und geometrische Vielfachheiten 
		\[ 1 = g_i = k_i, \text{ da } 1\leq g_i \leq k_i \text{ und } k_1+k_2 \leq 2; \]
	der Eigenwert
		\[ x=1 \text{ für } X = \begin{pmatrix} 2&-1\\1&0 \end{pmatrix} \]
	hat algebraische und geometrische Vielfachheiten
		\[ k = 2 \text{ und } g = 1 \]
	da
		\[ f\neq \id_V x = \id_V \]
	und $ \chi_f(t)=(t-x)^2 \in \mathbb{R}[t] $, da ein quadratisches Polynom zwei (relle oder komplex konjugierte) Nullstellen hat, oder aber eine doppelte reelle.

\subsection{Definition \& Lemma}\index{$ f $-invarianter Unterraum}
	Das Schlüsselargument im Beweis oben kann man verallgemeinern:

\begin{Definition}[$ f $-invarianter Unterraum]
	Sei $ f\in \End(V) $ und $ U\subset V $ ein \emph{$ f $-invarianter Unterraum}, d.h. $ f(U)\subset U $. 

\end{Definition}
\begin{Lemma}[]
	Ist dann $ V=U\oplus U' $ eine direkte Zerlegung und $ p,p'\in \End(V) $ die zugehörigen Projektionen, so gilt
		\[ \chi_f(t)=\chi_{f|_U}(t)\cdot \chi_{f'}(t), \]
	wobei
		\[ f':= p'\circ f|_{U'}\in \End(U'). \]

\end{Lemma}
\paragraph{Bemerkung}
	Man kann $ f|_U $ als Endomorphismus $ f|_U\in \End(U) $ auffassen, da $ f(U)\subset U $.
\paragraph{Beweis}
	Wie oben: Sei $ B=(b_1,\dots,b_n) $ Basis von $ V $, sodass
		\begin{itemize}
			\item $ C=(b_1,\dots,b_k) $ Basis von $ U $ und
			\item $ C'=(b_{k+1},\dots,b_n) $ Basis von $ U' $ ist.
		\end{itemize}
	Die Darstellungsmatrix von $ f $ bzgl. $ B $ hat dann Blockgestalt,
		\[ \xi_B^B(f) =
			\begin{pmatrix}
				X&Y\\0&X'
			\end{pmatrix}
		\text{ mit } X=\xi_C^C(f|_U), X' = \xi_{C'}^{C'}(f') \]
	Damit folgt die Behauptung (wie oben) mit der Leibniz-Formel.
\paragraph{Bemerkung}
	Alternativ kann man das Lemma mit der von $ f $ induzierten Quotientenabbildung $ f'\in \End(V/U) $ formulieren, wobei
		\[ f':V/U\to V/U, v+U\mapsto f'(v+U) := f(v)+U. \]
\subsection{Definition}\index{Diagonalisierbarkeit}\index{Triagonalisierbarkeit}
\begin{Definition}[Diagonalisierbarkeit, Triagonalisierbarkeit von Endomorphismen]
	Ein Endomorphismus $ f\in\End(f) $ heißt \emph{diagonalisierbar} bzw. \emph{trigonalisierbar}, falls es eine Basis $ B $ von $ V $ gibt, sodass $ \xi_B^B(f)=(x_{ij})_{i,j\in\{1,\dots,n\}} $ eine Diagonalmatrix 
		\[ i\neq j\Rightarrow x_{ij} = 0 \]
	bzw. obere Dreiecksmatrix ist,
		\[ i>j \Rightarrow x_{ij} = 0. \]
\end{Definition}
\paragraph{Bemerkung}
	Falls $ \dim V<\infty $, so ist $ f\in\End(V) $ genau dann diagonalisierbar, wenn $ V $ eine Basis aus Eigenvektoren von $ f $ besitzt. Damit kann man "`Diagonalisierbarkeit"' auch im Falle $ \dim V=\infty $ definieren.
\paragraph{Bemerkung}
	Ist $ f $ trigonalisierbar (oder gar diagonalisierbar), so zerfällt $ \chi_f (t) $ in Linearfaktoren: für geeignete $ x_1,\dots,x_n\in K $ ist
		\[ \chi_f(t)=\prod_{j=1}^{n}(t-x_j). \]
\subsection{Bemerkung \& Definition}
\begin{Definition}[Diagonalisierbarkeit, Triagonalisierbarkeit von Matrizen]
	Man nennt eine Matrix $ X\in K^{n\times n} $ diagonalisierbar (bzw. trigonalisierbar), falls $ f_X\in \End(K^n) $ diagonalisierbar (bzw. trigonalisierbar) ist.
\end{Definition}	

	Dies ist genau dann der Fall, falls es $ P\in Gl(n) $ gibt, sodass $ PXP^{-1} $ Diagonalmatrix (bzw. obere Dreiecksmatrix) ist.

% VO 2016-03-17

\subsection{Lemma}
	Frage: Was sind hinreichende Kriterien dafür? Notwendigkeit kennen wir: $ \chi_f(t) $ zerfällt in Linearfaktoren.
	
	\begin{Lemma}[Lineare Unabhängigkeit von Eigenvektoren]
		Eigenvektoren $ v_1,\dots,v_m\in V $ zu paarweise verschiedenen Eigenwerten $ x_1,\dots,x_m $ eines Endomorphismus $ f\in\End(V) $ sind linear unabhängig.
	\end{Lemma}
\paragraph{Bemerkung}
	Anders gesagt: Die Summe von Eigenräumen zu paarweise verschiedenen Eigenwerten ist direkt.
\paragraph{Beweis}
	Zu zeigen: Ist $ \sum_{i=1}^m v_iy_i = 0 $ für Koeffizienten $ y_1,\dots,y_m\in K $, so folgt $ y_1 = \dots = y_m = 0 $.
	
	Seien $ y_1,\dots,y_m \in K $ und $ w_i := v_iy_i $ und $ w:= \sum_{i=1}^{m}w_i = \sum_{i=1}^{m}v_iy_i$.
	Wiederholte Anwendung von $ f $ liefert, wegen $ f(w_i) = w_ix_i $
	
		\[ (f^{m-1}(w),\dots,f^2(w),f(w),w) = (w_1,\dots,w_m)
		\begin{pmatrix}
		 x_1^{m-1}&\cdots&x_1^2&x_1&1 \\
		 \vdots&\ddots&\vdots&\vdots&\vdots\\
		 x_m^{m-1}&\cdots&x_m^2&x_m & 1
		\end{pmatrix} \]
	mit der Vandermonde-Matrix $ X\in Gl(m) $, da
		\[ \det X = \prod_{i<j} (x_i - x_j)\neq 0 \]
	weil die Eigenwerte $ x_1,\dots,x_m $ paarweise verschieden sind.
	Damit folgt aus $ w=\sum_{i=1}^{m}v_iy_i = 0 $
		\[ (w_1,\dots,w_m)=(f^{m-1}(w),\dots,f(w),w)X^{-1} = (0,\dots,0) \]
	also
		\[ \forall i=1,\dots,m: 0 = w_i = v_iy_i \text{ und }v_i \neq 0 \Rightarrow y_i = 0.  \]
\subsection{Satz}
    \begin{Satz}[Diagonalisierbarkeit eines Endomorpismus]
		Ein Endomorphismus $ f\in \End(V) $ ist genau dann diagonalisierbar, wenn $ \chi_f(t) \in K[t] $ in Linearfaktoren zerfällt und die algebraischen und geometrischen Vielfachheiten aller Eigenwerte übereinstimmen,
			\[ \chi_f(t) = \prod_{i=1}^{m}(t-x_i)^{k_i} \text{ und } \forall i=1,\dots,m: k_i = g_i.\]
	\end{Satz}
\paragraph{Beweis}
	Ist $ f $ diagonalisierbar, so existiert eine Basis $ B $ aus Eigenvektoren von $ f $, also ist dann
		\[ \xi_B^B(f) =
			\begin{pmatrix}
				E_{g_1}x_1 &0& \cdots & 0 \\
				0 &E_{g_2}x_2& \ddots & \vdots\\
				\vdots & \ddots& \ddots & \vdots\\
				0 & 0 & \cdots & E_{g_m}x_m 
			\end{pmatrix} \]
	Damit ist
		\[ \chi_f(t)=\prod_{i=1}^{m}(t-x_i)^{g_i}. \]
	Hat andererseits das charakteristische Polynom diese Gestalt, so wähle man in jedem Eigenraum $ \ker(\id_Vx_i-f) $ eine Basis $ C_i,i=1,\dots,m $. Da Eigenvektoren zu verschiedenen Eigenwerten linear unabhängig sind, und wegen
		\[ g_1+\dots+g_m = k_1 + \dots + k_m = \dim V \]
	liefert $ B := \bigcup_{i=1}^mC_i $ eine Basis von $ V $.
\subsection{Korollar}
	\begin{Korollar}
		Ein Endomorphismus $ f\in\End(V) $ mit $ n=\dim V $ paarweise verschiedenen Eigenwerten ist diagonalisierbar.
	\end{Korollar}
\paragraph{Beweis}
	Für die geometrischen und algebraischen Vielfachheiten jedes Eigenwerts gilt
		\[ 1\leq g_i \leq k_i \text{ und } \sum_{i=1}^{n}k_i \leq n. \]
	Damit folgt
		\[ \forall i=1,\dots,n:k_i = 1 \text{ und } \sum_{i=1}^{n}k_i = n, \]
	d.h. das charakteristische Polynom zerfällt in Linearfaktoren und $ \forall i=1,\dots,n:k_i=g_i. $
\subsection{Satz}
\begin{Satz}[Trigonalisierbarkeit eines Endomorpismus]
	Ein Endomorpismus $ f\in\End(V) $ ist genau dann trigonalisierbar, wenn das charakteristische Polynom in Linearfaktoren zerfällt.
\end{Satz}
\paragraph{Bemerkung}
	Da Diagonalisierbarkeit bzw. Trigonalisierbarkeit durch die Existenz einer Darstellungsmatrix in spezieller Gestalt definiert wurde, wird in den Charakterisierungen immer (implizit) $ \dim V < \infty $ angenommen.
\paragraph{Beweis}
	Wir wissen schon: Ist $ f $ trigonalisierbar, so zerfällt $ \chi_f(t) $ in Linearfaktoren. Umkehrung: Beweis durch vollständige Induktion über $ n=\dim V $.
	
	Für $ n=1 $ ist nichts zu zeigen. Sei die Behauptung für $ n-1 $ bewiesen. Für $ n $ folgt dann:
	
	Da $ \chi_f(t) $ in Linearfaktoren zerfällt
		\[ \chi_f(t)=\prod_{i=1}^{n}(t-x_i) \]
	für geeignete $ x_1,\dots,x_n $, ist $ x_1 $ Eigenwert von $ f $. Nun seien
	\begin{itemize}
		\item $ b_1 $ ein Eigenvektor zum Eigenwert $ x_1 $ und $ U:= [\{b_1\}] $,
		\item $ U'\subset V $ ein zu $ U $ komplementärer Unterraum, und
		\item $ p,p'\in \End(V) $ die zur direkten Zerlegung $ V = U\oplus U' $ gehörenden Projektionen,
			\[ U = p(V) = \ker p' \text{ und } U' = p'(V) = \ker p, \]
		\item und $ f' := p'\circ f|_{U'}\in\End(U'). $
	\end{itemize}
	Da $ U (\neq \{0\}) $ $ f $-invarianter UR von $ V $ ist, faktorisiert das charakteristische Polynom
		\[ \chi_f(t)=\chi_{f|_U}(t)\cdot \chi_{f'}(t) = (t-x_1)\cdot \chi_{f'}(t); \]
	also zerfällt $ \chi_{f'}(t) $ in Linearfaktoren,
		\[ \chi_{f'}(t)=\prod_{i=2}^{n}(t-x_i). \]
	Nach Induktionsannahme existiert also eine Basis $ B' = (b_2,\dots,b_n) $ von $ U' $, sodass $ \xi_{B'}^{B'}(f) $ obere Dreiecksmatrix ist. Mit $ B=(b_1,\dots,b_n) $ als Basis von $ V $ gilt dann:
		\[ \xi_B^B (f) =
			\begin{pmatrix}
			x_1& Y\\
			0 & \xi_{B'}^{B'}(f')
			\end{pmatrix} \]
	ist obere Dreiecksmatrix.

\printindex
\end{document}
