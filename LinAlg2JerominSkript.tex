\documentclass[11pt, DIV=12, parskip=half]{scrreprt}
\usepackage{microtype}
\usepackage[utf8]{inputenc}
\usepackage[T1]{fontenc}
\usepackage{lmodern}%schoeneres Schriftbild
\usepackage[ngerman]{babel}%deutsche Silbentrennung
\usepackage[onehalfspacing]{setspace}
\usepackage{tikz}%Zeichnungen
\usetikzlibrary{calc,arrows.meta}
\usepackage{tikz-3dplot}
\usepackage{amsmath,amsfonts,amssymb}%Mathematik-Pakete
\usepackage{graphicx}
\usepackage{enumerate}%enumerate mit roemischen Zahlen
\usepackage{float}%fuer H Positionierung
\usepackage{multicol}
\usepackage{hyperref} %Verlinktes Inhaltsverzeichnis
\usepackage{makeidx} %Stichwortverzeichnis
\makeindex

\author{Studierendenmitschrift}
\title{Skript Lineare Algebra \& Geometrie 2, Hertrich-Jeromin}

\setcounter{tocdepth}{1}

% Eigene Operatoren:
\let\hom\relax
\DeclareMathOperator{\Char}{Char}
\DeclareMathOperator{\End}{End}
\DeclareMathOperator{\Aut}{Aut}
\DeclareMathOperator{\Iso}{Iso}
\DeclareMathOperator{\hom}{Hom}
\DeclareMathOperator{\rg}{rg}
\DeclareMathOperator{\dfkt}{def}
\DeclareMathOperator{\id}{id}
\DeclareMathOperator{\sgn}{sgn}
\DeclareMathOperator{\vol}{vol}
\DeclareMathOperator{\ggT}{ggT} 
\DeclareMathOperator{\tr}{tr} 

\newenvironment{Satz}[1][]{}{\par\addvspace{\baselineskip}}
\newenvironment{Lemma}[1][]{}{\par\addvspace{\baselineskip}}
\newenvironment{Definition}[1][]{}{\par\addvspace{\baselineskip}}
\newenvironment{Korollar}[1][]{}{\par\addvspace{\baselineskip}}

\begin{document}
\maketitle
\tableofcontents
% Einbinden der Kapitel
%VO1-2016-03-01
\setcounter{chapter}{3}
\chapter{Volumenmessung}
\setcounter{section}{2}
\section{Polynome \& Polynomfunktionen}
	Warum? (Vielleicht eher "`Algebra"' -- allgemein -- als "`lineare"' Algebra) Wichtig: das charakteristische Polynom eines Endomorphismus -- wichtiges Hilfsmittel im Kontext der Struktursätze.
\paragraph{Beispiel}
	Wir definieren Polynomfunktionen $ p,q: K\to K $ eines Körpers $ K $ in sich durch 
		\begin{align*}
		p:\ & K\to K,\ x\mapsto p(x):= 1+x+x^2\\
		q:\ & K\to K,\ x\mapsto q(x):= 1
		\end{align*}
	Falls $ K=\mathbb{Z}_2 $ so gilt dann
		\begin{align*}
		&\forall x\in K: x(x+1)=0\\
		\Rightarrow\ &\forall x\in K: p(x) = q(x)
		\end{align*}
	d.h., unterschiedliche "`Polynome"' liefern die gleiche Polynomfunktion: Koeffizientenvergleich funktioniert nicht.
\paragraph{Wiederholung}
	Auf dem Folgenraum $ K^\mathbb{N} $ betrachten wir die Familie $ (e_k)_{k\in \mathbb{N}} $ mit
		\[ e_k :\mathbb{N}\to K,\ j\mapsto e_k(j):= \delta_{jk}; \]
	wir wissen: $ (e_k)_{k\in \mathbb{N}} $ ist linear unabhängig, aber kein Erzeugendensystem:
		\[ \forall k\in \mathbb{N}: e_k \notin [(e_j)_{j\neq k}] \text{ und }
		[(e_j)_{j\in\mathbb{N}}]\neq K^{\mathbb{N}}\]
	Insbesondere gilt:
		\[ \forall x\in [(e_j)_{j\in \mathbb{N}}]\ \exists n\in \mathbb{N}\ \forall k>n : x_k = 0 \]
\subsection{Idee \& Definition} \index{Polynom}\index{Cauchyprodukt}
	\begin{Definition}[Cauchyprodukt]
		Wir fassen ein Polynom als (endliche) Koeffizientenfolge auf,
		\[ \sum_{k=0}^{n} t^ka_k \cong
		\sum_{k\in \mathbb{N}}e_ka_k \text{ mit } a_k = 0 \text{ für } k>n \]
	und führen darauf das \emph{Cauchyprodukt} (vgl. Analysis) als Multiplikation ein:
		\[ (a_k)_{k\in \mathbb{N}} \odot (b_k)_{k\in \mathbb{N}} := (c_k)_{k\in \mathbb{N}} \]
	wobei
		\[ c_k := \sum_{j=0}^{k}a_jb_{k-j}. \]
	\end{Definition}
	Insbesondere gilt damit
		\begin{gather*}
		\forall j,k\in \mathbb{N}: e_j \odot e_k = e_{j+k}
		\Rightarrow \forall k\in \mathbb{N}:
			\begin{cases}
				e_0 \odot e_k = e_k\\
				e_1^k = \underset{k \text{ mal}}{\underbrace{e_1 \odot \cdots \odot e_1}} = e_k
			\end{cases}
		\end{gather*}
	Mit $ 1:= e_0,\ t:= e_1 $ und $ t^0 := 1 $, wie üblich, liefert dies:
		\[ \sum_{k=0}^{n}t^ka_k = \sum_{k\in \mathbb{N}}e_ka_k \in [(e_k)_{k\in N}]\subset K^\mathbb{N} \]
\subsection{Definition}\index{Polynom!-algebra}\index{Polynom!Grad}\index{Polynom!normiertes}
		\begin{Definition}[Polynomalgebra]
			\[ K[t] := ([(e_k)_{k\in \mathbb{N}}],\odot) ,\]
	mit dem Cauchyprodukt $ \odot $, ist die \emph{Polynomalgebra} über dem Körper $ K $; die Elemente von $ K[t] $,
		\[ p(t) = \sum_{k=0}^{n}t^ka_k = \sum_{k\in\mathbb{N}}e_ka_k, \]
	heißen \emph{Polynome in der Variablen} $ t:= e_1 $.
	Der \emph{Grad} eines Polynoms ist
		\[  \deg\sum_{k=0}^{n}t^ka_k := \max \{k\in \mathbb{N}\mid a_k \neq 0\}  \quad \left( \text{bzw. } \deg 0 := -\infty \right) \]
	Ist (der "`höchste"' Koeffizient) $ a_n = 1 $ für $ \deg p(t) = n $, so heißt das Polynom $ p(t) $ \emph{normiert}.
		\end{Definition}
\paragraph{Notation}
	Mit $t^k = e_{k}$, also $ K[t] = [(e_k)_{k\in N}] $
	wird das Cauchyprodukt auf $ K[t] $ eine "`normale"' Multiplikation, gefolgt von einer Sortierung nach den Potenzen der Variablen $ t $. Wir werden das $ \odot $ daher oft unterdrücken, und z.B. $ p(t)q(t) $ schreiben, anstelle von $ p(t) \odot q(t) $.
\paragraph{Bemerkung (Koeffizientenvergleich)}
	Mit dieser Definition von "`Polynom"' gilt
		\begin{align*}
			p(t)=\sum_{k=0}^{n}t^ka_k = 0
			\Rightarrow \forall k\in \mathbb{N}: a_k = 0,
		\end{align*}
	da $ (t^k)_{k\in \mathbb{N}} = (e_k)_{k\in \mathbb{N}}$ linear unabhängig ist.
	Koeffizientenvergleich funktioniert!
\paragraph{Bemerkung}
	Die Polynomalgebra $ K[t] $ über $ K $ ist eine assoziative und kommutative $ K $-Algebra, weiters ist $ K[t] $ unitär mit Einselement $ 1=e_0 $.
\subsection{Definition}\index{Algebra}
	\begin{Definition}[Algebra]
		Eine $ K $-Algebra ist ein $ K $-VR mit einer \emph{bilinearen Abbildung},
		\[ \odot: V\times V \to V,\ (v,w)\mapsto v\odot w, \]
	d.h. es gilt
		\begin{enumerate}[(i)]
			\item $ \forall w\in V:\ V\ni v\mapsto v\odot w\in V $ ist linear;
			\item $ \forall v\in V: V\ni w\mapsto v\odot w\in V $ ist linear.
		\end{enumerate}
	
	Eine $ K $-Algebra heißt
		\begin{itemize}
			\item unitär (mit Einselement 1), falls  \hfill$ \exists 1\in V\forall v\in V: 1\odot v = v\odot 1 = v; $
			\item assoziativ, falls \hfill$ \forall u,v,w\in V: (u\odot v)\odot w = u\odot (v\odot w); $
			\item kommutativ, falls \hfill$ \forall v,w,\in V: v\odot w = w\odot v $
		\end{itemize}
	\end{Definition}
\paragraph{Beispiel}
	$ \End(V) $ ist (mit Komposition) eine unitäre assoziative Algebra.
\paragraph{Bemerkung}
	In jeder Algebra $ (V,\odot) $ gilt:
		\[ \forall v\in V: 0\odot v = v\odot 0 = 0 \]
	da z.B. für $ v\in V $ gilt
		\[ v\odot 0 = v\odot (0+0) = v\odot 0 + v\odot 0 \ \Rightarrow\  0=v\odot 0\]
	Ist $ (V,\odot) $ unitär, so folgt $ [1]\in V $ wegen $ 1\odot 1 = 1 $
		\[ ([1], +\mid_{[1]\times [1]},\odot\mid_{[1]\times [1]} ) \cong K \]
	vermöge $ K\ni x \mapsto 1 \cdot x\in [1] $ (siehe Aufgabe 5).
\subsection{Definition}\index{Algebra!-Homomorphismus}
	\begin{Definition}[Algebra-Homomorphismus]
		Ein \emph{Algebra-Homomorphismus} zwischen $ K $-Algebren $ (V,\odot) $ und $ (W,*) $ ist eine lineare Abbildung $ \psi\in \hom(V,W) $, für die gilt:
		\[ \forall v,v' \in V: \psi(v\odot v')=\psi(v)*\psi(v') \]
	\end{Definition}
\paragraph{Bemerkung}
	$ \hom(V,W) $ wird oft auch für den (Vektor-)Raum der Algebra-Homomorphismen verwendet. In dieser LVA bedeutet $ \hom(V,W) $ immer VR-Homomorphismen, bei allen "`anderen"' Homomorphismen wird extra erwähnt, was gemeint ist.

%VO2-2016-03-03

\subsection{Einsetzungssatz \& Definitionen}
	\begin{Satz}[Einsetzungssatz]\index{Einsetzungshomomorphismus}\index{Polynom!funktion}
		Seien $ (V,\odot) $ eine unitäre assoziative Algebra und $ v\in V $. Dann ist
			\[ \psi_v: K[t]\to V,\ \sum_{k=0}^{n}t^ka_k = p(t)\mapsto \psi_v(p(t)) := \sum_{k=0}^{n}v^ka_k \]
		-- wobei $ v^0 = 1 $ sinnvoll ist, da die Algebra unitär ist -- ein Algebra-Homomorphismus; $ \psi_v $ heißt \emph{Einsetzungshomomorphismus}. 
			\[ p:V\to V,\ v\mapsto p(v) := \psi_v(p(t)) \]
		heißt die zu $ p(t)\in K[t] $ gehörige \emph{Polynomfunktion} auf $ V $.
	\end{Satz}
\paragraph{Bemerkung}
	Wie üblich: $ v^k := \underset{k-\text{mal}}{\underbrace{v\odot\cdots \odot v}} $ und $ v^0 := 1 $.
\paragraph{Beweis}
	\begin{enumerate}
		\item $ \psi_v $ ist linear:
			\begin{itemize}
				\item für $ p(t) = \sum_{k\in\mathbb{N}} t^ka_k $ und $ a\in K $ gilt:
					\[ \psi_v(p(t)a) = \psi_v(\sum_{k\in\mathbb{N}} t^ka_ka) = \sum_{k\in\mathbb{N}} v^ka_ka = \psi_v(p(t))a; \]
				\item für $ p(t) = \sum_{k\in\mathbb{N}} t^ka_k $ und $ q(t) = \sum_{k\in\mathbb{N}} t^k b_k $ gilt:
					\[ \psi_v(p(t)+q(t)) = \psi_v(\sum_{k\in\mathbb{N}}t^k(a_k+b_k)) = \sum_{k\in\mathbb{N}} v^k(a_k+b_k) = \psi_v(p(t))+\psi_v(q(t)) \]
			\end{itemize}
		\item $ \psi_v $ ist verträglich mit der Multiplikation:
		
			Für die Vektoren der Basis $ (t^k)_{k\in\mathbb{N}} $ von $ K[t] $ gilt, da $ (V,\odot) $ assoziativ ist,
				\[ \psi_v(t^mt^n) = \psi_v(t^{m+n}) = v^{m+n} = v^m\odot v^n = \psi_v(t^m)\odot \psi_v(t^n). \]
			Da aber $ \psi_v $ linear und die Multiplikation in $ K[t] $ und in $ (V,\odot) $ bilinear sind, folgt die Behauptung.
	\end{enumerate}
\paragraph{Bemerkung (Fortsetzungssatz für bilineare Abbildungen)}
	Im Beweis haben wir verwendet: Die Abbildungen
		\[ K[t]\times K[t]\to V,\ (p(t),q(t))\mapsto
			\begin{cases}
			\psi_v(p(t)q(t))& \text{ (Cauchyprodukt)}\\
			\psi_v(p(t))\odot \psi_v(q(t)) & \text{(Produkt in $ (V,\odot) $)}
			\end{cases} \]
	sind bilinear (da $ \psi_v $ linear ist), sind also gleich, sobald sie auf einer Basis übereinstimmen.
	Dies ist die Eindeutigkeit eines Fortsetzungssatzes für bilineare Abbildungen:
	
	Sind $ V,W\ K$-VR, $ (b_i)_{i\in I} $ eine Basis von $ V $ und $ (\beta_{ij})_{i,j\in I} $ eine Familie in $ W $, so gibt es eine eindeutige bilineare Abbildung
		\[ \beta: V\times V\to W \]
	mit
		\[ \forall i,j\in I: \beta(b_i,b_j) = \beta_{ij} \]
	Dieser Fortsetzungssatz folgt direkt aus dem Fortsetzungssatz für lineare Abbildungen, da
		\[ \{\beta:V\times V\to W \text{ bilinear}\} \cong \hom(V,\hom(V,W))\]
	vermittels des Isomorphismus
		\[ \beta \mapsto \big(v\mapsto\underset{\in \hom(V,W)}{\underbrace{\beta(v,.)}}\big), \]
	d.h. durch Nacheinandereinsetzen der Argumente.
\paragraph{Bemerkung}
	Die Abbildung eines Polynoms auf seine Polynomfunktion auf dem Körper,
		\[ K[t]\ni p(t)\mapsto (x\mapsto p(x))=\psi_x(p(t))\in K^K \]
	ist für $ \Char K\neq 0 $ nicht injektiv, das heißt: Koeffizientenvergleich kann nur funktionieren, wenn $ \Char K = 0 $
\paragraph{Beispiel \& Bemerkung}
	Ist $ V\ K $-VR, so ist $ \End(V) $ eine $ K $-Algebra (mit Komposition $ \circ $). Man erhält also für $ f\in \End(V) $ einen Einsetzungshomomorphismus
		\[ \psi_f: K[t]\to \End(V),\ p(t) \mapsto \psi_f(p(t)) = p(f); \]
	und für jedes Polynom $ p(t)\in K[t] $ eine zugehörige Polynomfunktion
		\[ p: \End(V)\to\End(V),\ f\mapsto \psi_f(p(t))= p(f). \]
	Dieses Beispiel ist der Schlüssel zum Satz von Cayley-Hamilton (im nächsten Abschnitt).
\subsection{Lemma}
	\begin{Lemma}
		Für Polynome $ p(t), q(t)\in K[t] $ gilt:
			\begin{itemize}
				\item $ \deg p(t)\odot q(t) = \deg p(t)+\deg q(t) $,
				\item $ \deg p(t)+q(t) \leq \max\{\deg p(t), \deg q(t)\} $.
			\end{itemize}
	\end{Lemma}
\paragraph{Beweis}
	Für $ p(t) = \sum_{k\in\mathbb{N}}t^ka_k $ und $ q(t) = \sum_{k\in\mathbb{N}}t^kb_k $ ist
		\[ p(t)\odot q(t) = \sum_{k\in\mathbb{N}}t^kc_k \text{ mit } c_k = \sum_{j=0}^{k}a_jb_{k-j} \]
	Gilt nun $ \deg p(t) = n $ und $ \deg q(t) = m $, d.h.
		\[ a_n,b_n \neq 0 \land \forall k>n, k'>m:a_k = b_{k'}=0 \] 
	so folgt
		\[ \left.
		\begin{aligned}
		\forall k>m+n : c_k = 0\ \\
		        c_{m+n} = a_nb_m\ \\
		\end{aligned}
		 \right\}
		\Rightarrow \deg p(t)\odot q(t) = m+n \]
	Gilt andererseits $ \deg p(t) = -\infty $ oder $ \deg q(t) = -\infty $, also $ p(t) = 0 \lor q(t) = 0 $,
	so folgt
		\[ p(t)\odot q(t) = 0 \Rightarrow \deg p(t)\odot q(t) = -\infty. \]
	Die zweite Behauptung ist offensichtlich wahr.

\section{Das charakteristische Polynom}
\subsection{Definition}\index{Eigenwert,-vektor,-raum}
\begin{Definition}[Eigenwert,Eigenvektor,Eigenraum]
	Seien $ V $ ein $ K $-VR und $ f\in\End(V) $. Dann heißen
		\begin{enumerate}[(i)]
			\item $ x\in K $ ein Eigenwert von $ f $, falls
				\[ \exists v\in V^\times: f(v)=vx; \]
			\item $ v\in V^\times $ ein Eigenvektor von $ f $, falls
				\[ \exists x\in K:f(v)=vx; \]
			\item $ \ker(f-\id_Vx) \subset V $ ein Eigenraum, falls
				\[ \ker(f-\id_Vx) \neq \{0\}.\]
		\end{enumerate}
	\end{Definition}
\paragraph{Bemerkung}
	Der Skalar $ x\in K $ ist genau dann ein Eigenwert von $ f\in \End(V) $, wenn $ \ker(f-\id_Vx)\neq \{0\} $, d.h., wenn ein Eigenvektor $ v\in V^\times $ zu $ x $ existiert.
\paragraph{Beispiel}
	Für $ \frac{d}{ds} \in \End(C^\infty(\mathbb{R}))$ ist jedes $ x\in \mathbb{R} $ ein Eigenwert, da
		\[ \Big(\frac{d}{ds}-\id_Vx\Big)v = 0 \text{ für } v:\mathbb{R}\to\mathbb{R},s\mapsto v(s):= e^{xs}, \]
	wobei $ v\in C^\infty(\mathbb{R})\setminus \{0\} $, d.h. $ s\mapsto v(s)=e^{xs} $ ist ein Eigenvektor zum Eigenwert $ x\in\mathbb{R} $.
\paragraph{Beispiel}
	Ist $ \dim V < \infty $, so kann die Determinante zur Bestimmung von Eigenwerten von Endomorphismen $ f\in\End(V) $ benutzt werden, da
		\[ \ker(f-\id_Vx)\neq \{0\} \Leftrightarrow (f-\id_Vx) \text{ nicht injektiv}\Leftrightarrow \det(f-\id_Vx) = 0, \]
	d.h. das Auffinden von Eigenwerten $ x\in K $ von $ f $ ist reduziert auf die Bestimmung der Nullstellen der Funktion
		\[ K\ni x\mapsto \det(f-\id_Vx)\in K. \]
		
\paragraph{Beispiel}	
	Ist z.B. $ (b_1,b_2) $ Basis von $ V $ und $ f\in \End(V) $ durch $ f(B)=BX $ gegeben, so liefern die Nullstellen der Polynomfunktion
		\begin{gather*}
		\det(f-\id_Vx) = \det(X-E_2 x)= \det \begin{pmatrix}
		x_{11}-x & x_{12}\\
		x_{21} & x_{22} -x
		\end{pmatrix}\\
	= (x_{11}-x)(x_{22}-x)-x_{12}x_{21}
	= x^2 - x(x_{11}+x_{22}) + (x_{11}x_{22}-x_{12}x_{21})
		\end{gather*}
	die Eigenwerte von $ f $ -- beispielsweise erhalten wir für
		\[ X = \begin{pmatrix} 2 &3\\1 & 0 \end{pmatrix}:\ 
			\det(f-\id_Vx) = x^2-2x-4 = (x+1)(x-3), \]
	also Eigenwerte $ x_1 = -1 $ und $ x_2 = 3 $ mit zugehörigen Eigenvektoren als Lösungen von
		\[ v_i \in \ker(f-\id_Vx_i), \]
	also durch Lösungen der linearen Gleichungssysteme
		\[ \begin{pmatrix}
		2-(-1) & 3\\ 1 & -(-1)
		\end{pmatrix}
		\begin{pmatrix}
		v_1^1\\v_1^2
		\end{pmatrix} = \begin{pmatrix}
		3 & 3\\ 1 & 1
		\end{pmatrix}
		\begin{pmatrix}
		v_1^1\\v_1^2
		\end{pmatrix} \text{ und} \]
		\[ \begin{pmatrix}
		2-3 & 3\\ 1 & -3
		\end{pmatrix}
		\begin{pmatrix}
		v_2^1\\v_2^2
		\end{pmatrix}=
		\begin{pmatrix}
		-1 & 3\\ 1 & -3
		\end{pmatrix}
		\begin{pmatrix}
		v_2^1\\v_2^2
		\end{pmatrix}  \]
	sodass
		\[ v_1 = b_1-b_2 \text{ und } v_2 = b_13+b_2 \]
	Eigenvektoren zu den Eigenwerten $ x_1,x_2 $ liefert.

\paragraph{Rechenbeispiel 1}
	Für $ X = \begin{pmatrix}2&-1\\1&0\end{pmatrix} $ erhält man
		\[ \det(f-\id_Vx) = \det\begin{pmatrix}2-x&-1\\1&-x	\end{pmatrix} =x^2-2x+1 \]
	und Eigenvektoren zum Eigenwert $ x = 1 $ durch Lösung der LGS
		\[ \begin{pmatrix}
		2-1&-1\\1&-1
		\end{pmatrix}\begin{pmatrix}
		v_1^1\\v_1^2
		\end{pmatrix} =  \begin{pmatrix}
		1&-1\\1&-1
		\end{pmatrix}\begin{pmatrix}
		v_1^1\\v_1^2
		\end{pmatrix} \]
	d.h. der Eigenraum zum Eigenwert $ x $,
		\[ \ker(f-\id_V) = [\{b_1+b_2\}] \]
	hat
		\[ \dim \ker(f-\id_V)<\dim V. \]
\paragraph{Rechenbeispiel 2}
	Ist $ K=\mathbb{R} $ und
		\[ \det(f-\id_Vx)=x^2+1, \]
	so hat $ f $ keine Eigenwerte: z.B., wenn
		$ X=\begin{pmatrix} 0&1\\-1&0 \end{pmatrix} $.
		
\subsection{Definition} \index{Charakteristisches Polynom}
\begin{Definition}[Charakteristisches Polynom]
	Sei $ V $ ein $ K $-VR, für $ f\in\End(V) $ ist das \emph{charakteristische Polynom} von $ f $:
		\[ \chi_f(t) := \det (\id_Vt-f)\in K[t]. \]
	Analog definiert man für $ X\in K^{n\times n} $ das charakteristische Polynom
		\[ \chi_f(t) := \det (E_nt-X)\in K[t]. \]
\end{Definition}
\paragraph{Bemerkung}
	Oft wird auch das andere Vorzeichen in der Determinante verwendet, also $ \det(f-\id_Vt) $ bzw. $ \det(X-E_nt) $.
\paragraph{Bemerkung}
	\emph{Diese Definition ist erklärungsbedürftig!}
	
	Da $ t\notin K $ ist $ \id_Vt-f\notin \End(V) $, sondern $ \id_Vt-f\in\End(V)[t] $. Zwei Lösungsstrategien bieten sich an:
		\begin{enumerate}
			\item Erweiterung der Determinante auf $ \End(V)[t] $.
			\item Benutzung von Darstellungsmatrizen.
		\end{enumerate}
	Beide führen schließlich zur Leibniz-Formel:
	
	Ist $ B $ eine Basis von $ V $ und $ \xi_B^B(f) = X = (x_{ij})_{i,j\in\{1,\dots,n\}}$, so erhält man 
		\[ \chi_f(t)=\sum_{\sigma\in S_n}\sgn(\sigma)\prod_{j=1}^{n}\underset{\in K[t]}{\underbrace{\left(\delta_{\sigma(j)j}-x_{\sigma(j)j}\right)}} \in K[t]. \]
	Die Unabhängigkeit von der Basis $ B $ folgt aus der Transformationsformel für Darstellungsmatrizen und dem Determinanten-Multiplikationssatz (wie vorher für $ \det f = \det \xi_B^B(f) $).

% VO 2016-03-15

\subsection{Bemerkung \& Definition}\index{Spur}
\begin{Definition}[Spur]
	Ist $ \dim V=n $, so ist $ \chi_f(t) $ ein normiertes Polynom vom Grad $ \deg\left(\chi_f(t)\right)=n $,
		\[ \chi_f(t)=t^n-t^{n-1}\tr f + \dots + (-1)^n\det f,\] % = \det(-f) = \chi_f(0)
	wobei die \emph{Spur} $ \tr f $ (\glqq tr \grqq $\widehat{=}$ trace) von $ f $ durch diese Gleichung (wohl-)defininiert ist.
\end{Definition}	
	Ist $ (x_{ij})_{i,j\in\{1,\dots,n\}} = X = \xi_B^B(f) $ Darstellungsmatrix von $ f $, so gilt
		\[ \tr f = \sum_{j=1}^{n}x_{jj} = \sum_{j=1}^{n} b_j^*f(b_j). \]
	Oft wird $ \det(f-\id_vt)=(-1)^n\chi_f(t) $ als charakteristisches Polynom definiert -- dieses Polynom ist dann nur für gerade $ n $ normiert.
\subsection{Korollar}
\begin{Korollar}[Eigenwerte sind Nullstellen des char. Polynoms]
	Ein $ x\in K $ ist genau dann Eigenwert von $ f $, wenn $ \chi_f(x)=0 $.
	
	Also: Die Eigenwerte von $ f $ sind genau die Nullstellen des charakteristischen Polynoms $ \chi_f(t) $.
\end{Korollar}
\paragraph{Beweis}
	Klar -- das war die Idee hinter der Definition des charakteristischen Polynoms.
\subsection{Korollar \& Definition}\index{Algebraische/geometrische Vielfachheit}
\begin{Korollar}[Eigenwert ist Nullstelle des charakteristischen Polynoms]
	Ist $ x\in K $ Eigenwert von $ f\in\End(V) $, so ist $ (t-x) $ Teiler des charakteristischen Polynoms. Insbesondere gilt:
		\[ \exists!k\in \mathbb{N}^\times:
			\begin{cases}
				(t-x)^k\mid \chi_f(t)\\
				(t-x)^{k+1}\nmid \chi_f(t)
			\end{cases} \]
\end{Korollar}
\begin{Definition}[algebraische Vielfachheit, geometrische Vielfachheit]
	Diese Zahl $ k $ heißt die \emph{algebraische Vielfachheit} von $ x $;
		\[ g:= \dfkt(\id_Vx-f) \leq k \]
	ist die \emph{geometrische Vielfachheit} von $ x $.
\end{Definition}
\paragraph{Beweis}
	Da $ x $ Eigenwert von $ f $ ist, ist die Existenz und Eindeutigkeit von $ k $ klar. Außerdem gilt analog auch $ g\geq 1 $.
	
	Zu zeigen bleibt: $ g\leq k $, d.h. $ (t-x)^g \mid \chi_f(t) $:
	
	Für eine Basis $ B = (b_1,\dots,b_n) $ von $ V $ mit
	$ \ker (\id_v x - f) = [(b_1,\dots,b_g)]$
	hat
		\[ \xi_B^B(f) =
		\begin{pmatrix}
			E_gx & Y\\
			0 & X
		\end{pmatrix}
		\text{ mit } Y\in K^{g\times (n-g)}, X\in K^{(n-g)\times(n-g)} \]
	Blockgestalt, also ist
		\[ \chi_f(t)=(t-x)^g\cdot \chi_X(t), \]
	d.h. $ (t-x)^g \mid \chi_f(t)$, da $ (t-x)^{k+1}\nmid \chi_f(t) $, gilt also $ g\leq k $.
\paragraph{Beispiel}
	Ist $ f\in\End(V) $ wie oben durch $ f(B)=BX $ gegeben, so haben die Eigenwerte
		\[ x_1 = -1 \text{ und } x_2 = 3 \text{ für }
		X=\begin{pmatrix} 2 &3\\1 & 0 \end{pmatrix} \]
	algebraische und geometrische Vielfachheiten 
		\[ 1 = g_i = k_i, \text{ da } 1\leq g_i \leq k_i \text{ und } k_1+k_2 \leq 2; \]
	der Eigenwert
		\[ x=1 \text{ für } X = \begin{pmatrix} 2&-1\\1&0 \end{pmatrix} \]
	hat algebraische und geometrische Vielfachheiten
		\[ k = 2 \text{ und } g = 1 \]
	da
		\[ f\neq \id_V x = \id_V \]
	und $ \chi_f(t)=(t-x)^2 \in \mathbb{R}[t] $, da ein quadratisches Polynom zwei (relle oder komplex konjugierte) Nullstellen hat, oder aber eine doppelte reelle.

\subsection{Definition \& Lemma}\index{$ f $-invarianter Unterraum}
	Das Schlüsselargument im Beweis oben kann man verallgemeinern:

\begin{Definition}[$ f $-invarianter Unterraum]
	Sei $ f\in \End(V) $ und $ U\subset V $ ein \emph{$ f $-invarianter Unterraum}, d.h. $ f(U)\subset U $. 

\end{Definition}
\begin{Lemma}[]
	Ist dann $ V=U\oplus U' $ eine direkte Zerlegung und $ p,p'\in \End(V) $ die zugehörigen Projektionen, so gilt
		\[ \chi_f(t)=\chi_{f|_U}(t)\cdot \chi_{f'}(t), \]
	wobei
		\[ f':= p'\circ f|_{U'}\in \End(U'). \]

\end{Lemma}
\paragraph{Bemerkung}
	Man kann $ f|_U $ als Endomorphismus $ f|_U\in \End(U) $ auffassen, da $ f(U)\subset U $.
\paragraph{Beweis}
	Wie oben: Sei $ B=(b_1,\dots,b_n) $ Basis von $ V $, sodass
		\begin{itemize}
			\item $ C=(b_1,\dots,b_k) $ Basis von $ U $ und
			\item $ C'=(b_{k+1},\dots,b_n) $ Basis von $ U' $ ist.
		\end{itemize}
	Die Darstellungsmatrix von $ f $ bzgl. $ B $ hat dann Blockgestalt,
		\[ \xi_B^B(f) =
			\begin{pmatrix}
				X&Y\\0&X'
			\end{pmatrix}
		\text{ mit } X=\xi_C^C(f|_U), X' = \xi_{C'}^{C'}(f') \]
	Damit folgt die Behauptung (wie oben) mit der Leibniz-Formel.
\paragraph{Bemerkung}
	Alternativ kann man das Lemma mit der von $ f $ induzierten Quotientenabbildung $ f'\in \End(V/U) $ formulieren, wobei
		\[ f':V/U\to V/U, v+U\mapsto f'(v+U) := f(v)+U. \]
\subsection{Definition}\index{Diagonalisierbarkeit}\index{Triagonalisierbarkeit}
\begin{Definition}[Diagonalisierbarkeit, Triagonalisierbarkeit von Endomorphismen]
	Ein Endomorphismus $ f\in\End(f) $ heißt \emph{diagonalisierbar} bzw. \emph{trigonalisierbar}, falls es eine Basis $ B $ von $ V $ gibt, sodass $ \xi_B^B(f)=(x_{ij})_{i,j\in\{1,\dots,n\}} $ eine Diagonalmatrix 
		\[ i\neq j\Rightarrow x_{ij} = 0 \]
	bzw. obere Dreiecksmatrix ist,
		\[ i>j \Rightarrow x_{ij} = 0. \]
\end{Definition}
\paragraph{Bemerkung}
	Falls $ \dim V<\infty $, so ist $ f\in\End(V) $ genau dann diagonalisierbar, wenn $ V $ eine Basis aus Eigenvektoren von $ f $ besitzt. Damit kann man "`Diagonalisierbarkeit"' auch im Falle $ \dim V=\infty $ definieren.
\paragraph{Bemerkung}
	Ist $ f $ trigonalisierbar (oder gar diagonalisierbar), so zerfällt $ \chi_f (t) $ in Linearfaktoren: für geeignete $ x_1,\dots,x_n\in K $ ist
		\[ \chi_f(t)=\prod_{j=1}^{n}(t-x_j). \]
\subsection{Bemerkung \& Definition}
\begin{Definition}[Diagonalisierbarkeit, Triagonalisierbarkeit von Matrizen]
	Man nennt eine Matrix $ X\in K^{n\times n} $ diagonalisierbar (bzw. trigonalisierbar), falls $ f_X\in \End(K^n) $ diagonalisierbar (bzw. trigonalisierbar) ist.
\end{Definition}	

	Dies ist genau dann der Fall, falls es $ P\in Gl(n) $ gibt, sodass $ PXP^{-1} $ Diagonalmatrix (bzw. obere Dreiecksmatrix) ist.

% VO 2016-03-17

\subsection{Lemma}
	Frage: Was sind hinreichende Kriterien dafür? Notwendigkeit kennen wir: $ \chi_f(t) $ zerfällt in Linearfaktoren.
	
	\begin{Lemma}[Lineare Unabhängigkeit von Eigenvektoren]
		Eigenvektoren $ v_1,\dots,v_m\in V $ zu paarweise verschiedenen Eigenwerten $ x_1,\dots,x_m $ eines Endomorphismus $ f\in\End(V) $ sind linear unabhängig.
	\end{Lemma}
\paragraph{Bemerkung}
	Anders gesagt: Die Summe von Eigenräumen zu paarweise verschiedenen Eigenwerten ist direkt.
\paragraph{Beweis}
	Zu zeigen: Ist $ \sum_{i=1}^m v_iy_i = 0 $ für Koeffizienten $ y_1,\dots,y_m\in K $, so folgt $ y_1 = \dots = y_m = 0 $.
	
	Seien $ y_1,\dots,y_m \in K $ und $ w_i := v_iy_i $ und $ w:= \sum_{i=1}^{m}w_i = \sum_{i=1}^{m}v_iy_i$.
	Wiederholte Anwendung von $ f $ liefert, wegen $ f(w_i) = w_ix_i $
	
		\[ (f^{m-1}(w),\dots,f^2(w),f(w),w) = (w_1,\dots,w_m)
		\begin{pmatrix}
		 x_1^{m-1}&\cdots&x_1^2&x_1&1 \\
		 \vdots&\ddots&\vdots&\vdots&\vdots\\
		 x_m^{m-1}&\cdots&x_m^2&x_m & 1
		\end{pmatrix} \]
	mit der Vandermonde-Matrix $ X\in Gl(m) $, da
		\[ \det X = \prod_{i<j} (x_i - x_j)\neq 0 \]
	weil die Eigenwerte $ x_1,\dots,x_m $ paarweise verschieden sind.
	Damit folgt aus $ w=\sum_{i=1}^{m}v_iy_i = 0 $
		\[ (w_1,\dots,w_m)=(f^{m-1}(w),\dots,f(w),w)X^{-1} = (0,\dots,0) \]
	also
		\[ \forall i=1,\dots,m: 0 = w_i = v_iy_i \text{ und }v_i \neq 0 \Rightarrow y_i = 0.  \]
\subsection{Satz}
    \begin{Satz}[Diagonalisierbarkeit eines Endomorpismus]
		Ein Endomorphismus $ f\in \End(V) $ ist genau dann diagonalisierbar, wenn $ \chi_f(t) \in K[t] $ in Linearfaktoren zerfällt und die algebraischen und geometrischen Vielfachheiten aller Eigenwerte übereinstimmen,
			\[ \chi_f(t) = \prod_{i=1}^{m}(t-x_i)^{k_i} \text{ und } \forall i=1,\dots,m: k_i = g_i.\]
	\end{Satz}
\paragraph{Beweis}
	Ist $ f $ diagonalisierbar, so existiert eine Basis $ B $ aus Eigenvektoren von $ f $, also ist dann
		\[ \xi_B^B(f) =
			\begin{pmatrix}
				E_{g_1}x_1 &0& \cdots & 0 \\
				0 &E_{g_2}x_2& \ddots & \vdots\\
				\vdots & \ddots& \ddots & \vdots\\
				0 & 0 & \cdots & E_{g_m}x_m 
			\end{pmatrix} \]
	Damit ist
		\[ \chi_f(t)=\prod_{i=1}^{m}(t-x_i)^{g_i}. \]
	Hat andererseits das charakteristische Polynom diese Gestalt, so wähle man in jedem Eigenraum $ \ker(\id_Vx_i-f) $ eine Basis $ C_i,i=1,\dots,m $. Da Eigenvektoren zu verschiedenen Eigenwerten linear unabhängig sind, und wegen
		\[ g_1+\dots+g_m = k_1 + \dots + k_m = \dim V \]
	liefert $ B := \bigcup_{i=1}^mC_i $ eine Basis von $ V $.
\subsection{Korollar}
	\begin{Korollar}
		Ein Endomorphismus $ f\in\End(V) $ mit $ n=\dim V $ paarweise verschiedenen Eigenwerten ist diagonalisierbar.
	\end{Korollar}
\paragraph{Beweis}
	Für die geometrischen und algebraischen Vielfachheiten jedes Eigenwerts gilt
		\[ 1\leq g_i \leq k_i \text{ und } \sum_{i=1}^{n}k_i \leq n. \]
	Damit folgt
		\[ \forall i=1,\dots,n:k_i = 1 \text{ und } \sum_{i=1}^{n}k_i = n, \]
	d.h. das charakteristische Polynom zerfällt in Linearfaktoren und $ \forall i=1,\dots,n:k_i=g_i. $
\subsection{Satz}
\begin{Satz}[Trigonalisierbarkeit eines Endomorpismus]
	Ein Endomorpismus $ f\in\End(V) $ ist genau dann trigonalisierbar, wenn das charakteristische Polynom in Linearfaktoren zerfällt.
\end{Satz}
\paragraph{Bemerkung}
	Da Diagonalisierbarkeit bzw. Trigonalisierbarkeit durch die Existenz einer Darstellungsmatrix in spezieller Gestalt definiert wurde, wird in den Charakterisierungen immer (implizit) $ \dim V < \infty $ angenommen.
\paragraph{Beweis}
	Wir wissen schon: Ist $ f $ trigonalisierbar, so zerfällt $ \chi_f(t) $ in Linearfaktoren. Umkehrung: Beweis durch vollständige Induktion über $ n=\dim V $.
	
	Für $ n=1 $ ist nichts zu zeigen. Sei die Behauptung für $ n-1 $ bewiesen. Für $ n $ folgt dann:
	
	Da $ \chi_f(t) $ in Linearfaktoren zerfällt
		\[ \chi_f(t)=\prod_{i=1}^{n}(t-x_i) \]
	für geeignete $ x_1,\dots,x_n $, ist $ x_1 $ Eigenwert von $ f $. Nun seien
	\begin{itemize}
		\item $ b_1 $ ein Eigenvektor zum Eigenwert $ x_1 $ und $ U:= [\{b_1\}] $,
		\item $ U'\subset V $ ein zu $ U $ komplementärer Unterraum, und
		\item $ p,p'\in \End(V) $ die zur direkten Zerlegung $ V = U\oplus U' $ gehörenden Projektionen,
			\[ U = p(V) = \ker p' \text{ und } U' = p'(V) = \ker p, \]
		\item und $ f' := p'\circ f|_{U'}\in\End(U'). $
	\end{itemize}
	Da $ U (\neq \{0\}) $ $ f $-invarianter UR von $ V $ ist, faktorisiert das charakteristische Polynom
		\[ \chi_f(t)=\chi_{f|_U}(t)\cdot \chi_{f'}(t) = (t-x_1)\cdot \chi_{f'}(t); \]
	also zerfällt $ \chi_{f'}(t) $ in Linearfaktoren,
		\[ \chi_{f'}(t)=\prod_{i=2}^{n}(t-x_i). \]
	Nach Induktionsannahme existiert also eine Basis $ B' = (b_2,\dots,b_n) $ von $ U' $, sodass $ \xi_{B'}^{B'}(f) $ obere Dreiecksmatrix ist. Mit $ B=(b_1,\dots,b_n) $ als Basis von $ V $ gilt dann:
		\[ \xi_B^B (f) =
			\begin{pmatrix}
			x_1& Y\\
			0 & \xi_{B'}^{B'}(f')
			\end{pmatrix} \]
	ist obere Dreiecksmatrix.
\section{Der Satz von Cayley-Hamilton}
\subsection{Satz}
	Für $ f\in\End(V) $ gilt $ \chi_f(f) = 0$.
\paragraph{Unfug-Beweis}
	Durch direktes Einsetzen erhält man
		\[ \chi_f(f)=\det (\id_V f-f) = \det 0 = 0. \]
\paragraph{Zum Verständnis des Satzes}
	Ist $ V $ ein $ K $-VR mit $ n=\dim V < \infty $ und $ f\in \End (V) $, so ist
		\[ \chi_f(t) = \sum_{k=0}^{n}t^ka_k \in K[t] \]
	ein (abstraktes) Polynom in der Variablen $ t\ (= e_1\in K^\mathbb{N})$ und der Einsetzungshomomorphismus $ \psi_f:K[t]\to \End(V) $ (also ein Algebrahomomophismus) liefert
		\[ \chi_f(f) = \psi_f\left(\chi_f(t)\right) = \sum_{k=0}^{n}f^k a_k. \]
	Der Satz sagt, dass $ 0 = \chi_f(f)\in \End(V) $, d.h.
		\[ \forall v\in V: \chi_f(f)(v) = 0. \]
		
\subsection{Definition \& Lemma}\index{$ f $-zyklische Basis}
\begin{Definition}[$ f $-zyklische Basis]\label{fzykl}
	Seien $ f\in\End(V) $ und $ B $ eine \emph{$ f $-zyklische Basis} von $ V $, d.h. eine Basis der Form
		\[ B= (b_1,\dots,b_n) = \left(b,f(b),\dots,f^{n-1}(b)\right). \]		
\end{Definition}
\begin{Lemma}
	Dann existieren $ a_0,\dots,a_{n-1}\in K $ mit
		\[ f^n(b)+\sum_{k=0}^{n-1}f^k(b)a_k = 0, \]
	mit diesen Koeffizienten ist
		\[ \chi_f(t) = t^n+t^{n-1}a_{n-1}+\dots,+ta_1+a_0. \]
\end{Lemma}
\paragraph{Bemerkung}
	Im Allgemeinen existiert zu $ f\in \End(V) $ keine $ f $-zyklische Basis von $ V $, z.B. für $ f = \id_V $ und $ \dim V \geq 2 $.
\paragraph{Beweis}
	Da $ B= \left(b,f(b),\dots,f^{n-1}(b)\right) $ eine Basis ist, ist $ f^n(b)\in [B] $ und damit existieren die $ a_k $ mit
		\[ 0 = f^n(b) + \sum_{k=0}^{n-1}f^k(b)a_k. \]
	Damit ist die Darstellungsmatrix von $ f $
		\[ \xi_B^B(f) =
		\begin{pmatrix}
			0      & \cdots & \cdots & 0      & -a_0     \\
			1      & \ddots &        & \vdots & -a_1     \\
			0      & 1      & \ddots & \vdots & \vdots   \\
			\vdots & \ddots & \ddots & 0      & \vdots   \\
			0      & \cdots & 0      & 1      & -a_{n-1}
		\end{pmatrix} =: X
		\]
	und Entwicklung von $ \chi_f(t)=\det(E_nt-\xi_B^B(f)) $ nach der ersten Zeile (nach Laplaceschem Entwicklungssatz -- dieser Satz war "`nur"' eine Methode, die Terme in der Leibniz-Formel zu sortieren) liefert
	\begin{align*}
		\det(E_nt-X) &=\det 
			\begin{pmatrix}
				t      & 0      & \cdots & 0      & a_0       \\
				-1     & t      & \ddots & \vdots & a_1       \\
				0      & -1     & \ddots & 0	  & \vdots    \\
				\vdots & \ddots & \ddots & t      & \vdots    \\
				0      & \cdots & 0      & -1     & t+a_{n-1}
			\end{pmatrix}\\
		&= t\cdot\det
			\begin{pmatrix}
				t      & 0      & \cdots & 0      & a_1       \\
				-1     & t      & \ddots & \vdots & a_2       \\
				0      & -1     & \ddots & 0 	  & \vdots    \\
				\vdots & \ddots & \ddots & t      & \vdots    \\
				0      & \cdots & 0      & -1     & t+a_{n-1}
			\end{pmatrix}
		+ (-1)^{n+1} a_0 \underbrace{\det(X_{1n})}_{(-1)^{n-1}}\\
		\intertext{mittels vollständiger Induktion folgt}
		&= t (t(\dots(\underbrace{t(t+a_{n-1})+a_{n-2}}_{\det \begin{pmatrix}t&a_{n-2}\\-1&t+a_{n-1}\end{pmatrix}})\dots)+a_1)+a_0 \\
		&= t (t^{n-1}+t^{n-2}a_{n-1}+\dots+a_1)+a_0 \\
		&= t^n+t^{n-1}a_{n-1}+\dots,+ta_1+a_0,
	\end{align*}
	wie behauptet.
	
\paragraph{Beispiel}
	Zur Lösung des reellen \emph{Anfangswertproblems}
		\[ y'' + 2y' - 3y = 0,\ 
		\begin{cases}
			y(0)=4 \\
			y'(0)=0
		\end{cases} \]
	schreiben wir dieses als System erster Ordnung mit dem Ansatz $ y_1 = y $ und $y_2 = y' $:
	
	Daraus erhält man mit $ Y= (y_1,y_2) $
	  \begin{align*}
		 Y' &= (y_1',y_2') = (y',y'') = (y',-2y'+3y)\\
		 &= (y,y') \begin{pmatrix}
		 	0 & 3\\ 1 & -2
		 \end{pmatrix} = YX	\text{ mit }
		 X= \begin{pmatrix}
			0 & 3\\ 1 & -2
		\end{pmatrix},
		\end{align*}
	d.h. wir suchen eine $ \frac{d}{ds} $-zyklische Basis $ (y,\frac{d}{ds}y) = (y,y') $ eines 2-$ \dim $ UVR $ [(y,y')]\subset C^\infty(\R) $ bezüglich derer $ \frac{d}{ds}\in \End(C^\infty(\R)) $ Darstellungsmatrix $ X $ hat.
	
	Der Ansatz $ y(s) = e^{xs} (v_0,v_1)$ reduziert das AWP auf ein Eigenwertproblem.
		\[ 0 = \left(Y'-YX\right)(s) = \left(\frac{d}{ds}Y - YX\right)(s) = \underset{Y}{\underbrace{e^{xs}(v_0,v_1)}} \{E_2x-X\}\]
	bzw. (vgl. Abschnitt 3.1) mit dem zur transponierten Matrix $ X^t $ assoziierten Endomorphismus $ f_{X^t}\in \End(\R^2) $
		\[ f_{X^t}(v) = vx \text{ für }x\in\R \text{ und }v\in \R^2. \]
	Nach obigem Lemma sind die Eigenwerte Lösungen der Gleichung
		\[ 0 = \chi_{X^t}(x) = \chi_X(x) \overset{\text{Lemma}}{=} x^2+2x-3 = (x-1)(x+3). \]
	Also sind $ x_1 = 1 $ und $ x_2 = -3 $ die Eigenwerte; zugehörige Eigenvektoren erhält man als Lösungen der linearen Gleichungssysteme
		\[ (0,0) = (v_0,v_1)(E_2x_i-X)= (v_0,v_1)\begin{pmatrix}
		x_i&-3\\-1&x_i+2
		\end{pmatrix} = \begin{cases}
		(v_0,v_1)\begin{pmatrix}
		1&-3\\-1&3
		\end{pmatrix}& \text{ für } i = 1\\
		(v_0,v_1)\begin{pmatrix}
		-3&-3\\-1&-1
		\end{pmatrix}& \text{ für } i = 2
		\end{cases} \]
	Damit bekommt man Eigenvektoren $ (v_0,v_1) = (1,1) $ zum Eigenwert $ x=1 $ und $ (v_0,v_1) = (1,-3) $ zum Eigenwert $ x = -3 $.
	
	Die allgemeine, durch \emph{Superposition} (Linearkombination) erhaltene Lösung der Differentialgleichung ist also
		\[ s\mapsto Y(s) = e^s(1,1)c_1 + e^{-3s}(1,-3)c_2 \]
	mit Koeffizienten $ c_1,c_2 \in \R $. Abgleich der "`Integrationskonstanten"' $ c_1 $ und $ c_2 $ mit den Anfangsbedingungen liefert dann die Lösung
		\[ s \mapsto y(s) = 3e^s+e^{-3s}. \]
\paragraph{Bemerkung}
	Man bemerke: $ (y,y') $ ist linear unabhängig für die Lösung, ist also tatsächlich $ \frac{d}{ds} $-zyklische Basis eines 2-$ \dim $ URs $ [(y,y')]\subset C^\infty(\R) $ -- obwohl die den gleichen Raum aufspannenden "`Basislösungen"'
		\[ s\mapsto e^s \text{ und } s\mapsto e^{-3s} \]
	keine $ \frac{d}{ds} $-zyklischen Basen erzeugen, da sie lineare Differentialgleichungen erster Ordnung (mit konstanten Koeffizienten) lösen.
	
\subsection{Korollar}
\begin{Korollar}
	Besitzt $ V $ eine $ f $-zyklische Basis für $ f\in\End(V) $, so gilt $ \chi_f(f)=0 $.
\end{Korollar}
\paragraph{Beweis}
	Sei also $ B=(b_1,\dots,b_n) =(b,f(b),\dots,f^{n-1}(b)) $ $ f $-zyklische Basis von $ V $ und $ a_0,\dots,a_{n-1}\in K $ so, dass
		\[ 0 = f^n(b)+\sum_{k=0}^{n-1}f^k(b)a_k. \]
	Dann gilt
		\[ \chi_f(f)(b_1) = \chi_f(f)(b) = \left(f^n+\sum_{k=0}^{n-1}f^ka_k\right)(b) = f^n(b)+\sum_{k=0}^{n-1}f^k(b)a_k. \]
	Damit folgt für $ i=2,\dots,n $
		\[ \chi_f(f)(b_i) = \chi_f(f)\left(f^{i-1}(b)\right) \stackrel{\footnotemark}{=} f^{i-1}\left(\chi_f(f)(b) \right) = 0. \]
		\footnotetext{Aufgrund der Linearität der Endomorphismen $ \End(V) $ als unitäre Algebra.}
	Da also $ V=[B] \subset \ker {\chi_f(f)}$, folgt $ \chi_f(f) = 0. $
\paragraph{Bemerkung}
	Damit ist der Satz von Cayley-Hamilton bewiesen, sofern $ V $ eine $ f $-zyklische Basis besitzt.
	
\subsection{Lemma}
\begin{Lemma}[ $ f $-invarianter UVR endlicher Dimension besitzen eine $ f $-zyklische Basis]
	Für $ f\in\End(V) $ und $ v\in V^\times  $ sei
		\[ U := \left[\left(f^k(v)\right)_{k\in{\mathbb{N}}}\right]. \]
	Damit ist $ U $ ein $ f $-invarianter UVR von $ V $. Ist $ \dim V < \infty $, so besitzt $ U $ eine $ f $-zyklische Basis $ \left(v,f(v),\dots,f^{r-1}(v)\right) $.
\end{Lemma}
\paragraph{Beweis}
	Offenbar ist $ U $ $ f $-invarianter UR:
	\begin{itemize}
		\item $ U $ ist (als lineare Hülle einer Familie) ein UVR von $ V $;
		\item da gilt
			\[ \forall k\in \mathbb{N}: f\left(f^k(v)\right)=f^{k+1}(v)\in U \]
		folgt, dass
			\[f(U) = f\left(\left[\left(f^k(v)\right)_{k\in{\mathbb{N}}}\right]\right) = \left[\left(f^{k+1}(v)\right)_{k\in{\mathbb{N}}}\right]\subset U. \]
	\end{itemize}
	Ist $ \dim V < \infty $ und $ v\neq 0 $, so existiert $ r\in \mathbb{N} $, sodass
		\[ \left(v,\dots,f^{r-1}(v)\right) \text{ linear unabhängig und }f^r(v)\in\left[\left(v,\dots,f^{r-1}(v)\right)\right]; \]
	damit ist $ \left(v,f(v),\dots, f^{r-1}(v)\right) $ $ f $-zyklische Basis von $ U $:
	\begin{enumerate}
		\item $ \left(v,\dots,f^{r-1}(v) \right) $ ist linear unabhängig.
		\item $ f^r(v) \in \left[\left(v,\dots,f^{r-1}(v)\right)\right]$, damit gilt
		\[ \forall k\in \mathbb{N}: k\geq r \Rightarrow f^k(v)\in \left[\left(v,\dots,f^{r-1}(v)\right)\right] \]
		wie man z.B. mit Induktion sehen kann: ist
			\[ f^{k-1}(v) = \sum_{j=0}^{r-1}f^j(v)x_j \in \left[\left(v,\dots,f^{r-1}(v)\right) \right], \]
		so folgt
			\[ f^k(v) = \sum_{j=1}^{r}f^j(v)x_{j-1}=f^{r}(v)x_{r-1}+\sum_{j=1}^{r-1}f^j(v)x_{j-1}\in \left[\left(v,\dots,f^{r-1}(v)\right)\right] \]
		und damit
			\[ U = \left[\left(f^k(v)\right)_{k\in \mathbb{N}}\right]\subset \left[\left(v,\dots,f^{r-1}(v)\right)\right]. \]
	\end{enumerate}
		
\subsection{Beweis vom Satz von Cayley-Hamilton}
	Zu zeigen: für $ f\in \End(V) $ gilt $ \chi_f(f)=0 $, d.h.
		\[ \forall v\in V:\chi_f(f)(v) = 0. \]
	Seien also $ v\in V^\times $ und
		\[ U := \left[\left( f^k(v)_{k\in \mathbb{N}}\right)\right]\subset V. \]
	Mit einem zu $ U $ komplementären UVR $ U'\subset V $, $ V=U\oplus U' $, und den zugehörigen Projektionen
	
	\begin{multicols}{2}
	%------------------ Projektion ----------------
 	\begin{figure}[H]\centering
 		\tdplotsetmaincoords{0}{0} %-27
 		\begin{tikzpicture}[xscale=0.5,yscale=0.5,tdplot_main_coords]

 				\def\xstart{0} %x Koordinate der Startposition der Grafik
 				\def\ystart{-3} %y Koordinate der Startposition der Grafik
 				\def\myscale{0.5} %ändert die Größe der Grafik (Skalierung der Grafik)
                \def\myscalex{1.0}
                \def\myscaley{0.6}
                \def\maxlh{6.0}
                \def\maxlv{6.0}
                
 				\def\xstartdraw{(\xstart + \maxlh)} %xKoordinate des Referenzstartpunktes (in dieser Zeichnung: a)
 				\def\ystartdraw{(\ystart + \maxlv)}%yKoordinate des Referenzstartpunktes (in dieser Zeichnung: a)

 			    \node (pointro) at ({\xstartdraw+(\maxlh*\myscalex)},{\ystartdraw+(\maxlv*\myscaley)}) {};
 			    
			    \node (pointlu) at ({\xstartdraw-(\maxlh*\myscalex)},{\ystartdraw-(\maxlv*\myscaley)}) {};
			    \node (pointlo) at ({\xstartdraw-(\maxlh*\myscalex)},{\ystartdraw+(\maxlv*\myscaley)}) {};
                \node (pointru) at ({\xstartdraw+(\maxlh*\myscalex)},{\ystartdraw-(\maxlv*\myscaley)}) {};
                
 				%\node (pointo1) at ($(pointol)!0.2!(pointor)$) {};
 				%\node (pointo2) at ($(pointol)!0.9!(pointor)$) {};

 				\node (offsetx) at ({(3.0*\myscalex},{0.0}) {}; %just an offset
 				\node (offsety) at ({0.0},{3.0*\myscaley}) {}; %just an offset
                
                \node (unity) at ({\maxlh*0.25*\myscalex},{\maxlv*0.25*\myscaley}) {}; %just an offset
                \node (unitx) at ({\maxlh*0.25*\myscalex},{-\maxlv*0.25*\myscaley}) {}; %just an offset
                
               	\node (point00) at ($(pointlo) + 4.0*(unitx) - 1.0*(unity)$) {};
 				%\draw[fill,color=blue] (point00) circle [radius=0.18];
 				
                 
 				%Koordinatenkreuz
 				\draw[-,line width=0.2pt,color=black] ($(point00) -3.0*(unity)$) -- ($(point00) +5.0*(unity)$);
 				\draw[-,line width=0.2pt,color=black,shorten >=-20pt] ($(point00) -3.0*(unitx)$) -- ($(point00) + 3.0*(unitx)$);
 				
 				%Q2 dotted vertikal
 				\draw[-,line width=0.4pt,color=red,dotted] ($(point00) -2.0*(unitx) -2.0*(unity)$ + ) -- ($(point00) -2.0*(unitx) +3.0*(unity)$);
 				\draw[-,line width=0.4pt,color=red,dotted] ($(point00) -1.0*(unitx) -2.0*(unity)$ + ) -- ($(point00) -1.0*(unitx) +4.0*(unity)$);
 				
 				%Q1 dotted vertikal
 				\draw[-,line width=0.4pt,color=red,dotted] ($(point00) + 2.0*(unitx) -2.0*(unity)$ + ) -- ($(point00) + 2.0*(unitx) +5.0*(unity)$);
 				\draw[-,line width=0.4pt,color=red,dotted] ($(point00) +1.0*(unitx) -2.0*(unity)$ + ) -- ($(point00) +1.0*(unitx) +5.0*(unity)$);
 				
 				%Q1 und Q2 dotted horizontal
 				\draw[-,line width=0.4pt,color=red,dotted] ($(point00) - 1.0*(unitx) +4.0*(unity)$ + ) -- ($(point00) + 3.0*(unitx) +4.0*(unity)$);
 				\draw[-,line width=0.4pt,color=red,dotted] ($(point00) - 2.0*(unitx) +3.0*(unity)$ + ) -- ($(point00) + 4.0*(unitx) +3.0*(unity)$);
 				\draw[-,line width=0.4pt,color=red,dotted] ($(point00) - 3.0*(unitx) +2.0*(unity)$ + ) -- ($(point00) + 4.0*(unitx) +2.0*(unity)$);
 				\draw[-,line width=0.4pt,color=red,dotted] ($(point00) - 3.0*(unitx) +1.0*(unity)$ + ) -- ($(point00) + 4.0*(unitx) +1.0*(unity)$);	
 				\draw[-,line width=0.4pt,color=red,dotted] ($(point00) - 3.0*(unitx) -1.0*(unity)$ + ) -- ($(point00) + 3.0*(unitx) -1.0*(unity)$);	
 				
 				%Vektoren blau
 				\node (point03) at ($(point00) +3.0*(unity)$) {};
               	\node (point20) at ($(point00) +2.0*(unitx)$) {};
               	\node (point23) at ($(point00) +2.0*(unitx) +3.0*(unity) $){};
 				
 				\draw[-{>[scale=1,length=8,width=6]},shorten >=-5pt,line width=0.5pt,color=red] (point00) -- (point03);
 				\draw[-{>[scale=1,length=8,width=6]},shorten >=-5pt,line width=0.5pt,color=red] (point00) -- (point20);
 				\draw[-{>[scale=1,length=8,width=6]},shorten >=-5pt, shorten <=-5pt,line width=0.9pt,color=blue] (point00) -- (point23);
 				
 			    \node (pointvekw) at (point23) [above,color=blue]{$v$};
 			    \node (pointvekw) at (point20) [xshift=-20,color=red]{$p(v)$};
 			    \node (pointvekw) at (point03) [yshift= 10,xshift=-5,color=red]{$p'(v)$};
 			    \node (pointvekw) at ($(point00) -3.0*(unitx)$) [yshift= 8,color=green]{$U$};
 			    \node (pointvekw) at ($(point00) +5.0*(unity)$) [yshift=-5,xshift= 5,color=green]{$U'$};
 			\end{tikzpicture}
	\end{figure}
	%------------------ Projektion ----------------
	    \begin{align*}
		    &p: V\to V, p(V) = U, \ker p = U' \text{ bzw. } \\
		    &p':V\to V, p'(V) = U', \ker p' = U,
	    \end{align*}
	\end{multicols}
	ist dann $ \chi_f(t) = \chi_{f'}(t)\cdot \chi_{f\mid_U}(t) \text{ mit } f' := p'\circ f\mid_{U'}\in \End(U') $.
	
	Damit folgt
		\[ \chi_f(f)(v) = \chi_{f'}(f) \left(\chi_{f\mid_U}(f)(v) \right) = \chi_{f'}(f)(0)=0 \]
	nach Korollar oben, da $ U $ eine $ f $-zyklische Basis besitzt und $ v\in U $.
	
% % VO-12-04-2016 % % 
\subsection{Definition}\index{Annulatorpolynom}\index{Minimalpolynom}
\begin{Definition}[Annulatorpolynom, Minimalpolynom]
	Sei $ V $ ein $ K $-VR und $ f\in\End(V) $. Dann heißt $ p\in K[t] $
		\begin{itemize}
			\item \emph{Annulatorpolynom von $ f $}, falls $ p(f)=0 $;
			\item \emph{Minimalpolynom von $ f $}, falls $ p(t) $ normiertes Annulatorpolynom minimalen Grades ist.
		\end{itemize}
\end{Definition}
\paragraph{Bemerkung}
	Jedes (polynomiale) Vielfache 
		\[ p(t) = q(t)\mu_f(t)\in K[t] \]
	eines Minimalpolynoms $ \mu_f(t) $ von $ f $ ist ein Annulatorpolynom, da
		\[ \forall v\in V: p(f)(v) = \left(q(f)\circ \mu_f(f)\right)(v) = q(f)\left(\mu_f(f)(v)\right) = q(f)(0) = 0 \]
\paragraph{Bemerkung}
	Nach dem Satz von Cayley-Hamilton hat jeder Endomorphismus $ f\in\End(f) $ ein Annulatorpolynom, also auch ein Minimalpolynom -- wenn $ \dim V < \infty $.
	
\subsection{Lemma}
\begin{Lemma}[Jedes Minimalpolynom ist Teiler des Annulatorpolynom von $ f\in\End(V) $]
	Ist $ p(t)\in K[t] $ Annulatorpolynom von $ f\in\End(V) $, so ist jedes Minimalpolynom $ \mu_f(t)\in K[t] $ Teiler von $ p(t) $. 
\end{Lemma}

\paragraph{Beweis}
	Seien $ q(t),r(t)\in K[t] $ die (nach dem euklidischen Divisionsalgorithmus) eindeutigen Polynome mit
		\[ p(t) = q(t)\mu_f(t)+r(t) \text{ und }\deg r(t)<\deg \mu_f(t). \]
	Dies liefert
		\[ r(f) = p(f)-q(f)\circ \mu_f(f) = 0-q(f)(0) = 0, \]
	also $ r(t) = 0 $, denn andernfalls wäre $ \mu_f(t) $ nicht normiertes Annulatorpolynom minimalen Grades.
	
\subsection{Korollar}
\begin{Korollar}
	Das Minimalpolynom $ \mu_f(t)\in K[t] $ eines Endomorphismus $ f\in \End(V) $ ist eindeutig.
\end{Korollar}
\paragraph{Beweis}
	Sind $ \mu_f(t),\tilde{\mu}_f(t)\in K[t] $ Minimalpolynome von $ f\in \End(V) $, so gilt
		\[ \exists! q(t)\in K[t] : \tilde{\mu}_f(t) = q(t)\mu_f(t) \] % Nach Lemma 4.5.7
	wobei
		\begin{itemize}
			\item $ \deg q(t) = 0 $, da $ \deg \tilde{\mu}_f(t) \leq \deg \mu_f(t) $,
			\item $ q(t) = 1$, da $ \tilde{\mu}_f(t) $ und $ \mu_f(t) $ normiert sind.
		\end{itemize}
	Daher ist
		\[ \tilde{\mu}_f(t) = 1\cdot \mu_f(t) = \mu_f(t). \]
\paragraph{Bemerkung}
	Wie für Endomorphismen kann man Annulatorpolynome, Minimalpolynome, usw. auch für Matrizen $ X\in K^{n\times n} $ definieren:
		\begin{itemize}
			\item mithilfe der assoziierten Endomorphismen $ f_X\in \End(K^n) $, oder 
			\item mithilfe des Einsetzungshomomorphismus $ \psi_X: K[t] \to K^{n\times n}. $ % in der Algebra der quadratischen Matrizen
		\end{itemize}
	Beide Methoden liefern das gleiche Ergebnis durch den Algebrahomomorphismus zwischen den Endomorphismen und den quadratischen Matrizen.

\paragraph{Bemerkung \& Beispiel}
	Zerfällt das charakteristische Polynom in Linearfaktoren, so zerfällt auch das Minimalpolynom in dieselben Linearfaktoren:
		\[ \chi_f(t)= \prod_{i=1}^{m}(t-x_i)^{k_i} \Rightarrow \mu_f(t) = \prod_{i=1}^{m}(t-x_i)^{m_i}, \]
	wobei für $ i= 1,\dots,m $ gilt $ 1\leq m_i\leq k_i $.
	
	Zum Beispiel: 
		\begin{itemize}
			\item $ X = \begin{pmatrix}
			1&0\\0&0
			\end{pmatrix} $: $ \chi_{f_X} = t(t-1) = \mu_{f_X}(t)$
			\item $ X = \begin{pmatrix}
			1&0\\0&1
			\end{pmatrix} $: $ \chi_{f_X} = (t-1)^2 \Rightarrow \mu_{f_X}(t) = (t-1) $
			\item $ X = \begin{pmatrix}
			1&1\\0&1
			\end{pmatrix} $: $ \chi_{f_X} = (t-1)^2 = \mu_{f_X}(t)$.
		\end{itemize}

\paragraph{Bemerkung}
	Die Definition des charakteristischen Polynoms ist etwas problematisch:
		\[ \chi_f(t) := \det (\id_Vt-f) \]
	ist "`gut"' für Polynomfunktionen, aber "`nicht korrekt"' für abstrakte Polynome; die Definition 
		\[ \chi_f(t) := \sum_{\sigma\in S_n}\sgn(\sigma)\prod_{i=1}^{n}\left(\delta_{\sigma(j)j}-x_{\sigma(j)j} \right)\in K[t] \]
	mithilfe der Darstellungsmatrix $ X = (x_{ij})_{i,j\in \{1,\dots,n\}} = \xi_B^B(f) $
	von $ f $ bzgl. einer Basis $ B $ und der Leibniz-Formel ist nicht sehr übersichtlich. Vergleiche auch [Axler, Kap. 8] zum Thema.
	
	Im Gegensatz dazu: Definitionen von "`Annulatorpolynom"' und "`Minimalpolynom"' etc. sind einfach (konzeptionell).
	
	Frage: Braucht man das charakteristische Polynom überhaupt?
	Man kommt auch ohne das charakteristische Polynom "`recht weit"':
		\begin{itemize}
			\item Für $ \dim V <\infty $ folgt die Existenz eines Annulatorpolynoms, und damit des Minimalpolynoms recht einfach wegen $ \dim \End(V) <\infty $.
			\item Durch Einsetzen: Jeder Eigenwert eines Endomorphismus ist Nullstelle seines Minimalpolynoms.
			\item Umgekehrt ist auch jede Nullstelle des Minimalpolynoms Eigenwert -- ist $ \mu_f(x) = 0 $, so existiert $ q(t)\in K[t] $ mit
				\[ \mu_f(t) = q(t)(t-x); \]
			wäre $ x $ kein Eigenwert, also $ f-\id_V x \in Gl(V) $, so gälte
				\[ (f-\id_Vx)(V) = V \Rightarrow \{0\} = \mu_f(f)(V) = q(f)(V), \]
			d.h. $ \mu_f(t) $ wäre nicht Minimal-Polynom.
			\item Ein Endomorphismus ist diagonalisierbar, wenn sein Minimal-Polynom in paarweise verschiedene Linearfaktoren zerfällt.
		\end{itemize}
	
	Nachteil des Minimal-Polynoms: schwierig berechenbar?

% VO 14-04-2016 %
\chapter{Längen- und Winkelmessung}
Plan: 
	Längen und Winkel (in "`Punkträumen"' $ \cong $ affinen Räumen) verstehen.
	
Algebraisch:
	via Produkte (bilineare -- oder fast bilineare -- Abbildungen).
	
%TODO schönere Grafik...; dann auslagern.
	\definecolor{qqwuqq}{rgb}{0.,0.39215686274509803,0.}
	\definecolor{qqqqff}{rgb}{0.,0.,1.}
	\begin{tikzpicture}[line cap=round,line join=round,>=triangle 45,scale=1.8]
	\clip(0,0) rectangle (10,3);
	\draw [shift={(5.635,1.07)},color=qqwuqq,fill=qqwuqq,fill opacity=0.1] (0,0) -- (34.85:0.14) arc (34.85:143.6:0.14) -- cycle;
	\draw [->] (2.58,1) -- (0.56,2.3);
	\draw [->] (6.6,0.35) -- (4.25,2.1);
	\draw [->] (4.5,0.26) -- (6.8,1.9);

	\draw [fill=qqqqff] (2.58,1) circle (1pt);
	\draw[color=qqqqff] (2.6,1) node[below] {$A$};
	\draw [fill=qqqqff] (0.56,2.3) circle (1pt);
	\draw[color=blue] (0.5,2.3) node[above] {$B$};
	\draw (1.5,0.1) node {Abstand a bis b $ \cong $ Länge b-a};
	\draw (5.4,0.1) node {Winkel $ \cong $ Winkel zwischen Richtungsvektoren};
	\end{tikzpicture}

\section{Bilinearformen \& Sesquilinearformen}
\paragraph{Zur Erinnerung}
	Sind $ V $ und  $W$ $ K $-VR, so nennt man eine Abbildung
		\[ \beta: V\times V\to W \]
	\emph{bilinear} oder ein \emph{Produkt}, wenn sie in jedem Argument linear ist:
		\begin{enumerate}[(i)]
			\item $ \forall w\in V :V\ni v \mapsto \beta(v,w)\in W $ ist linear;
			\item $ \forall v\in V: V\ni w\mapsto \beta(v,w)\in W $ ist linear.
		\end{enumerate}
	Zu vorgegebenen Werten $ \beta_{ij} \in W$ auf einer Basis $ (b_i)_{i\in I} $ von $ V $ existiert dann eine eindeutige Bilinearform $ \beta $ (Fortsetzungssatz Abschnitt 4.3):
		\[ \exists! \beta:V\times V\to W \text{ bilinear}: \forall i,j\in I: \beta(b_i,b_j) = \beta_{ij}. \]
\paragraph{Bemerkung}
	Man kann auch bilineare Abbildungen $ V\times V'\to W $ betrachten und, zum Beispiel, auch einen Fortsetzungssatz beweisen.
	
	Wir benötigen eine Verallgemeinerung in eine andere Richtung:
\subsection{Definition} \index{Sesquilinearform}\index{Semilinearität}
\begin{Definition}[Sesquilinearform]
Seien $ V $ ein $ K $-VR und $ K\ni x\mapsto \overline{x}\in K $ ein (Körper-) Automorphismus, d.h. eine bijektive Abbildung mit
		\[ \overline{x+y} = \overline{x}+\overline{y} \text{ und } \overline{xy} = \overline{x}\cdot \overline{y} \]
	für alle $ x,y\in K $. Eine Abbildung $ \sigma: V\times V \to K $ heißt dann \emph{Sesquilinearform} (bzgl. $ \bar{.} $), falls
		\begin{enumerate}[(i)]
			\item $ \forall v\in V: V\ni w \mapsto \sigma(v,w)\in K $ ist linear, d.h. $ \sigma(v,.)\in V^* $;
			\item $ \forall w\in V: V\ni v \mapsto \sigma(v,w)\in K $ ist \emph{semilinear}, d.h.
				\begin{enumerate}[(a)]
					\item $ \forall v,v' \in V: \sigma(v+v',w) = \sigma(v,w)+\sigma(v',w) $ und
					\item $ \forall v\in V\forall x\in K: \sigma(vx,w) = \overline{x}\sigma(v,w) $.
				\end{enumerate}
		\end{enumerate}
\end{Definition}

\paragraph{Beispiel}
	Die Identität $ K\ni x\mapsto \overline{x}:= x\in K $ ist offensichtlich ein Körperautomorphismus für jeden Körper $ K $. \emph{Bilinearformen} sind genau die Sesquilinearformen bezüglich $ \id_K $.
\paragraph{Beispiel}
	Für $ K = \mathbb{C} $ liefert \emph{komplexe Konjugation} einen Körperautomorphismus (keinen VR-Automorphismus, vgl. Abschnitt 1.4):
		\[ \mathbb{C}\ni x+iy \mapsto \overline{x+iy}:= x-iy \in \mathbb{C}. \]
	Dieses Beispiel ist unser Grund für die Einführung des Begriffs der Sesquilinearform.
\paragraph{Bemerkung}
	Ist $ \sigma $ Bilinearform und Sesquilinearform bezüglich $ \bar{.} $, so ist $ \sigma $ oder $ \bar{.} $ trivial:
		\[ \forall x\in K\forall v,w\in V: 0 = \sigma(vx,w) - \sigma(vx,w) = (x-\overline{x})\sigma(v,w)  \]
		\[ \Rightarrow \begin{cases}
		\forall v,w\in V: \sigma(v,w) = 0 \text{ oder}\\
		\exists v,w\in V: \sigma(v,w)\neq 0 \land \forall x\in K: \overline{x} = x.
		\end{cases} \]
\paragraph{Bemerkung}
	In $ \mathbb{Z}_p, \mathbb{Q} $ und $ \mathbb{R} $ gibt es nur \emph{einen} Körperautomorphismus: $ \id_K $. Ein Automorphismus $ \bar{.} $ von $ \mathbb{C} $ mit $ \overline{\mathbb{R}} = \mathbb{R} $ ist trivial, $ \bar{.} = \id_\mathbb{C} $ oder die komplexe Konjugation.
	
\subsection{Fortsetzungssatz für Sesquilinearformen}
\begin{Satz}[Fortsetzungssatz für Sesquilinearformen]
	Sind $ V $ ein $ K $-VR und $ K\ni x\mapsto \overline{x}\in K $ ein Körperautomorphismus, $ (b_i)_{i\in I} $ Basis von $ V $ und $ (s_{ij})_{i,j\in I} $ eine Familie in $ K $, so existiert eine eindeutige Sesquilinearform $ \sigma $ mit
		\[ \forall i,j\in I:\sigma(b_i,b_j) = s_{ij}. \]
\end{Satz}

% VO 19-04-2016 % 
\paragraph{Beweis}
	Wir imitieren den Beweis unseres ersten Fortsetzungssatzes für lineare Abbildungen:
	
	{Eindeutigkeit:}
	Sei $ \sigma $ eine Sesquilinearform mit der gewünschten Eigenschaft oben; gilt
		\[ v = \sum_{i\in I}b_ix_i \text{ und }w = \sum_{i\in I}b_i y_i \]
	so folgt
		\[ \sigma(v,w) = \sum_{i,j\in I}\overline{x_i}\sigma(b_i,b_j)y_j = \sum_{i,j\in I}\overline{x_i}s_{ij}y_j \]
	d.h. $ \sigma $ ist durch die Familie $ (s_{ij})_{i,j\in I} $ eindeutig bestimmt.
	
	{Existenz:}
	Da jeder Vektor $ v\in V $ eine eindeutige Basisdarstellung $ v=\sum_{i\in I}b_ix_i $ hat, wird durch
	\[ \sigma:V\times V \to K, (v,w)= \left(\sum_{i\in I}b_ix_i, \sum_{j\in I}b_jy_j\right) \]
	\[ \mapsto \sigma(v,w) := \sum_{i,j\in I}\overline{x_i}s_{ij}y_j \]
	eine Abbildung wohldefiniert. Offenbar (nachrechnen) ist $ \sigma $ dann sesquilinear. 

\paragraph{Bemerkung}
	Jede Sesquilinearform $ \sigma: V\times V\to K $ liefert eine semi-lineare Abbildung
		\[ V\ni v\mapsto \sigma(v,.)\in V^*. \]
	Mit einem "`Fortsetzungssatz für semi-lineare Abbildungen"' (Aufgabe 34) hätte man auch den früher skizzierten Beweis für bilineare Abbildungen imitieren können.

\subsection{Buchhaltung}\index{Gramsche Matrix}
\paragraph{Gramsche Matrix}
	Ist $ n=\dim V < \infty $ und $ B=(b_1,\dots,b_n) $ Basis von $ V $, so kann man eine Sesquilinearform $ \sigma: V\times V\to K $ durch eine Matrix $ S $ beschreiben:
	\[ \begin{array}{c|ccc}
	\sigma & b_1 & \dots & b_n \\ \hline
	b_1 & s_{11} &  & s_{1n} \\ 
	\vdots &  & \ddots &  \\ 
	b_n & s_{n1} &  & s_{nn}
	\end{array}  \]
	Diese Matrix
		\[ \Gamma_B(\sigma) = S = \left(\sigma(b_i,b_j)\right)_{i,j\in \{1,\dots,n\}} \]
	heißt die Darstellungsmatrix oder \emph{Gramsche Matrix} von $ \sigma $ bezüglich $ B $. Für Vektoren
		\[ v = \sum_{i=1}^{n}b_ix_i = BX \text{ und } w = \sum_{j=1}^{n}b_jy_j = BY \]
	ist dann
		\[ \sigma(v,w) = \sum_{i,j=1}^{n}\overline{x_i}s_{ij}y_j = \overline{X}^tSY \]
		\[ = (\overline{x_1},\dots,\overline{x_n})\begin{pmatrix}
		\sum_{i=1}^{n}s_{1j}y_j\\ \vdots\\ \sum_{j=1}^{n}s_{nj}y_j
		\end{pmatrix} = \sum_{i=1}^{n}\overline{x_i}\sum_{j=1}^{n}s_{ij}y_j. \]
\paragraph{Transformationsformel}
	Ein Basiswechsel $ B' = BP $ mit $ P = \xi_{B'}^B \in Gl(n)$ liefert dann
		\[ v = BX = (B'P^{-1})X = B'\underset{X'}{\underbrace{(P^{-1}X)}} \text{ und } w = B'\underset{Y'}{\underbrace{(P^{-1}Y)}} \]
	und damit für $ X,Y \in K^{n\times 1} $
		\[ \overline{X}^tSY = \overline{X'}^t\underset{S'}{\underbrace{(\overline{P}^tSP)}}Y' \]
	woraus die \emph{Transformationsformel für Gramsche Matrizen} folgt
		\[ S' = \overline{P}^tSP, \]
	wobei $ \overline{P}^t $ die Transponierte der Matrix mit Einträgen $ \overline{p_{ij}} $ ist.
\paragraph{Äquivalenz von Matrizen}
Dies liefert einen weiteren Äquivalenzbegriff für quadratische Matrizen $ S\in K^{n\times n} $:
		\[ S' \sim S :\Leftrightarrow \exists  P\in Gl(n): S' = \overline{P}^tSP. \]
	Die verschiedenen Begriffe der Äquivalenz von Matrizen (vgl. 3.1 \& 4.2) spiegeln die verschiedenen Funktionen/Bedeutungen von Matrizen wider.
	
\paragraph{Bemerkung}
	Die Menge der Sesquilinearformen auf einem $ K $-VR ist selbst ein $ K $-VR. Ist $ n=\dim V< \infty $ und $ B $ Basis von $ V $, so erhält man (Fortsetzungssatz) einen Isomorphismus
		\[ K^{V\times V}\supset \{\sigma:V\times V\to K \text{ Sesquilinearform}\}\ni \sigma \mapsto \Gamma_B(\sigma)\in K^{n\times n}. \]
\subsection{Beispiel \& Definition} \index{Sesquilinearform!kanonische}\index{Sesquilinearform!assoziierte}
\begin{Definition}[assoziierte Sesquilinearform]
	Sei $ \bar{.}:K\to K $ Körperautomorphismus; jedes $ S\in K^{n\times n} $ liefert dann eine eindeutige Sesquilinearform
		\[ \sigma_S:K^n\times K^n \to K \text{ mit } (e_i,e_j)\mapsto \sigma_S(e_i,e_j):= s_{ij}, \] 
	die zu \emph{$ S $ assoziierte Sesquilinearform}.

	Für $ S = E_n $ bezeichnet man $ \sigma_S $ auch als \emph{kanonische Sesquilinearform}.
\end{Definition}

\subsection{Definition}\index{Sesquilinearform!(schief-)symmetrische}
\begin{Definition}[symmetrische, schiefsymmetrische und alternierende Sesquilinearformen]
	Eine Sesquilinearform $ \sigma:V\times V\to K $ auf einem $ K $-VR bzgl. eines Automorphismus $ \bar{.}:K\to K $ nennen wir
		\begin{enumerate}[(i)]
			\item \emph{symmetrisch}, falls
				\[ \forall v,w\in V: \sigma(w,v) = \overline{\sigma(v,w)} \]
			\item \emph{schiefsymmetrisch}, falls
				\[ \forall v,w\in V: \sigma(w,v) = - \overline{\sigma(v,w)} \]
			\item \emph{alternierend}, falls
				\[ \forall v \in V: \sigma(v,v) = 0. \]
		\end{enumerate}
\end{Definition}

\begin{Definition}[Hermitesche Sesquilinearform]\index{Sesquilinearform!Hermitesche}
	Falls $ K =\mathbb{C} $ und $ \bar{.} $ komplexe Konjugation sind, so nennt man eine symmetrische Sesquilinearform auch \emph{Hermitesche Sesquilinearform}.
\end{Definition}

\paragraph{Bemerkung}
	Ist $ \sigma $ nicht-trivial und (schief-)symmetrisch, so muss $ \bar{.} $ eine Involution sein.
	
	Nämlich: Wähle $ v,w\in V $ mit $ \sigma(v,w) = 1$; dann gilt
		\[ \forall x\in K: \overline{\overline{x}} = \overline{\sigma(vx,w)} = \pm \sigma(w,vx) = \overline{\overline{x}\sigma(v,w)} = \pm \sigma(w,v)x = \overline{\sigma(v,w)}x = x. \]

% VO 21-04-2016 %
	Ist $ \Char(K) \neq 2 $ und $ \bar{.}  $ Involution, so kann jede Sesquilinearform in einen symmetrischen und einen schiefsymmetrischen Anteil zerlegt werden:
		\[ \forall v,w\in V: \sigma(v,w) = \frac{1}{2}\left(\sigma(v,w)+\overline{\sigma(w,v)}\right) +\frac{1}{2}\left(\sigma(v,w)-\overline{\sigma(w,v)} \right).\]
\paragraph{Bemerkung}
	Ist $ \Char(K)\neq 2 $ und $ \bar{.} = \id_K $, so sind "`alternierend"' und "`schiefsymmetrisch"' äquivalent für eine Sesquilinearform $ \sigma $.
	
	Andererseits ist jede alternierende Sesquilinearform bilinear, d.h. $ \bar{.} = \id_K $ oder $ \sigma = 0 $.
	
\paragraph{Buchhaltung}
	Unter den folgenden Annahmen:
		\begin{itemize}
			\item $ \Char(K)\neq 2 $ und $ \bar{.} $ Involution;
			\item $ n=\dim V <\infty $ und $ B $ ist Basis von $ V $;
		\end{itemize}
	gilt für die Gramsche Matrix $ S = \Gamma_B(\sigma) $ einer Sesquilinearform $ \sigma $ auf $ V $:
		\begin{itemize}
			\item $ 0 = \overline{S}^t-S\Leftrightarrow \sigma $ symmetrisch;\footnote{bis auf Faktor 2: Gramsche Matrix des schiefsymmetrischen Anteils von $ \sigma $}
			\item $ 0 = S + \overline{S}^t \Leftrightarrow \sigma $ schiefsymmetrisch.
		\end{itemize}
	Nämlich:
		\[ \overline{S}^t = \begin{pmatrix}
		\overline{\sigma(b_1,b_1)} & \overline{\sigma(b_1,b_2)} &\cdots& \overline{\sigma(b_1,b_n)} \\ 
		\overline{\sigma(b_2,b_1)} &  & & \vdots \\ 
		\vdots &  & & \vdots \\ 
		\overline{\sigma(b_n,b_1)} & \cdots & & \overline{\sigma(b_n,b_n)}
		\end{pmatrix}^t =  \begin{pmatrix}
		\overline{\sigma(b_1,b_1)} & \overline{\sigma(b_2,b_1)} &\cdots& \overline{\sigma(b_n,b_1)} \\ 
		\overline{\sigma(b_1,b_2)} &  & & \vdots \\ 
		\vdots &  & & \vdots \\ 
		\overline{\sigma(b_1,b_n)} & \cdots & & \overline{\sigma(b_n,b_n)}
		\end{pmatrix} \]
		\[ S = \begin{pmatrix}
		\sigma(b_1,b_1) & \sigma(b_1,b_2) &\cdots& \sigma(b_1,b_n) \\ 
		\sigma(b_2,b_1) &  & & \vdots \\ 
		\vdots &  & & \vdots \\ 
		\sigma(b_n,b_1) & \cdots & & \sigma(b_n,b_n)
		\end{pmatrix} \]
		
\subsection{Definition} \index{Orthogonal!-raum}\index{Orthogonal}
\begin{Definition}[orthogonal, Orthogonalraum]
	Sei $ \sigma $ symmetrische Sesquilinearform auf einem Vektorraum $ V $. Zwei Vektoren $ v,w\in V $ heißen \emph{orthogonal} (bzgl. $ \sigma $),
		\[ w \perp v, \text{ falls } \sigma(v,w) = 0. \]
	Der \emph{Orthogonalraum} einer Menge $ \emptyset \neq S\subset V $ ist der UVR
		\[ S^\perp := \bigcap_{s\in S} \ker \underset{\in V^*}{\underbrace{\sigma(s,.)}}. \]
\end{Definition}
\paragraph{Bemerkung}
	Wegen der Symmetrie von $ \sigma $ ist die \emph{Orthogonalitätsrelation} symmetrisch,
		\[ w \perp v \Leftrightarrow v \perp w. \]
\paragraph{Bemerkung}
	Da $ \forall v\in V: \sigma(v,.) \in V^* $, ist der Orthogonalraum wohldefiniert und (als Schnitt von UVR) ein UVR. Offenbar gilt
		\[ \tilde{S} \subset S \Rightarrow \tilde{S}^\perp \supset S^\perp. \]
	Damit folgt direkt $ S^\perp \supset [S]^\perp $, sind andererseits $ w\in S^\perp $ und $ v\in [S] $, so gilt
		\[ v = \sum_{s\in S}sx_s \Rightarrow \sigma(v,w)= \sum_{s\in S}\overline{x_s}\sigma(s,w) = 0, \text{ da } \forall s\in S: w\perp s \]
	d.h. $ w\in S^\perp \Rightarrow w\in [S]^\perp. $ Insgesamt ist also
		\[ \forall S \subset V: [S]^\perp=S^\perp.\]
	Ähnlich zeigt man für jede Familie $ (U_i)_{i\in I} $ von UVR $ U_i\subset V $:
		\[ \left(\sum_{i\in I}U_i \right)^\perp= \bigcap_{i\in I} U_i^\perp. \]
\paragraph{Bemerkung \& Beispiel}
	Für $ S\subset V $ kann man $ S^{\perp\perp} = \left(S^\perp\right)^\perp $ betrachten; im Allgemeinen gilt
		\[ S\subset S^{\perp\perp} \text{ aber } S\neq S^{\perp\perp}. \]
	Ist etwa $ \sigma = 0 $, so ist $ S^\perp = V $ für jede Menge $ \emptyset \neq S\subsetneq V $; also ist
		\[ S^{\perp\perp} = V^\perp s = V \neq S. \]

\subsection{Definition}\index{Radikal!-raum}\index{Radikal!-frei}
\begin{Definition}[Radikal(-raum),radikalfrei,nicht-degeneriert,degeneriert]
$ V^\perp $ ist der \emph{Radikal(-raum)} eines VR mit symmetrischer Sesquilinearform $ \sigma $; ist $ V^\perp = \{0 \} $, so heißt $ \sigma $ \emph{radikalfrei} oder \emph{nicht-degeneriert}, andernfalls \emph{degeneriert}.
\end{Definition}
\paragraph{Beispiel}
	Betrachte $ V=\mathbb{R}^2 $ mit Standardbasis $ (e_1,e_2) $.
	
	Ist für eine symmetrische Sesquilinearform (Bilinearform) $ \sigma $ auf $ V $
		\[ \sigma(e_1,e_1) = 0, \sigma(e_1,e_2) = 1, \sigma(e_2,e_2) = 0 \]
	so ist $ \sigma $ nicht-degeneriert, $ V^\perp = \{0\} $, da
		\[ v = e_1x_1 + e_2x_2 \perp e_1,e_2 \Rightarrow
		\begin{cases}
		0 = \sigma(e_1,v) = x_2\\
		0 = \sigma(e_2,v) = x_1
		\end{cases} \]
		\[ \Rightarrow x_1 = x_2 = 0, \]
	also $ V^\perp = \{0\} $, d.h. $ \sigma $ ist nicht-degeneriert.
	
	% Grafik R^2
	
	Ist aber
		\[ \sigma(e_1,e_1) = 1, \sigma(e_1,e_2) = 1, \sigma(e_2,e_2) = 1, \]
	so ist $ V^\perp = [e_1-e_2] $, d.h. $ \sigma $ ist degeneriert.
	
	% Grafik R^2 mit UVR [e_1-e_2]-> Gerade y = -x

\subsection{Lemma}
\begin{Lemma}[]
	Ist $ U\subset V $ ein zum Radikal von $ (V,\sigma) $ komplementärer UVR, $ V = V^\perp \oplus U $, so ist
		\[ \sigma\big|_{U\times U}:U\times U \to K \]
	radikalfrei.
\end{Lemma}
\paragraph{Beweis}
	Sei $ u\in U $ im Radikal von $ (U,\sigma\big|_{U\times U}) $, d.h. es gelte $ \forall v\in U: \sigma(v,u) = 0 $.
	Weil
		\[ \forall v\in V^\perp\forall w\in V: v\perp w \Rightarrow \forall v\in V^\perp: v\perp u \]
	erhalten wir $ u\in U\cap V^\perp = \{0\} $.
\paragraph{Beispiel}
	Die Einschränkung von $ \sigma $ mit (wie oben)
		\[ \forall i,j \in \{1,2\}: \sigma(e_i,e_j) = 1 \]
	auf jeden UVR $ U = [e_1x_1 + e_2x_2] $ mit $ x_1 + x_2 \neq 0 $ ist radikalfrei, denn
			\[ \sigma(e_1x_1+e_2x_2, e_1x_1+e_2x_2) = (x_1+x_2)^2 \neq 0. \]
	

\printindex
\end{document}
