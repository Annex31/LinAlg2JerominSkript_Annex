\documentclass[a4paper,11pt, DIV=12, parskip=half]{scrreprt}
\usepackage{microtype}
\usepackage[utf8]{inputenc}
\usepackage[T1]{fontenc}
\usepackage{lmodern}%schoeneres Schriftbild
\usepackage[ngerman]{babel}%deutsche Silbentrennung
\usepackage[onehalfspacing]{setspace}
\usepackage{tikz}%Zeichnungen
\usetikzlibrary{calc,arrows.meta,intersections,positioning,calc}
\usepackage{tikz-3dplot}
\usepackage{amsmath,amsfonts,amssymb}%Mathematik-Pakete
\usepackage{graphicx}
\usepackage{enumerate}%enumerate mit roemischen Zahlen
\usepackage{float}%fuer H Positionierung
\usepackage{multicol}
\usepackage{hyperref} %Verlinktes Inhaltsverzeichnis
\usepackage{makeidx} %Stichwortverzeichnis
\makeindex

\author{Studierendenmitschrift}
\title{Skript Lineare Algebra \& Geometrie 2, Hertrich-Jeromin}

\setcounter{tocdepth}{1}

% Eigene Operatoren:
\let\hom\relax
\DeclareMathOperator{\Char}{Char}
\DeclareMathOperator{\End}{End}
\DeclareMathOperator{\Aut}{Aut}
\DeclareMathOperator{\Iso}{Iso}
\DeclareMathOperator{\hom}{Hom}
\DeclareMathOperator{\rg}{rg}
\DeclareMathOperator{\dfkt}{def}
\DeclareMathOperator{\id}{id}
\DeclareMathOperator{\sgn}{sgn}
\DeclareMathOperator{\vol}{vol}
\DeclareMathOperator{\ggT}{ggT} 
\DeclareMathOperator{\tr}{tr} 

\newcommand{\R}{\mathbb{R}}
\newcommand{\C}{\mathbb{C}}
\newcommand{\K}{\mathbb{K}}
\newcommand{\N}{\mathbb{N}}
\renewcommand{\Re}{\operatorname{Re}}
\newcommand{\Skl}[2]{\langle #1,#2 \rangle}
\newcommand{\SSkl}[2]{\langle\! \Skl{#1}{#2}\! \rangle}

\newenvironment{Satz}[1][]{}{\par\addvspace{\baselineskip}}
\newenvironment{Lemma}[1][]{}{\par\addvspace{\baselineskip}}
\newenvironment{Definition}[1][]{}{\par\addvspace{\baselineskip}}
\newenvironment{Korollar}[1][]{}{\par\addvspace{\baselineskip}}

\begin{document}
\maketitle
\tableofcontents
% Einbinden der Kapitel
%VO1-2016-03-01
\setcounter{chapter}{3}
\chapter{Volumenmessung}
\setcounter{section}{2}
\section{Polynome \& Polynomfunktionen}
	Warum? (Vielleicht eher "`Algebra"' -- allgemein -- als "`lineare"' Algebra) Wichtig: das charakteristische Polynom eines Endomorphismus -- wichtiges Hilfsmittel im Kontext der Struktursätze.
\paragraph{Beispiel}
	Wir definieren Polynomfunktionen $ p,q: K\to K $ eines Körpers $ K $ in sich durch 
		\begin{align*}
		p:\ & K\to K,\ x\mapsto p(x):= 1+x+x^2\\
		q:\ & K\to K,\ x\mapsto q(x):= 1
		\end{align*}
	Falls $ K=\mathbb{Z}_2 $ so gilt dann
		\begin{align*}
		&\forall x\in K: x(x+1)=0\\
		\Rightarrow\ &\forall x\in K: p(x) = q(x)
		\end{align*}
	d.h., unterschiedliche "`Polynome"' liefern die gleiche Polynomfunktion: Koeffizientenvergleich funktioniert nicht.
\paragraph{Wiederholung}
	Auf dem Folgenraum $ K^\mathbb{N} $ betrachten wir die Familie $ (e_k)_{k\in \mathbb{N}} $ mit
		\[ e_k :\mathbb{N}\to K,\ j\mapsto e_k(j):= \delta_{jk}; \]
	wir wissen: $ (e_k)_{k\in \mathbb{N}} $ ist linear unabhängig, aber kein Erzeugendensystem:
		\[ \forall k\in \mathbb{N}: e_k \notin [(e_j)_{j\neq k}] \text{ und }
		[(e_j)_{j\in\mathbb{N}}]\neq K^{\mathbb{N}}\]
	Insbesondere gilt:
		\[ \forall x\in [(e_j)_{j\in \mathbb{N}}]\ \exists n\in \mathbb{N}\ \forall k>n : x_k = 0 \]
\subsection{Idee \& Definition} \index{Polynom}\index{Cauchyprodukt}
	\begin{Definition}[Cauchyprodukt]
		Wir fassen ein Polynom als (endliche) Koeffizientenfolge auf,
		\[ \sum_{k=0}^{n} t^ka_k \cong
		\sum_{k\in \mathbb{N}}e_ka_k \text{ mit } a_k = 0 \text{ für } k>n \]
	und führen darauf das \emph{Cauchyprodukt} (vgl. Analysis) als Multiplikation ein:
		\[ (a_k)_{k\in \mathbb{N}} \odot (b_k)_{k\in \mathbb{N}} := (c_k)_{k\in \mathbb{N}} \]
	wobei
		\[ c_k := \sum_{j=0}^{k}a_jb_{k-j}. \]
	\end{Definition}
	Insbesondere gilt damit
		\begin{gather*}
		\forall j,k\in \mathbb{N}: e_j \odot e_k = e_{j+k}
		\Rightarrow \forall k\in \mathbb{N}:
			\begin{cases}
				e_0 \odot e_k = e_k\\
				e_1^k = \underset{k \text{ mal}}{\underbrace{e_1 \odot \cdots \odot e_1}} = e_k
			\end{cases}
		\end{gather*}
	Mit $ 1:= e_0,\ t:= e_1 $ und $ t^0 := 1 $, wie üblich, liefert dies:
		\[ \sum_{k=0}^{n}t^ka_k = \sum_{k\in \mathbb{N}}e_ka_k \in [(e_k)_{k\in N}]\subset K^\mathbb{N} \]
\subsection{Definition}\index{Polynom!-algebra}\index{Polynom!Grad}\index{Polynom!normiertes}
		\begin{Definition}[Polynomalgebra]
			\[ K[t] := ([(e_k)_{k\in \mathbb{N}}],\odot) ,\]
	mit dem Cauchyprodukt $ \odot $, ist die \emph{Polynomalgebra} über dem Körper $ K $; die Elemente von $ K[t] $,
		\[ p(t) = \sum_{k=0}^{n}t^ka_k = \sum_{k\in\mathbb{N}}e_ka_k, \]
	heißen \emph{Polynome in der Variablen} $ t:= e_1 $.
	Der \emph{Grad} eines Polynoms ist
		\[  \deg\sum_{k=0}^{n}t^ka_k := \max \{k\in \mathbb{N}\mid a_k \neq 0\}  \quad \left( \text{bzw. } \deg 0 := -\infty \right) \]
	Ist (der "`höchste"' Koeffizient) $ a_n = 1 $ für $ \deg p(t) = n $, so heißt das Polynom $ p(t) $ \emph{normiert}.
		\end{Definition}
\paragraph{Notation}
	Mit $t^k = e_{k}$, also $ K[t] = [(e_k)_{k\in N}] $
	wird das Cauchyprodukt auf $ K[t] $ eine "`normale"' Multiplikation, gefolgt von einer Sortierung nach den Potenzen der Variablen $ t $. Wir werden das $ \odot $ daher oft unterdrücken, und z.B. $ p(t)q(t) $ schreiben, anstelle von $ p(t) \odot q(t) $.
\paragraph{Bemerkung (Koeffizientenvergleich)}
	Mit dieser Definition von "`Polynom"' gilt
		\begin{align*}
			p(t)=\sum_{k=0}^{n}t^ka_k = 0
			\Rightarrow \forall k\in \mathbb{N}: a_k = 0,
		\end{align*}
	da $ (t^k)_{k\in \mathbb{N}} = (e_k)_{k\in \mathbb{N}}$ linear unabhängig ist.
	Koeffizientenvergleich funktioniert!
\paragraph{Bemerkung}
	Die Polynomalgebra $ K[t] $ über $ K $ ist eine assoziative und kommutative $ K $-Algebra, weiters ist $ K[t] $ unitär mit Einselement $ 1=e_0 $.
\subsection{Definition}\index{Algebra}
	\begin{Definition}[Algebra]
		Eine $ K $-Algebra ist ein $ K $-VR mit einer \emph{bilinearen Abbildung},
		\[ \odot: V\times V \to V,\ (v,w)\mapsto v\odot w, \]
	d.h. es gilt
		\begin{enumerate}[(i)]
			\item $ \forall w\in V:\ V\ni v\mapsto v\odot w\in V $ ist linear;
			\item $ \forall v\in V: V\ni w\mapsto v\odot w\in V $ ist linear.
		\end{enumerate}
	
	Eine $ K $-Algebra heißt
		\begin{itemize}
			\item unitär (mit Einselement 1), falls  \hfill$ \exists 1\in V\forall v\in V: 1\odot v = v\odot 1 = v; $
			\item assoziativ, falls \hfill$ \forall u,v,w\in V: (u\odot v)\odot w = u\odot (v\odot w); $
			\item kommutativ, falls \hfill$ \forall v,w,\in V: v\odot w = w\odot v $
		\end{itemize}
	\end{Definition}
\paragraph{Beispiel}
	$ \End(V) $ ist (mit Komposition) eine unitäre assoziative Algebra.
\paragraph{Bemerkung}
	In jeder Algebra $ (V,\odot) $ gilt:
		\[ \forall v\in V: 0\odot v = v\odot 0 = 0 \]
	da z.B. für $ v\in V $ gilt
		\[ v\odot 0 = v\odot (0+0) = v\odot 0 + v\odot 0 \ \Rightarrow\  0=v\odot 0\]
	Ist $ (V,\odot) $ unitär, so folgt $ [1]\in V $ wegen $ 1\odot 1 = 1 $
		\[ ([1], +\mid_{[1]\times [1]},\odot\mid_{[1]\times [1]} ) \cong K \]
	vermöge $ K\ni x \mapsto 1 \cdot x\in [1] $ (siehe Aufgabe 5).
\subsection{Definition}\index{Algebra!-Homomorphismus}
	\begin{Definition}[Algebra-Homomorphismus]
		Ein \emph{Algebra-Homomorphismus} zwischen $ K $-Algebren $ (V,\odot) $ und $ (W,*) $ ist eine lineare Abbildung $ \psi\in \hom(V,W) $, für die gilt:
		\[ \forall v,v' \in V: \psi(v\odot v')=\psi(v)*\psi(v') \]
	\end{Definition}
\paragraph{Bemerkung}
	$ \hom(V,W) $ wird oft auch für den (Vektor-)Raum der Algebra-Homomorphismen verwendet. In dieser LVA bedeutet $ \hom(V,W) $ immer VR-Homomorphismen, bei allen "`anderen"' Homomorphismen wird extra erwähnt, was gemeint ist.

%VO2-2016-03-03

\subsection{Einsetzungssatz \& Definitionen}
	\begin{Satz}[Einsetzungssatz]\index{Einsetzungshomomorphismus}\index{Polynom!funktion}
		Seien $ (V,\odot) $ eine unitäre assoziative Algebra und $ v\in V $. Dann ist
			\[ \psi_v: K[t]\to V,\ \sum_{k=0}^{n}t^ka_k = p(t)\mapsto \psi_v(p(t)) := \sum_{k=0}^{n}v^ka_k \]
		-- wobei $ v^0 = 1 $ sinnvoll ist, da die Algebra unitär ist -- ein Algebra-Homomorphismus; $ \psi_v $ heißt \emph{Einsetzungshomomorphismus}. 
			\[ p:V\to V,\ v\mapsto p(v) := \psi_v(p(t)) \]
		heißt die zu $ p(t)\in K[t] $ gehörige \emph{Polynomfunktion} auf $ V $.
	\end{Satz}
\paragraph{Bemerkung}
	Wie üblich: $ v^k := \underset{k-\text{mal}}{\underbrace{v\odot\cdots \odot v}} $ und $ v^0 := 1 $.
\paragraph{Beweis}
	\begin{enumerate}
		\item $ \psi_v $ ist linear:
			\begin{itemize}
				\item für $ p(t) = \sum_{k\in\mathbb{N}} t^ka_k $ und $ a\in K $ gilt:
					\[ \psi_v(p(t)a) = \psi_v(\sum_{k\in\mathbb{N}} t^ka_ka) = \sum_{k\in\mathbb{N}} v^ka_ka = \psi_v(p(t))a; \]
				\item für $ p(t) = \sum_{k\in\mathbb{N}} t^ka_k $ und $ q(t) = \sum_{k\in\mathbb{N}} t^k b_k $ gilt:
					\[ \psi_v(p(t)+q(t)) = \psi_v(\sum_{k\in\mathbb{N}}t^k(a_k+b_k)) = \sum_{k\in\mathbb{N}} v^k(a_k+b_k) = \psi_v(p(t))+\psi_v(q(t)) \]
			\end{itemize}
		\item $ \psi_v $ ist verträglich mit der Multiplikation:
		
			Für die Vektoren der Basis $ (t^k)_{k\in\mathbb{N}} $ von $ K[t] $ gilt, da $ (V,\odot) $ assoziativ ist,
				\[ \psi_v(t^mt^n) = \psi_v(t^{m+n}) = v^{m+n} = v^m\odot v^n = \psi_v(t^m)\odot \psi_v(t^n). \]
			Da aber $ \psi_v $ linear und die Multiplikation in $ K[t] $ und in $ (V,\odot) $ bilinear sind, folgt die Behauptung.
	\end{enumerate}
\paragraph{Bemerkung (Fortsetzungssatz für bilineare Abbildungen)}
	Im Beweis haben wir verwendet: Die Abbildungen
		\[ K[t]\times K[t]\to V,\ (p(t),q(t))\mapsto
			\begin{cases}
			\psi_v(p(t)q(t))& \text{ (Cauchyprodukt)}\\
			\psi_v(p(t))\odot \psi_v(q(t)) & \text{(Produkt in $ (V,\odot) $)}
			\end{cases} \]
	sind bilinear (da $ \psi_v $ linear ist), sind also gleich, sobald sie auf einer Basis übereinstimmen.
	Dies ist die Eindeutigkeit eines Fortsetzungssatzes für bilineare Abbildungen:
	
	Sind $ V,W\ K$-VR, $ (b_i)_{i\in I} $ eine Basis von $ V $ und $ (\beta_{ij})_{i,j\in I} $ eine Familie in $ W $, so gibt es eine eindeutige bilineare Abbildung
		\[ \beta: V\times V\to W \]
	mit
		\[ \forall i,j\in I: \beta(b_i,b_j) = \beta_{ij} \]
	Dieser Fortsetzungssatz folgt direkt aus dem Fortsetzungssatz für lineare Abbildungen, da
		\[ \{\beta:V\times V\to W \text{ bilinear}\} \cong \hom(V,\hom(V,W))\]
	vermittels des Isomorphismus
		\[ \beta \mapsto \big(v\mapsto\underset{\in \hom(V,W)}{\underbrace{\beta(v,.)}}\big), \]
	d.h. durch Nacheinandereinsetzen der Argumente.
\paragraph{Bemerkung}
	Die Abbildung eines Polynoms auf seine Polynomfunktion auf dem Körper,
		\[ K[t]\ni p(t)\mapsto (x\mapsto p(x))=\psi_x(p(t))\in K^K \]
	ist für $ \Char K\neq 0 $ nicht injektiv, das heißt: Koeffizientenvergleich kann nur funktionieren, wenn $ \Char K = 0 $
\paragraph{Beispiel \& Bemerkung}
	Ist $ V\ K $-VR, so ist $ \End(V) $ eine $ K $-Algebra (mit Komposition $ \circ $). Man erhält also für $ f\in \End(V) $ einen Einsetzungshomomorphismus
		\[ \psi_f: K[t]\to \End(V),\ p(t) \mapsto \psi_f(p(t)) = p(f); \]
	und für jedes Polynom $ p(t)\in K[t] $ eine zugehörige Polynomfunktion
		\[ p: \End(V)\to\End(V),\ f\mapsto \psi_f(p(t))= p(f). \]
	Dieses Beispiel ist der Schlüssel zum Satz von Cayley-Hamilton (im nächsten Abschnitt).
\subsection{Lemma}
	\begin{Lemma}
		Für Polynome $ p(t), q(t)\in K[t] $ gilt:
			\begin{itemize}
				\item $ \deg p(t)\odot q(t) = \deg p(t)+\deg q(t) $,
				\item $ \deg p(t)+q(t) \leq \max\{\deg p(t), \deg q(t)\} $.
			\end{itemize}
	\end{Lemma}
\paragraph{Beweis}
	Für $ p(t) = \sum_{k\in\mathbb{N}}t^ka_k $ und $ q(t) = \sum_{k\in\mathbb{N}}t^kb_k $ ist
		\[ p(t)\odot q(t) = \sum_{k\in\mathbb{N}}t^kc_k \text{ mit } c_k = \sum_{j=0}^{k}a_jb_{k-j} \]
	Gilt nun $ \deg p(t) = n $ und $ \deg q(t) = m $, d.h.
		\[ a_n,b_n \neq 0 \land \forall k>n, k'>m:a_k = b_{k'}=0 \] 
	so folgt
		\[ \left.
		\begin{aligned}
		\forall k>m+n : c_k = 0\ \\
		        c_{m+n} = a_nb_m\ \\
		\end{aligned}
		 \right\}
		\Rightarrow \deg p(t)\odot q(t) = m+n \]
	Gilt andererseits $ \deg p(t) = -\infty $ oder $ \deg q(t) = -\infty $, also $ p(t) = 0 \lor q(t) = 0 $,
	so folgt
		\[ p(t)\odot q(t) = 0 \Rightarrow \deg p(t)\odot q(t) = -\infty. \]
	Die zweite Behauptung ist offensichtlich wahr.

\section{Das charakteristische Polynom}
\subsection{Definition}\index{Eigenwert,-vektor,-raum}
\begin{Definition}[Eigenwert,Eigenvektor,Eigenraum]
	Seien $ V $ ein $ K $-VR und $ f\in\End(V) $. Dann heißen
		\begin{enumerate}[(i)]
			\item $ x\in K $ ein Eigenwert von $ f $, falls
				\[ \exists v\in V^\times: f(v)=vx; \]
			\item $ v\in V^\times $ ein Eigenvektor von $ f $, falls
				\[ \exists x\in K:f(v)=vx; \]
			\item $ \ker(f-\id_Vx) \subset V $ ein Eigenraum, falls
				\[ \ker(f-\id_Vx) \neq \{0\}.\]
		\end{enumerate}
	\end{Definition}
\paragraph{Bemerkung}
	Der Skalar $ x\in K $ ist genau dann ein Eigenwert von $ f\in \End(V) $, wenn $ \ker(f-\id_Vx)\neq \{0\} $, d.h., wenn ein Eigenvektor $ v\in V^\times $ zu $ x $ existiert.
\paragraph{Beispiel}
	Für $ \frac{d}{ds} \in \End(C^\infty(\mathbb{R}))$ ist jedes $ x\in \mathbb{R} $ ein Eigenwert, da
		\[ \Big(\frac{d}{ds}-\id_Vx\Big)v = 0 \text{ für } v:\mathbb{R}\to\mathbb{R},s\mapsto v(s):= e^{xs}, \]
	wobei $ v\in C^\infty(\mathbb{R})\setminus \{0\} $, d.h. $ s\mapsto v(s)=e^{xs} $ ist ein Eigenvektor zum Eigenwert $ x\in\mathbb{R} $.
\paragraph{Beispiel}
	Ist $ \dim V < \infty $, so kann die Determinante zur Bestimmung von Eigenwerten von Endomorphismen $ f\in\End(V) $ benutzt werden, da
		\[ \ker(f-\id_Vx)\neq \{0\} \Leftrightarrow (f-\id_Vx) \text{ nicht injektiv}\Leftrightarrow \det(f-\id_Vx) = 0, \]
	d.h. das Auffinden von Eigenwerten $ x\in K $ von $ f $ ist reduziert auf die Bestimmung der Nullstellen der Funktion
		\[ K\ni x\mapsto \det(f-\id_Vx)\in K. \]
		
\paragraph{Beispiel}	
	Ist z.B. $ (b_1,b_2) $ Basis von $ V $ und $ f\in \End(V) $ durch $ f(B)=BX $ gegeben, so liefern die Nullstellen der Polynomfunktion
		\begin{gather*}
		\det(f-\id_Vx) = \det(X-E_2 x)= \det \begin{pmatrix}
		x_{11}-x & x_{12}\\
		x_{21} & x_{22} -x
		\end{pmatrix}\\
	= (x_{11}-x)(x_{22}-x)-x_{12}x_{21}
	= x^2 - x(x_{11}+x_{22}) + (x_{11}x_{22}-x_{12}x_{21})
		\end{gather*}
	die Eigenwerte von $ f $ -- beispielsweise erhalten wir für
		\[ X = \begin{pmatrix} 2 &3\\1 & 0 \end{pmatrix}:\ 
			\det(f-\id_Vx) = x^2-2x-4 = (x+1)(x-3), \]
	also Eigenwerte $ x_1 = -1 $ und $ x_2 = 3 $ mit zugehörigen Eigenvektoren als Lösungen von
		\[ v_i \in \ker(f-\id_Vx_i), \]
	also durch Lösungen der linearen Gleichungssysteme
		\[ \begin{pmatrix}
		2-(-1) & 3\\ 1 & -(-1)
		\end{pmatrix}
		\begin{pmatrix}
		v_1^1\\v_1^2
		\end{pmatrix} = \begin{pmatrix}
		3 & 3\\ 1 & 1
		\end{pmatrix}
		\begin{pmatrix}
		v_1^1\\v_1^2
		\end{pmatrix} \text{ und} \]
		\[ \begin{pmatrix}
		2-3 & 3\\ 1 & -3
		\end{pmatrix}
		\begin{pmatrix}
		v_2^1\\v_2^2
		\end{pmatrix}=
		\begin{pmatrix}
		-1 & 3\\ 1 & -3
		\end{pmatrix}
		\begin{pmatrix}
		v_2^1\\v_2^2
		\end{pmatrix}  \]
	sodass
		\[ v_1 = b_1-b_2 \text{ und } v_2 = b_13+b_2 \]
	Eigenvektoren zu den Eigenwerten $ x_1,x_2 $ liefert.

\paragraph{Rechenbeispiel 1}
	Für $ X = \begin{pmatrix}2&-1\\1&0\end{pmatrix} $ erhält man
		\[ \det(f-\id_Vx) = \det\begin{pmatrix}2-x&-1\\1&-x	\end{pmatrix} =x^2-2x+1 \]
	und Eigenvektoren zum Eigenwert $ x = 1 $ durch Lösung der LGS
		\[ \begin{pmatrix}
		2-1&-1\\1&-1
		\end{pmatrix}\begin{pmatrix}
		v_1^1\\v_1^2
		\end{pmatrix} =  \begin{pmatrix}
		1&-1\\1&-1
		\end{pmatrix}\begin{pmatrix}
		v_1^1\\v_1^2
		\end{pmatrix} \]
	d.h. der Eigenraum zum Eigenwert $ x $,
		\[ \ker(f-\id_V) = [\{b_1+b_2\}] \]
	hat
		\[ \dim \ker(f-\id_V)<\dim V. \]
\paragraph{Rechenbeispiel 2}
	Ist $ K=\mathbb{R} $ und
		\[ \det(f-\id_Vx)=x^2+1, \]
	so hat $ f $ keine Eigenwerte: z.B., wenn
		$ X=\begin{pmatrix} 0&1\\-1&0 \end{pmatrix} $.
		
\subsection{Definition} \index{Charakteristisches Polynom}
\begin{Definition}[Charakteristisches Polynom]
	Sei $ V $ ein $ K $-VR, für $ f\in\End(V) $ ist das \emph{charakteristische Polynom} von $ f $:
		\[ \chi_f(t) := \det (\id_Vt-f)\in K[t]. \]
	Analog definiert man für $ X\in K^{n\times n} $ das charakteristische Polynom
		\[ \chi_f(t) := \det (E_nt-X)\in K[t]. \]
\end{Definition}
\paragraph{Bemerkung}
	Oft wird auch das andere Vorzeichen in der Determinante verwendet, also $ \det(f-\id_Vt) $ bzw. $ \det(X-E_nt) $.
\paragraph{Bemerkung}
	\emph{Diese Definition ist erklärungsbedürftig!}
	
	Da $ t\notin K $ ist $ \id_Vt-f\notin \End(V) $, sondern $ \id_Vt-f\in\End(V)[t] $. Zwei Lösungsstrategien bieten sich an:
		\begin{enumerate}
			\item Erweiterung der Determinante auf $ \End(V)[t] $.
			\item Benutzung von Darstellungsmatrizen.
		\end{enumerate}
	Beide führen schließlich zur Leibniz-Formel:
	
	Ist $ B $ eine Basis von $ V $ und $ \xi_B^B(f) = X = (x_{ij})_{i,j\in\{1,\dots,n\}}$, so erhält man 
		\[ \chi_f(t)=\sum_{\sigma\in S_n}\sgn(\sigma)\prod_{j=1}^{n}\underset{\in K[t]}{\underbrace{\left(\delta_{\sigma(j)j}-x_{\sigma(j)j}\right)}} \in K[t]. \]
	Die Unabhängigkeit von der Basis $ B $ folgt aus der Transformationsformel für Darstellungsmatrizen und dem Determinanten-Multiplikationssatz (wie vorher für $ \det f = \det \xi_B^B(f) $).

% VO 2016-03-15

\subsection{Bemerkung \& Definition}\index{Spur}
\begin{Definition}[Spur]
	Ist $ \dim V=n $, so ist $ \chi_f(t) $ ein normiertes Polynom vom Grad $ \deg\left(\chi_f(t)\right)=n $,
		\[ \chi_f(t)=t^n-t^{n-1}\tr f + \dots + (-1)^n\det f,\] % = \det(-f) = \chi_f(0)
	wobei die \emph{Spur} $ \tr f $ (\glqq tr \grqq $\widehat{=}$ trace) von $ f $ durch diese Gleichung (wohl-)defininiert ist.
\end{Definition}	
	Ist $ (x_{ij})_{i,j\in\{1,\dots,n\}} = X = \xi_B^B(f) $ Darstellungsmatrix von $ f $, so gilt
		\[ \tr f = \sum_{j=1}^{n}x_{jj} = \sum_{j=1}^{n} b_j^*f(b_j). \]
	Oft wird $ \det(f-\id_vt)=(-1)^n\chi_f(t) $ als charakteristisches Polynom definiert -- dieses Polynom ist dann nur für gerade $ n $ normiert.
\subsection{Korollar}
\begin{Korollar}[Eigenwerte sind Nullstellen des char. Polynoms]
	Ein $ x\in K $ ist genau dann Eigenwert von $ f $, wenn $ \chi_f(x)=0 $.
	
	Also: Die Eigenwerte von $ f $ sind genau die Nullstellen des charakteristischen Polynoms $ \chi_f(t) $.
\end{Korollar}
\paragraph{Beweis}
	Klar -- das war die Idee hinter der Definition des charakteristischen Polynoms.
\subsection{Korollar \& Definition}\index{Algebraische/geometrische Vielfachheit}
\begin{Korollar}[Eigenwert ist Nullstelle des charakteristischen Polynoms]
	Ist $ x\in K $ Eigenwert von $ f\in\End(V) $, so ist $ (t-x) $ Teiler des charakteristischen Polynoms. Insbesondere gilt:
		\[ \exists!k\in \mathbb{N}^\times:
			\begin{cases}
				(t-x)^k\mid \chi_f(t)\\
				(t-x)^{k+1}\nmid \chi_f(t)
			\end{cases} \]
\end{Korollar}
\begin{Definition}[algebraische Vielfachheit, geometrische Vielfachheit]
	Diese Zahl $ k $ heißt die \emph{algebraische Vielfachheit} von $ x $;
		\[ g:= \dfkt(\id_Vx-f) \leq k \]
	ist die \emph{geometrische Vielfachheit} von $ x $.
\end{Definition}
\paragraph{Beweis}
	Da $ x $ Eigenwert von $ f $ ist, ist die Existenz und Eindeutigkeit von $ k $ klar. Außerdem gilt analog auch $ g\geq 1 $.
	
	Zu zeigen bleibt: $ g\leq k $, d.h. $ (t-x)^g \mid \chi_f(t) $:
	
	Für eine Basis $ B = (b_1,\dots,b_n) $ von $ V $ mit
	$ \ker (\id_v x - f) = [(b_1,\dots,b_g)]$
	hat
		\[ \xi_B^B(f) =
		\begin{pmatrix}
			E_gx & Y\\
			0 & X
		\end{pmatrix}
		\text{ mit } Y\in K^{g\times (n-g)}, X\in K^{(n-g)\times(n-g)} \]
	Blockgestalt, also ist
		\[ \chi_f(t)=(t-x)^g\cdot \chi_X(t), \]
	d.h. $ (t-x)^g \mid \chi_f(t)$, da $ (t-x)^{k+1}\nmid \chi_f(t) $, gilt also $ g\leq k $.
\paragraph{Beispiel}
	Ist $ f\in\End(V) $ wie oben durch $ f(B)=BX $ gegeben, so haben die Eigenwerte
		\[ x_1 = -1 \text{ und } x_2 = 3 \text{ für }
		X=\begin{pmatrix} 2 &3\\1 & 0 \end{pmatrix} \]
	algebraische und geometrische Vielfachheiten 
		\[ 1 = g_i = k_i, \text{ da } 1\leq g_i \leq k_i \text{ und } k_1+k_2 \leq 2; \]
	der Eigenwert
		\[ x=1 \text{ für } X = \begin{pmatrix} 2&-1\\1&0 \end{pmatrix} \]
	hat algebraische und geometrische Vielfachheiten
		\[ k = 2 \text{ und } g = 1 \]
	da
		\[ f\neq \id_V x = \id_V \]
	und $ \chi_f(t)=(t-x)^2 \in \mathbb{R}[t] $, da ein quadratisches Polynom zwei (relle oder komplex konjugierte) Nullstellen hat, oder aber eine doppelte reelle.

\subsection{Definition \& Lemma}\index{$ f $-invarianter Unterraum}
	Das Schlüsselargument im Beweis oben kann man verallgemeinern:

\begin{Definition}[$ f $-invarianter Unterraum]
	Sei $ f\in \End(V) $ und $ U\subset V $ ein \emph{$ f $-invarianter Unterraum}, d.h. $ f(U)\subset U $. 

\end{Definition}
\begin{Lemma}[]
	Ist dann $ V=U\oplus U' $ eine direkte Zerlegung und $ p,p'\in \End(V) $ die zugehörigen Projektionen, so gilt
		\[ \chi_f(t)=\chi_{f|_U}(t)\cdot \chi_{f'}(t), \]
	wobei
		\[ f':= p'\circ f|_{U'}\in \End(U'). \]

\end{Lemma}
\paragraph{Bemerkung}
	Man kann $ f|_U $ als Endomorphismus $ f|_U\in \End(U) $ auffassen, da $ f(U)\subset U $.
\paragraph{Beweis}
	Wie oben: Sei $ B=(b_1,\dots,b_n) $ Basis von $ V $, sodass
		\begin{itemize}
			\item $ C=(b_1,\dots,b_k) $ Basis von $ U $ und
			\item $ C'=(b_{k+1},\dots,b_n) $ Basis von $ U' $ ist.
		\end{itemize}
	Die Darstellungsmatrix von $ f $ bzgl. $ B $ hat dann Blockgestalt,
		\[ \xi_B^B(f) =
			\begin{pmatrix}
				X&Y\\0&X'
			\end{pmatrix}
		\text{ mit } X=\xi_C^C(f|_U), X' = \xi_{C'}^{C'}(f') \]
	Damit folgt die Behauptung (wie oben) mit der Leibniz-Formel.
\paragraph{Bemerkung}
	Alternativ kann man das Lemma mit der von $ f $ induzierten Quotientenabbildung $ f'\in \End(V/U) $ formulieren, wobei
		\[ f':V/U\to V/U, v+U\mapsto f'(v+U) := f(v)+U. \]
\subsection{Definition}\index{Diagonalisierbarkeit}\index{Triagonalisierbarkeit}
\begin{Definition}[Diagonalisierbarkeit, Triagonalisierbarkeit von Endomorphismen]
	Ein Endomorphismus $ f\in\End(f) $ heißt \emph{diagonalisierbar} bzw. \emph{trigonalisierbar}, falls es eine Basis $ B $ von $ V $ gibt, sodass $ \xi_B^B(f)=(x_{ij})_{i,j\in\{1,\dots,n\}} $ eine Diagonalmatrix 
		\[ i\neq j\Rightarrow x_{ij} = 0 \]
	bzw. obere Dreiecksmatrix ist,
		\[ i>j \Rightarrow x_{ij} = 0. \]
\end{Definition}
\paragraph{Bemerkung}
	Falls $ \dim V<\infty $, so ist $ f\in\End(V) $ genau dann diagonalisierbar, wenn $ V $ eine Basis aus Eigenvektoren von $ f $ besitzt. Damit kann man "`Diagonalisierbarkeit"' auch im Falle $ \dim V=\infty $ definieren.
\paragraph{Bemerkung}
	Ist $ f $ trigonalisierbar (oder gar diagonalisierbar), so zerfällt $ \chi_f (t) $ in Linearfaktoren: für geeignete $ x_1,\dots,x_n\in K $ ist
		\[ \chi_f(t)=\prod_{j=1}^{n}(t-x_j). \]
\subsection{Bemerkung \& Definition}
\begin{Definition}[Diagonalisierbarkeit, Triagonalisierbarkeit von Matrizen]
	Man nennt eine Matrix $ X\in K^{n\times n} $ diagonalisierbar (bzw. trigonalisierbar), falls $ f_X\in \End(K^n) $ diagonalisierbar (bzw. trigonalisierbar) ist.
\end{Definition}	

	Dies ist genau dann der Fall, falls es $ P\in Gl(n) $ gibt, sodass $ PXP^{-1} $ Diagonalmatrix (bzw. obere Dreiecksmatrix) ist.

% VO 2016-03-17

\subsection{Lemma}
	Frage: Was sind hinreichende Kriterien dafür? Notwendigkeit kennen wir: $ \chi_f(t) $ zerfällt in Linearfaktoren.
	
	\begin{Lemma}[Lineare Unabhängigkeit von Eigenvektoren]
		Eigenvektoren $ v_1,\dots,v_m\in V $ zu paarweise verschiedenen Eigenwerten $ x_1,\dots,x_m $ eines Endomorphismus $ f\in\End(V) $ sind linear unabhängig.
	\end{Lemma}
\paragraph{Bemerkung}
	Anders gesagt: Die Summe von Eigenräumen zu paarweise verschiedenen Eigenwerten ist direkt.
\paragraph{Beweis}
	Zu zeigen: Ist $ \sum_{i=1}^m v_iy_i = 0 $ für Koeffizienten $ y_1,\dots,y_m\in K $, so folgt $ y_1 = \dots = y_m = 0 $.
	
	Seien $ y_1,\dots,y_m \in K $ und $ w_i := v_iy_i $ und $ w:= \sum_{i=1}^{m}w_i = \sum_{i=1}^{m}v_iy_i$.
	Wiederholte Anwendung von $ f $ liefert, wegen $ f(w_i) = w_ix_i $
	
		\[ (f^{m-1}(w),\dots,f^2(w),f(w),w) = (w_1,\dots,w_m)
		\begin{pmatrix}
		 x_1^{m-1}&\cdots&x_1^2&x_1&1 \\
		 \vdots&\ddots&\vdots&\vdots&\vdots\\
		 x_m^{m-1}&\cdots&x_m^2&x_m & 1
		\end{pmatrix} \]
	mit der Vandermonde-Matrix $ X\in Gl(m) $, da
		\[ \det X = \prod_{i<j} (x_i - x_j)\neq 0 \]
	weil die Eigenwerte $ x_1,\dots,x_m $ paarweise verschieden sind.
	Damit folgt aus $ w=\sum_{i=1}^{m}v_iy_i = 0 $
		\[ (w_1,\dots,w_m)=(f^{m-1}(w),\dots,f(w),w)X^{-1} = (0,\dots,0) \]
	also
		\[ \forall i=1,\dots,m: 0 = w_i = v_iy_i \text{ und }v_i \neq 0 \Rightarrow y_i = 0.  \]
\subsection{Satz}
    \begin{Satz}[Diagonalisierbarkeit eines Endomorpismus]
		Ein Endomorphismus $ f\in \End(V) $ ist genau dann diagonalisierbar, wenn $ \chi_f(t) \in K[t] $ in Linearfaktoren zerfällt und die algebraischen und geometrischen Vielfachheiten aller Eigenwerte übereinstimmen,
			\[ \chi_f(t) = \prod_{i=1}^{m}(t-x_i)^{k_i} \text{ und } \forall i=1,\dots,m: k_i = g_i.\]
	\end{Satz}
\paragraph{Beweis}
	Ist $ f $ diagonalisierbar, so existiert eine Basis $ B $ aus Eigenvektoren von $ f $, also ist dann
		\[ \xi_B^B(f) =
			\begin{pmatrix}
				E_{g_1}x_1 &0& \cdots & 0 \\
				0 &E_{g_2}x_2& \ddots & \vdots\\
				\vdots & \ddots& \ddots & \vdots\\
				0 & 0 & \cdots & E_{g_m}x_m 
			\end{pmatrix} \]
	Damit ist
		\[ \chi_f(t)=\prod_{i=1}^{m}(t-x_i)^{g_i}. \]
	Hat andererseits das charakteristische Polynom diese Gestalt, so wähle man in jedem Eigenraum $ \ker(\id_Vx_i-f) $ eine Basis $ C_i,i=1,\dots,m $. Da Eigenvektoren zu verschiedenen Eigenwerten linear unabhängig sind, und wegen
		\[ g_1+\dots+g_m = k_1 + \dots + k_m = \dim V \]
	liefert $ B := \bigcup_{i=1}^mC_i $ eine Basis von $ V $.
\subsection{Korollar}
	\begin{Korollar}
		Ein Endomorphismus $ f\in\End(V) $ mit $ n=\dim V $ paarweise verschiedenen Eigenwerten ist diagonalisierbar.
	\end{Korollar}
\paragraph{Beweis}
	Für die geometrischen und algebraischen Vielfachheiten jedes Eigenwerts gilt
		\[ 1\leq g_i \leq k_i \text{ und } \sum_{i=1}^{n}k_i \leq n. \]
	Damit folgt
		\[ \forall i=1,\dots,n:k_i = 1 \text{ und } \sum_{i=1}^{n}k_i = n, \]
	d.h. das charakteristische Polynom zerfällt in Linearfaktoren und $ \forall i=1,\dots,n:k_i=g_i. $
\subsection{Satz}
\begin{Satz}[Trigonalisierbarkeit eines Endomorpismus]
	Ein Endomorpismus $ f\in\End(V) $ ist genau dann trigonalisierbar, wenn das charakteristische Polynom in Linearfaktoren zerfällt.
\end{Satz}
\paragraph{Bemerkung}
	Da Diagonalisierbarkeit bzw. Trigonalisierbarkeit durch die Existenz einer Darstellungsmatrix in spezieller Gestalt definiert wurde, wird in den Charakterisierungen immer (implizit) $ \dim V < \infty $ angenommen.
\paragraph{Beweis}
	Wir wissen schon: Ist $ f $ trigonalisierbar, so zerfällt $ \chi_f(t) $ in Linearfaktoren. Umkehrung: Beweis durch vollständige Induktion über $ n=\dim V $.
	
	Für $ n=1 $ ist nichts zu zeigen. Sei die Behauptung für $ n-1 $ bewiesen. Für $ n $ folgt dann:
	
	Da $ \chi_f(t) $ in Linearfaktoren zerfällt
		\[ \chi_f(t)=\prod_{i=1}^{n}(t-x_i) \]
	für geeignete $ x_1,\dots,x_n $, ist $ x_1 $ Eigenwert von $ f $. Nun seien
	\begin{itemize}
		\item $ b_1 $ ein Eigenvektor zum Eigenwert $ x_1 $ und $ U:= [\{b_1\}] $,
		\item $ U'\subset V $ ein zu $ U $ komplementärer Unterraum, und
		\item $ p,p'\in \End(V) $ die zur direkten Zerlegung $ V = U\oplus U' $ gehörenden Projektionen,
			\[ U = p(V) = \ker p' \text{ und } U' = p'(V) = \ker p, \]
		\item und $ f' := p'\circ f|_{U'}\in\End(U'). $
	\end{itemize}
	Da $ U (\neq \{0\}) $ $ f $-invarianter UR von $ V $ ist, faktorisiert das charakteristische Polynom
		\[ \chi_f(t)=\chi_{f|_U}(t)\cdot \chi_{f'}(t) = (t-x_1)\cdot \chi_{f'}(t); \]
	also zerfällt $ \chi_{f'}(t) $ in Linearfaktoren,
		\[ \chi_{f'}(t)=\prod_{i=2}^{n}(t-x_i). \]
	Nach Induktionsannahme existiert also eine Basis $ B' = (b_2,\dots,b_n) $ von $ U' $, sodass $ \xi_{B'}^{B'}(f) $ obere Dreiecksmatrix ist. Mit $ B=(b_1,\dots,b_n) $ als Basis von $ V $ gilt dann:
		\[ \xi_B^B (f) =
			\begin{pmatrix}
			x_1& Y\\
			0 & \xi_{B'}^{B'}(f')
			\end{pmatrix} \]
	ist obere Dreiecksmatrix.
\section{Der Satz von Cayley-Hamilton}
\subsection{Satz}
	Für $ f\in\End(V) $ gilt $ \chi_f(f) = 0$.
\paragraph{Unfug-Beweis}
	Durch direktes Einsetzen erhält man
		\[ \chi_f(f)=\det (\id_V f-f) = \det 0 = 0. \]
\paragraph{Zum Verständnis des Satzes}
	Ist $ V $ ein $ K $-VR mit $ n=\dim V < \infty $ und $ f\in \End (V) $, so ist
		\[ \chi_f(t) = \sum_{k=0}^{n}t^ka_k \in K[t] \]
	ein (abstraktes) Polynom in der Variablen $ t\ (= e_1\in K^\mathbb{N})$ und der Einsetzungshomomorphismus $ \psi_f:K[t]\to \End(V) $ (also ein Algebrahomomophismus) liefert
		\[ \chi_f(f) = \psi_f\left(\chi_f(t)\right) = \sum_{k=0}^{n}f^k a_k. \]
	Der Satz sagt, dass $ 0 = \chi_f(f)\in \End(V) $, d.h.
		\[ \forall v\in V: \chi_f(f)(v) = 0. \]
		
\subsection{Definition \& Lemma}\index{$ f $-zyklische Basis}
\begin{Definition}[$ f $-zyklische Basis]\label{fzykl}
	Seien $ f\in\End(V) $ und $ B $ eine \emph{$ f $-zyklische Basis} von $ V $, d.h. eine Basis der Form
		\[ B= (b_1,\dots,b_n) = \left(b,f(b),\dots,f^{n-1}(b)\right). \]		
\end{Definition}
\begin{Lemma}
	Dann existieren $ a_0,\dots,a_{n-1}\in K $ mit
		\[ f^n(b)+\sum_{k=0}^{n-1}f^k(b)a_k = 0, \]
	mit diesen Koeffizienten ist
		\[ \chi_f(t) = t^n+t^{n-1}a_{n-1}+\dots,+ta_1+a_0. \]
\end{Lemma}
\paragraph{Bemerkung}
	Im Allgemeinen existiert zu $ f\in \End(V) $ keine $ f $-zyklische Basis von $ V $, z.B. für $ f = \id_V $ und $ \dim V \geq 2 $.
\paragraph{Beweis}
	Da $ B= \left(b,f(b),\dots,f^{n-1}(b)\right) $ eine Basis ist, ist $ f^n(b)\in [B] $ und damit existieren die $ a_k $ mit
		\[ 0 = f^n(b) + \sum_{k=0}^{n-1}f^k(b)a_k. \]
	Damit ist die Darstellungsmatrix von $ f $
		\[ \xi_B^B(f) =
		\begin{pmatrix}
			0      & \cdots & \cdots & 0      & -a_0     \\
			1      & \ddots &        & \vdots & -a_1     \\
			0      & 1      & \ddots & \vdots & \vdots   \\
			\vdots & \ddots & \ddots & 0      & \vdots   \\
			0      & \cdots & 0      & 1      & -a_{n-1}
		\end{pmatrix} =: X
		\]
	und Entwicklung von $ \chi_f(t)=\det(E_nt-\xi_B^B(f)) $ nach der ersten Zeile (nach Laplaceschem Entwicklungssatz -- dieser Satz war "`nur"' eine Methode, die Terme in der Leibniz-Formel zu sortieren) liefert
	\begin{align*}
		\det(E_nt-X) &=\det 
			\begin{pmatrix}
				t      & 0      & \cdots & 0      & a_0       \\
				-1     & t      & \ddots & \vdots & a_1       \\
				0      & -1     & \ddots & 0	  & \vdots    \\
				\vdots & \ddots & \ddots & t      & \vdots    \\
				0      & \cdots & 0      & -1     & t+a_{n-1}
			\end{pmatrix}\\
		&= t\cdot\det
			\begin{pmatrix}
				t      & 0      & \cdots & 0      & a_1       \\
				-1     & t      & \ddots & \vdots & a_2       \\
				0      & -1     & \ddots & 0 	  & \vdots    \\
				\vdots & \ddots & \ddots & t      & \vdots    \\
				0      & \cdots & 0      & -1     & t+a_{n-1}
			\end{pmatrix}
		+ (-1)^{n+1} a_0 \underbrace{\det(X_{1n})}_{(-1)^{n-1}}\\
		\intertext{mittels vollständiger Induktion folgt}
		&= t (t(\dots(\underbrace{t(t+a_{n-1})+a_{n-2}}_{\det \begin{pmatrix}t&a_{n-2}\\-1&t+a_{n-1}\end{pmatrix}})\dots)+a_1)+a_0 \\
		&= t (t^{n-1}+t^{n-2}a_{n-1}+\dots+a_1)+a_0 \\
		&= t^n+t^{n-1}a_{n-1}+\dots,+ta_1+a_0,
	\end{align*}
	wie behauptet.
	
\paragraph{Beispiel}
	Zur Lösung des reellen \emph{Anfangswertproblems}
		\[ y'' + 2y' - 3y = 0,\ 
		\begin{cases}
			y(0)=4 \\
			y'(0)=0
		\end{cases} \]
	schreiben wir dieses als System erster Ordnung mit dem Ansatz $ y_1 = y $ und $y_2 = y' $:
	
	Daraus erhält man mit $ Y= (y_1,y_2) $
	  \begin{align*}
		 Y' &= (y_1',y_2') = (y',y'') = (y',-2y'+3y)\\
		 &= (y,y') \begin{pmatrix}
		 	0 & 3\\ 1 & -2
		 \end{pmatrix} = YX	\text{ mit }
		 X= \begin{pmatrix}
			0 & 3\\ 1 & -2
		\end{pmatrix},
		\end{align*}
	d.h. wir suchen eine $ \frac{d}{ds} $-zyklische Basis $ (y,\frac{d}{ds}y) = (y,y') $ eines 2-$ \dim $ UVR $ [(y,y')]\subset C^\infty(\R) $ bezüglich derer $ \frac{d}{ds}\in \End(C^\infty(\R)) $ Darstellungsmatrix $ X $ hat.
	
	Der Ansatz $ y(s) = e^{xs} (v_0,v_1)$ reduziert das AWP auf ein Eigenwertproblem.
		\[ 0 = \left(Y'-YX\right)(s) = \left(\frac{d}{ds}Y - YX\right)(s) = \underset{Y}{\underbrace{e^{xs}(v_0,v_1)}} \{E_2x-X\}\]
	bzw. (vgl. Abschnitt 3.1) mit dem zur transponierten Matrix $ X^t $ assoziierten Endomorphismus $ f_{X^t}\in \End(\R^2) $
		\[ f_{X^t}(v) = vx \text{ für }x\in\R \text{ und }v\in \R^2. \]
	Nach obigem Lemma sind die Eigenwerte Lösungen der Gleichung
		\[ 0 = \chi_{X^t}(x) = \chi_X(x) \overset{\text{Lemma}}{=} x^2+2x-3 = (x-1)(x+3). \]
	Also sind $ x_1 = 1 $ und $ x_2 = -3 $ die Eigenwerte; zugehörige Eigenvektoren erhält man als Lösungen der linearen Gleichungssysteme
		\[ (0,0) = (v_0,v_1)(E_2x_i-X)= (v_0,v_1)\begin{pmatrix}
		x_i&-3\\-1&x_i+2
		\end{pmatrix} = \begin{cases}
		(v_0,v_1)\begin{pmatrix}
		1&-3\\-1&3
		\end{pmatrix}& \text{ für } i = 1\\
		(v_0,v_1)\begin{pmatrix}
		-3&-3\\-1&-1
		\end{pmatrix}& \text{ für } i = 2
		\end{cases} \]
	Damit bekommt man Eigenvektoren $ (v_0,v_1) = (1,1) $ zum Eigenwert $ x=1 $ und $ (v_0,v_1) = (1,-3) $ zum Eigenwert $ x = -3 $.
	
	Die allgemeine, durch \emph{Superposition} (Linearkombination) erhaltene Lösung der Differentialgleichung ist also
		\[ s\mapsto Y(s) = e^s(1,1)c_1 + e^{-3s}(1,-3)c_2 \]
	mit Koeffizienten $ c_1,c_2 \in \R $. Abgleich der "`Integrationskonstanten"' $ c_1 $ und $ c_2 $ mit den Anfangsbedingungen liefert dann die Lösung
		\[ s \mapsto y(s) = 3e^s+e^{-3s}. \]
\paragraph{Bemerkung}
	Man bemerke: $ (y,y') $ ist linear unabhängig für die Lösung, ist also tatsächlich $ \frac{d}{ds} $-zyklische Basis eines 2-$ \dim $ URs $ [(y,y')]\subset C^\infty(\R) $ -- obwohl die den gleichen Raum aufspannenden "`Basislösungen"'
		\[ s\mapsto e^s \text{ und } s\mapsto e^{-3s} \]
	keine $ \frac{d}{ds} $-zyklischen Basen erzeugen, da sie lineare Differentialgleichungen erster Ordnung (mit konstanten Koeffizienten) lösen.
	
\subsection{Korollar}
\begin{Korollar}
	Besitzt $ V $ eine $ f $-zyklische Basis für $ f\in\End(V) $, so gilt $ \chi_f(f)=0 $.
\end{Korollar}
\paragraph{Beweis}
	Sei also $ B=(b_1,\dots,b_n) =(b,f(b),\dots,f^{n-1}(b)) $ $ f $-zyklische Basis von $ V $ und $ a_0,\dots,a_{n-1}\in K $ so, dass
		\[ 0 = f^n(b)+\sum_{k=0}^{n-1}f^k(b)a_k. \]
	Dann gilt
		\[ \chi_f(f)(b_1) = \chi_f(f)(b) = \left(f^n+\sum_{k=0}^{n-1}f^ka_k\right)(b) = f^n(b)+\sum_{k=0}^{n-1}f^k(b)a_k. \]
	Damit folgt für $ i=2,\dots,n $
		\[ \chi_f(f)(b_i) = \chi_f(f)\left(f^{i-1}(b)\right) \stackrel{\footnotemark}{=} f^{i-1}\left(\chi_f(f)(b) \right) = 0. \]
		\footnotetext{Aufgrund der Linearität der Endomorphismen $ \End(V) $ als unitäre Algebra.}
	Da also $ V=[B] \subset \ker {\chi_f(f)}$, folgt $ \chi_f(f) = 0. $
\paragraph{Bemerkung}
	Damit ist der Satz von Cayley-Hamilton bewiesen, sofern $ V $ eine $ f $-zyklische Basis besitzt.
	
\subsection{Lemma}
\begin{Lemma}[ $ f $-invarianter UVR endlicher Dimension besitzen eine $ f $-zyklische Basis]
	Für $ f\in\End(V) $ und $ v\in V^\times  $ sei
		\[ U := \left[\left(f^k(v)\right)_{k\in{\mathbb{N}}}\right]. \]
	Damit ist $ U $ ein $ f $-invarianter UVR von $ V $. Ist $ \dim V < \infty $, so besitzt $ U $ eine $ f $-zyklische Basis $ \left(v,f(v),\dots,f^{r-1}(v)\right) $.
\end{Lemma}
\paragraph{Beweis}
	Offenbar ist $ U $ $ f $-invarianter UR:
	\begin{itemize}
		\item $ U $ ist (als lineare Hülle einer Familie) ein UVR von $ V $;
		\item da gilt
			\[ \forall k\in \mathbb{N}: f\left(f^k(v)\right)=f^{k+1}(v)\in U \]
		folgt, dass
			\[f(U) = f\left(\left[\left(f^k(v)\right)_{k\in{\mathbb{N}}}\right]\right) = \left[\left(f^{k+1}(v)\right)_{k\in{\mathbb{N}}}\right]\subset U. \]
	\end{itemize}
	Ist $ \dim V < \infty $ und $ v\neq 0 $, so existiert $ r\in \mathbb{N} $, sodass
		\[ \left(v,\dots,f^{r-1}(v)\right) \text{ linear unabhängig und }f^r(v)\in\left[\left(v,\dots,f^{r-1}(v)\right)\right]; \]
	damit ist $ \left(v,f(v),\dots, f^{r-1}(v)\right) $ $ f $-zyklische Basis von $ U $:
	\begin{enumerate}
		\item $ \left(v,\dots,f^{r-1}(v) \right) $ ist linear unabhängig.
		\item $ f^r(v) \in \left[\left(v,\dots,f^{r-1}(v)\right)\right]$, damit gilt
		\[ \forall k\in \mathbb{N}: k\geq r \Rightarrow f^k(v)\in \left[\left(v,\dots,f^{r-1}(v)\right)\right] \]
		wie man z.B. mit Induktion sehen kann: ist
			\[ f^{k-1}(v) = \sum_{j=0}^{r-1}f^j(v)x_j \in \left[\left(v,\dots,f^{r-1}(v)\right) \right], \]
		so folgt
			\[ f^k(v) = \sum_{j=1}^{r}f^j(v)x_{j-1}=f^{r}(v)x_{r-1}+\sum_{j=1}^{r-1}f^j(v)x_{j-1}\in \left[\left(v,\dots,f^{r-1}(v)\right)\right] \]
		und damit
			\[ U = \left[\left(f^k(v)\right)_{k\in \mathbb{N}}\right]\subset \left[\left(v,\dots,f^{r-1}(v)\right)\right]. \]
	\end{enumerate}
		
\subsection{Beweis vom Satz von Cayley-Hamilton}
	Zu zeigen: für $ f\in \End(V) $ gilt $ \chi_f(f)=0 $, d.h.
		\[ \forall v\in V:\chi_f(f)(v) = 0. \]
	Seien also $ v\in V^\times $ und
		\[ U := \left[\left( f^k(v)_{k\in \mathbb{N}}\right)\right]\subset V. \]
	Mit einem zu $ U $ komplementären UVR $ U'\subset V $, $ V=U\oplus U' $, und den zugehörigen Projektionen
	
	\begin{multicols}{2}
	%------------------ Projektion ----------------
 	\begin{figure}[H]\centering
 		\tdplotsetmaincoords{0}{0} %-27
 		\begin{tikzpicture}[xscale=0.5,yscale=0.5,tdplot_main_coords]

 				\def\xstart{0} %x Koordinate der Startposition der Grafik
 				\def\ystart{-3} %y Koordinate der Startposition der Grafik
 				\def\myscale{0.5} %ändert die Größe der Grafik (Skalierung der Grafik)
                \def\myscalex{1.0}
                \def\myscaley{0.6}
                \def\maxlh{6.0}
                \def\maxlv{6.0}
                
 				\def\xstartdraw{(\xstart + \maxlh)} %xKoordinate des Referenzstartpunktes (in dieser Zeichnung: a)
 				\def\ystartdraw{(\ystart + \maxlv)}%yKoordinate des Referenzstartpunktes (in dieser Zeichnung: a)

 			    \node (pointro) at ({\xstartdraw+(\maxlh*\myscalex)},{\ystartdraw+(\maxlv*\myscaley)}) {};
 			    
			    \node (pointlu) at ({\xstartdraw-(\maxlh*\myscalex)},{\ystartdraw-(\maxlv*\myscaley)}) {};
			    \node (pointlo) at ({\xstartdraw-(\maxlh*\myscalex)},{\ystartdraw+(\maxlv*\myscaley)}) {};
                \node (pointru) at ({\xstartdraw+(\maxlh*\myscalex)},{\ystartdraw-(\maxlv*\myscaley)}) {};
                
 				%\node (pointo1) at ($(pointol)!0.2!(pointor)$) {};
 				%\node (pointo2) at ($(pointol)!0.9!(pointor)$) {};

 				\node (offsetx) at ({(3.0*\myscalex},{0.0}) {}; %just an offset
 				\node (offsety) at ({0.0},{3.0*\myscaley}) {}; %just an offset
                
                \node (unity) at ({\maxlh*0.25*\myscalex},{\maxlv*0.25*\myscaley}) {}; %just an offset
                \node (unitx) at ({\maxlh*0.25*\myscalex},{-\maxlv*0.25*\myscaley}) {}; %just an offset
                
               	\node (point00) at ($(pointlo) + 4.0*(unitx) - 1.0*(unity)$) {};
 				%\draw[fill,color=blue] (point00) circle [radius=0.18];
 				
                 
 				%Koordinatenkreuz
 				\draw[-,line width=0.2pt,color=black] ($(point00) -3.0*(unity)$) -- ($(point00) +5.0*(unity)$);
 				\draw[-,line width=0.2pt,color=black,shorten >=-20pt] ($(point00) -3.0*(unitx)$) -- ($(point00) + 3.0*(unitx)$);
 				
 				%Q2 dotted vertikal
 				\draw[-,line width=0.4pt,color=red,dotted] ($(point00) -2.0*(unitx) -2.0*(unity)$ + ) -- ($(point00) -2.0*(unitx) +3.0*(unity)$);
 				\draw[-,line width=0.4pt,color=red,dotted] ($(point00) -1.0*(unitx) -2.0*(unity)$ + ) -- ($(point00) -1.0*(unitx) +4.0*(unity)$);
 				
 				%Q1 dotted vertikal
 				\draw[-,line width=0.4pt,color=red,dotted] ($(point00) + 2.0*(unitx) -2.0*(unity)$ + ) -- ($(point00) + 2.0*(unitx) +5.0*(unity)$);
 				\draw[-,line width=0.4pt,color=red,dotted] ($(point00) +1.0*(unitx) -2.0*(unity)$ + ) -- ($(point00) +1.0*(unitx) +5.0*(unity)$);
 				
 				%Q1 und Q2 dotted horizontal
 				\draw[-,line width=0.4pt,color=red,dotted] ($(point00) - 1.0*(unitx) +4.0*(unity)$ + ) -- ($(point00) + 3.0*(unitx) +4.0*(unity)$);
 				\draw[-,line width=0.4pt,color=red,dotted] ($(point00) - 2.0*(unitx) +3.0*(unity)$ + ) -- ($(point00) + 4.0*(unitx) +3.0*(unity)$);
 				\draw[-,line width=0.4pt,color=red,dotted] ($(point00) - 3.0*(unitx) +2.0*(unity)$ + ) -- ($(point00) + 4.0*(unitx) +2.0*(unity)$);
 				\draw[-,line width=0.4pt,color=red,dotted] ($(point00) - 3.0*(unitx) +1.0*(unity)$ + ) -- ($(point00) + 4.0*(unitx) +1.0*(unity)$);	
 				\draw[-,line width=0.4pt,color=red,dotted] ($(point00) - 3.0*(unitx) -1.0*(unity)$ + ) -- ($(point00) + 3.0*(unitx) -1.0*(unity)$);	
 				
 				%Vektoren blau
 				\node (point03) at ($(point00) +3.0*(unity)$) {};
               	\node (point20) at ($(point00) +2.0*(unitx)$) {};
               	\node (point23) at ($(point00) +2.0*(unitx) +3.0*(unity) $){};
 				
 				\draw[-{>[scale=1,length=8,width=6]},shorten >=-5pt,line width=0.5pt,color=red] (point00) -- (point03);
 				\draw[-{>[scale=1,length=8,width=6]},shorten >=-5pt,line width=0.5pt,color=red] (point00) -- (point20);
 				\draw[-{>[scale=1,length=8,width=6]},shorten >=-5pt, shorten <=-5pt,line width=0.9pt,color=blue] (point00) -- (point23);
 				
 			    \node (pointvekw) at (point23) [above,color=blue]{$v$};
 			    \node (pointvekw) at (point20) [xshift=-20,color=red]{$p(v)$};
 			    \node (pointvekw) at (point03) [yshift= 10,xshift=-5,color=red]{$p'(v)$};
 			    \node (pointvekw) at ($(point00) -3.0*(unitx)$) [yshift= 8,color=green]{$U$};
 			    \node (pointvekw) at ($(point00) +5.0*(unity)$) [yshift=-5,xshift= 5,color=green]{$U'$};
 			\end{tikzpicture}
	\end{figure}
	%------------------ Projektion ----------------
	    \begin{align*}
		    &p: V\to V, p(V) = U, \ker p = U' \text{ bzw. } \\
		    &p':V\to V, p'(V) = U', \ker p' = U,
	    \end{align*}
	\end{multicols}
	ist dann $ \chi_f(t) = \chi_{f'}(t)\cdot \chi_{f\mid_U}(t) \text{ mit } f' := p'\circ f\mid_{U'}\in \End(U') $.
	
	Damit folgt
		\[ \chi_f(f)(v) = \chi_{f'}(f) \left(\chi_{f\mid_U}(f)(v) \right) = \chi_{f'}(f)(0)=0 \]
	nach Korollar oben, da $ U $ eine $ f $-zyklische Basis besitzt und $ v\in U $.
	
% % VO-12-04-2016 % % 
\subsection{Definition}\index{Annulatorpolynom}\index{Minimalpolynom}
\begin{Definition}[Annulatorpolynom, Minimalpolynom]
	Sei $ V $ ein $ K $-VR und $ f\in\End(V) $. Dann heißt $ p\in K[t] $
		\begin{itemize}
			\item \emph{Annulatorpolynom von $ f $}, falls $ p(f)=0 $;
			\item \emph{Minimalpolynom von $ f $}, falls $ p(t) $ normiertes Annulatorpolynom minimalen Grades ist.
		\end{itemize}
\end{Definition}
\paragraph{Bemerkung}
	Jedes (polynomiale) Vielfache 
		\[ p(t) = q(t)\mu_f(t)\in K[t] \]
	eines Minimalpolynoms $ \mu_f(t) $ von $ f $ ist ein Annulatorpolynom, da
		\[ \forall v\in V: p(f)(v) = \left(q(f)\circ \mu_f(f)\right)(v) = q(f)\left(\mu_f(f)(v)\right) = q(f)(0) = 0 \]
\paragraph{Bemerkung}
	Nach dem Satz von Cayley-Hamilton hat jeder Endomorphismus $ f\in\End(f) $ ein Annulatorpolynom, also auch ein Minimalpolynom -- wenn $ \dim V < \infty $.
	
\subsection{Lemma}
\begin{Lemma}[Jedes Minimalpolynom ist Teiler des Annulatorpolynom von $ f\in\End(V) $]
	Ist $ p(t)\in K[t] $ Annulatorpolynom von $ f\in\End(V) $, so ist jedes Minimalpolynom $ \mu_f(t)\in K[t] $ Teiler von $ p(t) $. 
\end{Lemma}

\paragraph{Beweis}
	Seien $ q(t),r(t)\in K[t] $ die (nach dem euklidischen Divisionsalgorithmus) eindeutigen Polynome mit
		\[ p(t) = q(t)\mu_f(t)+r(t) \text{ und }\deg r(t)<\deg \mu_f(t). \]
	Dies liefert
		\[ r(f) = p(f)-q(f)\circ \mu_f(f) = 0-q(f)(0) = 0, \]
	also $ r(t) = 0 $, denn andernfalls wäre $ \mu_f(t) $ nicht normiertes Annulatorpolynom minimalen Grades.
	
\subsection{Korollar}
\begin{Korollar}
	Das Minimalpolynom $ \mu_f(t)\in K[t] $ eines Endomorphismus $ f\in \End(V) $ ist eindeutig.
\end{Korollar}
\paragraph{Beweis}
	Sind $ \mu_f(t),\tilde{\mu}_f(t)\in K[t] $ Minimalpolynome von $ f\in \End(V) $, so gilt
		\[ \exists! q(t)\in K[t] : \tilde{\mu}_f(t) = q(t)\mu_f(t) \] % Nach Lemma 4.5.7
	wobei
		\begin{itemize}
			\item $ \deg q(t) = 0 $, da $ \deg \tilde{\mu}_f(t) \leq \deg \mu_f(t) $,
			\item $ q(t) = 1$, da $ \tilde{\mu}_f(t) $ und $ \mu_f(t) $ normiert sind.
		\end{itemize}
	Daher ist
		\[ \tilde{\mu}_f(t) = 1\cdot \mu_f(t) = \mu_f(t). \]
\paragraph{Bemerkung}
	Wie für Endomorphismen kann man Annulatorpolynome, Minimalpolynome, usw. auch für Matrizen $ X\in K^{n\times n} $ definieren:
		\begin{itemize}
			\item mithilfe der assoziierten Endomorphismen $ f_X\in \End(K^n) $, oder 
			\item mithilfe des Einsetzungshomomorphismus $ \psi_X: K[t] \to K^{n\times n}. $ % in der Algebra der quadratischen Matrizen
		\end{itemize}
	Beide Methoden liefern das gleiche Ergebnis durch den Algebrahomomorphismus zwischen den Endomorphismen und den quadratischen Matrizen.

\paragraph{Bemerkung \& Beispiel}
	Zerfällt das charakteristische Polynom in Linearfaktoren, so zerfällt auch das Minimalpolynom in dieselben Linearfaktoren:
		\[ \chi_f(t)= \prod_{i=1}^{m}(t-x_i)^{k_i} \Rightarrow \mu_f(t) = \prod_{i=1}^{m}(t-x_i)^{m_i}, \]
	wobei für $ i= 1,\dots,m $ gilt $ 1\leq m_i\leq k_i $.
	
	Zum Beispiel: 
		\begin{itemize}
			\item $ X = \begin{pmatrix}
			1&0\\0&0
			\end{pmatrix} $: $ \chi_{f_X} = t(t-1) = \mu_{f_X}(t)$
			\item $ X = \begin{pmatrix}
			1&0\\0&1
			\end{pmatrix} $: $ \chi_{f_X} = (t-1)^2 \Rightarrow \mu_{f_X}(t) = (t-1) $
			\item $ X = \begin{pmatrix}
			1&1\\0&1
			\end{pmatrix} $: $ \chi_{f_X} = (t-1)^2 = \mu_{f_X}(t)$.
		\end{itemize}

\paragraph{Bemerkung}
	Die Definition des charakteristischen Polynoms ist etwas problematisch:
		\[ \chi_f(t) := \det (\id_Vt-f) \]
	ist "`gut"' für Polynomfunktionen, aber "`nicht korrekt"' für abstrakte Polynome; die Definition 
		\[ \chi_f(t) := \sum_{\sigma\in S_n}\sgn(\sigma)\prod_{i=1}^{n}\left(\delta_{\sigma(j)j}-x_{\sigma(j)j} \right)\in K[t] \]
	mithilfe der Darstellungsmatrix $ X = (x_{ij})_{i,j\in \{1,\dots,n\}} = \xi_B^B(f) $
	von $ f $ bzgl. einer Basis $ B $ und der Leibniz-Formel ist nicht sehr übersichtlich. Vergleiche auch [Axler, Kap. 8] zum Thema.
	
	Im Gegensatz dazu: Definitionen von "`Annulatorpolynom"' und "`Minimalpolynom"' etc. sind einfach (konzeptionell).
	
	Frage: Braucht man das charakteristische Polynom überhaupt?
	Man kommt auch ohne das charakteristische Polynom "`recht weit"':
		\begin{itemize}
			\item Für $ \dim V <\infty $ folgt die Existenz eines Annulatorpolynoms, und damit des Minimalpolynoms recht einfach wegen $ \dim \End(V) <\infty $.
			\item Durch Einsetzen: Jeder Eigenwert eines Endomorphismus ist Nullstelle seines Minimalpolynoms.
			\item Umgekehrt ist auch jede Nullstelle des Minimalpolynoms Eigenwert -- ist $ \mu_f(x) = 0 $, so existiert $ q(t)\in K[t] $ mit
				\[ \mu_f(t) = q(t)(t-x); \]
			wäre $ x $ kein Eigenwert, also $ f-\id_V x \in Gl(V) $, so gälte
				\[ (f-\id_Vx)(V) = V \Rightarrow \{0\} = \mu_f(f)(V) = q(f)(V), \]
			d.h. $ \mu_f(t) $ wäre nicht Minimal-Polynom.
			\item Ein Endomorphismus ist diagonalisierbar, wenn sein Minimal-Polynom in paarweise verschiedene Linearfaktoren zerfällt.
		\end{itemize}
	
	Nachteil des Minimal-Polynoms: schwierig berechenbar?

% VO 14-04-2016 %
\chapter{Längen- und Winkelmessung}
Plan: 
	Längen und Winkel (in "`Punkträumen"' $ \cong $ affinen Räumen) verstehen.
	
Algebraisch:
	via Produkte (bilineare -- oder fast bilineare -- Abbildungen).
	
%TODO schönere Grafik...; dann auslagern.
	\definecolor{qqwuqq}{rgb}{0.,0.39215686274509803,0.}
	\definecolor{qqqqff}{rgb}{0.,0.,1.}
	\begin{tikzpicture}[line cap=round,line join=round,>=triangle 45,scale=1.8]
	\clip(0,0) rectangle (10,3);
	\draw [shift={(5.635,1.07)},color=qqwuqq,fill=qqwuqq,fill opacity=0.1] (0,0) -- (34.85:0.14) arc (34.85:143.6:0.14) -- cycle;
	\draw [->] (2.58,1) -- (0.56,2.3);
	\draw [->] (6.6,0.35) -- (4.25,2.1);
	\draw [->] (4.5,0.26) -- (6.8,1.9);

	\draw [fill=qqqqff] (2.58,1) circle (1pt);
	\draw[color=qqqqff] (2.6,1) node[below] {$A$};
	\draw [fill=qqqqff] (0.56,2.3) circle (1pt);
	\draw[color=blue] (0.5,2.3) node[above] {$B$};
	\draw (1.5,0.1) node {Abstand a bis b $ \cong $ Länge b-a};
	\draw (5.4,0.1) node {Winkel $ \cong $ Winkel zwischen Richtungsvektoren};
	\end{tikzpicture}

\section{Bilinearformen \& Sesquilinearformen}
\paragraph{Zur Erinnerung}
	Sind $ V $ und  $W$ $ K $-VR, so nennt man eine Abbildung
		\[ \beta: V\times V\to W \]
	\emph{bilinear} oder ein \emph{Produkt}, wenn sie in jedem Argument linear ist:
		\begin{enumerate}[(i)]
			\item $ \forall w\in V :V\ni v \mapsto \beta(v,w)\in W $ ist linear;
			\item $ \forall v\in V: V\ni w\mapsto \beta(v,w)\in W $ ist linear.
		\end{enumerate}
	Zu vorgegebenen Werten $ \beta_{ij} \in W$ auf einer Basis $ (b_i)_{i\in I} $ von $ V $ existiert dann eine eindeutige Bilinearform $ \beta $ (Fortsetzungssatz Abschnitt 4.3):
		\[ \exists! \beta:V\times V\to W \text{ bilinear}: \forall i,j\in I: \beta(b_i,b_j) = \beta_{ij}. \]
\paragraph{Bemerkung}
	Man kann auch bilineare Abbildungen $ V\times V'\to W $ betrachten und, zum Beispiel, auch einen Fortsetzungssatz beweisen.
	
	Wir benötigen eine Verallgemeinerung in eine andere Richtung:
\subsection{Definition} \index{Sesquilinearform}\index{Semilinearität}
\begin{Definition}[Sesquilinearform]
Seien $ V $ ein $ K $-VR und $ K\ni x\mapsto \overline{x}\in K $ ein (Körper-) Automorphismus, d.h. eine bijektive Abbildung mit
		\[ \overline{x+y} = \overline{x}+\overline{y} \text{ und } \overline{xy} = \overline{x}\cdot \overline{y} \]
	für alle $ x,y\in K $. Eine Abbildung $ \sigma: V\times V \to K $ heißt dann \emph{Sesquilinearform} (bzgl. $ \bar{.} $), falls
		\begin{enumerate}[(i)]
			\item $ \forall v\in V: V\ni w \mapsto \sigma(v,w)\in K $ ist linear, d.h. $ \sigma(v,.)\in V^* $;
			\item $ \forall w\in V: V\ni v \mapsto \sigma(v,w)\in K $ ist \emph{semilinear}, d.h.
				\begin{enumerate}[(a)]
					\item $ \forall v,v' \in V: \sigma(v+v',w) = \sigma(v,w)+\sigma(v',w) $ und
					\item $ \forall v\in V\forall x\in K: \sigma(vx,w) = \overline{x}\sigma(v,w) $.
				\end{enumerate}
		\end{enumerate}
\end{Definition}

\paragraph{Beispiel}
	Die Identität $ K\ni x\mapsto \overline{x}:= x\in K $ ist offensichtlich ein Körperautomorphismus für jeden Körper $ K $. \emph{Bilinearformen} sind genau die Sesquilinearformen bezüglich $ \id_K $.
\paragraph{Beispiel}
	Für $ K = \mathbb{C} $ liefert \emph{komplexe Konjugation} einen Körperautomorphismus (keinen VR-Automorphismus, vgl. Abschnitt 1.4):
		\[ \mathbb{C}\ni x+iy \mapsto \overline{x+iy}:= x-iy \in \mathbb{C}. \]
	Dieses Beispiel ist unser Grund für die Einführung des Begriffs der Sesquilinearform.
\paragraph{Bemerkung}
	Ist $ \sigma $ Bilinearform und Sesquilinearform bezüglich $ \bar{.} $, so ist $ \sigma $ oder $ \bar{.} $ trivial:
		\[ \forall x\in K\forall v,w\in V: 0 = \sigma(vx,w) - \sigma(vx,w) = (x-\overline{x})\sigma(v,w)  \]
		\[ \Rightarrow \begin{cases}
		\forall v,w\in V: \sigma(v,w) = 0 \text{ oder}\\
		\exists v,w\in V: \sigma(v,w)\neq 0 \land \forall x\in K: \overline{x} = x.
		\end{cases} \]
\paragraph{Bemerkung}
	In $ \mathbb{Z}_p, \mathbb{Q} $ und $ \mathbb{R} $ gibt es nur \emph{einen} Körperautomorphismus: $ \id_K $. Ein Automorphismus $ \bar{.} $ von $ \mathbb{C} $ mit $ \overline{\mathbb{R}} = \mathbb{R} $ ist trivial, $ \bar{.} = \id_\mathbb{C} $ oder die komplexe Konjugation.
	
\subsection{Fortsetzungssatz für Sesquilinearformen}
\begin{Satz}[Fortsetzungssatz für Sesquilinearformen]
	Sind $ V $ ein $ K $-VR und $ K\ni x\mapsto \overline{x}\in K $ ein Körperautomorphismus, $ (b_i)_{i\in I} $ Basis von $ V $ und $ (s_{ij})_{i,j\in I} $ eine Familie in $ K $, so existiert eine eindeutige Sesquilinearform $ \sigma $ mit
		\[ \forall i,j\in I:\sigma(b_i,b_j) = s_{ij}. \]
\end{Satz}

% VO 19-04-2016 % 
\paragraph{Beweis}
	Wir imitieren den Beweis unseres ersten Fortsetzungssatzes für lineare Abbildungen:
	
	{Eindeutigkeit:}
	Sei $ \sigma $ eine Sesquilinearform mit der gewünschten Eigenschaft oben; gilt
		\[ v = \sum_{i\in I}b_ix_i \text{ und }w = \sum_{i\in I}b_i y_i \]
	so folgt
		\[ \sigma(v,w) = \sum_{i,j\in I}\overline{x_i}\sigma(b_i,b_j)y_j = \sum_{i,j\in I}\overline{x_i}s_{ij}y_j \]
	d.h. $ \sigma $ ist durch die Familie $ (s_{ij})_{i,j\in I} $ eindeutig bestimmt.
	
	{Existenz:}
	Da jeder Vektor $ v\in V $ eine eindeutige Basisdarstellung $ v=\sum_{i\in I}b_ix_i $ hat, wird durch
	\[ \sigma:V\times V \to K, (v,w)= \left(\sum_{i\in I}b_ix_i, \sum_{j\in I}b_jy_j\right) \]
	\[ \mapsto \sigma(v,w) := \sum_{i,j\in I}\overline{x_i}s_{ij}y_j \]
	eine Abbildung wohldefiniert. Offenbar (nachrechnen) ist $ \sigma $ dann sesquilinear. 

\paragraph{Bemerkung}
	Jede Sesquilinearform $ \sigma: V\times V\to K $ liefert eine semi-lineare Abbildung
		\[ V\ni v\mapsto \sigma(v,.)\in V^*. \]
	Mit einem "`Fortsetzungssatz für semi-lineare Abbildungen"' (Aufgabe 34) hätte man auch den früher skizzierten Beweis für bilineare Abbildungen imitieren können.

\subsection{Buchhaltung}\index{Gramsche Matrix}
\paragraph{Gramsche Matrix}
	Ist $ n=\dim V < \infty $ und $ B=(b_1,\dots,b_n) $ Basis von $ V $, so kann man eine Sesquilinearform $ \sigma: V\times V\to K $ durch eine Matrix $ S $ beschreiben:
	\[ \begin{array}{c|ccc}
	\sigma & b_1 & \dots & b_n \\ \hline
	b_1 & s_{11} &  & s_{1n} \\ 
	\vdots &  & \ddots &  \\ 
	b_n & s_{n1} &  & s_{nn}
	\end{array}  \]
	Diese Matrix
		\[ \Gamma_B(\sigma) = S = \left(\sigma(b_i,b_j)\right)_{i,j\in \{1,\dots,n\}} \]
	heißt die Darstellungsmatrix oder \emph{Gramsche Matrix} von $ \sigma $ bezüglich $ B $. Für Vektoren
		\[ v = \sum_{i=1}^{n}b_ix_i = BX \text{ und } w = \sum_{j=1}^{n}b_jy_j = BY \]
	ist dann
		\[ \sigma(v,w) = \sum_{i,j=1}^{n}\overline{x_i}s_{ij}y_j = \overline{X}^tSY \]
		\[ = (\overline{x_1},\dots,\overline{x_n})\begin{pmatrix}
		\sum_{i=1}^{n}s_{1j}y_j\\ \vdots\\ \sum_{j=1}^{n}s_{nj}y_j
		\end{pmatrix} = \sum_{i=1}^{n}\overline{x_i}\sum_{j=1}^{n}s_{ij}y_j. \]
\paragraph{Transformationsformel}
	Ein Basiswechsel $ B' = BP $ mit $ P = \xi_{B'}^B \in Gl(n)$ liefert dann
		\[ v = BX = (B'P^{-1})X = B'\underset{X'}{\underbrace{(P^{-1}X)}} \text{ und } w = B'\underset{Y'}{\underbrace{(P^{-1}Y)}} \]
	und damit für $ X,Y \in K^{n\times 1} $
		\[ \overline{X}^tSY = \overline{X'}^t\underset{S'}{\underbrace{(\overline{P}^tSP)}}Y' \]
	woraus die \emph{Transformationsformel für Gramsche Matrizen} folgt
		\[ S' = \overline{P}^tSP, \]
	wobei $ \overline{P}^t $ die Transponierte der Matrix mit Einträgen $ \overline{p_{ij}} $ ist.
\paragraph{Äquivalenz von Matrizen}
Dies liefert einen weiteren Äquivalenzbegriff für quadratische Matrizen $ S\in K^{n\times n} $:
		\[ S' \sim S :\Leftrightarrow \exists  P\in Gl(n): S' = \overline{P}^tSP. \]
	Die verschiedenen Begriffe der Äquivalenz von Matrizen (vgl. 3.1 \& 4.2) spiegeln die verschiedenen Funktionen/Bedeutungen von Matrizen wider.
	
\paragraph{Bemerkung}
	Die Menge der Sesquilinearformen auf einem $ K $-VR ist selbst ein $ K $-VR. Ist $ n=\dim V< \infty $ und $ B $ Basis von $ V $, so erhält man (Fortsetzungssatz) einen Isomorphismus
		\[ K^{V\times V}\supset \{\sigma:V\times V\to K \text{ Sesquilinearform}\}\ni \sigma \mapsto \Gamma_B(\sigma)\in K^{n\times n}. \]
\subsection{Beispiel \& Definition} \index{Sesquilinearform!kanonische}\index{Sesquilinearform!assoziierte}
\begin{Definition}[assoziierte Sesquilinearform]
	Sei $ \bar{.}:K\to K $ Körperautomorphismus; jedes $ S\in K^{n\times n} $ liefert dann eine eindeutige Sesquilinearform
		\[ \sigma_S:K^n\times K^n \to K \text{ mit } (e_i,e_j)\mapsto \sigma_S(e_i,e_j):= s_{ij}, \] 
	die zu \emph{$ S $ assoziierte Sesquilinearform}.

	Für $ S = E_n $ bezeichnet man $ \sigma_S $ auch als \emph{kanonische Sesquilinearform}.
\end{Definition}

\subsection{Definition}\index{Sesquilinearform!(schief-)symmetrische}
\begin{Definition}[symmetrische, schiefsymmetrische und alternierende Sesquilinearformen]
	Eine Sesquilinearform $ \sigma:V\times V\to K $ auf einem $ K $-VR bzgl. eines Automorphismus $ \bar{.}:K\to K $ nennen wir
		\begin{enumerate}[(i)]
			\item \emph{symmetrisch}, falls
				\[ \forall v,w\in V: \sigma(w,v) = \overline{\sigma(v,w)} \]
			\item \emph{schiefsymmetrisch}, falls
				\[ \forall v,w\in V: \sigma(w,v) = - \overline{\sigma(v,w)} \]
			\item \emph{alternierend}, falls
				\[ \forall v \in V: \sigma(v,v) = 0. \]
		\end{enumerate}
\end{Definition}

\begin{Definition}[Hermitesche Sesquilinearform]\index{Sesquilinearform!Hermitesche}
	Falls $ K =\mathbb{C} $ und $ \bar{.} $ komplexe Konjugation sind, so nennt man eine symmetrische Sesquilinearform auch \emph{Hermitesche Sesquilinearform}.
\end{Definition}

\paragraph{Bemerkung}
	Ist $ \sigma $ nicht-trivial und (schief-)symmetrisch, so muss $ \bar{.} $ eine Involution sein.
	
	Nämlich: Wähle $ v,w\in V $ mit $ \sigma(v,w) = 1$; dann gilt
		\[ \forall x\in K: \overline{\overline{x}} = \overline{\sigma(vx,w)} = \pm \sigma(w,vx) = \overline{\overline{x}\sigma(v,w)} = \pm \sigma(w,v)x = \overline{\sigma(v,w)}x = x. \]

% VO 21-04-2016 %
	Ist $ \Char(K) \neq 2 $ und $ \bar{.}  $ Involution, so kann jede Sesquilinearform in einen symmetrischen und einen schiefsymmetrischen Anteil zerlegt werden:
		\[ \forall v,w\in V: \sigma(v,w) = \frac{1}{2}\left(\sigma(v,w)+\overline{\sigma(w,v)}\right) +\frac{1}{2}\left(\sigma(v,w)-\overline{\sigma(w,v)} \right).\]
\paragraph{Bemerkung}
	Ist $ \Char(K)\neq 2 $ und $ \bar{.} = \id_K $, so sind "`alternierend"' und "`schiefsymmetrisch"' äquivalent für eine Sesquilinearform $ \sigma $.
	
	Andererseits ist jede alternierende Sesquilinearform bilinear, d.h. $ \bar{.} = \id_K $ oder $ \sigma = 0 $.
	
\paragraph{Buchhaltung}
	Unter den folgenden Annahmen:
		\begin{itemize}
			\item $ \Char(K)\neq 2 $ und $ \bar{.} $ Involution;
			\item $ n=\dim V <\infty $ und $ B $ ist Basis von $ V $;
		\end{itemize}
	gilt für die Gramsche Matrix $ S = \Gamma_B(\sigma) $ einer Sesquilinearform $ \sigma $ auf $ V $:
		\begin{itemize}
			\item $ 0 = \overline{S}^t-S\Leftrightarrow \sigma $ symmetrisch;\footnote{bis auf Faktor 2: Gramsche Matrix des schiefsymmetrischen Anteils von $ \sigma $}
			\item $ 0 = S + \overline{S}^t \Leftrightarrow \sigma $ schiefsymmetrisch.
		\end{itemize}
	Nämlich:
		\[ \overline{S}^t = \begin{pmatrix}
		\overline{\sigma(b_1,b_1)} & \overline{\sigma(b_1,b_2)} &\cdots& \overline{\sigma(b_1,b_n)} \\ 
		\overline{\sigma(b_2,b_1)} &  & & \vdots \\ 
		\vdots &  & & \vdots \\ 
		\overline{\sigma(b_n,b_1)} & \cdots & & \overline{\sigma(b_n,b_n)}
		\end{pmatrix}^t =  \begin{pmatrix}
		\overline{\sigma(b_1,b_1)} & \overline{\sigma(b_2,b_1)} &\cdots& \overline{\sigma(b_n,b_1)} \\ 
		\overline{\sigma(b_1,b_2)} &  & & \vdots \\ 
		\vdots &  & & \vdots \\ 
		\overline{\sigma(b_1,b_n)} & \cdots & & \overline{\sigma(b_n,b_n)}
		\end{pmatrix} \]
		\[ S = \begin{pmatrix}
		\sigma(b_1,b_1) & \sigma(b_1,b_2) &\cdots& \sigma(b_1,b_n) \\ 
		\sigma(b_2,b_1) &  & & \vdots \\ 
		\vdots &  & & \vdots \\ 
		\sigma(b_n,b_1) & \cdots & & \sigma(b_n,b_n)
		\end{pmatrix} \]
		
\subsection{Definition} \index{Orthogonal!-raum}\index{Orthogonal}
\begin{Definition}[orthogonal, Orthogonalraum]
	Sei $ \sigma $ symmetrische Sesquilinearform auf einem Vektorraum $ V $. Zwei Vektoren $ v,w\in V $ heißen \emph{orthogonal} (bzgl. $ \sigma $),
		\[ w \perp v, \text{ falls } \sigma(v,w) = 0. \]
	Der \emph{Orthogonalraum} einer Menge $ \emptyset \neq S\subset V $ ist der UVR
		\[ S^\perp := \bigcap_{s\in S} \ker \underset{\in V^*}{\underbrace{\sigma(s,.)}}. \]
\end{Definition}
\paragraph{Bemerkung}
	Wegen der Symmetrie von $ \sigma $ ist die \emph{Orthogonalitätsrelation} symmetrisch,
		\[ w \perp v \Leftrightarrow v \perp w. \]
\paragraph{Bemerkung}
	Da $ \forall v\in V: \sigma(v,.) \in V^* $, ist der Orthogonalraum wohldefiniert und (als Schnitt von UVR) ein UVR. Offenbar gilt
		\[ \tilde{S} \subset S \Rightarrow \tilde{S}^\perp \supset S^\perp. \]
	Damit folgt direkt $ S^\perp \supset [S]^\perp $, sind andererseits $ w\in S^\perp $ und $ v\in [S] $, so gilt
		\[ v = \sum_{s\in S}sx_s \Rightarrow \sigma(v,w)= \sum_{s\in S}\overline{x_s}\sigma(s,w) = 0, \text{ da } \forall s\in S: w\perp s \]
	d.h. $ w\in S^\perp \Rightarrow w\in [S]^\perp. $ Insgesamt ist also
		\[ \forall S \subset V: [S]^\perp=S^\perp.\]
	Ähnlich zeigt man für jede Familie $ (U_i)_{i\in I} $ von UVR $ U_i\subset V $:
		\[ \left(\sum_{i\in I}U_i \right)^\perp= \bigcap_{i\in I} U_i^\perp. \]
\paragraph{Bemerkung \& Beispiel}
	Für $ S\subset V $ kann man $ S^{\perp\perp} = \left(S^\perp\right)^\perp $ betrachten; im Allgemeinen gilt
		\[ S\subset S^{\perp\perp} \text{ aber } S\neq S^{\perp\perp}. \]
	Ist etwa $ \sigma = 0 $, so ist $ S^\perp = V $ für jede Menge $ \emptyset \neq S\subsetneq V $; also ist
		\[ S^{\perp\perp} = V^\perp s = V \neq S. \]

\subsection{Definition}\index{Radikal!-raum}\index{Radikal!-frei}
\begin{Definition}[Radikal(-raum),radikalfrei,nicht-degeneriert,degeneriert]
$ V^\perp $ ist der \emph{Radikal(-raum)} eines VR mit symmetrischer Sesquilinearform $ \sigma $; ist $ V^\perp = \{0 \} $, so heißt $ \sigma $ \emph{radikalfrei} oder \emph{nicht-degeneriert}, andernfalls \emph{degeneriert}.
\end{Definition}
\paragraph{Beispiel}
	Betrachte $ V=\mathbb{R}^2 $ mit Standardbasis $ (e_1,e_2) $.
	
	Ist für eine symmetrische Sesquilinearform (Bilinearform) $ \sigma $ auf $ V $
		\[ \sigma(e_1,e_1) = 0, \sigma(e_1,e_2) = 1, \sigma(e_2,e_2) = 0 \]
	so ist $ \sigma $ nicht-degeneriert, $ V^\perp = \{0\} $, da
		\[ v = e_1x_1 + e_2x_2 \perp e_1,e_2 \Rightarrow
		\begin{cases}
		0 = \sigma(e_1,v) = x_2\\
		0 = \sigma(e_2,v) = x_1
		\end{cases} \]
		\[ \Rightarrow x_1 = x_2 = 0, \]
	also $ V^\perp = \{0\} $, d.h. $ \sigma $ ist nicht-degeneriert.
	
	% Grafik R^2
	
	Ist aber
		\[ \sigma(e_1,e_1) = 1, \sigma(e_1,e_2) = 1, \sigma(e_2,e_2) = 1, \]
	so ist $ V^\perp = [e_1-e_2] $, d.h. $ \sigma $ ist degeneriert.
	
	% Grafik R^2 mit UVR [e_1-e_2]-> Gerade y = -x

\subsection{Lemma}
\begin{Lemma}[]
	Ist $ U\subset V $ ein zum Radikal von $ (V,\sigma) $ komplementärer UVR, $ V = V^\perp \oplus U $, so ist
		\[ \sigma\big|_{U\times U}:U\times U \to K \]
	radikalfrei.
\end{Lemma}
\paragraph{Beweis}
	Sei $ u\in U $ im Radikal von $ (U,\sigma\big|_{U\times U}) $, d.h. es gelte $ \forall v\in U: \sigma(v,u) = 0 $.
	Weil
		\[ \forall v\in V^\perp\forall w\in V: v\perp w \Rightarrow \forall v\in V^\perp: v\perp u \]
	erhalten wir $ u\in U\cap V^\perp = \{0\} $.
\paragraph{Beispiel}
	Die Einschränkung von $ \sigma $ mit (wie oben)
		\[ \forall i,j \in \{1,2\}: \sigma(e_i,e_j) = 1 \]
	auf jeden UVR $ U = [e_1x_1 + e_2x_2] $ mit $ x_1 + x_2 \neq 0 $ ist radikalfrei, denn
			\[ \sigma(e_1x_1+e_2x_2, e_1x_1+e_2x_2) = (x_1+x_2)^2 \neq 0. \]
	
% VO 26-04-2016 %
\section{Der Satz von Sylvester}
\paragraph{Beispiel}
	Ist $ \sigma $ symmetrische Sesquilinearform auf $ V=\mathbb{Z}^2_2 $ mit
		\[ \sigma(e_1,e_1)=0, \sigma(e_1,e_2) = 1, \sigma(e_2,e_2) = 0, \]
	so ist $ \sigma $ (wie vorher) nicht-degeneriert, $ V^\perp =\{0\}$; trotzdem gilt
		\[ \forall v\in V: \sigma(v,v) = 0. \]
	Das folgende Lemma zeigt, dass dies ein degenerierter Fall ist:
	
\subsection{Lemma \& Definition (Polarisation)}\index{quadratische Form}
	Ist $ \sigma $ symmetrische Bilinearform auf einem $ K $-VR $ V $ über einem Körper $ K $ mit $ \Char K \neq 2 $, so gilt
		\[ \forall v,w\in V: \sigma(v,w)=\frac{1}{2}\left(q(v+w)-q(v)-q(w)\right), \]
	wobei
		\[ q:V\to K, v\mapsto q(v):= \sigma(v,v) \]
	die zu $ \sigma $ gehörige \emph{quadratische Form} bezeichnet.
\paragraph{Beweis}
	Ausrechnen: sind $ v,w\in V $, so gilt
	\begin{align*}
	q(v+w) &= \sigma(v+w,v+w)\\
			&= \sigma(v,v) + \sigma(v,w)+\sigma(w,v)+\sigma(w,w)\\
			&= q(v)+2\sigma(v,w)+q(w)
	\end{align*}
	diese Gleichung kann (da $ \Char K\neq 2 $) nach $ \sigma(v,w) $ aufgelöst werden.
\paragraph{Bemerkung}
	Ist $ \Char K=0 $ so kann man statt
		\[ q(v+w)=q(v)+2\sigma(v,w)+q(w) \]
	auch
		\[ q(v+w)-q(v-w) = 4 \sigma(v,w) \]
	für die Polarisation verwenden.
	
\subsection{Lemma}
	Ist $ \sigma $ symmetrische Sesquilinearform auf einem $ K $-VR $ V $ über einem Körper $ K $ mit $ \Char K \neq 2 $, so gilt
		\[ \sigma = 0 \Leftrightarrow \forall v\in V: \sigma(v,v) = 0. \]
\paragraph{Bemerkung}
	Im Falle einer Bilinearform folgt dies direkt mit Polarisation.
	
	Im Falle eines nicht-trivialen Körperautomorphismus $ \bar{.} $ liefert $ v\mapsto \sigma(v,v) $ wegen
		\[ K\ni x \mapsto \sigma(vx,vx)-x^2\sigma(v,v) = (\overline{x}x-x^2)\sigma(v,v)\neq 0 \]
	im Allgemeinen \emph{keine} quadratische Form:
		\[ \exists x\in K: \exists v\in V: \sigma(vx,vx) = \overline{x}\sigma(v,v)x \neq x^2\sigma(v,v). \]
\paragraph{Beweis}
	Ist $ \sigma = 0 $, so folgt trivialerweise
		\[ \forall v\in V: \sigma(v,v) = 0. \]
	Sei nun $ \sigma \neq 0 $, d.h.
		\[ \exists v,w\in V: \sigma(v,w)\neq 0. \]
	Wie vorher berechnet man für $ v,w\in V $
		\[ \sigma(v+w,v+w) = \sigma(v,v)+\sigma(v,w)+\overline{\sigma(v,w)}+\sigma(w,w). \]
	Wähle nun $ v,w\in V $ mit $ \sigma(v,w)\neq 0 $, o.B.d.A. $ \sigma(v,w) = 1 $.
	\footnote{Ggf. ersetzt man $ w $ durch $ \frac{w}{\sigma(v,w)} $.}
	Ist $ \sigma(v,v) \neq 0 $ oder $ \sigma(w,w)\neq 0 $, so sind wir fertig.
	
	Gilt jedoch $ \sigma(v,v) = \sigma(w,w) = 0 $, so liefert
		\[ \sigma(v+w,v+w) = 0 + 1 + 1 + 0 \neq 0 \]
	wieder die Behauptung, da $ \Char K \neq 2 $.
\paragraph{Vereinbarung}
	Im Folgenden schließen wir $ \Char K = 2 $ aus.
	
\subsection{Lemma}
	Für eine symmetrische Sesquilinearform $ \sigma $ auf $ V $ und $ b\in V $ mit $ \sigma(b,b)\neq 0 $ gilt
		\[ V = [b]\oplus \{b\}^\perp. \]
\paragraph{Beweis}
	Es gilt $ V = [b]+\{b\}^\perp $, da für $ v\in V $
		\[ v = u + b\frac{\sigma(b,v)}{\sigma(b,b)} \text{ mit } u := v-b\frac{\sigma(b,v)}{\sigma(b,b)}\perp b;\footnote{denn $ \sigma(b,u) = \sigma(b,v-b\frac{\sigma(b,v)}{\sigma(b,b)}) = \sigma(b,v)-\sigma(b,b)\frac{\sigma(b,v)}{\sigma(b,b)} = 0 $} \]
	ist $ v\in [b]\cap \{b\}^\perp $, so gilt
		\[ v = bx \text{ für ein }x\in K \text{ und} \]
		\[ 0 = \sigma(b,v) = \sigma(b,bx) = \sigma(b,b)x \]
			\[ \Rightarrow x = 0 \land v = 0, \]
	d.h. $ [b]\cap \{b\}^\perp = \{0\}$ und damit folgt die Behauptung.
\paragraph{Bemerkung}
	Ist $ \sigma(b,b) = 0 $ für $ b \in V $, so gilt
		\[ b\in [b]\cap \{b\}^\perp, \]
	d.h. ist $ b\neq 0 $, so ist $ [b]\cap \{b\}^\perp \neq \{0\}$. Außerdem ist dann $ \sigma\big|_{U\times U} $ für $ U:= \{b\}^\perp $ degeneriert, da
		\[ \exists v = b \in U^\times \forall u\in U: u\perp b.\]
		
\subsection{Diagonalisierungslemma}
	Zu jeder symmetrischen Sesquilinearform $ \sigma $ auf einem endlichdimensionalen VR $ V $, also $ n = \dim V < \infty $,  gibt es eine Basis $ B = (b_1,\dots,b_n) $ von $ V $, die $ \sigma $ \emph{diagonalisiert}, d.h. für die gilt
		\[ \sigma(b_i,b_j)=0, \text{ falls } i\neq j. \]
\paragraph{Beweis}
	Durch Induktion über $ n $.
	
	Für $ n = 1 $ ist die Behauptung trivial (denn $ i\neq j $ existiert nicht).
	
	Sei die Behauptung also für $\dim V = n $ bewiesen. Ist $ \sigma $ symmetrische Sesquilinearform auf $ V $ mit $ \dim V = n+1 $ und o.B.d.A. $ \sigma \neq 0 $, also
		\[ \exists b\in V: \sigma(b,b) \neq 0 \]
	nach obigem Lemma lässt sich also $ V $ aufspalten in 
		\[ V = [b]\oplus U \text{ mit } U:=\{b\}^\perp \]
	und $ \dim U = n $. Nach Annahme existiert eine Basis $ (b_1,\dots,b_n) $ von $ U $, die $ \sigma|_{U\times U} $ diagonalisiert. Da $ b \perp b_1,\dots,b_n \in U$ liefert $ B := (b,b_1,\dots,b_n) $ eine $ \sigma $-diagonalisierende Basis von $ V $. 
\paragraph{Bemerkung}
	Ist $ B = (b_1,\dots,b_n) $ eine $ \sigma $-diagonalisierende Basis, also
		\[ s_{ij} = \sigma(b_i,b_j) = 0 \text{ für } i\neq j \]
	so ist
		\[ \sigma(v,v) = \sum_{i=1}^{n}\overline{x_i}s_{ii}x_i \text{ für } v = \sum_{i=1}^{n}b_ix_i. \]
	Sind $ a_1,\dots,a_n\in K^\times $ und $ b_i' = b_ia_i $, so zeigt
		\[ s_{ij}' = \sigma(b_i',b_j') = \overline{a_i}\sigma(b_i,b_j)a_j = \overline{a_i}s_{ij}a_j, \]
	dass $ B' = (b_1',\dots,b_n') $ eine weitere $ \sigma $-diagonalisierende Basis ist.
	Man kann also die $ s_{ii} $ "`adjustieren"', sofern man die (unabhängigen) Gleichungen
		\[ s_{ii}' = \overline{a_i}s_{ii}a_i \]
	für gegebene $ s_{ii}' $ (nach den $ a_i $) lösen kann. Zum Beispiel:
\subsection{Korollar}
	Ist $ \sigma $ symmetrische Bilinearform auf einem $ \mathbb{C} $-VR $ V $ mit $ \dim V < \infty $, so besitzt $ V $ eine Basis $ B = (b_1,\dots,b_n) $, sodass
		\[ \exists r\in \mathbb{N}: s_{ij} = \sigma(b_i,b_j) = \begin{cases}
		1 & \text{für } i = j \leq r\\
		0 & \text{sonst}.
		\end{cases} \]
		
% VO 28-04-2016 %
\paragraph{Bemerkung}
	D.h.
		\[ \begin{pmatrix}
		E_r & 0 \\ 0 & 0
		\end{pmatrix} \]
\paragraph{Beweis}
	Sei (nach Diagonalisierungslemma) $ B' = (b_1',\dots,b_n') $ eine $ \sigma$-diagonalisierende Basis von $ V $; durch Umsortierung der Basisvektoren kann man erreichen, dass
		\[ s_{11}' ,\dots, s_{rr}' \neq 0 \text{ und } s_{r+1,r+1}' = \dots = s_{nn}' = 0 \]
	für ein $ r\in \{0,\dots,n\} $. Mit einer Wahl der Wurzel bilden die Vektoren 
		\[ b_i := \begin{cases}
		{b_i'}\cdot \frac{1}{\sqrt{s_{ii}'}} & \text{ für } i = 1,\dots,r\\
		b_i' = 0 & \text{ für } i = r+1,\dots,n 
		\end{cases} \]
	dann eine Basis $ B $ mit der gewünschten Eigenschaft:
		\[ \sigma(b_i,b_i) = \underset{s_{ii}'}{\underbrace{\sigma(b_i',b_i')}} \cdot \left(\frac{1}{\sqrt{s_{ii}'}}\right)^2 = 1 \text{ für } i = 1,\dots,r\]
		\[ \sigma(b_i,b_i) = \sigma(b_i',b_i') = 0 \text{ für } i = r+1,\dots, n. \]
		
\subsection{Korollar}
	Ist $ V $ ein $ K $-VR mit $ \dim V <\infty $ und $ \sigma $ entweder
		\begin{itemize}
			\item symmetrische Bilinearform, wenn $ K=\mathbb{R} $, oder
			\item Hermitesche Sesquilinearform, wenn $ K = \mathbb{C} $,
		\end{itemize}
	so besitzt $ V $ eine Basis $ B = (b_1,\dots,b_n) $, sodass
		\[ \exists r\in \mathbb{N}: s_{ij} = \sigma(b_i,b_j) =
		\begin{cases}
			\pm 1 & \text{ für }i = j \leq r\\
			0 & \text{ sonst}
		\end{cases} \]
\paragraph{Beweis}
	Wie oben -- aber:
	In diesen beiden Fällen gilt für eine diagonalisierende Basis $ B'=(b_1',\dots,b_n') $ und $ b_i = b_i'\cdot \frac{1}{a_i} $ mit $ a_i\in K $ für $ i=1,\dots,n $:
		\[ s_{ii}' = \sigma(b_i',b_i')\in \mathbb{R} \text{ und }
		\begin{cases}
			a_i^2 \geq 0 & \text{falls } K = \mathbb{R},\\
			\overline{a_i}a_i \geq 0 & \text{falls } K =\mathbb{C}.
		\end{cases} \]
	Also kann man die $ s_{ii}' $ (nur) positiv reskalieren und so $ s_{ii} = 0 $ oder $ s_{ii} = \pm 1 $ erreichen.
\paragraph{Notation}
	Im Folgenden bezeichnet $ \mathbb{K} $ entweder $ \mathbb{K} = \mathbb{R} $ oder $ \mathbb{K}=\mathbb{C} $.
\paragraph{Motivation}
	Für die obige Basis $ B $ von $ V $ mit den Eigenschaften des Korollars gilt offenbar:
		\[ v\perp b_1,\dots,b_r \Rightarrow v\in [\{b_{r+1},\dots,b_{n}\}] \]
	und
		\[ b_{r+1},\dots,b_n \perp V, \]
	also ist $ (b_{r+1},\dots,b_n) $ Basis des Radikalraums $ V^\perp $ von $ (V,\sigma) $,
		\[ V^\perp = [\{b_{r+1},\dots,b_n\} ] \Rightarrow r = \dim V-\dim V^\perp. \]
	Insbesondere ist $ \dim V^\perp $ und damit $ r $ unabhängig von der Basis $ B $.

\subsection{Satz von Sylvester}\index{Signatur}
	Sei $ V $ ein $ \mathbb{K} $-VR, $ \dim V <\infty $, und $ \sigma $
		\begin{itemize}
			\item symmetrische Bilinearform, wenn $ \mathbb{K}=\mathbb{R} $, oder
			\item Hermitesche Sesquilinearform, wenn $ \mathbb{K}=\mathbb{C} $.
		\end{itemize}
	Dann gibt es eine direkte Zerlegung von $ V $ mit UVR $ V_{\pm}\subset V $,
		\[ V= V_+ \oplus_\perp V_- \oplus_\perp V^\perp,  \]
	wobei
		\[ V_+ \perp V_- \text{ und } \forall v\in V^\times_\pm: \pm \sigma(v,v) > 0. \]
	Die \emph{Signatur} $ \sgn(\sigma):= (\dim V_+,\dim V_-,\dim V^\perp) $ von $ \sigma $ ist unabhängig von der direkten Zerlegung von $ V $.
\paragraph{Bemerkung \& Definition}\index{Trägheitsindex}\index{Positivitätsindex}\index{Negativitätsindex}
	Ist $ \sigma $ nicht-degeneriert, $ V^\perp \{0\} $, so bezeichnet man auch\footnote{Die Reihenfolge kann bei verschiedenen Autoren auch jeweils $ - $ vor $ + $ sein.}
		\begin{itemize}
			\item das Paar $ \sgn (\sigma) =(\dim V_+,\dim V_-)$ als Signatur von $ \sigma $, und
			\item die Differenz $ \dim V_+ - \dim V_- $ als \emph{Trägheitsindex} von $ \sigma $.
		\end{itemize}
	Die Dimension $ \dim V_\pm $ ist auch der \emph{Positivitäts-} bzw. \emph{Negativitätsindex} von $ \sigma $.
	
	Der Satz von Sylvester wird auch "`Trägheitssatz von Sylvester"' genannt.
\paragraph{Beweis}
	Sei $ B=(b_1,\dots,b_n) $ eine Basis von $ V $ und $ p,r\in \mathbb{N} $, sodass (siehe Korollar oben)
		\[ \sigma(b_i,b_j) =
		\begin{cases}
			+1 & \text{ für } 0 < i=j\leq p\\
			-1 & \text{ für } p < i=j\leq r\\
			0 & \text{ sonst. }
		\end{cases} \]
	Mit
		\[ V_+ := [\{b_1,\dots,b_p \}] \text{ und } V_- := [\{b_{p+1},\dots,b_r\}] \]
	erhält man die gewünschte direkte orthogonale Zerlegung von $ V $,
		\[ V = V_+ \oplus_\perp V_- \oplus_\perp V^\perp. \]
	Zur Eindeutigkeit der Signatur $ \sgn(\sigma) = (p,r-p,n-r) $:
	
	Seien
	\[ V=V_+ \oplus_\perp V_- \oplus_\perp V^\perp
	= \tilde{V}_+\oplus_\perp\tilde{V}_-\oplus_\perp\tilde{V}^\perp \]
	direkte orthogonale Zerlegungen von $ V $ mit
		\[ \pm \sigma(v,v)>0 \text{ für }
			\begin{cases}
				v\in V_\pm^\times\\
				v\in \tilde{V}_\pm^\times.
			\end{cases} \]
	Nun gilt
	\begin{align*}
		&\forall v\in V_-^\times: \sigma(v,v)< 0 \\
		&\Rightarrow \forall v\in V_- \oplus V^\perp: \sigma(v,v) \leq 0
	\end{align*}
	und damit, da $ \sigma(v,v)>0 $ für $ v\in \tilde{V}_+^\times $,
		\[ v\in (V_-\oplus V^\perp)\cap \tilde{V}_+ \Rightarrow v= 0. \]
	Es folgt, mit dem Dimensionssatz, $ \tilde{p}\leq p $, da
		\[ \tilde{p}+(n-p) = \dim \tilde{V}_+ + \dim(V_-\oplus V^\perp) \leq \dim V = n. \]
	Vertauscht man die Rollen der Zerlegungen, so erhält man die Ungleichung $ p\leq \tilde{p} $ und damit also
		\[ p = \tilde{p}. \]
\paragraph{Bemerkung}
	Diese Zerlegung $ V= V_+ \oplus_\perp V_- \oplus_\perp V^\perp $ ist im Allgemeinen \emph{nicht} eindeutig!
\paragraph{Beispiel}
	Betrachte eine durch ihre Werte auf der Standardbasis $ E=(e_1,e_2) $ gegebene symmetrische Bilinearform $ \sigma: \mathbb{R}^2\times \mathbb{R}^2 \to \mathbb{R} $.
		\begin{enumerate}
			\item $ S=(\sigma(e_i,e_j))_{i,j\in \{1,2\}} =
			\begin{pmatrix}
				0&1\\1& 0
			\end{pmatrix} $. Mit $ P:=\begin{pmatrix}
			1&1\\ 1& -1
			\end{pmatrix}\in Gl(2) $ liefert ein Basiswechsel $ B=EP $
				\[ (\sigma(b_i,b_j))_{i,j\in \{1,2\}} = P^tSP = \begin{pmatrix}
				2 & 0 \\ 0 & -2
				\end{pmatrix} \]
			die Signatur $ \sgn(\sigma) = (1,1,0)\cong (1,1) $.
			Jeder weitere Basiswechsel
				\[ \tilde{B}=BQ \text{ mit } Q = \begin{pmatrix}
				\cosh(s) & \sinh(s)\\ \sinh(s)& \cosh(s) 
				\end{pmatrix}, s\in \mathbb{R}, \]
			liefert eine andere Zerlegung, ohne die Gramsche Matrix zu ändern.
			\item $ S=(\sigma(e_i,e_j))_{i,j\in \{1,2\}}= \begin{pmatrix}
			1 & 1\\ 1 & 1
			\end{pmatrix}. $ Der Basiswechsel $ B=EP $ wie oben liefert hier
				\[ (\sigma(b_i,b_j))_{i,j\in \{1,2\}} = P^tSP = \begin{pmatrix}
				4 & 0 \\ 0 & 0
				\end{pmatrix}, \]
			also die Signatur $ \sgn(\sigma) = (1,0,1) $ von $ \sigma $. Hier ist $ V^\perp = [\{b_2\}]$ durch $ \sigma $ festgelegt, aber jeder Basiswechsel
				\[ \tilde{B} = BQ \text{ mit } Q= \begin{pmatrix}
				1 & 0 \\ s & 1
				\end{pmatrix}, s\in \mathbb{R} \]
			ändert die der Basis zugeordnete Zerlegung -- wieder ohne Änderung der Gramschen Matrix.
		\end{enumerate}
\section{Euklidische \& unitäre Vektorräume}

% GRAFIK-MOTIVATION %
	% (A,V,\tau) reelle affine Ebene
	% (o; e_1,e_2) affines Bezugssystem
	% Abstand/Länge von b-a ist (falls e_1 \perp e_2 und |e_1| = |e_2| = 1):
	% d(a,b) = \|b-a\| = \sqrt{(y_1-x_1)^2+(y_2-x_2)^2}

\paragraph{Bemerkung}
	Die folgende Definition ist nur für Skalarprodukte $ \Skl{.}{.} $ sinnvoll, für die
		\[ v\mapsto \Skl{ v}{v} \in T \]
	mit einem angeordneten Teilkörper $ T \subset K $ des Körpers $ K $ (vgl. Abschnitt 1.2).
	Ein nicht-triviales Beispiel, mit $ T=\R\subset \C = K $, ist ein Hermitesches Skalarprodukt:
		\[ \forall v\in V: \Skl{ v}{v} = \overline{\Skl{ v}{v}} \Rightarrow \forall v\in V: \Skl{ v}{v} \in \R\subset \C. \]
\paragraph{Vereinbarung}
	Im Folgenden beschränken wir uns bis auf Weiteres auf $ \K $-VR mit $ \Skl{.}{.} $ Hermitsche Sesquilinearform, falls $ \K = \C $ (vgl. Satz von Sylvester).
	
\subsection{Definition}\index{positiv definit}\index{induzierte Norm}\index{Vektorraum!Euklidischer}\index{Vektorraum!unitärer}
\begin{Definition}[positiv definit, euklidischer-, unitärer Vektorraum ]
	Ein Skalarprodukt $ \Skl{.}{.} $ auf einem $ \K $-VR $ V $ heißt \emph{positiv definit}, falls
		\[ \forall v\in V^\times:\Skl{ v}{v} >0; \]
	die \emph{induzierte Norm} eines positiv-definiten Skalarprodukts $ \Skl{.}{.} $ ist die Abbildung
		\[ \|.\|: V\to \R, v\mapsto \|v\| := \sqrt{\Skl{ v}{v}}\geq 0. \]
	Ein $ \K $-VR $ (V,\Skl{.}{.} ) $ mit positiv-definitem Skalarprodukt ist
		\begin{itemize}
			\item ein \emph{Euklidischer Vektorraum}, falls $ \K=\R $, und
			\item ein \emph{unitärer Vektorraum}, falls $ \K = \C $ und $ \Skl{.}{.} $ Hermitesche Sesquilinearform ist. 
		\end{itemize}
\end{Definition}

\subsection{Bemerkung \& Definition}
\begin{Definition}[negativ definit, indefinit]
	Ebenso definiert man ein Skalarprodukt als \emph{negativ definit}, falls
		\[ \forall v\in V^\times: \Skl{ v}{v} < 0; \]
	$ \Skl{.}{.} $ heißt \emph{indefinit}, falls es weder positiv, noch negativ definit ist.
	Die Definition der induzierten Norm ist nur im positiv definiten Fall sinnvoll.
\end{Definition}

% VO 10-05-2016 %

\paragraph{Beispiel}	
	Der \emph{Betrag} einer komplexen Zahl
		$ z=x+iy\in \C\cong_\R \R^2 $
			% Vektorraum-Isomorphismus - R^2 ist kein Körper! %
	ist 
		\[ |z| = \sqrt{\overline{z}z} = \sqrt{x^2+y^2} \]
	die vom Standardskalarprodukt auf $ \R^2 $ induzierte Norm.
	Insbesondere gilt
		\[ \forall z\in \C: |\Re z| \leq |z| \]	
\subsection{Bemerkung zum Zusammenhang von reellen und komplexen VR}
	Da $ \R\subset \C $ ein Teilkörper ist, kann jeder $ \C $-VR $ V $ auch als $ \R $-VR aufgefasst werden (Einschränkung der Skalarmultiplikation).
	
	Ist nun $ S\subset V $ linear unabhängig über $ \C $ (in $ V $ als $ \C $-VR), so ist
		\[ S' := S\cup Si = S \cup \{si\mid s\in S\} \]
	linear unabhängig über $ \R $, denn
		\[ 0 = \sum_{s\in S}sx_s + \sum_{s\in S}siy_s = \sum_{s\in S} s (x_s+iy_s) \]
		\[ \Rightarrow \forall s\in S: x_s + iy_s = 0\Rightarrow \forall s\in S: x_s = y_s = 0, \]
	d.h. $ S' = S\cup S_i $ ist linear unabhängig über $ \R $.
	Insbesondere folgt
		\[ \dim_\R V = 2\dim_\C V.  \]

	Weiters definiert für ein Hermitesches Skalarprodukt $ \Skl{ .}{. } $ auf $ V $ (als $ \C $-VR)
		\[ \Skl{ .}{. } :V\times V\to \R, (v,w)\mapsto \Skl{ v}{w }_\R := \Re \Skl{v}{w } \]
	ein reelles Skalarprodukt auf $ V $ (als $ \R $-VR), das genau dann positiv definit ist, wenn $ \Skl{.}{. } $ positiv definit ist:
		\[ \forall v\in V: \Skl{v}{v }_\R = \Skl{v}{v }. \]
	Damit kann man jeden unitären Vektorraum als Euklidischen Vektorraum auffassen:
		\begin{itemize}
			\item mit verschiedenen Skalarprodukten $ \Skl{.}{.} $ bzw. $ \Skl{.}{.}_\R $, aber
			\item mit gleichen induzierten Normen. % denn diese sind ohnehin immer reell
		\end{itemize}
\index{Komplexifizierung}
\paragraph{Komplexifizierung}
	Fasst man einen $ \C $-VR $ V $ als $ \R $-VR auf, so liefert Multiplikation mit $ i\in \C $ einen Endomorphismus
		\[ J:V\to V, v\mapsto J(v):= vi \] % Skalarmultiplikation in $ V $ als $ \C $-VR.
	mit
		\[ J^2 = -\id_V. \]
	Insbesondere besitzt $ J $ keine reellen Eigenwerte;
	ist $ \dim V < \infty $, so folgt damit
		\[ \dim V = \deg{\chi_J}(t) = 0 \mod 2. \]
	Umgekehrt: Ist $ V $ ein $ \R $-VR und $ J\in \End(V) $ mit $ J^2 = -\id_V $ gegeben, so erhält man eine komplexe Skalarmultiplikation
		\[ \cdot :\C\times V \to V, (z,v)\mapsto vz := vx+J(v)y, \]
	für $ z = x+iy $.
	Ist weiter $ \Skl{.}{.} $ ein (reelles) Skalarprodukt auf $ V $, das von $ J $ erhalten wird, 
		\[ \forall v,w\in V: \Skl{Jv}{Jw} = \Skl{v}{w}, \]
	so definiert
		\[ \Skl{v}{w}_\C := \Skl{v}{w} -i \Skl{v}{Jw} \]
	ein Hermitesches Skalarprodukt auf dem so konstruierten $ \C $-VR.

\paragraph{Beispiel}\label{JDrehung}
	Ist $ \Skl{.}{.} $ das kanonische Skalarprodukt auf $ \R^2 $, mit der Standardbasis $ (e_1,e_2) $ als ONB, so definiert (Fortsetzungssatz)
		\[ J(e_1) = e_2 \text{ und } J(e_2) = -e_1, \]
	einen Endomorphismus $ J\in\End(\R^2) $ mit
		\[ J^2 = -\id_{\R^2} \text{ und } \Skl{Je_i}{Je_j} = \Skl{e_i}{e_j}. \]
	Vermöge
		\[ e_1i := J(e_1) = e_2\ \text{ und }\ e_2i := J(e_2) = J^2(e_1) = -e_1 = e_1 i^2 \]
	wird $ \R^2 $ zu einem eindimensionalen $ \C $-VR, $ \R^2 = [\{e_1\}]_\C $, da
		\[ e_1x+e_2y = e_1x+J(e_1)y = e_1 (x+iy); \]
	und
		\[ \Skl{e_1x+e_2y}{e_1x'+e_2y'}_\C = \Skl{e_1(x+iy)}{e_1(x'+iy')}_\C = \overline{(x+iy)}(x'+iy') \]
	liefert das kanonische Skalarprodukt auf $ \C $, mit dem kanonischen Euklidischen Skalarprodukt von $ \R^2 $ als Realteil.

\subsection{Komplexifizierungslemma}
\begin{Lemma}[Komplexifizierungslemma]
	Ist $ (V,\Skl{.}{.}) $ ein Euklidischer Vektorraum, so liefert
		\[ (v,w)(x+iy) := (vx-wy,wx+vy) \]
	eine komplexe Skalarmultiplikation auf $ V_\C := V\times V $, und
		\[ \SSkl{((v,w))}{(v',w')}_\C := \left(\Skl{v}{v'}+\Skl{w}{w'} \right)+i\left(\Skl{v}{w'}-\Skl{w}{v'} \right) \]
	ein Hermitesches Skalarprodukt, das $ (V_\C,\SSkl{.}{.}_\C) $ zu einem unitären VR macht.
\end{Lemma}

\paragraph{Beweis}
	Auf dem Euklidischen VR $ (V^2,\SSkl{.}{.}) $, wobei
		\[ \SSkl{.}{.}:V^2\times V^2 \to \R, \left((v,w),(v',w')\right)\mapsto \SSkl{(v,w)}{(v',w')} := \Skl{v}{v'}+\Skl{w}{w'}, \]
	definiere $ J\in \End(V^2) $ durch
		\[ J:V^2 \to V^2, (v,w)\mapsto J\left((v,w)\right):= (-w,v). \]
	Offenbar gilt $ J^2 = -\id_{V^2} $ und
		\[ \SSkl{J(v,w)}{J(v',w')} = \Skl{w}{w'}+\Skl{v}{v'} = \SSkl{(v,w)}{(v',w')}, \]
	sodass
		\[ (v,w)(x+iy) = (v,w)x+J(v,w)y= (vx-wy,wx+vy) \]
	und
		\[ \SSkl{(v,w)}{(v',w')}_\C = \SSkl{(v,w)}{(v',w')}-i\SSkl{(v,w)}{J(v',w')}  \]
		\[ = \left(\Skl{v}{v'}+\Skl{w}{w'}\right) -i\left(-\Skl{v}{w'}+\Skl{w}{v'}\right) \]
	$ (V^2,\SSkl{.}{.}) $ zu einem unitären VR machen, wie vorher.
\paragraph{Bemerkung}
	Mit dem "`Komplexifizierungslemma"' kann man jeden Euklidischen VR in einen unitären VR gleicher (komplexer) Dimension einbetten:
		\[ \dim_\C V_\C = \frac{1}{2}\dim_\R V^2 = \dim_\R V. \]
	Wichtig für den Zusammenhang zwischen unitären und Euklidischen VR:
	Die induzierte Norm des Hermitschen Skalarprodukts kann als die eines Euklidischen Skalarprodukts aufgefasst werden.
	
% VO 12-05-2016 %
\subsection{Definition}\index{Norm}
\begin{Definition}[Norm , normierter Vektorraum ]
	Eine Abbildung $ \|.\|:V\to \R $ auf einem $ \K $-VR $ V $ heißt \emph{Norm}, falls
	\begin{enumerate}[(i)]
		\item $ \forall v\in V^\times: \|v\| > 0 $, d.h. $ \|.\| $ ist \emph{positiv definit};
		\item $ \forall v\in V \forall x\in \K: \|vx\| = \|v\|\cdot |x| $, d.h. $ \|.\| $ \emph{positiv homogen};
		\item $ \forall v,w\in V: \| v+w\|\leq \|v\| + \|w\| $, d.h. $ \|.\| $ erfüllt die \emph{Dreiecksungleichung}.
	\end{enumerate}
	Ein Vektorraum mit Norm, $ (V,\|.\|) $ heißt \emph{normierter Vektorraum}.
\end{Definition}

\paragraph{Bemerkung}
	Die von einem positiv definiten Skalarprodukt $ \Skl{.}{.} $ induzierte Norm $ \|.\| $ erfüllt offenbar (i) und (ii); die Dreiecksungleichung zeigen wir unten.
\paragraph{Cauchy-Schwarzsche Ungleichung}
\begin{Satz}[Cauchy-Schwarzsche Ungleichung]
	Ist $ (V,\Skl{.}{.}) $ Euklidisch oder unitär, so gilt\footnote{Im Euklidischen Fall ist der Betrag offenbar überflüssig.}
		\[ \forall v,w\in V: |\Skl{v}{w}|^2 \leq \Skl{v}{v}\Skl{w}{w} \]
\end{Satz}
\paragraph{Beweis}
	Seien $ v,w\in V $, o.B.d.A $ v\neq 0 $. Wir bestimmen das Minimum der Funktion im Euklidischen Fall (unitärer Fall in der Übung)
		\[ \R \ni s\mapsto g(s):= \Skl{vs-w}{vs-w}. \]
	Einsetzen des kritischen Punktes,
		\[ 0 = g'(s) = 2\Skl{v}{v}s -(\Skl{v}{w}+\Skl{w}{v}) = 2(\Skl{v}{v}s-\Re \Skl{v}{w}) \]
		\[ \Rightarrow s = \frac{\Skl{v}{w}}{\Skl{v}{v}} \]
	liefert
		\[ 0\leq g(s) = \Skl{v}{v}\frac{\Skl{v}{w}^2}{\Skl{v}{v}^2}-2\Skl{v}{w}\frac{\Skl{v}{w}}{\Skl{v}{v}}+\Skl{w}{w} \]
		\[ = \frac{1}{\Skl{v}{v}}\left(-\Skl{v}{w}^2+\Skl{v}{v}\Skl{w}{w} \right) \Leftrightarrow 0\leq -\Skl{v}{w}^2+\Skl{v}{v}\Skl{w}{w}. \ \]
		
\subsection{Korollar}
\begin{Korollar}[]
	Die induzierte Norm in $ (V,\Skl{.}{.}) $ erfüllt die Dreiecksungleichung.
\end{Korollar}
\paragraph{Beweis}
	Ist $ (V,\Skl{.}{.}) $ Euklidischer VR, so gilt für $ v,w\in V $:
		\begin{align*}
		\|v+w\|^2 = \Skl{v+w}{v+w} &= \|v\|^2+2\Skl{v}{w}+\|w\|^2 \\
		&\overset{C.S.}{\leq} \|v\|^2+2\|v\|\cdot \|w\|+\|w\|^2 = (\|v\|+\|w\|)^2.
		\end{align*}
\paragraph{Bemerkung}
	Das Skalarprodukt eines Euklidischen VR $ (V,\Skl{.}{.}) $ kann (Polarisation) aus seiner induzierten Norm rekonstruiert werden.
	
	Nicht jede Norm ist jedoch von einem Skalarprodukt induziert (vgl. Aufgabe 56).
	Hinreichende (Satz von Jordan-von Neumann) und notwendige Bedingung ist die Parallelogrammgleichung:
	
\subsection{Parallelogrammgleichung}
\begin{Satz}[Parallelogrammgleichung]
	Für die induzierte Norm $ \|.\| $ von $ (V,\Skl{.}{.}) $ gilt:
		\[ \forall v,w\in V: \|v+w\|^2+\|v-w\|^2=2\|v\|^2+2\|w\|^2 \]
\end{Satz}		

	%------------------ Parallelogrammgleichung ----------------
 	\begin{figure}[H]\centering
 		\tdplotsetmaincoords{0}{0} %-27
 	\begin{tikzpicture}[yscale=1,tdplot_main_coords]

 		\def\xstart{0} %x Koordinate der Startposition der Grafik
 		\def\ystart{0} %y Koordinate der Startposition der Grafik
 		\def\myscale{1.0} %�ndert die Gr��e der Grafik (Skalierung der Grafik)
        \def\myscalex{(\myscale)}
        \def\myscaley{(\myscale)}
                
 		\def\xstartdraw{(\xstart + 1.5)} %xKoordinate des Referenzstartpunktes (in dieser Zeichnung: a)
 		\def\ystartdraw{(\ystart + 1.0)}%yKoordinate des Referenzstartpunktes (in dieser Zeichnung: a)

 		\def\balkenhoehe{(4.0)}% L�nge des vertikalen blauen Balkens
 		\def\balkenlaenge{(7)}% L�nge des horizontalen blauen Balkens
 		\def\balkenbreite{0.4} %Balkenbreite

 		%---------Begin Balken----------
 		\def\drehwinkel{0}
 		\node (VekV) at ({\xstart+0.2*cos(\drehwinkel)-\balkenbreite*sin(\drehwinkel)},{\ystart+0.5*sin(\drehwinkel)+\balkenbreite*cos(\drehwinkel)})[right, xshift=1,color=blue] {$V$};
 		\node (AffA) at ({\xstart+(\balkenlaenge-1)*cos(\drehwinkel)},{\ystart+(\balkenlaenge-1)*sin(\drehwinkel)+\balkenbreite*cos(\drehwinkel)})[color=red] {$A$};

 		\path[ shade, top color=white, bottom color=blue, opacity=.6]
 		({\xstart},{\ystart},0)  -- ({\xstart - \balkenbreite * cos(\drehwinkel)- (-\balkenbreite+0)*sin(\drehwinkel)},{\ystart - \balkenbreite * sin(\drehwinkel)+ (-\balkenbreite+0)*cos(\drehwinkel)},0)  -- ({\xstart - \balkenbreite * cos(\drehwinkel)- (\balkenhoehe+0.5)*sin(\drehwinkel)},{\ystart - \balkenbreite * sin(\drehwinkel)+ (\balkenhoehe+0.5)*cos(\drehwinkel)},0) -- ({\xstart - 0 * cos(\drehwinkel)- (\balkenhoehe+0)*sin(\drehwinkel)},{\ystart - 0 * sin(\drehwinkel)+ (\balkenhoehe+0)*cos(\drehwinkel)},0) -- cycle;

 		\path[ shade, right color=white, left color=blue, opacity=.6]
 		({\xstart},{\ystart},0)  -- ({\xstart - \balkenbreite * cos(\drehwinkel)- (-\balkenbreite+0)*sin(\drehwinkel)},{\ystart - \balkenbreite * sin(\drehwinkel)+ (-\balkenbreite+0)*cos(\drehwinkel)},0) --
 		({\xstart + (\balkenlaenge+0.5) * cos(\drehwinkel)- (-\balkenbreite+0)*sin(\drehwinkel)},{\ystart + (\balkenlaenge+0.5) * sin(\drehwinkel)+ (-\balkenbreite+0)*cos(\drehwinkel)},0) --
 		({\xstart + \balkenlaenge * cos(\drehwinkel)},{\ystart + \balkenlaenge * sin(\drehwinkel)},0)--
 		cycle;
 		%---------End Balken----------
 		\def\lightoffset{0.2*\myscale} %offeset der Vektoren

 		% rote Punkte Definition
 		
 		\node (offsetx) at ({(3.5*\myscalex},{0.0}) {}; %just an offset
 		\node (offsety) at ({0.0},{2.5*\myscaley}) {}; %just an offset
 		
 		\node (pointa1) at ({\xstartdraw},{\ystartdraw}) {};
 		\node[ xshift=-2mm, yshift=-3mm,color=red] (labela1) at (pointa1) {$0$};
 		
 		\node (pointa2) at ($(pointa1) + 0.1*(offsetx) + 1.0*(offsety)$) {};
 		
 		\node (pointb1) at ($(pointa1) + 1.0*(offsetx) + 0.1*(offsety)$ ) {};
 		\node (pointb2) at ($(pointb1) + 0.1*(offsetx) + 1.0*(offsety)$) {};
 	
 		%Vektoren blau
 	    %waagrecht
 		\draw[-{>[scale=1,length=10,width=6]},shorten >=2pt, shorten <=2pt,line width=0.2pt,color=blue] (pointa1) -- (pointb1);
 		\draw[-{>[scale=1,length=10,width=6]},shorten >=2pt, shorten <=2pt,line width=0.2pt,color=blue,dashed] (pointa2) -- (pointb2);
 		
 		%senkrecht
 		\draw[-{>[scale=1,length=10,width=6]},shorten >=2pt, shorten <=2pt,line width=0.2pt,color=blue] (pointa1) -- (pointa2);
 		\draw[-{>[scale=1,length=10,width=6]},shorten >=2pt, shorten <=2pt,line width=0.2pt,color=blue,dashed] (pointb1) -- (pointb2);
 		
 		%diagonal
 		\draw[-{>[scale=1,length=10,width=6]},shorten >=2pt, shorten <=2pt,line width=0.2pt,color=blue] (pointa2) -- (pointb1);
 		\draw[-{>[scale=1,length=10,width=6]},shorten >=2pt, shorten <=2pt,line width=0.2pt,color=blue] (pointa1) -- (pointb2);
 		
 		%Beschriftung der Vektoren
 		\node [color=blue] (pointlabelvu) at ($(pointa1)!0.5!(pointb1)$) [above, xshift=0, yshift=-5mm] {$v$} ;
 		\node [color=blue] (pointlabelvo) at ($(pointa2)!0.5!(pointb2)$) [above, xshift=0, yshift=0mm] {$v$} ;
 		
 		\node [color=blue] (pointlabelwl) at ($(pointa1)!0.5!(pointa2)$) [above, xshift=-4mm, yshift=-5mm] {$w$} ;
 		\node [color=blue] (pointlabelwr) at ($(pointb1)!0.5!(pointb2)$) [above, xshift=2mm, yshift=-5mm] {$w$} ;
 		
 		%Beschriftung der Diagonalvektoren
 		def\drehwinkel{30}
 		\node [color=blue,rotate=35] (pointlabeldo) at ($(pointa1)!0.2!(pointb2)$) [above, xshift=3mm, yshift=-0.5mm] {$v+w$} ;
        \node [color=blue,rotate=-35] (pointlabeldu) at ($(pointa2)!0.3!(pointb1)$) [above, xshift=2mm, yshift=-1mm] {$v-w$} ;


 		%Punkte malen
 		\draw[fill,color=red] (pointa1) circle [x=1cm,y=1cm,radius=0.08]node[above, xshift=0, yshift=0]{};
 		\draw[fill,color=red] (pointb1) circle [x=1cm,y=1cm,radius=0.08]node[above, xshift=0, yshift=0]{};
 		\draw[fill,color=red] (pointa2) circle [x=1cm,y=1cm,radius=0.08]node[below, xshift=5, yshift=0]{};
 		\draw[fill,color=red] (pointb2) circle [x=1cm,y=1cm,radius=0.08]node[below, xshift=5, yshift=0]{};
 		
\end{tikzpicture}
	\end{figure}
	%------------------ Parallelogrammgleichung ----------------
		
\paragraph{Beweis}
	Rechnung, wie bei Polarisation.
\paragraph{Beispiel}
	Für die induzierte Norm des kanonischen Skalarprodukts auf $ \R^n $ gilt die Parallelogrammgleichung:
		\[ \sum_{i=1}^{n}(x_i+y_i)^2 + \sum_{i=1}^{n}(x_i-y_i)^2 = 2\sum_{i=1}^{n}x_i^2+2\sum_{i=1}^{n}y_i^2. \]
	Für die durch
		\[ \|(x_i)_{i\in \{1,\dots,n\}}\|_1 = \sum_{i=1}^{n}|x_i| \]
	auf $ \R^n $ definierte Norm $ \|.\|_1 $ gilt sie nicht; diese Norm ist also nicht induzierte Norm eines Euklidischen Skalarprodukts auf $ \R^n $.
\paragraph{Beispiel}
	Auf dem Raum $ C^0([0,1]) $ der stetigen Funktionen auf $ [0,1] $ definiert
		\[ \|.\|_\infty: C^0([0,1]) \to \R, f\mapsto \|f\|_\infty := \max_{x\in [0,1]}|f(x)| \]
	die \emph{Maximumsnorm} (vgl. gleichmäßige Konvergenz).
	
	Für $ f,g\in C^0([0,1]) $,
		\[ f(x):= 1-x \text{ und } g(x) = x \]
	ist dann
		\[ \|f\|_\infty = \|g\|_\infty = \|f+g\|_\infty = \|f-g\|_\infty = 1 \]
	womit die Parallelogrammgleichung offenbar nicht erfüllt, und die Norm keine induzierte Norm eines Skalarprodukts ist.
\section{Euklidische Geometrie}
\subsection{Definition}\index{Euklidischer Raum}\index{Länge}\index{Abstand}\index{Winkel}
\begin{Definition}[Euklidischer Raum, Abstand, Länge, Winkel ]
	Ein \emph{Euklidischer Raum} ist eine affiner Raum $ (A,V,\tau) $ über einem Euklidischen Vektorraum $ (V,\Skl{.}{.}) $ mit induzierter Norm $ \|.\| $.
		\begin{itemize}
			\item Die \emph{Länge} eines Vektors $ v\in V $ ist seine Norm, der \emph{Abstand} zweier Punkte $ a,b\in A $ ist die Länge ihres Verbindungsvektors,
				\[ d(a,b) := \|b-a\| = \sqrt{\Skl{b-a}{b-a}}. \]
			\item Der \emph{Winkel} $ \alpha\in [0,\pi] $ zweier Vektoren $ v,w\in V^\times $ ist durch die Gleichung
				\[ \Skl{v}{w} = \|v\|\cdot \|w\|\cdot \cos \alpha \]
			definiert; der \emph{Winkel} (am Punkt $ a $) in einem nicht-degenerierten Dreieck $ \{a,b,c\} \subset A$ ist der Winkel der beiden Seitenvektoren $ v=b-a $ und $ w = c-a $.
		\end{itemize}
\end{Definition}

\begin{figure}[H]
\centering	
\definecolor{qqwuqq}{rgb}{0.,0.39215686274509803,0.}
\definecolor{qqqqff}{rgb}{0.,0.,1.}
\begin{tikzpicture}[line cap=round,line join=round,>=triangle 45,x=1.0cm,y=1.0cm]
\draw[->,color=black] (-1.9743290273232554,0.) -- (5.524000640987365,0.);
\foreach \x in {-1.,1.,2.,3.,4.,5.}
\draw[shift={(\x,0)},color=black] (0pt,2pt) -- (0pt,-2pt) node[below] {\footnotesize $\x$};
\draw[->,color=black] (0.,-1.133095836146613) -- (0.,4.526265243390068);
\foreach \y in {-1.,1.,2.,3.,4.}
\draw[shift={(0,\y)},color=black] (2pt,0pt) -- (-2pt,0pt) node[left] {\footnotesize $\y$};
\draw[color=black] (0pt,-10pt) node[right] {\footnotesize $0$};
\clip(-1.9743290273232554,-1.133095836146613) rectangle (5.524000640987365,4.526265243390068);
\draw [shift={(1.,1.)},color=qqwuqq,fill=qqwuqq,fill opacity=0.1] (0,0) -- (13.366930696316846:0.38851449058604254) arc (13.366930696316846:70.38212059188172:0.38851449058604254) -- cycle;
\draw [->] (1.,1.) -- (1.72,3.02);
\draw [->] (1.,1.) -- (3.02,1.48);
\begin{scriptsize}
\draw [fill=qqqqff] (1.,1.) circle (2.5pt);
\draw[color=qqqqff] (0.9654306181111331,0.848328065842202) node {$A$};
\draw [fill=qqqqff] (1.72,3.02) circle (2.5pt);
\draw[color=qqqqff] (1.8072120143808919,3.257117907475663) node {$B$};
\draw [fill=qqqqff] (3.02,1.48) circle (2.5pt);
\draw[color=qqqqff] (3.115210799353902,1.7160104281510296) node {$C$};
\draw[color=black] (1.1726383464236891,2.130425884776141) node {$v$};
\draw[color=black] (2.014419742693448,1.1202882092524316) node {$w$};
\draw[color=qqwuqq] (1.5093509049315927,1.495852216818939) node {$\alpha$};
\end{scriptsize}
\end{tikzpicture}
\end{figure}

\paragraph{Bemerkung}
	Nach der Cauchy-Schwarzschen Ungleichung ist für $ v,w\in V^\times $
		\[ \frac{\Skl{v}{w}}{\|v\|\cdot \|w\|}\in [-1,1]; \]
	andererseits ist 
		\[ \cos:[0,\pi]\to [-1,1]\text{ bijektiv} \]
	Damit ist der Winkel von Vektoren bzw. im Dreieck wohldefiniert.

\subsection{Definition}\index{Kongruenzabbildung}\index{Isometrie}\index{Ähnlichkeitstransformation}
\begin{Definition}[Kongruenzabbildung, Ähnlichkeitstransformation]
    Eine affine Transformation eines Euklidischen Raumes heißt
		\begin{itemize}
			\item \emph{Kongruenzabbildung} oder \emph{Isometrie}, falls sie Abstandstreu ist,
			\item \emph{Ähnlichkeitstransformation}, falls sie winkeltreu ist.
		\end{itemize}
\end{Definition}
\paragraph{Bemerkung}
	Jede Kongruenzabbildung ist Ähnlichkeitstransformation (Polarisation).
\paragraph{Bemerkung}
	Offenbar bilden die Kongruenz- bzw. Ähnlichkeitsabbildungen eines Euklidischen Raumes $ A $ auf $ A $ operierende (Transformations-)Gruppen.
	
\subsection{Definition (Geometrie)}\index{Euklidische Geometrie}\index{Ähnlichkeitsgeometrie}
\begin{Definition}[Euklidische Geometrie, Ähnlichkeitsgeometrie]
	Die auf einem Euklidischen Raum operierende Gruppe der Kongruenzabbildungen bestimmt eine Euklidische Geometrie.
	
	Die Gruppe der Ähnlichkeitstransformationen eines Euklidischen Raumes $ A $ bestimmt eine Ähnlichkeitsgeometrie.
\end{Definition}

% VO 19-05-2016 %

\paragraph{Beispiel}
Jede Translation $ \tau_v:A\to A $ ist eine Isometrie:

\begin{minipage}[t]{0.55\linewidth}
	Für $ a,b\in A $ gilt
		\[ \exists!w\in V: b=\tau_w(a) \]
	d.h. $ w=b-a $; also
		\[ \tau_v(b) = \tau_v(\tau_w(a)) = \tau_{v+w}(a) = \tau_w(\tau_v (a)) \]
	d.h. $ w = \tau_v(b)-\tau_v(a) $.
	Damit folgt:
\end{minipage}
\hfill
\begin{minipage}[t]{0.4\linewidth}
	%------------------ IsometrieTranslation ----------------
 	\begin{figure}[H]\centering
 		\tdplotsetmaincoords{0}{0} %-27
 	\begin{tikzpicture}[yscale=1,tdplot_main_coords]

 		\def\xstart{0} %x Koordinate der Startposition der Grafik
 		\def\ystart{0} %y Koordinate der Startposition der Grafik
 		\def\myscale{1.0} %ändert die Größe der Grafik (Skalierung der Grafik)
        \def\myscalex{(\myscale)}
        \def\myscaley{(\myscale)}
                
 		\def\xstartdraw{(\xstart + 1.3)} %xKoordinate des Referenzstartpunktes (in dieser Zeichnung: a)
 		\def\ystartdraw{(\ystart + 1.0)}%yKoordinate des Referenzstartpunktes (in dieser Zeichnung: a)

 		\def\balkenhoehe{(3.0)}% Länge des vertikalen blauen Balkens
 		\def\balkenlaenge{(5.5)}% Länge des horizontalen blauen Balkens
 		\def\balkenbreite{0.4} %Balkenbreite

 		%---------Begin Balken----------
 		\def\drehwinkel{0}
 		\node (VekV) at ({\xstart+0.2*cos(\drehwinkel)-\balkenbreite*sin(\drehwinkel)},{\ystart+0.5*sin(\drehwinkel)+\balkenbreite*cos(\drehwinkel)})[right, xshift=1,color=blue] {$V$};
 		\node (AffA) at ({\xstart+(\balkenlaenge-0.5)*cos(\drehwinkel)},{\ystart+(\balkenlaenge-0.5)*sin(\drehwinkel)+\balkenbreite*cos(\drehwinkel)})[color=red] {$A$};

 		\path[ shade, top color=white, bottom color=blue, opacity=.6]
 		({\xstart},{\ystart},0)  -- ({\xstart - \balkenbreite * cos(\drehwinkel)- (-\balkenbreite+0)*sin(\drehwinkel)},{\ystart - \balkenbreite * sin(\drehwinkel)+ (-\balkenbreite+0)*cos(\drehwinkel)},0)  -- ({\xstart - \balkenbreite * cos(\drehwinkel)- (\balkenhoehe+0.5)*sin(\drehwinkel)},{\ystart - \balkenbreite * sin(\drehwinkel)+ (\balkenhoehe+0.5)*cos(\drehwinkel)},0) -- ({\xstart - 0 * cos(\drehwinkel)- (\balkenhoehe+0)*sin(\drehwinkel)},{\ystart - 0 * sin(\drehwinkel)+ (\balkenhoehe+0)*cos(\drehwinkel)},0) -- cycle;

 		\path[ shade, right color=white, left color=blue, opacity=.6]
 		({\xstart},{\ystart},0)  -- ({\xstart - \balkenbreite * cos(\drehwinkel)- (-\balkenbreite+0)*sin(\drehwinkel)},{\ystart - \balkenbreite * sin(\drehwinkel)+ (-\balkenbreite+0)*cos(\drehwinkel)},0) --
 		({\xstart + (\balkenlaenge+0.5) * cos(\drehwinkel)- (-\balkenbreite+0)*sin(\drehwinkel)},{\ystart + (\balkenlaenge+0.5) * sin(\drehwinkel)+ (-\balkenbreite+0)*cos(\drehwinkel)},0) --
 		({\xstart + \balkenlaenge * cos(\drehwinkel)},{\ystart + \balkenlaenge * sin(\drehwinkel)},0)--
 		cycle;
 		%---------End Balken----------
 		\def\lightoffset{0.2*\myscale} %offeset der Vektoren

 		% rote Punkte Definition
 		
 		\node (offsetx) at ({(2.5*\myscalex},{0.0}) {}; %just an offset
 		\node (offsety) at ({0.0},{1.5*\myscaley}) {}; %just an offset
 		
 		\node (pointa1) at ({\xstartdraw},{\ystartdraw}) {};
 		\node[ xshift=-2mm, yshift=-3mm,color=red] (labela1) at (pointa1) {$a$};
 		
 		\node (pointa2) at ($(pointa1) - 0.15*(offsetx) + 1.0*(offsety)$) {};
 		\node (pointb1) at ($(pointa1) + 1.0*(offsetx) + 0.1*(offsety)$ ) {};
 		\node (pointb2) at ($(pointb1) - 0.15*(offsetx) + 1.0*(offsety)$) {};
 		
 		\node[ xshift=-3mm, yshift=3mm,color=red] (labela2) at (pointa2) {$b$};
 		\node[ xshift=0mm, yshift=4mm,color=red] (labelataub) at (pointb2) {$\tau_v(b)$};
 		\node[ xshift=0mm, yshift=-4mm,color=red] (labelataua) at (pointb1) {$\tau_v(a)$};
 	
 		%Vektoren blau
 	    %waagrecht
 		\draw[-{>[scale=1,length=10,width=6]},shorten >=2pt, shorten <=2pt,line width=0.2pt,color=blue] (pointa1) -- (pointb1);
 		\draw[-{>[scale=1,length=10,width=6]},shorten >=2pt, shorten <=2pt,line width=0.2pt,color=blue] (pointa2) -- (pointb2);
 		
 		%senkrecht
 		\draw[-{>[scale=1,length=10,width=6]},shorten >=2pt, shorten <=2pt,line width=0.2pt,color=blue] (pointa1) -- (pointa2);
 		\draw[-{>[scale=1,length=10,width=6]},shorten >=2pt, shorten <=2pt,line width=0.2pt,color=blue] (pointb1) -- (pointb2);
 		
 	
 		%Beschriftung der Vektoren
 		\node [color=blue] (pointlabelvu) at ($(pointa1)!0.5!(pointb1)$) [above, xshift=0, yshift=-5mm] {$v$} ;
 		\node [color=blue] (pointlabelvo) at ($(pointa2)!0.5!(pointb2)$) [above, xshift=0, yshift=0mm] {$v$} ;
 		
 		\node [color=blue] (pointlabelwl) at ($(pointa1)!0.5!(pointa2)$) [ xshift=5mm, yshift=-2mm] {\small $w=b-a$} ;
 		\node [color=blue] (pointlabelwr) at ($(pointb1)!0.5!(pointb2)$) [ xshift=7mm, yshift=0mm] {\small $w = \tau_v(b) - \tau_v(a)$} ;
 		
 
 		%Punkte malen
 		\draw[fill,color=red] (pointa1) circle [x=1cm,y=1cm,radius=0.08]node[above, xshift=0, yshift=0]{};
 		\draw[fill,color=red] (pointb1) circle [x=1cm,y=1cm,radius=0.08]node[above, xshift=0, yshift=0]{};
 		\draw[fill,color=red] (pointa2) circle [x=1cm,y=1cm,radius=0.08]node[below, xshift=5, yshift=0]{};
 		\draw[fill,color=red] (pointb2) circle [x=1cm,y=1cm,radius=0.08]node[below, xshift=5, yshift=0]{};
 		
\end{tikzpicture}
	\end{figure}
	%------------------ IsometrieTranslation ----------------
\end{minipage}
	\[ \|\tau_v(b)-\tau_v(a)\| = \|w\| = \|b-a\| \]
	d.h. $ \tau_v $ ist abstandstreu, da $ a,b\in A $ beliebig waren.

\paragraph{Beispiel}
	Die Streckung mit Zentrum $ o\in A $ um den Faktor $ s\in \mathbb{R}^\times $,
	
	\begin{minipage}[t]{0.55\linewidth}
		\[ o+v=a\overset{\delta_s}{\mapsto}\delta_s(a) = \delta_s(o+v):= o+vs \]
	ist winkeltreu, denn für $ a=o+v, b=o+w $ gilt
		\[ \delta_s(b)-\delta_s(a) = (o+ws)-(o+vs) = \dots = (w-v)s \]
	und damit für drei paarweise verschiedene Punkte $ a,b,c\in A $
\end{minipage}
\hfill
\begin{minipage}[t]{0.4\linewidth}
	%------------------ IsometrieTranslation ----------------
 	\begin{figure}[H]\centering
 		\tdplotsetmaincoords{0}{0} %-27
 	\begin{tikzpicture}[yscale=1,tdplot_main_coords]

 		\def\xstart{0} %x Koordinate der Startposition der Grafik
 		\def\ystart{0} %y Koordinate der Startposition der Grafik
 		\def\myscale{1.0} %ändert die Größe der Grafik (Skalierung der Grafik)
        \def\myscalex{(\myscale)}
        \def\myscaley{(\myscale)}
                
 		\def\xstartdraw{(\xstart + 2.0)} %xKoordinate des Referenzstartpunktes (in dieser Zeichnung: a)
 		\def\ystartdraw{(\ystart + 1.5)}%yKoordinate des Referenzstartpunktes (in dieser Zeichnung: a)

 		\def\balkenhoehe{(3.5)}% Länge des vertikalen blauen Balkens
 		\def\balkenlaenge{(5.5)}% Länge des horizontalen blauen Balkens
 		\def\balkenbreite{0.4} %Balkenbreite

 		%---------Begin Balken----------
 		\def\drehwinkel{0}
 		\node (VekV) at ({\xstart+0.2*cos(\drehwinkel)-\balkenbreite*sin(\drehwinkel)},{\ystart+0.5*sin(\drehwinkel)+\balkenbreite*cos(\drehwinkel)})[right, xshift=1,color=blue] {$V$};
 		\node (AffA) at ({\xstart+(\balkenlaenge-0.5)*cos(\drehwinkel)},{\ystart+(\balkenlaenge-0.5)*sin(\drehwinkel)+\balkenbreite*cos(\drehwinkel)})[color=red] {$A$};

 		\path[ shade, top color=white, bottom color=blue, opacity=.6]
 		({\xstart},{\ystart},0)  -- ({\xstart - \balkenbreite * cos(\drehwinkel)- (-\balkenbreite+0)*sin(\drehwinkel)},{\ystart - \balkenbreite * sin(\drehwinkel)+ (-\balkenbreite+0)*cos(\drehwinkel)},0)  -- ({\xstart - \balkenbreite * cos(\drehwinkel)- (\balkenhoehe+0.5)*sin(\drehwinkel)},{\ystart - \balkenbreite * sin(\drehwinkel)+ (\balkenhoehe+0.5)*cos(\drehwinkel)},0) -- ({\xstart - 0 * cos(\drehwinkel)- (\balkenhoehe+0)*sin(\drehwinkel)},{\ystart - 0 * sin(\drehwinkel)+ (\balkenhoehe+0)*cos(\drehwinkel)},0) -- cycle;

 		\path[ shade, right color=white, left color=blue, opacity=.6]
 		({\xstart},{\ystart},0)  -- ({\xstart - \balkenbreite * cos(\drehwinkel)- (-\balkenbreite+0)*sin(\drehwinkel)},{\ystart - \balkenbreite * sin(\drehwinkel)+ (-\balkenbreite+0)*cos(\drehwinkel)},0) --
 		({\xstart + (\balkenlaenge+0.5) * cos(\drehwinkel)- (-\balkenbreite+0)*sin(\drehwinkel)},{\ystart + (\balkenlaenge+0.5) * sin(\drehwinkel)+ (-\balkenbreite+0)*cos(\drehwinkel)},0) --
 		({\xstart + \balkenlaenge * cos(\drehwinkel)},{\ystart + \balkenlaenge * sin(\drehwinkel)},0)--
 		cycle;
 		%---------End Balken----------
 		\def\lightoffset{0.2*\myscale} %offeset der Vektoren

 		% rote Punkte Definition
 		
 		\node (offsetx) at ({(2.5*\myscalex},{0.0}) {}; %just an offset
 		\node (offsety) at ({0.0},{1.5*\myscaley}) {}; %just an offset
 		
 		\node (pointintersection) at ({\xstartdraw},{\ystartdraw}) {};
 		
 		
 	%	\draw[red] (fov) -- ++(295:2cm);
    %\draw[red] (fov) -- ++(335:2cm);
       %\coordinate (B) at (45:2cm) ;
        
        \node (pointa2) at ($(pointintersection) + (70:2.5)$) {};
 		\node (pointb2) at ($(pointintersection) + (25:4)$) {};
 		
 		\node (pointa1) at ($(pointintersection) + (250:1.25)$) {};
 		\node (pointb1) at ($(pointintersection) + (205:2)$) {};
 		
 	%	\node (pointa2) at ($(pointa1) - 0.15*(offsetx) + 1.0*(offsety)$) {};
 		
 	
 		\node[ xshift=3mm, yshift=0mm,color=red] (labela1) at (pointa1) {$a$};
 		\node[ xshift=6mm, yshift=-1mm,color=red] (labela2) at (pointa2) {$\delta_{\text{\tiny  -2}} (a)$};
 		\node[ xshift=1mm, yshift=4mm,color=red] (labelataub) at (pointb2) {$\delta_{\text{\tiny  -2}} (b)$};
 		\node[ xshift=0mm, yshift=4mm,color=red] (labelataua) at (pointb1) {$b$};
 	
 	%    \draw[name path=line 1] (0,0) -- (2,2);
     %   \draw[name path=line 2] (2,0) -- (0,2);
%\fill[red,name intersections={of=line 1 and line 2,total=\t}]
 %   \foreach \s in {1,...,\t}{(intersection-\s) circle (2pt) node {\footnotesize\s}};
    
    
 		%Vektoren blau
 	    %waagrecht
 		\draw[name path=a--da,{<[scale=1,length=6,width=6]}-{>[scale=1,length=6,width=6]},shorten >=2pt, shorten <=2pt,line width=0.2pt,color=blue] (pointa1) -- (pointa2);
 		\draw[name path=b--db,{<[scale=1,length=6,width=6]}-{>[scale=1,length=6,width=6]},shorten >=2pt, shorten <=2pt,line width=0.2pt,color=blue] (pointb1) -- (pointb2);
 		
 		\draw[line width=0.2pt,color=red] ($(pointintersection) + (28:0.7)$) arc[radius=0.7, start angle=28, end angle=67] ($(pointintersection) + (67:0.7)$);
 		\draw[line width=0.2pt,color=red] ($(pointintersection) + (28:0.62)$) arc[radius=0.62, start angle=28, end angle=67] ($(pointintersection) + (67:0.62)$);
 		
 		\draw[line width=0.2pt,color=red] ($(pointintersection) + (208:0.7)$) arc[radius=0.7, start angle=208, end angle=247] ($(pointintersection) + (247:0.7)$);
 		\draw[line width=0.2pt,color=red] ($(pointintersection) + (208:0.62)$) arc[radius=0.62, start angle=208, end angle=247] ($(pointintersection) + (247:0.62)$);
 	
 		%\path [name intersections={of=a--da and b--db,by=E}];
 		
 	
 		%Beschriftung der Vektoren
 		\node [color=blue] (pointlabelvu) at ($(pointintersection)!0.5!(pointa1)$) [ xshift=2mm, yshift=0mm] {\small $v$} ;
 		\node [color=blue] (pointlabelvo) at ($(pointintersection)!0.5!(pointb1)$) [above, xshift=0, yshift=0mm] {\small $w$} ;
 		
 		\node [color=blue] (pointlabelwl) at ($(pointintersection)!0.5!(pointa2)$) [ xshift=-8mm, yshift=0mm] {\small $v \cdot (-2)$} ;
 		\node [color=blue] (pointlabelwr) at ($(pointintersection)!0.5!(pointb2)$) [ xshift=7mm, yshift=-2mm] {\small $w \cdot (-2)$} ;
 		
 	

 		%Punkte malen
 		\draw[fill,color=red] (pointa1) circle [x=1cm,y=1cm,radius=0.08]node[above, xshift=0, yshift=0]{};
 		\draw[fill,color=red] (pointb1) circle [x=1cm,y=1cm,radius=0.08]node[above, xshift=0, yshift=0]{};
 		\draw[fill,color=red] (pointa2) circle [x=1cm,y=1cm,radius=0.08]node[below, xshift=5, yshift=0]{};
 		\draw[fill,color=red] (pointb2) circle [x=1cm,y=1cm,radius=0.08]node[below, xshift=5, yshift=0]{};
 		
 		\draw[fill,color=white] (pointintersection) circle [x=1cm,y=1cm,radius=0.18];
 		
 		\draw[fill,color=red] (pointintersection) circle [x=1cm,y=1cm,radius=0.08]node[below, xshift=5, yshift=0]{};
 		
 		
 		
\end{tikzpicture}
	\end{figure}
	%------------------ IsometrieTranslation ----------------
\end{minipage}

		\[ \cos \alpha = \frac{\Skl{\delta_s(b)-\delta_s(a)}{\delta_s(c)-\delta_s(a)}}{\|\delta_s(b)-\delta_s(a)\|\|\delta_s(c)-\delta_s(a)\|} = \frac{\Skl{(b-a)s}{(c-a)s}}{\|(b-a)s\|\|(c-a)s\|} =\frac{s^2}{|s^2|} \cdot \frac{\Skl{b-a}{c-a}}{\|b-a\|\|c-a\|} \]
		
\begin{minipage}[t]{0.5\linewidth}
    \vspace{2cm}
	d.h. $ \delta_s $ ist winkeltreu; andererseits ist $ \delta_s $ für $ s\neq \pm 1 $ nicht abstandstreu.
	Ist $ a \neq b $, so gilt dann
		\[ \|\delta_s(b)-\delta_s(a)\| = \|b-a\|\cdot |s| \neq \|b-a\|. \] 
\end{minipage}
\hfill
\begin{minipage}[t]{0.45\linewidth}
	%------------------ WinkeltreuAberNichtAbstandstreu ----------------
 	\begin{figure}[H]\centering
 		\tdplotsetmaincoords{0}{0} %-27
 	\begin{tikzpicture}[yscale=0.9,tdplot_main_coords]

 		\def\xstart{0} %x Koordinate der Startposition der Grafik
 		\def\ystart{0} %y Koordinate der Startposition der Grafik
 		\def\myscale{1.0} %ändert die Größe der Grafik (Skalierung der Grafik)
        \def\myscalex{(\myscale)}
        \def\myscaley{(\myscale)}
                
 		\def\xstartdraw{(\xstart + 2.5)} %xKoordinate des Referenzstartpunktes (in dieser Zeichnung: a)
 		\def\ystartdraw{(\ystart + 2.0)}%yKoordinate des Referenzstartpunktes (in dieser Zeichnung: a)

 		\def\balkenhoehe{(4.0)}% Länge des vertikalen blauen Balkens
 		\def\balkenlaenge{(6.5)}% Länge des horizontalen blauen Balkens
 		\def\balkenbreite{0.4} %Balkenbreite

 		%---------Begin Balken----------
 		\def\drehwinkel{0}
 		\node (VekV) at ({\xstart+0.2*cos(\drehwinkel)-\balkenbreite*sin(\drehwinkel)},{\ystart+0.5*sin(\drehwinkel)+\balkenbreite*cos(\drehwinkel)})[right, xshift=1,color=blue] {$V$};
 		\node (AffA) at ({\xstart+(\balkenlaenge-0.5)*cos(\drehwinkel)},{\ystart+(\balkenlaenge-0.5)*sin(\drehwinkel)+\balkenbreite*cos(\drehwinkel)})[color=red] {$A$};

 		\path[ shade, top color=white, bottom color=blue, opacity=.6]
 		({\xstart},{\ystart},0)  -- ({\xstart - \balkenbreite * cos(\drehwinkel)- (-\balkenbreite+0)*sin(\drehwinkel)},{\ystart - \balkenbreite * sin(\drehwinkel)+ (-\balkenbreite+0)*cos(\drehwinkel)},0)  -- ({\xstart - \balkenbreite * cos(\drehwinkel)- (\balkenhoehe+0.5)*sin(\drehwinkel)},{\ystart - \balkenbreite * sin(\drehwinkel)+ (\balkenhoehe+0.5)*cos(\drehwinkel)},0) -- ({\xstart - 0 * cos(\drehwinkel)- (\balkenhoehe+0)*sin(\drehwinkel)},{\ystart - 0 * sin(\drehwinkel)+ (\balkenhoehe+0)*cos(\drehwinkel)},0) -- cycle;

 		\path[ shade, right color=white, left color=blue, opacity=.6]
 		({\xstart},{\ystart},0)  -- ({\xstart - \balkenbreite * cos(\drehwinkel)- (-\balkenbreite+0)*sin(\drehwinkel)},{\ystart - \balkenbreite * sin(\drehwinkel)+ (-\balkenbreite+0)*cos(\drehwinkel)},0) --
 		({\xstart + (\balkenlaenge+0.5) * cos(\drehwinkel)- (-\balkenbreite+0)*sin(\drehwinkel)},{\ystart + (\balkenlaenge+0.5) * sin(\drehwinkel)+ (-\balkenbreite+0)*cos(\drehwinkel)},0) --
 		({\xstart + \balkenlaenge * cos(\drehwinkel)},{\ystart + \balkenlaenge * sin(\drehwinkel)},0)--
 		cycle;
 		%---------End Balken----------
 		\def\lightoffset{0.2*\myscale} %offeset der Vektoren

 		% rote Punkte Definition
 		
 		\node (offsetx) at ({(2.5*\myscalex},{0.0}) {}; %just an offset
 		\node (offsety) at ({0.0},{1.5*\myscaley}) {}; %just an offset
 		
 		\node (pointintersection) at ({\xstartdraw},{\ystartdraw}) {};
 		
 		
  		\node (pointa2) at ($(pointintersection) + (80:2)$) {};
 		\node (pointb2) at ($(pointintersection) + (35:3)$) {};
 		\node (pointc2) at ($(pointintersection) + (5:2.25)$) {};
 		
 		\node (pointa1) at ($(pointintersection) + (260:1.25)$) {};
 		\node (pointb1) at ($(pointintersection) + (215:2)$) {};
 		\node (pointc1) at ($(pointintersection) + (185:1.5)$) {};
 
 		
 	
 		\node[ xshift=3mm, yshift=0mm,color=red] (labela1) at (pointa1) {$a$};
 		\node[ xshift=-6mm, yshift=-1mm,color=red] (labela2) at (pointa2) {$\delta_{\text{\tiny  -2}} (a)$};
 		\node[ xshift=1mm, yshift=-4mm,color=red] (labelataub) at (pointb2) {$\delta_{\text{\tiny  -2}} (b)$};
 		\node[ xshift=1mm, yshift=-4mm,color=red] (labelatauc) at (pointc2) {$\delta_{\text{\tiny  -2}} (c)$};
 		\node[ xshift=0mm, yshift=4mm,color=red] (labelb) at (pointb1) {$b$};
 		\node[ xshift=-1mm, yshift=2mm,color=red] (labelc) at (pointc1) {$c$};
 		
 	
 		%Vektoren blau
 		\draw[name path=a--da,-{>[scale=1,length=6,width=6]},shorten >=2pt, shorten <=2pt,line width=0.2pt,color=blue] (pointa2) -- (pointb2);
 		\draw[name path=b--db,-{>[scale=1,length=6,width=6]},shorten >=2pt, shorten <=2pt,line width=0.2pt,color=blue] (pointa2) -- (pointc2);
 		
 		\draw[name path=a--da,-{>[scale=1,length=6,width=6]},shorten >=2pt, shorten <=2pt,line width=0.2pt,color=blue] (pointa1) -- (pointb1);
 		\draw[name path=b--db,-{>[scale=1,length=6,width=6]},shorten >=2pt, shorten <=2pt,line width=0.2pt,color=blue] (pointa1) -- (pointc1);
 	
 	    %punktierte Linien	
 		\draw[line width=0.2pt,color=blue,dotted] (pointa1) -- (pointa2);
 		\draw[line width=0.2pt,color=blue,dotted] (pointb1) -- (pointb2);
 		\draw[line width=0.2pt,color=blue,dotted] (pointc1) -- (pointc2);
 		
 		\draw[line width=0.2pt,color=red] ($(pointintersection) + (38:0.6)$) arc[radius=0.6, start angle=38, end angle=77] ($(pointintersection) + (77:0.6)$);
 		\draw[line width=0.2pt,color=red] ($(pointintersection) + (38:0.52)$) arc[radius=0.52, start angle=38, end angle=77] ($(pointintersection) + (77:0.52)$);
 		
 		\draw[line width=0.2pt,color=red] ($(pointintersection) + (218:0.6)$) arc[radius=0.6, start angle=218, end angle=257] ($(pointintersection) + (257:0.6)$);
 		\draw[line width=0.2pt,color=red] ($(pointintersection) + (218:0.52)$) arc[radius=0.52, start angle=218, end angle=257] ($(pointintersection) + (257:0.52)$);
 	
 	    \draw[line width=0.2pt,color=red] ($(pointintersection) + (8:0.9)$) arc[radius=0.9, start angle=8, end angle=33] ($(pointintersection) + (33:0.9)$);
 		\draw[line width=0.2pt,color=red] ($(pointintersection) + (8:0.82)$) arc[radius=0.82, start angle=8, end angle=33] ($(pointintersection) + (33:0.82)$);
 		
 		\draw[line width=0.2pt,color=red] ($(pointintersection) + (188:0.9)$) arc[radius=0.9, start angle=188, end angle=213] ($(pointintersection) + (213:0.9)$);
 		\draw[line width=0.2pt,color=red] ($(pointintersection) + (188:0.82)$) arc[radius=0.82, start angle=188, end angle=213] ($(pointintersection) + (213:0.82)$);
 		
 		%Beschriftung der Vektoren
 		
 		\node [color=blue] (pointlabelvr) at ($(pointa1)!0.5!(pointb1)$) [ xshift=0mm, yshift=-3mm] {\footnotesize $b- a$} ;
 		
 		\node [color=blue] (pointlabelwr) at ($(pointa2)!0.5!(pointb2)$) [ xshift=0mm, yshift=4mm] {\footnotesize $ \delta_{\text{\tiny  -2}} (b) - \delta_{\text{\tiny  -2}} (a) $} ;
 	
 	
 		%Punkte malen
 		\draw[fill,color=red] (pointa1) circle [x=1cm,y=1cm,radius=0.08]node[above, xshift=0, yshift=0]{};
 		\draw[fill,color=red] (pointb1) circle [x=1cm,y=1cm,radius=0.08]node[above, xshift=0, yshift=0]{};
 		\draw[fill,color=red] (pointa2) circle [x=1cm,y=1cm,radius=0.08]node[below, xshift=5, yshift=0]{};
 		\draw[fill,color=red] (pointb2) circle [x=1cm,y=1cm,radius=0.08]node[below, xshift=5, yshift=0]{};
 		\draw[fill,color=red] (pointc1) circle [x=1cm,y=1cm,radius=0.08]node[above, xshift=0, yshift=0]{};
 		\draw[fill,color=red] (pointc2) circle [x=1cm,y=1cm,radius=0.08]node[above, xshift=0, yshift=0]{};
 		
 		\draw[fill,color=white] (pointintersection) circle [x=1cm,y=1cm,radius=0.18];
 		
 		\draw[fill,color=red] (pointintersection) circle [x=1cm,y=1cm,radius=0.08]node[below, xshift=5, yshift=0]{};
 		\node[ xshift=-2mm, yshift=3mm,color=red] (label0) at (pointintersection) {\small $0$};
 		
 		
 		
\end{tikzpicture}
	\end{figure}
	%------------------ WinkeltreuAberNichtAbstandstreu ----------------
\end{minipage}		

\paragraph{Zur Erinnerung}
	Jede affine Abbildung $ \alpha:A\to A' $ besitzt einen (eindeutigen) \emph{linearen Anteil} $ \lambda:V\to V' $, sodass
		\[ \forall a\in A\forall v\in V: \alpha(a+v) = \alpha(a)+\lambda(v); \]
	ist $ \alpha $ eine affine Transformation, so ist $ \lambda \in Gl(V) $.
\paragraph{Bemerkung}
	Jede Ähnlichkeitstransformation ist Komposition einer Streckung und einer Kongruenzabbildung.
	
	Nämlich: Ist $ \alpha $ Ähnlichkeitstransformation mit linearem Anteil $ \lambda\in Gl(V) $, so erhält $ \lambda $ Winkel von Vektoren, insbesondere also Orthogonalität.
	Nun wähle $ w\in V^\times $ und setze
		\[ s := \frac{\|w\|}{\|\lambda w\|}. \]
	Ist dann $ v\in V $ mit $ \|v\|=\|w\| $, so folgt
		\[ v+w \perp v-w \Rightarrow \lambda(v+w)\perp \lambda(v-w) \Rightarrow \|\lambda(v)\| = \|\lambda(w)\|, \]
	also 
		\[ \forall v\in V^\times: \frac{\|\lambda(v)\|}{\|v\|} = \|\lambda(v\frac{\|w\|}{\|v\|})\|\frac{1}{\|w\|} = \frac{\|\lambda (w)\|}{\|w\|} = \frac{1}{s}. \]
	Mit einem beliebigen Streckungszentrum $ o\in A $ erhält man also eine Isometrie durch
		\[ \delta_s\circ \alpha :A\to A. \]
\paragraph{Beispiel}
	Eine \emph{nicht-triviale} Scherung ist \emph{keine} Ähnlichkeitstransformation. Beweis in der Übung. % 3 Zeilen Rechnung, 5 Zeilen Begründung!

\subsection{Lemma \& Definition}
\begin{Lemma}
	Eine affine Transformation $ \alpha:A\to A $ eines Euklidischen Raumes $ A $ ist genau dann eine Kongruenzabbildung, wenn ihr linearer Anteil $ \lambda $ \emph{orthogonal} ist:
\end{Lemma}
\begin{Definition}[orthonogale Gruppe]
		\[ \lambda\in O(V):= \{f\in Gl(V)\mid \forall v,w\in V: \Skl{f(v)}{f(w)} = \Skl{v}{w}\}. \]
	$ O(V) $ heißt die \emph{orthonogale Gruppe} von $ (V,\Skl{.}{.}) $.
\end{Definition}

\paragraph{Bemerkung}
	$ O(V)\subset Gl(V) $ ist eine Gruppe. Beweis in der Übung.
\paragraph{Bemerkung}
	Ist $ f\in \End(V) $, so folgt die Injektivität von $ f $ aus
		\[ \forall v,w\in V: \Skl{f(v)}{f(w)} = \Skl{v}{w}. \]
	Aus $ f(v) = 0 $ folgt nämlich
		\[ 0 = \|f(v)\| = 0 = \|v\| \Rightarrow v = 0, \text{ da } \Skl{.}{.}\text{ pos. definit.} \]
	Ist $ \dim V <\infty $, so folgt mit dem Rangsatz, $ \dim V = \rg f + \dfkt f = \rg $, dass $ f\in Gl(V) $.
	
	Im Fall $ \dim V = \infty $ ist $ f $ nicht notwendigerweise surjektiv, wie der \emph{Shiftoperator}
		\[ f\in \End(\R^\N), \forall n\in \N: f(e_n) = e_{n+1} \]
	zeigt.
\paragraph{Beweis (Lemma)}
	Sei $ (A,V,\tau) $ Euklidischer Raum über einem Euklidischen VR $ (V,\Skl{.}{.}) $ und $\alpha:A\to A $ Affinität mit linearem Anteil $ \lambda\in Gl(V) $. Dann ist $ \alpha $ genau dann Isometrie, wenn
		\[ \forall a,b\in A: \|\lambda(b-a)\| = \|\alpha(b)-\alpha(a)\| = \|b-a\|,  \]
	also (Polarisation), wenn $ \lambda\in O(V) $.
\subsection{Definition}
\begin{Definition}[unitäre Gruppe]
	Ist $ (V,\Skl{.}{.}) $ unitärer VR, so heißt $ f\in Gl(V) $ mit
		\[ \forall v,w\in V: \Skl{f(v)}{f(w)} = \Skl{v}{w} \]
	\emph{unitär}; die \emph{unitäre Gruppe} von $ (V,\Skl{.}{.}) $ ist die Gruppe
		\[ U(V) := \{f\in Gl(V)\mid \forall v,w\in V: \Skl{f(v)}{f(w)} = \Skl{v}{w} \}. \]
\end{Definition}

% VO 24-05-2016 %

\subsection{Schulgeometrie}
	Betrachte eine Euklidische Ebene $ A^2 $ über Euklidischem VR $ (\R^2,\Skl{.}{.}) $ mit kanonischem Skalarprodukt
		\[ \forall i,j\in \{1,2\}:\Skl{e_i}{e_j} = \delta_{ij}. \]
	Weiter (vgl. Abschnitt 5.3) bezeichne $ J\in \End(\R^2) $ den durch
		\[ J(e_1) = e_2 \text{ und } J(e_2) = 1 \]
	definierten Endomorphismus, also eine "`$ 90^{\circ}$-Drehung"',
	bzw. die $ \R^2 $ mit $ \C $ identifizierende komplexe Multiplikation mit $ i $,
		\[ v(x+iy) = vx+J(v)y \text{ für }\begin{cases}
		v\in \R^2\\ (x+iy)\in \C.
		\end{cases} \]
	Man bemerke: Für $ v\in \R^2\setminus \{0\} $ ist $ \{Jv\}^\perp = [v] $ und damit
		\[ \forall w\in \R^2: w\perp Jv \Leftrightarrow w\parallel v. \]
     So ermöglicht $ J $ einen einfachen Wechsel zwischen \emph{parametrischer} und \emph{impliziter Darstellung} (e.g. \emph{Hessesche Normalform}) einer Geraden
     
	\begin{minipage}[t]{0.5\linewidth}
	 \vspace{1.5cm}
   		\begin{align*}
		    g = \{p = o + vx\mid x\in \R\} \Leftrightarrow \\
		    g = \{p\in A^2\mid \Skl{p-o}{Jv} = 0\} 
		\end{align*} 
    \end{minipage}
    \hfill
    \begin{minipage}[t]{0.45\linewidth}
    	%------------------ WinkeltreuAberNichtAbstandstreu ----------------
     	\begin{figure}[H]\centering
     		\tdplotsetmaincoords{0}{0} %-27
 	\begin{tikzpicture}[yscale=1,tdplot_main_coords]

 		\def\xstart{0} %x Koordinate der Startposition der Grafik
 		\def\ystart{0} %y Koordinate der Startposition der Grafik
 		\def\myscale{1.0} %ändert die Größe der Grafik (Skalierung der Grafik)
        \def\myscalex{(\myscale)}
        \def\myscaley{(\myscale)}
                
 		\def\xstartdraw{(\xstart + 3.0)} %xKoordinate des Referenzstartpunktes (in dieser Zeichnung: a)
 		\def\ystartdraw{(\ystart + 2.0)}%yKoordinate des Referenzstartpunktes (in dieser Zeichnung: a)

 		\def\balkenhoehe{(3.5)}% Länge des vertikalen blauen Balkens
 		\def\balkenlaenge{(5.5)}% Länge des horizontalen blauen Balkens
 		\def\balkenbreite{0.4} %Balkenbreite

 		%---------Begin Balken----------
 		\def\drehwinkel{0}
 		\node (VekV) at ({\xstart+0.2*cos(\drehwinkel)-\balkenbreite*sin(\drehwinkel)},{\ystart+0.5*sin(\drehwinkel)+\balkenbreite*cos(\drehwinkel)})[right, xshift=1,color=blue] {$\mathbb{R}^2$};
 		\node (AffA) at ({\xstart+(\balkenlaenge-0.5)*cos(\drehwinkel)},{\ystart+(\balkenlaenge-0.5)*sin(\drehwinkel)+\balkenbreite*cos(\drehwinkel)})[color=red] {$A$};

 		\path[ shade, top color=white, bottom color=blue, opacity=.6]
 		({\xstart},{\ystart},0)  -- ({\xstart - \balkenbreite * cos(\drehwinkel)- (-\balkenbreite+0)*sin(\drehwinkel)},{\ystart - \balkenbreite * sin(\drehwinkel)+ (-\balkenbreite+0)*cos(\drehwinkel)},0)  -- ({\xstart - \balkenbreite * cos(\drehwinkel)- (\balkenhoehe+0.5)*sin(\drehwinkel)},{\ystart - \balkenbreite * sin(\drehwinkel)+ (\balkenhoehe+0.5)*cos(\drehwinkel)},0) -- ({\xstart - 0 * cos(\drehwinkel)- (\balkenhoehe+0)*sin(\drehwinkel)},{\ystart - 0 * sin(\drehwinkel)+ (\balkenhoehe+0)*cos(\drehwinkel)},0) -- cycle;

 		\path[ shade, right color=white, left color=blue, opacity=.6]
 		({\xstart},{\ystart},0)  -- ({\xstart - \balkenbreite * cos(\drehwinkel)- (-\balkenbreite+0)*sin(\drehwinkel)},{\ystart - \balkenbreite * sin(\drehwinkel)+ (-\balkenbreite+0)*cos(\drehwinkel)},0) --
 		({\xstart + (\balkenlaenge+0.5) * cos(\drehwinkel)- (-\balkenbreite+0)*sin(\drehwinkel)},{\ystart + (\balkenlaenge+0.5) * sin(\drehwinkel)+ (-\balkenbreite+0)*cos(\drehwinkel)},0) --
 		({\xstart + \balkenlaenge * cos(\drehwinkel)},{\ystart + \balkenlaenge * sin(\drehwinkel)},0)--
 		cycle;
 		%---------End Balken----------
 	
 		\node (pointintersection) at ({\xstartdraw},{\ystartdraw}) {};
 		
 		
        \node (pointa1) at ($(pointintersection) + (230:2.0)$) {};
 		\node (pointb1) at ($(pointintersection) + (140:2.5)$) {};
 		\node (pointb11) at ($(pointintersection) + (140:2.0)$) {};
 		\node (pointb12) at ($(pointintersection) + (140:3.0)$) {};
 		\node (pointb2) at ($(pointintersection) + (140:-2.0)$) {};
 		
 		\node[ xshift=1mm, yshift=4mm,color=red] (labelataub) at (pointb2) {$g$};
 		\node[ xshift=12mm, yshift=0mm,color=red] (labelataua) at (pointb1) {\small $p=o+vx$};
 	
        \draw[name path=b--db,-,color=red] (pointb12) -- (pointb2);
 		%Vektoren blau
 		
 	    \draw[name path=a--da,-{>[scale=1,length=6,width=6]},shorten >=0pt, shorten <=0pt,line width=0.2pt,color=blue] (pointintersection) -- (pointa1);
 	    \draw[name path=a--da,-{>[scale=1,length=6,width=6]},shorten >=0pt, shorten <=0pt,line width=0.2pt,color=blue] (pointintersection) -- (pointb11);
    
    	\draw[line width=0.2pt,color=blue,dashed] ($(pointintersection) + (145:1.8)$) arc[radius=1.8, start angle=145, end angle=225] ($(pointintersection) + (225:1.8)$);
 		
 		\draw[line width=0.2pt,color=blue] ($(pointintersection) + (145:0.4)$) arc[radius=0.4, start angle=145, end angle=225] ($(pointintersection) + (225:0.4)$);
 	
 		%Beschriftung der Vektoren
 		\node [color=blue] (pointlabelvu) at ($(pointintersection)!0.5!(pointa1)$) [ xshift=4mm, yshift=-1mm] {\small $J(v)$} ;
 		\node [color=blue] (pointlabelvo) at ($(pointintersection)!0.5!(pointb1)$) [above, xshift=4mm, yshift=-3mm] {\small $v$} ;

 		%Punkte malen
 		
 		\draw[fill,color=white] (pointintersection) circle [radius=0.11]node[below, xshift=5, yshift=0]{};
 	    \draw[fill,color=white] (pointb1) circle [radius=0.11]node[below, xshift=5, yshift=0]{};
 	    
 		\draw[fill,color=red] (pointb1) circle [radius=0.06]node[above, xshift=0, yshift=0]{};
 		
 		\node (pointrw) at ($(pointintersection) + (185:0.2)$) {};
 		\draw[fill,color=blue] (pointrw) circle [x=0.2cm,y=0.2cm,radius=0.08]node[above, xshift=0, yshift=0]{};
 		
 
 		\draw[fill,color=red] (pointintersection) circle [radius=0.06]node[below, xshift=5, yshift=0]{};
 		\node[ xshift=2mm, yshift=1mm,color=red] (labelato) at (pointintersection) {$o$};
 		
 		
\end{tikzpicture}
    	\end{figure}
    	%------------------ WinkeltreuAberNichtAbstandstreu ----------------
    \end{minipage}	

\subsection{Definition}\index{Kreis}
\begin{Definition}[Kreis, Mittelpunkt]
    Ein \emph{Kreis} mit \emph{Mittelpunkt} $ z\in A^2 $ und \emph{Radius} $ r\geq 0 $ ist die Menge
		\[ k = \{p\in A^2\mid \|p-z\| = r\}. \]
\end{Definition}
\paragraph{Bemerkung}
	Es ist mitunter sinnvoll, Punkte als Kreise mit Radius $ r=0 $ zu betrachten.
	
\subsection{Umkreissatz}\index{Umkreis}\index{Streckensymmetrale}
\begin{Satz}[Umkreissatz]
	Sei $ \{a,b,c\} \subset A^2 $ ein nicht-degeneriertes Dreieck. Dann gibt es genau einen Kreis $ k\subset A^2 $, den \emph{Umkreis} des Dreiecks, der die Eckpunkte $ a,b $ und $ c $ des Dreiecks enthält.
	
	Sei Mittelpunkt ist der Schnittpunkt der drei \emph{Streckensymmetralen}/\emph{Mittelsenkrechten} $ m_{ab}, m_{bc}$ und $ m_{ca} $ des Dreiecks, wobei
\end{Satz}

    \begin{minipage}[t]{0.41\linewidth}
        \vspace{2cm}
    		\[ m_{ab} = \{p\in A^2\mid \Skl{p-s_{ab}}{b-a} = 0\} \]
    	mit $ s_{ab} = a\frac{1}{2}+b\frac{1}{2} $ etc.
    	
    \end{minipage}
    \hfill
    \begin{minipage}[t]{0.58\linewidth}
    	%------------------ WinkeltreuAberNichtAbstandstreu ----------------
     	\begin{figure}[H]\centering
     		\tikzset{
        sum/.style={circle,draw=red,fill=red,thick, inner sep=0pt,minimum size=5pt},
        gain/.style={regular polygon,regular polygon sides=3, draw = black, fill=white,thick, minimum size=6mm},
        none/.style={draw=none,fill=none}}

\tdplotsetmaincoords{0}{0} %-27
 	\begin{tikzpicture}[xscale=0.7,yscale=0.7,tdplot_main_coords]

 		\def\xstart{0} %x Koordinate der Startposition der Grafik
 		\def\ystart{0} %y Koordinate der Startposition der Grafik
 		\def\myscale{1.0} %ändert die Größe der Grafik (Skalierung der Grafik)
        \def\myscalex{(\myscale)}
        \def\myscaley{(\myscale)}
                
 		\def\xstartdraw{(\xstart + 3.7)} %xKoordinate des Referenzstartpunktes (in dieser Zeichnung: a)
 		\def\ystartdraw{(\ystart + 2.7)}%yKoordinate des Referenzstartpunktes (in dieser Zeichnung: a)

 		\def\balkenhoehe{(5.5)}% Länge des vertikalen blauen Balkens
 		\def\balkenlaenge{(7.5)}% Länge des horizontalen blauen Balkens
 		\def\balkenbreite{0.4} %Balkenbreite

 		%---------Begin Balken----------
 		\def\drehwinkel{0}
 		\node (VekV) at ({\xstart+0.2*cos(\drehwinkel)-\balkenbreite*sin(\drehwinkel)},{\ystart+0.5*sin(\drehwinkel)+\balkenbreite*cos(\drehwinkel)})[right, xshift=1,color=blue] {$\mathbb{R}^2$};
 		\node (AffA) at ({\xstart+(\balkenlaenge-0.5)*cos(\drehwinkel)},{\ystart+(\balkenlaenge-0.5)*sin(\drehwinkel)+\balkenbreite*cos(\drehwinkel)})[color=red] {$A$};

 		\path[ shade, top color=white, bottom color=blue, opacity=.6]
 		({\xstart},{\ystart},0)  -- ({\xstart - \balkenbreite * cos(\drehwinkel)- (-\balkenbreite+0)*sin(\drehwinkel)},{\ystart - \balkenbreite * sin(\drehwinkel)+ (-\balkenbreite+0)*cos(\drehwinkel)},0)  -- ({\xstart - \balkenbreite * cos(\drehwinkel)- (\balkenhoehe+0.5)*sin(\drehwinkel)},{\ystart - \balkenbreite * sin(\drehwinkel)+ (\balkenhoehe+0.5)*cos(\drehwinkel)},0) -- ({\xstart - 0 * cos(\drehwinkel)- (\balkenhoehe+0)*sin(\drehwinkel)},{\ystart - 0 * sin(\drehwinkel)+ (\balkenhoehe+0)*cos(\drehwinkel)},0) -- cycle;

 		\path[ shade, right color=white, left color=blue, opacity=.6]
 		({\xstart},{\ystart},0)  -- ({\xstart - \balkenbreite * cos(\drehwinkel)- (-\balkenbreite+0)*sin(\drehwinkel)},{\ystart - \balkenbreite * sin(\drehwinkel)+ (-\balkenbreite+0)*cos(\drehwinkel)},0) --
 		({\xstart + (\balkenlaenge+0.5) * cos(\drehwinkel)- (-\balkenbreite+0)*sin(\drehwinkel)},{\ystart + (\balkenlaenge+0.5) * sin(\drehwinkel)+ (-\balkenbreite+0)*cos(\drehwinkel)},0) --
 		({\xstart + \balkenlaenge * cos(\drehwinkel)},{\ystart + \balkenlaenge * sin(\drehwinkel)},0)--
 		cycle;
 		%---------End Balken----------
 	
 		\node (pointz1) at ({\xstartdraw},{\ystartdraw}) {};
 		
 		
        \def\cradius{2.5}
 		
 		\draw[color=green] (pointz1) circle [x=1cm,y=1cm,radius=\cradius cm]node[below, xshift=5, yshift=0]{};
 		
 		\node (pointa1) at ($(pointz1) + (210:\cradius)$) {};
 		%\node (pointb1) at ($(pointz1) + (140:2.5)$) {};
 		\node (pointb1) at ($(pointz1) + (-10:\cradius)$) {};
 		\node (pointc1) at ($(pointz1) + (80:\cradius)$) {};
 		%\node (pointb12) at ($(pointz1) + (140:3.0)$) {};
 		%\node (pointb2) at ($(pointz1) + (140:-2.0)$) {};
 		\node (pointsa1b1) at ($(pointa1)!(pointz1)!(pointb1)$) {};
 		\node (pointsb1c1) at ($(pointb1)!(pointz1)!(pointc1)$) {};
 		\node (pointsa1c1) at ($(pointa1)!(pointz1)!(pointc1)$) [] {};
 		
 		\node (offset_for_p) at ($(pointsa1b1) - (pointz1)$){};
 		\node (pointp) at ($(pointz1) - (offset_for_p)$){};
 		
  		
 	
 		\draw[name path=a1--b1,-,shorten >=-30pt, shorten <=-30pt,line width=0.1pt,color=red] (pointa1) -- (pointb1);
 	    \draw[name path=c1--b1,-,shorten >=-30pt, shorten <=-30pt,line width=0.1pt,color=red] (pointc1) -- (pointb1);
 	    \draw[name path=a1--c1,-,shorten >=-30pt, shorten <=-30pt,line width=0.1pt,color=red] (pointa1) -- (pointc1);
 	    
 	    \draw [color=red, -, line width=0.4pt, shorten >=-30mm, shorten <=-15mm, ] (pointsa1b1) -- (pointz1);
 	    \draw [color=red, dotted, line width=0.3pt, shorten >=-5mm, shorten <=-13mm, ] (pointsb1c1) -- (pointz1);
 	    \draw [color=red, dotted, line width=0.3pt, shorten >=-5mm, shorten <=-20mm, ] (pointsa1c1) -- (pointz1);
 	    
 	    %Vektoren blau
 	    \draw[name path=z1--sb1c1,-{>[scale=1,length=8,width=8]},shorten >=-0pt, shorten <=0pt,line width=0.2pt,color=blue] (pointsa1b1) -- (pointsb1c1);
 	    
 	    \draw[name path=z1--sb1c1,-{>[scale=1,length=8,width=8]},shorten >=-0pt, shorten <=0pt,line width=0.2pt,color=blue] (pointsa1b1) -- (pointp);
 	    
 	     \draw[name path=z1--sb1c1,-{>[scale=1,length=8,width=8]},shorten >=4pt, shorten <=4pt,line width=0.2pt,color=blue] ($(pointa1) + 0.1*(offset_for_p)$) -- ($(pointb1) + 0.1*(offset_for_p)$);
 	    
 	    %rechter Winkel
 	    \draw[line width=0.2pt,color=blue] ($(pointsa1b1) + (100:0.3)$) arc[radius=0.3, start angle=100, end angle=190] ($(pointsa1b1) + (225:0.3)$);
 	    
 	    \draw[line width=0.2pt,color=blue] ($(pointsa1c1) + (60:0.3)$) arc[radius=0.3, start angle=60, end angle=140] ($(pointsa1c1) + (140:0.3)$);
 	    
 		%Punkte malen
 		%rechter Winkel
 		\node (pointrw1) at ($(pointsa1b1) + (145:0.15)$) {};
 		\node (pointrw2) at ($(pointsa1c1) + (100:0.15)$) {};
 		
 		
 		\draw[fill,color=blue] (pointrw1) circle [radius=0.02]node[above, xshift=0, yshift=0]{};
 		\draw[fill,color=blue] (pointrw2) circle [radius=0.02]node[above, xshift=0, yshift=0]{};
 		
 		\draw[fill,color=white] (pointp) circle [radius=0.11] node[below, xshift=5, yshift=0]{};\textbf{}
 		\draw[fill,color=white] (pointz1) circle [radius=0.11] node[below, xshift=5, yshift=0]{};
 		\draw[fill,color=white] (pointa1) circle [radius=0.11] node[below, xshift=5, yshift=0]{};
 		\draw[fill,color=white] (pointb1) circle [radius=0.11] node[below, xshift=5, yshift=0]{};
 		\draw[fill,color=white] (pointc1) circle [radius=0.11] node[below, xshift=5, yshift=0]{};
 		\draw[fill,color=white] (pointsb1c1) circle [radius=0.11] node[below, xshift=5, yshift=0]{};
 		\draw[fill,color=white] (pointsa1b1) circle [radius=0.11] node[below, xshift=5, yshift=0]{};
 		
 
        \draw[fill,color=red] (pointp) circle [radius=0.06]node[below, xshift=5, yshift=0]{};\textbf{}
 		\draw[fill,color=red] (pointz1) circle [radius=0.06]node[below, xshift=5, yshift=0]{};
 		\draw[fill,color=red] (pointa1) circle [radius=0.06]node[below, xshift=5, yshift=0]{};
 		\draw[fill,color=red] (pointb1) circle [radius=0.06]node[below, xshift=5, yshift=0]{};
 		\draw[fill,color=red] (pointc1) circle [radius=0.06]node[below, xshift=5, yshift=0]{};
 		\draw[fill,color=red] (pointsb1c1) circle [radius=0.06]node[below, xshift=5, yshift=0]{};
 		\draw[fill,color=red] (pointsa1b1) circle [radius=0.06]node[below, xshift=5, yshift=0]{};
 		
 		%Beschriftung der Punkte
 		\node[ xshift=-3mm, yshift=1mm,color=red] (labela1) at (pointa1) {$a$};
 		\node[ xshift=2mm, yshift=3mm,color=red] (labelb1) at (pointb1) {$b$};
 		\node[ xshift=0mm, yshift=3.5mm,color=red] (labelc1) at (pointc1) {$c$};
 		\node[ xshift=2mm, yshift=-2mm,color=red] (labelz) at (pointz1) {\small $z$};
 		\node[ xshift=3mm, yshift=0mm,color=red] (labelz) at (pointp) {$p$};
 		\node[ xshift=4mm, yshift=-1mm,color=red] (labelsbc) at (pointsb1c1) {$s_{bc}$};
 		\node[ xshift=13mm, yshift=-4mm,color=red,rotate=-5] (labelsab) at (pointsa1b1) {\small $s_{ab} = a\frac{1}{2}+b\frac{1}{2} $};
 		\node[ xshift=-4mm, yshift=0mm,color=red,rotate=0] (labelsab) at ($(pointsa1b1) + 1.6*(offset_for_p) $) {{\small $m_{ab}$}};
 		%\node[ xshift=-4mm, yshift=0mm,color=red] (labelsab) at (pointsa1c1) {$s_{ca}$};
 		
 		%Beschriftung der Vektoren
 		\node[ xshift=2mm, yshift=-3mm,color=blue,rotate=10] (labelbma) at ($(pointa1)!0.3!(pointb1)$) {\small $b-a$};
 		\node[ xshift=7mm, yshift=-2mm,color=blue] (labelbma) at ($(pointsa1b1)!0.5!(pointsb1c1)$) {\small $\frac{1}{2}(c-a)$};
 		\node[ xshift=-7mm, yshift=-3mm,color=blue,rotate=0] (labelbma) at ($(pointz1)$) {\small $p-s_{ab}$};
 		
 		\node[ xshift=10mm, yshift=-1mm,color=green] (labelkreis) at (pointc1) {$k$};
\end{tikzpicture}
    	\end{figure}
    	%------------------ WinkeltreuAberNichtAbstandstreu ----------------
    \end{minipage}		
	
\paragraph{Beweis}
	Definiere
		\[ g_{ab} : A^2\to \R, p\mapsto g_{ab}(p):= 2\Skl{p-s_{ab}}{b-a} \]
	und analog $ g_{bc} $ und $ g_{ca} $ (zyklische Vertauschung).
	Für $ p\in A^2 $ gilt dann mit
		\[ g_{ab}(p) \overset{!}{=} \Skl{(p-a)+(p-b)}{(p-a)-(p-b)}  = \|p-a\|^2-\|p-b\|^2 \tag{$ * $} \]
	damit folgt
		\[ \forall p\in A^2: (g_{ab}+g_{bc}+g_{ca})(p) = 0, \]
	also
		\[ p\in m_{ab}\cap m_{bc} \Rightarrow p\in m_{ca}. \]
	Nun ist
		\[ m_{ab} = \{p(x) = s_{ab}+J(b-a)x\mid x\in \R\} \]
	mit $ J(b-a)\not\perp b-c $, da das Dreieck $ \{a,b,c\} $ nicht-degeneriert ist. Dies liefert einen eindeutigen Schnittpunkt $ z\in p(x)\in m_{ab}\cap m_{bc} $ als Lösung der linearen Gleichung
		\[ 0 = g_{bc}(p(x)) = 2\Skl{s_{ab}+J(b-a)x-s_{bc}}{c-b} \]
		\[ = 2\Skl{J(b-a)}{c-b}x+\Skl{a-c}{c-b}. \]
	Wegen $ (*) $ gilt nun für diesen Schnittpunkt $ z $
		\[ \|z-a\| = \|z-b\| = \|z-c\| \tag{$ ** $} \]
	d.h. $ a,b $ und $ c $ liegen auf einem Kreis mit Mittelpunkt $ z $.
	Andererseits: Wegen $ (*) $ impliziert $ (**) $, dass $ z\in m_{ab}\cap m_{bc} $, womit die Eindeutigkeit von $ z $ und damit des Umkreises folgt.

\subsection{Höhensatz}\index{Höhen!-schnittpunkt}\index{Höhen}
\begin{Satz}[Höhensatz]
	Die \emph{Höhen} $ h_a,h_b $ und $ h_c $ eines nicht-degenerierten Dreiecks $ \{a,b,c\} \subset A^2 $ schneiden sich in einem Punkt, dem \emph{Höhenschnittpunkt}, wobei
		\[ h_a = \{p\in A^2\mid \Skl{p-a}{b-c} = 0 \},\text{ etc.} \]
\end{Satz}
	Beweis in der Übung, analog zum Umkreissatz.

\subsection{Euler-Gerade}\index{Euler-Gerade}
\begin{Satz}[Euler-Gerade]
	Seien $ s, h $ und $ z $ Schwerpunkt, Höhenschnittpunkt und Umkreismittelpunkt eines nicht-degenerierten Dreiecks $ a,b,c\subset A^2 $.
	Dann gilt
		\[ s=z\frac{2}{3}+h\frac{1}{3}. \]
	Ist $ s\neq z $, so liegen die drei Punkte also auf einer eindeutig bestimmten Geraden, \emph{Euler-Geraden}, mit einem Teilverhältnis $ (zs:hs)=-\frac{1}{2} $.
\end{Satz}
	Beweis in der Übung.

\subsection{Satz von Pythagoras}
\begin{Satz}[Satz von Pythagoras]
	 In einem Dreieck $ \{a,b,c\}\subset A^2 $ mit einem rechten Winkel $ \alpha = \frac{\pi}{2} $ bei $ a $ gilt stets
		 \[ \|c-a\|^2+\|a-b\|^2 = \|c-b\|^2. \]
\end{Satz}
\paragraph{Beweis}
	Offenbar gilt $ c-b = (c-a)+(a-b) $, daher
		\[ \|c-b\|^2 = \|c-a\|^2 + 2\Skl{c-a}{a-b}+\|a-b\|^2 = \|c-a\|^2+\|a-b\|^2. \]
\paragraph{Bemerkung}
	Für allgemeine Dreiecke liefert die gleiche Rechnung den Cosinussatz:
		\[ \|b-c\|^2 = \|c-a\|^2+\|a-b\|^2- 2\|c-a\|\|a-b\|\cos \alpha. \]
\paragraph{Bemerkung}
	Ist $ (o;e_1,e_2) $ ein affines Bezugssystem in $ A^2 $ mit
		\[ e_1 \perp e_2 \text{ und } \|e_1\| = \|e_2\| = 1, \]
	so ist jeder Punkt $ a\in A^2 $ Eckpunkt eines \emph{rechtwinkligen} Dreiecks
		\[ \{o,i+e_1x_1,o+e_1x_1+e_2x_2\} \text{ für } a = o+e_1x_1+e_2x_2; \]
	der Abstand vom Ursprung ist also (Pythagoras)
		\[ \|a-o\| = \sqrt{x_1^2+x_2^2}. \]
	Wegen seiner Translationsinvarianz kann der Abstand zwischen beliebigen Punkten genau so berechnet werden.
	
\subsection{Definition}\index{Kartesisches Bezugssystem}
\begin{Definition}[Kartesisches Bezugssystem]
	Ein \emph{kartesisches Bezugssystem} $ (o,E) $ eines Euklidischen Raumes $ (A,V,\tau) $ über einem Euklidischen VR $ (V,\Skl{.}{.}) $ besteht aus einem Ursprung $ o\in A $ und einer ONB $ E $ von $ (V,\Skl{.}{.}) $.
\end{Definition}
\paragraph{Bemerkung}
	In jedem endlichdimensionalen Euklidischen Raum gibt es ein kartesisches Bezugssystem, im Allgemeinen ist dies nicht so (vgl. Abschnitt 5.2).
	
\subsection{Lemma}
\begin{Lemma}[]
	Ist $ (o;E) $ mit $ E=(e_i)_{i\in I} $ kartesisches Bezugssystem eines Euklidischen Raumes $ (A,V,\tau) $ über $ (V,\Skl{.}{.}) $, so ist 
		\[ \forall a\in A: a = o + \sum_{i\in I} e_i \Skl{e_i}{a-o} \]
\end{Lemma}
\paragraph{Beweis}
	Da $ E $ Basis ist, existiert zu $ a\in A $ eine Familie $ (x_i)_{i\in I} $ in $ \R $ mit
		\[ a = o + \sum_{i\in I}e_ix_i, \]
	wobei
		\[ \forall i\in I: \Skl{e_i}{a-o} = \Skl{e_i}{\sum_{j\in I}e_jx_j} = \sum_{j\in I}\delta_{ij}x_j = x_i. \]
\printindex
\end{document}
